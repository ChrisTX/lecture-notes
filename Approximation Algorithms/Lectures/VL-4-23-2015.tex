\documentclass[../skript.tex]{subfiles}

\begin{document}
The \emph{complement} of a decision problem $P \subseteq \{ 0, 1\}^*$ is the decision problem $\{ 0, 1\}^* \setminus P$.
The class \coNP is the class of all decision problems, \ac{st} their complement is in \NP.
Famous example: \emph{Prime?} \\
\begin{tabular}{ll}
\textit{Input:} & $n \in \N$ \\
\textit{Question:} & Is $n$ prime?
\end{tabular} \par
Prime? is in \NP{} [Pratl, 1975].
A problem in $\NP \cap \coNP$ is said to have a \emph{good characterization}.
Prime? is also in $P$ [Agrawal, Kayal, Saxena, 2004].

A \emph{discrete optimization problem} is a four-tuple $(X, S, c, \goal)$ with
\begin{itemize}
\item $X \subseteq \{ 0, 1\}^*$ can be decided in polynomial time. The elements of $X$ are \emph{instances}.
\item $S \, : \, X \to 2^{\{ 0, 1\}^*}$ where $\emptyset \neq S(x) \subseteq \{ 0, 1\}^*$ is the set of \emph{solutions} of instance $x$, \ac{st} there exists a polynomial $p$ with
\[
	|y| \leq p(|x|) \quad \forall y \in S(x) \, \forall x \in X.
\]
\item $c : \{ (x, y) \mid x \in X, y \in S(x)\} \to \Z$ is a function computable in polynomial time.
\item $\goal \in \{ \min, \max \}$.
\end{itemize}
We denote by \emph{$\OPT(x)$} the value $\goal \{ c((x, y)) \mid y \in S(x) \}$.
An \emph{optimum solution} for an instance $x$ is a $y \in S(x)$ with $c((x, y)) = \OPT(x)$. An algorithm for an optimization problem $(X, S, c, \goal)$ is an algorithm $A$ which computes for each input $x \in X$ a $y\in S(x)$.
We write $\emph{A(x)} \coloneqq c((x, y))$ if $A$ returns $y$ as a solution to $x$.
If $A(x) = \OPT(x)$ for all $x \in X$ then $A$ is an \emph{exact algorithm}.

An algorithm is called \emph{pseudo-polynomial} if its runtime is polynomially bounded (in the largest number appearing in the input) if the input is given unary.
\begin{theorem} % Thm 19
\label{thm:19}
There exists a pseudo-polynomial algorithm for \textsc{Subset Sum}.
\end{theorem}
\begin{proof}
See the algorithm for \textsc{Component Grouping}, with runtime $\mathcal{O}(n \cdot K)$, where $K$ is the size of the largest number in unary notation.
\end{proof}
An \NP-complete (\NP-hard) problem is \emph{strongly \NP-complete (\NP-hard)} if it stays \NP-complete (\NP-hard) if the input is given unary.
\begin{theorem} % Thm 20
\label{thm:20}
Unless $P = \NP$ there exists no exact pseudo-polynomial algorithm for a strongly \NP-complete (\NP-hard) problem.
\end{theorem}
\begin{proof}
Follows from the definition.
\end{proof}
\chapter{Approximation Algorithms} % Ch 2
\label{ch:2}
An \emph{approximation algorithm} $A$ for an optimization problem $P$ is a polynomial time algorithm for $P$.
An algorithm $A$ has \emph{approximation ratio $c$} if
\begin{IEEEeqnarray*}{rCl'l}
	A(I) &\leq& c \cdot \OPT(I), & P \text{ is a minimization problem}, \\
	A(I) &\geq& c \cdot \OPT(I), & P \text{ is a maximization problem}.
\end{IEEEeqnarray*}
If $P$ is a minimization problem, then $1 \leq c < \infty$. If $P$ is a maximization problem, then $0 \leq c \leq 1$ (sometimes \sfrac{1}{$c$} is being used instead in order to always use a number greater than one).
\begin{problem}[Minimum Weight Set Cover Problem]
\begin{tabular}{ll}
\textit{Input:} & A family $S_1, \ldots, S_m$ of subsets of a finite set $U$ with $\bigcup_{i=1}^m S_i = U$ and \\
& weights $c : \{ S_1, \ldots, S_m \} \to \R_+$.\\
\textit{Output:} & A subfamily $S_{i_1}, \ldots, S_{i_k}$ \ac{st} $\bigcup_{j=1}^k S_{i_j} = U$ and \\
& $\sum_{j=1}^k c(S_{i_j})$ is minimum.
\end{tabular}
\end{problem}
\begin{algorithm}[Greedy Weighted Set Cover]
\begin{algorithmic}[1]
\State $C \coloneqq \emptyset$, $F \coloneqq \emptyset$
\While{$C \neq U$}
\State Choose $S \in \{S_1, \ldots, S_m\}$ \ac{st} $\frac{c(S)}{|S\setminus C|}$ is minimum.
\State $C \coloneqq C \cup S$
\State $F \coloneqq F \cup \{ S \}$
\EndWhile
\end{algorithmic}
\end{algorithm}
\begin{theorem} % Thm 21
\label{thm:21}
The \textsc{Greedy Weighted Set Cover} algorithm has approximation ratio
\[
	H_n \coloneqq \sum_{i=1}^n \frac{1}{i},
\]
where $n \coloneqq |U|$.
\end{theorem}
\begin{proof}
\ac{wlog} assume that $F = \{ S_1, \ldots, S_k \}$ is the solution of the greedy algorithm in this ordering. For each $x \in U$ assign the cost
\[
\tilde{c}(x) \coloneqq \frac{c(S_i)}{|S_i \setminus (S_1 \cup \ldots \cup S_{i-1})|},
\]
where $i$ is chosen minimum \ac{st} $x \in S_i$. \\
\underline{Claim:} For $S \in \{S_1, \ldots, S_m\}$ we have
\[
	\sum_{x \in S} \tilde{c}(x) \leq c(s) \cdot H_{|S|}.
\]
Using this claim, we get for an optimum solution $F^*$:
\begin{IEEEeqnarray*}{rCl}
	\sum_{S \in F} c(S) &=& \sum_{x \in U} \tilde{c}(x) \\
	&\leq& \sum_{S \in F^*} \underbrace{ \sum_{x \in S} \tilde{c}(x) }_{\text{Claim: }\leq c(S) \cdot H_{|S|}} \\
	&\leq& \sum_{S \in F} c(S) \cdot H_{|S|} \\
	&\leq& H_{n} \cdot \sum_{S \in F^*} c(S).
\end{IEEEeqnarray*}
\begin{proof}[Claim]
Let $S = \{ x_1, \ldots, x_p \}$ be one of the sets $\{ S_1, \ldots, S_m\}$. Assume $x_i$ are ordered in the way they are covered by the greedy algorithm. The greedy algorithm could have taken $S$ to cover element $x_i$.
Hence,
\[
	\tilde{c}(x_i) \leq \frac{c(S)}{|S| - (i-1)},
\]
as we know that at least $|S| - (i-1)$ elements of $S$ are not yet covered.
Thus,
\begin{IEEEeqnarray*}{rCl}
	\sum_{x \in S} \tilde{c}(x) &\leq& \sum_{i = 1}^{|S|} \frac{c(S)}{|S| - (i-1)} \\
	&\leq& c(S) \cdot \sum_{i=1}^{|S|} \frac{1}{i} \\
	&=& c(S) \cdot H_{|S|}.
\end{IEEEeqnarray*}
This proves the claim.
Due to the proof of the claim, the theorem has been proven.
\end{proof}
\NoEndMark
\end{proof}
The result is best possible:
Take $S = \{ 1, \ldots, n \}$ and let $S_i = \{ i \}$ with costs $\frac{1}{i}$. Additionally add $S_{n+1} = S$ with weight $1 + \varepsilon$. The greedy algorithm will select all singletons, whereas $S_{n+1}$ would be optimal. This gives an approximation ratio of $H_n$.
The same boundary can be proven for the unweighted case. \\
\begin{remark}
\begin{itemize}
\item Unless $P = \NP$ no \textit{better} algorithm exists.
\item The proof shows:
\[
	\sum_{S \in F} c(S) \leq H_l \cdot \sum_{S \in F^*} c(S),
\]
where
\[
	l \coloneqq \max_{S \in \{S_1, \ldots, S_m\}} \{ |S| \}.
\]
\item Runtime: $\mathcal{O}(m^2 + \sum_{i=1}^m |S_i|)$.
\item Negative weights? Add all sets with negative weight. In general, we cannot find a boundary on the approximation ratio, as the optimum solution could have weight zero, in which case we would have to find an optimum solution.
\item Special case: $|S_i| = 2$: \textsc{Edge Cover} can be solved exactly in polynomial time (\textsc{Minimum-Weight-Perfect-Matching}).
\item Special case: \textsc{Minimum Size Vertex Cover}.
\end{itemize}
\end{remark}
\begin{theorem} % Thm 22
There exists a 2-approximation algorithm for \textsc{Minimum Size Vertex Cover}.
\end{theorem}
\begin{proof}
Compute a \textit{maximal} matching $M$. Set
\[
	S \coloneqq \{ x \in V(G) \mid \exists \{ x, y\} \in M\}.
\]
Then $S$ is a vertex cover and $|S| = 2 \cdot |M|$. \\
\underline{Observation:} $\OPT \geq |M|$, as not two edges in a matching are incident.
Thus, $|S| \leq 2 \cdot \OPT$.
\end{proof}
\begin{remark}
\begin{itemize}
\item This is the best ratio known for \textsc{Minimum Size Vertex Cover}.
\item $1.36$ is a lower bound unless $P = \NP$ [Dinur, Sufu, 2005].
\item $2-\varepsilon$ is a lower bound if the unique games conjecture is true [Knot, Reger, 2008].
\end{itemize}
\end{remark}
\begin{problem}[$k$-Center]
\begin{tabular}{ll}
\textit{Input:} & A finite set $X$, a metric $\dist : X \times X \to \R$, $k \in \N$. \\
\textit{Output:} & A set $C \subseteq X$ with $|C| = k$, \ac{st} \\
&$\max_{x \in X} \min_{c \in C} \{\dist(x, c)\}$ is minimum.
\end{tabular}
\end{problem}
\textsc{$k$-Center} can be solved in polynomial time for fixed $k$: enumerate all $C$. There are
\[
	\begin{pmatrix}
	|X| \\ k
	\end{pmatrix} 
\]
possibilities, thus we have a runtime of $\mathcal{O}(|X|^{k+1} k)$.
View this as a problem on graphs:
\begin{IEEEeqnarray*}{rCl}
X \, &:& \, \text{vertices of a complete graph } G. \\
\dist \, &:& \, E(G) \to \R_{\geq 0} \text{ metric}.
\end{IEEEeqnarray*}
Let $G = (V, E)$ be a graph. $D \subseteq V(G)$ is a \emph{dominating set} if
\[
	\forall v \in V(G) \setminus D \, \exists d \in D \, : \, \{ v, d \} \in E(G).
\]
\end{document}