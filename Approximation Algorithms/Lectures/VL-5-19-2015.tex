\documentclass[../skript.tex]{subfiles}

\begin{document}
\begin{samepage}
\begin{algorithmbox}[Next Fit]
\begin{tabular}{@{}ll}
\textit{Input:} & $0 \leq a_1, \ldots, a_n \leq 1$. \\
\textit{Output:} & $k \in \N$, $f : \{ 1, \ldots, n\} \to \{ 1, \ldots, k\}$, \ac{st}\ $\sum_{f(i) = j} a_i \leq 1 \;\; \forall j$.
\end{tabular}
\end{algorithmbox}
\vspace{-7pt}
\begin{algorithm}[H]
$k \coloneqq 1$, $S \coloneqq 0$\;
\For{$i \coloneqq 1$ \KwTo $n$}{
	\lIf{$S + a_i > 1$}{
		$k \coloneqq k + 1$, $S \coloneqq 0$
	}
	$f(i) \coloneqq k$, $S \coloneqq S + a_i$.
}
\end{algorithm}
\vspace{-7pt}
\EndAlgorithmLine
\end{samepage}
For an instance $I = 0 \leq a_1, \ldots, a_n \leq 1$, we use the notation $\NF(I)$ to denote the number of bins \textsc{Next Fit} returns.
\begin{theorem} % Thm 52
\label{thm:52}
\textsc{Next Fit} computes in linear time a solution with
\[
	\NF(I) \leq 2 \cdot \left\lceil \sum_{i=1}^n a_i \right\rceil - 1 \leq 2 \cdot \OPT - 1.
\]
\end{theorem}
\begin{proof}
The runtime is obvious. Set $k \coloneqq \NF(I)$. For $j = 1, \ldots, \left\lfloor \frac{k}{2} \right\rfloor$, we have:
\[
	\sum_{f(i) \in \{ 2j, 2j - 1 \}} a_i > 1.
\]
Therefore:
\[
	\sum_{i=1}^n a_i \geq \sum_{j=1}^{\left\lfloor \frac{k}{2} \right\rfloor} \sum_{f(i) \in \{ 2j, 2j-1 \}} a_i > \left\lfloor \frac{k}{2} \right\rfloor. 
\]
The right hand side is integral and thus we have
\[
	\left\lceil \sum_{i=1}^n a_i \right\rceil - 1 \geq \left\lfloor \frac{k}{2} \right\rfloor \geq \frac{k-1}{2}.
\]
\end{proof}
\begin{remark}
The analysis is best possible: For the instance $a_1, \ldots, a_{2n}$ where
\begin{IEEEeqnarray*}{rClClClCl}
a_1 &=& a_3 &=& a_5 &=& \ldots &=& \frac{1}{n} \\
a_2 &=& a_4 &=& a_6 &=& \ldots &=& 1-\frac{1}{2n}
\end{IEEEeqnarray*}
Then $\NF(I) = 2n$, but $\OPT = n + 1$.
\end{remark}
\textsc{Next Fit} is ``better'' if all items are small:
\begin{theorem} % Thm 53
\label{thm:53}
Let $0 < \gamma < 1$ and $I = a_1, \ldots, a_n$ be an instance with $a_i \leq \gamma$ for all $i$.
Then
\[
	\NF(I) \leq \left\lceil \frac{\sum_{i=1}^n a_i}{1 - \gamma} \right\rceil \leq \left\lceil \frac{\OPT(I)}{1-\gamma} \right\rceil.
\]
\end{theorem}
\begin{proof}\tabularnewline
We have
\[
	\sum_{i \midcolon f(i) = j} a_i > 1-\gamma \quad \text{for } j = 1, \ldots, \NF(I) -1.
\]
This implies
\[
	\sum_{i=1}^n a_i > (\NF(I) - 1)(1-\gamma),
\]
and thus
\[
	\NF(I) - 1 < \left\lceil \frac{\sum_{i=1}^n a_i}{1-\gamma} \right\rceil.
\]
\end{proof}
Is \textsc{Bin Packing} hard if $a_i < \gamma$ for all $i$, $\gamma$ fixed? No, it remains \NP-hard: Use a reduction from \textsc{Partition} as in the proof of \cref{thm:50}. Scale all $a_i$ by $\frac{1}{8k + 2}$ for some integer $k$ and add $2k$ items of size $\frac{4}{4k+1}$ each. Then the total size of all items is 2, each item has size at most $\frac{4}{4k + 1}$ and the \textsc{Partition} instance has a solution if any only if the \textsc{Bin Packing} instance has a solution with 2 bins.

\underline{Exercise:} \textsc{Bin Packing} is in $P$ if $a_i \geq \frac{1}{3}$. For $a_i > \frac{1}{2}$ it is obviously solvable in linear time.
\begin{samepage}
\begin{algorithmbox}[First Fit]
\begin{tabular}{@{}ll}
\textit{Input:} & $0 \leq a_1, \ldots, a_n \leq 1$. \\
\textit{Output:} & $k \in \N$, $f : \{ 1, \ldots, n\} \to \{ 1, \ldots, k\}$, \ac{st}\ $\sum_{f(i) = j} a_i \leq 1 \;\; \forall j$.
\end{tabular}
\end{algorithmbox}
\vspace{-7pt}
\begin{algorithm}[H]
\For{$i \coloneqq 1$ \KwTo $n$}{
	$f(i) \coloneqq \min \left\{ j \in \N \midcolon a_i + \sum_{\substack{f(l) = j \\ l < i}} a_l \leq 1 \right\}$.\;
}
\end{algorithm}
\vspace{-7pt}
\EndAlgorithmLine
\end{samepage}
We use $\FF(I)$ to denote the number of bins \textsc{First Fit} requires. Obviously, $\FF(I) \leq \NF(I)$.
It can be shown that
\[
	\FF(I) \leq \left\lceil \frac{17}{10} \cdot \OPT(I) \right\rceil.
\]
Furthermore, one can construct instances such that
\[
	\FF(I) \geq \frac{17}{10} (\OPT(I) - 1).
\]
\begin{samepage}
\begin{algorithmbox}[First Fit Decreasing]
\begin{tabular}{@{}ll}
\textit{Input:} & $0 \leq a_1, \ldots, a_n \leq 1$. \\
\textit{Output:} & $k \in \N$, $f : \{ 1, \ldots, n\} \to \{ 1, \ldots, k\}$, \ac{st}\ $\sum_{f(i) = j} a_i \leq 1 \;\; \forall j$.
\end{tabular}
\end{algorithmbox}
\vspace{-7pt}
\begin{algorithm}[H]
Sort the $a_i$ \ac{st}\ $a_1 \geq a_2 \geq \ldots \geq a_n$.\;
Apply \textsc{First Fit}.
\end{algorithm}
\vspace{-7pt}
\EndAlgorithmLine
\end{samepage}
For this algorithm, we use $\FFD(I)$ to denote its number of bins used.
\begin{theorem} % Thm 54
\label{thm:54}
\textsc{First Fit Decreasing} has an approximation ratio of \sfrac{3}{2}.
\end{theorem}
\begin{proof}
Let $I$ be a \textsc{Bin Packing} instance and $k \coloneqq \FFD(I)$.
Consider the $j$-th bin used by the algorithm with $j \coloneqq \left\lceil \frac{2}{3} k \right\rceil$.

\underline{Case 1:} The $j$-th bin contains an item of size greater than $\frac{1}{2}$. Because of the sorting we can conclude that there was not enough space in any of the earlier bins $1, \ldots, j -1$, which have to contain at least one item of size greater than $\frac{1}{2}$. Thus, $\OPT(I) \geq j \geq \frac{2}{3} k$.

\underline{Case 2:} The $j$-th bin does \textit{not} contain an item of size greater than $\frac{1}{2}$.
Again we can conclude due to the sorting step that the bins $j, j+1, \ldots, k$ do not contain such an item, either.
Hence, the bins $j, j + 1, \ldots, k$ have to contain at least $2(k-j) + 1$ items. We observe that
\[
	2 \left( k - \left\lceil \frac{2}{3} k \right\rceil \right) + 1 \geq \left\lceil \frac{2}{3} k \right\rceil -1.
\]
Hence
\[
	\OPT(I) \geq \sum_{i=1}^n a_i > j - 1 = \left\lceil \frac{2}{3} k \right\rceil - 1,
\]
giving us,
\[
	\OPT(I) \geq \left\lceil \frac{2}{3} k \right\rceil \geq \frac{2}{3} k.
\]
\end{proof}
There exists a linear time approximation algorithm for \textsc{Bin Packing} with ratio \sfrac{3}{2} (Berghammer, Reuter [2003]). One can show (Dósa [2007])
\[
	\FFD(I) \leq \frac{11}{9} \cdot \OPT(I) + \frac{2}{3}.
\]
Moreover, one can show that
\[
	\FFD(I) \geq \frac{11}{9} \cdot \OPT(I),
\]
consider the following example: Let $I = \{ a_1, \ldots, a_{30m} \}$, $\varepsilon > 0$ and set
\begin{IEEEeqnarray*}{rCl}
a_1, \ldots, a_{6m} &=& \frac{1}{2} + \varepsilon \\
a_{6m+1}, \ldots, a_{12m} &=& \frac{1}{4} + 2 \varepsilon \\
a_{12m+1}, \ldots, a_{18m} &=& \frac{1}{4} + \varepsilon \\
a_{18m+1}, \ldots, a_{30m} &=& \frac{1}{4} - 2 \varepsilon
\end{IEEEeqnarray*}
We can pack these optimally using $9m = 6m + 3m$ many bins:
\begin{IEEEeqnarray*}{u"l}
$6m$ bins containing & \frac{1}{2} + \varepsilon, \frac{1}{4} + \varepsilon, \frac{1}{4} - 2 \varepsilon \\
$3m$ bins containing & \frac{1}{4} + 2 \varepsilon, \frac{1}{4} + 2 \varepsilon, \frac{1}{4} - 2 \varepsilon, \frac{1}{4} - 2 \varepsilon
\end{IEEEeqnarray*}
However, $\FFD(I) = 11m$:
\begin{IEEEeqnarray*}{u"l}
$6m$ bins containing & \frac{1}{2} + \varepsilon, \frac{1}{4} + 2 \varepsilon\\
$2m$ bins containing & \frac{1}{4} + \varepsilon, \frac{1}{4} + \varepsilon, \frac{1}{4} + \varepsilon\\
$3m$ bins containing & \frac{1}{4} - 2 \varepsilon, \frac{1}{4} - 2 \varepsilon, \frac{1}{4} - 2 \varepsilon, \frac{1}{4} - 2 \varepsilon
\end{IEEEeqnarray*}
Note that \textsc{Next Fit} and \textsc{First Fit} are \emph{online algorithms}, whereas \textsc{First Fit Decreasing} is not due to the sorting step. The best rate for an online algorithm is $1.59$ and one can prove a lower bound of $1.54$ for online algorithms.

An \emph{\ac{APTAS}} (or \emph{asymptotic approximation scheme}) is a family of approximation algorithms $A_\varepsilon$, $\varepsilon > 0$, \ac{st}
\begin{IEEEeqnarray*}{rCl"u}
A_\varepsilon(I) &\leq& (1 + \varepsilon) \cdot \OPT(I) + c_\varepsilon & (for minimization problems) \\
A_\varepsilon(I) &\geq& (1 - \varepsilon) \cdot \OPT(I) - c_\varepsilon & (for maximization problems)
\end{IEEEeqnarray*}
for all instances $I$ where $c_\varepsilon$ may only depend of $\varepsilon$.

Our goal is now to find a $(1 + \varepsilon) \cdot \OPT(I) + 1$ approximation for \textsc{Bin Packing}. Let
\begin{IEEEeqnarray*}{r"s}
\BP_\delta & \textsc{Bin Packing} problem where all items have size greater or equal to $\delta$. \\
\BP_{\delta, m} & \textsc{Bin Packing} problem where all items have size greater or equal to $\delta$ \\ 
& and at most $m$ different item sizes appear.
\end{IEEEeqnarray*}
\begin{theorem} % Thm 55
\label{thm:55}
For fixed $\delta$ and $m$ there exists a polynomial time algorithm for $\BP_{\delta, m}$.
\end{theorem}
\begin{proof}
All items are greater or equal to $\delta$ and therefore each bin contains at most $\frac{1}{\delta}$ items and at most $m$ different values appear.
Thus, there are at most $(m+1)^\frac{1}{\delta}$ many possibilities to pack a bin.
An optimum solution can be interpreted as a vector of length $(m+1)^\frac{1}{\delta}$ with entries between $0$ and $n$. This gives us
\[
\mathcal{O} \left( (n+1)^{\left( (m+1)^\frac{1}{\delta} \right)} \right)
\]
possibilities. This requires a runtime of
\[
\mathcal{O} \left( n^{\mathcal{O}\left( m^\frac{1}{\delta} \right)} \right).
\]
\end{proof}
\begin{theorem} % Thm 56
\label{thm:56}
There exists an \ac{APTAS} $A_\varepsilon^\delta$ for $\BP_\delta$ with $A_\varepsilon^\delta(I) \leq (1+ \varepsilon) \cdot \OPT(I) + 1$. 
\end{theorem}
\end{document}