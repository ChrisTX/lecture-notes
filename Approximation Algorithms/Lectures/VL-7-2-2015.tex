\documentclass[../skript.tex]{subfiles}

\begin{document}
\begin{theorem} % Thm 96
\label{thm:96}
Let $V \subset [0, 2^N] \times [0, 2^N]$ be a well-rounded \textsc{Euclidean TSP} instance.
If $a$ and $b$ are randomly chosen from $\{ 0, 2, \ldots, 2^N - 2 \}$, then with probability at least \sfrac{1}{2} there exists a light Steiner tour of length at most $(1 + \varepsilon) \cdot \OPT(V)$.
\end{theorem}
\begin{proof}
Let $T$ be an optimum TSP tour for $V$. Add a Steiner point whenever we cross a grid line.
Move the Steiner points to portals.
The nearest portal for a grid line in level $i$ has distance at most $\frac{2^{N-i-1}}{p}$.
Consider a line $\ell \in G_{N-1}$ which is in level $i$ with probability
\[
	p(\ell, i) \coloneqq \begin{cases}
	2^{i-N} & \text{if } i > 1 \\
	2^{2-N} & \text{if } i = 1
	\end{cases}.
\]
In $G_i$, there are $2^i$ lines, of which half are in the $i$th level, thus we have a probability of
\[
	\frac{2^{i-1}}{2^{N-1}} = 2^{i-N}.
\]
The expected increase of the length for a line $\ell$ is:
\begin{IEEEeqnarray*}{rCl}
\sum_{i=1}^{N-1} p(\ell, i) \cdot \crossings(T, \ell) \cdot 2 \cdot \frac{2^{N-i-1}}{p} &=& N \cdot \frac{\crossings(T, \ell)}{p}
\end{IEEEeqnarray*}

Next modification to get a light Steiner tour:
We define as \emph{segment} of a horizontal or vertical line in $G_i^{(a, b)}$ the segment between
\[
	\left(a + j \cdot 2^{N-i}, b + k \cdot 2^{N-i} \right)
\]
and
\[
	\left( a + (j + 1) \cdot 2^{N-i}, b + k \cdot 2^{N-i} \right)
\]
taken $\operatorname{mod} 2$.

\begin{samepage}
\EndAlgorithmLine
\begin{algorithm}[H]
\For{$i \coloneqq N - 1$ \KwSty{down to} $1$}{
	apply the Patching \cref{thm:95} to each segment of a horizontal line of $G_i^{(a, b)}$ that contains more than $C - 4$ crossings\;
	perform the same action for vertical segments\;
}
\end{algorithm}
\end{samepage}
\EndAlgorithmLine
\textit{Remark:} For segments consisting of $2$ parts, we apply the Patching \cref{thm:95} twice. This gives at most $4$ crossings.

For each line $\ell$ the Patching \cref{thm:95} is applied at most $\frac{\crossings(T, \ell)}{C - 7}$.
Let for a line $\ell$ in $G_{N-1}^{(a, b)}$ denote $c(\ell, i, a, b)$ the number of applications of the Patching \cref{thm:95} in iteration $i$.
Then, $c(\ell, i, a, b) = 0$ if the level of $\ell$ is greater than $i$.
Hence, the total increase in the length of a tour for line $\ell$ is:
\begin{IEEEeqnarray*}{rCl}
\sum_{i \geq \level(\ell)} c(\ell, i, a, b) \cdot 6 \cdot 2^{N-i}
\end{IEEEeqnarray*}
and
\[
	\sum_{i \geq \level(\ell)} c(\ell, i, a, b) \leq \frac{\crossings(T, \ell)}{C - 7}.
\]
Recall that $p(\ell, j)$ was the probability that $\ell$ is in level $j$. This gives us that the total increase is at most
\begin{IEEEeqnarray*}{rCl}
	\sum_{j=1}^{N-1} p(\ell, j) \cdot \sum_{i \geq j} c(\ell, i, a, b) \cdot 6 \cdot 2^{N-i} &=& 6 \cdot \sum_{i=1}^{N-1} c(\ell, i, a, b) \cdot 2^{N-i} \cdot \underbrace{ \sum_{j=1}^i p(\ell, j) }_{\leq 2 \cdot p(\ell, i)} \\
	&\leq& 12 \cdot \frac{\crossings(T, \ell)}{C - 7}
\end{IEEEeqnarray*}
We remove pairs of edges if a corner portal is used more than twice. This reduces amount of crossings appropriately and we have a light Steiner tour.
By \cref{thm:94} we have:
\begin{IEEEeqnarray*}{rCl}
\sum_{\ell \in G_{N-1}} \crossings(T, \ell) &\leq& \OPT(V)
\end{IEEEeqnarray*}
Our total increase of the previous steps can be bounded with this result:
\begin{IEEEeqnarray*}{rCl}
\sum_{\ell \in G_{N-1}} N \cdot \frac{\crossings(T, \ell)}{p} + \sum_{\ell \in G_{N-1}} \frac{12 \cdot \crossings(T, \ell)}{C - 7} &\leq& \OPT(V) \cdot \left( \frac{N}{p} + \frac{12}{C - 7} \right) \\
&\leq& \OPT(V) \cdot \frac{\varepsilon}{2}.
\end{IEEEeqnarray*}
Hence, by Markov's inequality
\[
\Pr[\text{increase} > \varepsilon \cdot \OPT(V)] \leq \frac{1}{2}.
\]
\end{proof}
A region has at most $4 \cdot C$ portals used by the light Steiner tour.
There are $P + 2$ possibilities for a portal: Within a region there are $p + 1 \leq (p+2)^{C4}$ possibilities to choose the portal.
Hence, there are $(4C)!$ possibilities to run through the portals.
\begin{algorithmbox}[Arora's Algorithm]
\begin{tabular}{@{}ll}
\textit{Input:} & A well-rounded \textsc{Euclidean TSP} instance $V \subseteq [0, 2^N] \times [0, 2^N]$, \\
& $0 < \varepsilon < 1$ \\
\textit{Output:} & A tour that has length at most $(1 + \varepsilon) \cdot \OPT(V)$
\end{tabular}
\end{algorithmbox}
\vspace{-7pt}
\begin{algorithm}[H]
Choose $a$ and $b$ randomly from $\{ 0, 2, \ldots, 2^N - 2 \}$.\;
Set $R_0 \coloneqq \{ ([0, 2^N] \times [0, 2^N], V) \}$.\;
\For{$i \coloneqq 1$ \KwTo $N-1$}{
	Compute $G_i^{(a, b)}$, set $R_i \coloneqq \emptyset$.\;
	\For{$(r, Vr) \in R_i$ \KwSty{with} $|V_r| \geq 2$}{
		add $4$ subregions of $r$ to $R_i$\;
	}
}
\For{$i \coloneqq N - 1$ \KwSty{down to} 1}{
	\For{$r \in R_i$}{
		Compute an optimum solution for $(r, Vr)$
	}
}
\end{algorithm}
\vspace{-7pt}
\EndAlgorithmLine
\begin{theorem} % Thm 97
\label{thm:97}
\textsc{Arora's Algorithm} finds with probability at least \sfrac{1}{2} a tour of length at most $(1 + \varepsilon) \cdot \OPT(V)$.
The runtime is $\mathcal{O}(n \cdot (\log n)^\alpha)$, where $\alpha$ is a constant that linearly depends on $\frac{1}{\varepsilon}$, i.e.\ $\mathcal{O}(n \cdot (\log n)^{\mathcal{O}\left( \frac{1}{\varepsilon} \right)})$.
\end{theorem}
\begin{proof}
\underline{Correctness:} $a, b$ are randomly chosen. Thus \cref{thm:96} gives us this result.

\underline{Runtime:} We create a tree rooted at $R_0$ where each node is connected to vertices of the subregions that it was split into in the algorithm.
Denote by $S$ the set of all vertices in the tree that have $4$ children, which are all leaves.
Each such vertex represents a region with at least $2$ points, and thus
\[
	|S| \leq \frac{n}{2},
\]
where $n \coloneqq |V|$.
We have at most $N \cdot \frac{n}{2} + 4 \cdot N \cdot \frac{n}{2} = \frac{5}{2} \cdot N \cdot n$ vertices all together.
For each fixed region, we have $(P+2)^{4C} \cdot (4C)!$ possibilities.
If $|V_r| = 1$, we can compute an optimum solution in $\mathcal{O}(C)$ time.
Otherwise, if $|V_r| > 1$, we have at most $4C$ portals on the inner lines and hence $8C$ portals in total.
We have $(P+2)^{4C}$ possibilities and there are $(8C)!$ possibilities for a tour to run through the portals.

For each region the runtime is at most
\[
	(P+2)^{4C} \cdot (4C)! \cdot (P+2)^{4c} \cdot (8C)!.
\]
We have as already discusses, at most $\frac{5}{2} \cdot N \cdot n$ many regions and had
\[
	N = \mathcal{O} \left( \log \frac{n}{\varepsilon} \right), \quad C = \mathcal{O}\left( \frac{1}{\varepsilon} \right), \quad P = \mathcal{O} \left( \frac{N}{\varepsilon} \right).
\]
By plugging these in, we obtain a total runtime of
\[
\mathcal{O} \left( n \log \frac{n}{\varepsilon} \cdot (P+2)^{8C} \cdot (8C)^{12C} \right) = \mathcal{O} \left( n \cdot (\log n)^{\mathcal{O}\left( \frac{1}{\varepsilon} \right)} \cdot \mathcal{O}\left( \frac{1}{\varepsilon} \right)^{\mathcal{O}\left( \frac{1}{\varepsilon} \right)} \right).
\]
\end{proof}
\begin{theorem} % Thm 98
\label{thm:98}
\textsc{Euclidean TSP} has a \ac{PTAS} with runtime $\mathcal{O}\left(n^{3.01} \right)$.
\end{theorem}
\begin{proof}
For $a$ and $b$ there are $\mathcal{O} \left( \frac{n^2}{\varepsilon^2} \right)$ possibilities. We can try them all.
\end{proof}
This can be generalized to the $d$-dimensional case:
\[
	\mathcal{O}(n \log n^{\mathcal{O} \left( d \cdot \frac{1}{\varepsilon} \right)^d}).
\]
This approach also works for the \textsc{Steiner Tree Problem}.
\chapter{The \texorpdfstring{\PCP{}}{PCP} Theorem} % Ch 7
\label{sec:c7}
So far we have seen for example
\begin{itemize}
\item \ac{PTAS} for \textsc{Euclidean TSP}
\item \sfrac{3}{2} approximation for \textsc{Metric TSP} (lower bound $1 + \varepsilon$)
\item 2-approximation for \textsc{Vertex Cover} (lower bound $1 + \varepsilon$)
\item $\frac{n}{\log n}$-approximation for \textsc{Coloring} (lower bound $n^{1-\varepsilon}$)
\item $\log n$-approximation for \textsc{Set Cover} ($\Omega(\log n)$)
\end{itemize}
Can't we find a PTAS in all these cases? The answer is negative as we will see in this chapter.
\begin{theorem} % Thm 99
\label{thm:99}
$\NP = \PCP(\log n, 1)$.
\end{theorem}
PCP stands for probabilistically checkable proofs. This is a common generalization of \NP{} and \coRP{}.

\NP{}: For a language $L$ and $x \in L$, there exists a certificate $c$ with $|c| \leq \operatorname{poly}(|X|)$ such that Turing machine $M$ accepts $(x, c)$ and if $x \notin L$ for all certificates $c$, Turing machine $M$ rejects $(x, c)$.
\end{document}