\documentclass[../skript.tex]{subfiles}

\begin{document}
\underline{Observation:} We can assume $|\delta(v)| \geq 2 \quad \forall v \in V(G)$. For isolated vertices this is clear, for vertices of degree one, we can replace them by their neighbor, as that neighbor has to be part of any solution.

In the following we assume we have a \emph{metric Steiner tree instance:} the shortest path between the two endpoints of an edge is greater or equal to the edge length. For $K \subseteq V(G)$ we denote by
\begin{itemize}
\item $\SMT(K)$: an optimum Steiner tree for the terminal set $K$.
\item $\smt(K)$: length of $\SMT(K)$.
\item $\smt_v(K)$, $v \in V(G)$: shortest possible length of a Steiner tree for $K$, where all leaves are in $K$ and vertex $v$ has degree at least 2. Obviously, $\smt_v(K) \geq \smt(K)$.
\end{itemize}
\begin{lemma} % Lemma 65
\label{thm:65}
Let $G = (V, E)$ be a graph with edge weights, $R \subseteq V$ a terminal set, $\emptyset \neq K \subseteq R$ and $v \in V \setminus K$. Then:
\begin{IEEEeqnarray*}{rCl}
	\smt_v(K + v) &=& \min_{\emptyset \neq K' \subset K} \{ \smt(K' + v) + \smt(K \setminus K' + v )\} \\
	\smt(K + v) &=& \min \left\{ \min_{w \in K} \left\{ \dist(v, w) + \smt(K) \right\}, \min_{w \notin K} \left\{ \dist(v, w) + \smt_w(K+w) \right\} \right\}
\end{IEEEeqnarray*}
(here, $K + v \coloneqq K \cup \{ v \}$).
\end{lemma}
\begin{proof}
Consider a Steiner tree for $K + v$, where $v$ has degree at least 2. Then we can decompose this tree into 2 subtrees on terminal sets $K_1 + v$ and $K_2 + v$ such that $K_1 \cup K_2 = K$, $K_1 \cap K_2 = \emptyset$. Then,
\[
	\smt_v(K + v) = \smt(K_1 + v) + \smt(K_2 + v).
\]

Now consider a \ac{SMT} $T$ for $K + v$. Trivially, the left hand side is smaller or equal to the right hand side, as those are the sizes of special Steiner tree.
If vertex $v$ has degree at least 2 in this tree, then
\[
	\smt(K + v) = \smt_v(K + v) \quad \text{(choose $w = v$ in the second expression)}.
\]
If vertex $v$ has degree 1: choose a longest parth starting in $v$ in $T$ such that all interior vertices of this path are Steiner vertices of degree 2.
Let $w$ be the other end vertex of this path.
If $|\delta_T(w)| \geq 3$, $w \notin K$, then
\[
	\smt(K + v) = \dist(v, w) + \smt_w(K + w).
\]
If $w \in K$, then
\[
	\smt(K + v) = \dist(v, w) + \smt(K).
\]
\end{proof}
\begin{algorithmbox}[Dreyfus-Wagner Algorithm (1971)]
\begin{tabular}{@{}ll}
\textit{Input:} & $G = (V, E)$, $R \subseteq V(G)$, $c : E(G) \to \R_+$\\
\textit{Output:} & $\SMT(R)$
\end{tabular}
\end{algorithmbox}
\vspace{-7pt}
\begin{algorithm}[H]
Compute $\dist(v, w)$ for all $v, w \in V(G)$\label{alg:DWA-1}\;
\For{$i = 2$ \KwTo $|R| - 1$}{
	\ForAll{$K \subset R$ with $|K| = i$ and all $v \in V \setminus K$}{
		$\smt_v(K+v) \coloneqq \min_{\emptyset \neq K' \subset K} \left\{ \smt(K' + v) + \smt(K \setminus K' + v) \right\}$\label{alg:DWA-4}\;
	}
	\ForAll{$K \subset R$ with $|K| = i$ and all $v \in V \setminus K$\label{alg:DWA-5}}{
		$\begin{IEEEeqnarraybox}[][c]{rCll} \smt(K + v) &\coloneqq& \min \left\{ \vphantom{\min_{w \notin K}} \right. & \min_{w \in K} \left\{ \dist(v, w) + \smt(K) \right\}, \\ &&& \min_{w \notin K} \left\{ \dist(v, w) + \smt_w(K + w) \right\} \left. \vphantom{\min_{w \notin K}} \right\} \end{IEEEeqnarraybox}$\;
	}
}
\end{algorithm}
\vspace{-7pt}
\EndAlgorithmLine
\begin{theorem} % Thm 66
\label{thm:66}
The \textsc{Dreyfus-Wagner Algorithm} is correct and has runtime
\[
	\mathcal{O} \left( 3^{|R|} \cdot |V| + 2^{|R|} \cdot |V|^2 + |V|^2 \log |V| + |V||E| \right).
\]
\end{theorem}
\begin{proof}
Correctness follows immediately from the previous \cref{thm:65}.
\begin{itemize}
\item \cref{alg:DWA-1}: All pairs shortest paths: $|V| \times \text{Dijkstra}$.
\item \cref{alg:DWA-4}: $\smt(K' + v) + \smt(K \setminus K' + v)$. Here, we have $\emptyset \subset K' \subset K \subset R$ and thus at most $3^{|R|}$ possibilities for each $v \in V \setminus K$, for which we have at most $|V|$ possibilities.
\item \cref{alg:DWA-5}: $K \subset R$, $v \in V \setminus K$, $w \in V$.
\end{itemize}
\end{proof}
In theory it is possible for $\varepsilon > 0$ to improve $3^{|R|}$ to $(2+\varepsilon)^{|R|}$.

A Steiner tree is \emph{full} if all terminals are leaves. If a Steiner tree is not full, it can be decomposed into its \emph{full components}: maximal subtrees which are full Steiner trees. If all full components of a Steiner tree contain at most $k$ terminals, then the three is called \emph{$k$-Steiner tree}.
\begin{lemma} % Lemma 67
\label{thm:67}
Let $T$ be a Steiner tree with $k$ terminals and $\varepsilon > 0$. Then there exists a subset of at most $1/\varepsilon$ Steiner vertices in $T$ such that by transforming these Steiner vertices into terminals, $T$ becomes a $\lceil \varepsilon k + 1 \rceil$-Steiner tree.
\end{lemma}
\begin{proof}
Let $T$ be a $k$-Steiner tree that is \ac{wlog}\ full. The case $k = 2$ is trivial, so let $k \geq 3$.
For a Steiner vertex $x \in V(T)$, we denote by $k_x$ the smallest number such that after transforming $x$ into a terminal we get  $k_x$-Steiner tree.

\underline{Claim:} There exists a Steiner vertex $y \in V(T)$ such that $k_y \leq \frac{k}{2} + 1$.
\begin{proof}
Choose $y$ arbitrarily. Assume $k_y  > \frac{k}{2} + 1$. By moving $y$ into a subtree that has more than $\frac{k}{2}$ terminals, then the rest will have less than $\frac{k}{2}$ terminals.
\end{proof}
$k_y$ is an integer, thus
\[
	k_y \leq \left\lfloor \frac{k}{2 + 1} \right\rfloor.
\]
For $\varepsilon > \frac{1}{2} - \frac{1}{k}$ we have
\[
	\varepsilon k + 1 > \left( \frac{1}{2} - \frac{1}{k} \right) k + 1 = \frac{k}{2}.
\]
So assume $\varepsilon \leq \frac{1}{2} - \frac{1}{k}$. We perform ``induction'' on $\varepsilon$ to prove the statement.
Assume the statement was not true. Choose the ``largest'' $\varepsilon$ for which the statement is false, but is true for $\frac{\varepsilon}{1-\varepsilon} > \varepsilon$.
As $\varepsilon \leq \frac{1}{2} - \frac{1}{k}$, we have
\[
	(1-\varepsilon) \cdot k \geq \frac{k}{2} + 1.
\]
Choose a Steiner vertex $x^*$ in $V(T)$ such that $k_{x^*} \leq (1-\varepsilon) k$ and $k_{x^*}$ is largest possible. Transform $x^*$ into a terminal. This way, we obtain some full component $K$ with $k_{x^*}$ terminals. All other full components have to contain at most $\varepsilon \cdot k + 1$ terminals. By the same argument as in the proof of the claim, we can see that otherwise we could find a better $x^*$ in regard to $k_{x^*}$.
Apply the Lemma to $K$: $\varepsilon' \coloneqq \frac{\varepsilon}{1- \varepsilon}$ (we know the statement holds for $\varepsilon'$ by the choice of $\varepsilon$).
Then there exists a set of at most 
\[
	1/\varepsilon' = \frac{1 - \varepsilon}{\varepsilon} = \frac{1}{\varepsilon} - 1
\]
Steiner vertices in $K$, such that each full component contains at most
\begin{IEEEeqnarray*}{rCl}
\left\lceil \varepsilon' \cdot k_{x^*} + 1 \right\rceil &\leq& \left\lceil \frac{\varepsilon}{1-\varepsilon} (1-\varepsilon) \cdot k + 1 \right\rceil \\
&=& \left\lceil \varepsilon \cdot k + 1 \right\rceil
\end{IEEEeqnarray*}
many terminals. In total we have used $\frac{1}{\varepsilon} - 1 + 1 = \frac{1}{\varepsilon}$ Steiner vertices.
\end{proof}
\begin{algorithmbox}[Attach Small Components-Algorithm (Fuchs et al.\ \lbrack{}2007\rbrack{})]
\begin{tabular}{@{}ll}
\textit{Input:} & $G = (V, E)$, $R \subseteq V(G)$, $c : E(G) \to \R_+$, $\varepsilon > 0$\\
\textit{Output:} & $\SMT(R)$
\end{tabular}
\end{algorithmbox}
\vspace{-7pt}
\begin{algorithm}[H]
\For{$R \subset \tilde{R} \subseteq V$, $|\tilde{R}| = |R| + \left\lfloor \frac{1}{\varepsilon} \right\rfloor$}{
	\For{$K \subseteq \tilde{R}$, $|K| \leq \varepsilon \cdot |R|+1$}{
		Compute $\SMT(K)$.\label{alg:ASC-3}\;
	}
	\For{$K \subseteq \tilde{R}$, $|K| > \varepsilon \cdot |R| + 1$}{
		$\SMT(K) \coloneqq \min_{\substack{v \in K \\ K' \subset K \setminus \{ v \} \\ 0 < |K'| \leq \varepsilon \cdot |R| }} \left\{ \SMT(K' + v) \cup \SMT(K \setminus K') \right\}$\;
	}
}
\end{algorithm}
\vspace{-7pt}
\EndAlgorithmLine
\begin{theorem} % Thm 68
\label{thm:68}
The \textsc{Attach Small Components Algorithm} correctly computes $\SMT(R)$ and has runtime
\[
\mathcal{O} \left( n^\frac{1}{\varepsilon} \cdot |R| \cdot \left( \left( \frac{e}{\varepsilon} \right)^\varepsilon \cdot 2  \right)^{|R|} \right).
\]
\end{theorem}
\begin{proof}
Correctness: Let $T \coloneqq \SMT(R)$ and $A \subseteq V(T) \setminus R$ with $|A| \leq \frac{1}{\varepsilon}$ such that after transforming $A$ into terminals, the tree $T$ becomes a $\lceil \varepsilon \cdot |R| + 1 \rceil$-Steiner tree.
Let $\tilde{R} \coloneqq R \cup A$, then $\SMT(R) = \SMT(\tilde{R})$.

We have $\mathcal{O}\left( n^\frac{1}{\varepsilon} \right)$ possible choices for $\tilde{R}$ and at most
\[
	|R| \begin{pmatrix}
	|\tilde{R}| \\ \varepsilon \cdot |R| + 1
	\end{pmatrix}
\]
possibles choices for $K$ in \cref{alg:ASC-3}. Using the \textsc{Dreyfus-Wagner Algorithm} to compute $\SMT(K)$, we get $3^{\varepsilon \cdot |R|} \cdot n^3$. Overall we get $\mathcal{O} \left( 2^{|R|} \cdot \operatorname{poly}(n) \right)$.

On the other hand, we have $\mathcal{O}\left( n^\frac{1}{\varepsilon} \right)$ possible choices for $\tilde{R}$. As $K \subset \tilde{R}$, we have $2^{|\tilde{R}|}$ possibilities for $K$.
For $K' \subset \tilde{R}$, we have
\[
	|K'| \leq \varepsilon \cdot |R|
\]
and thus, have
\[
	\varepsilon \cdot |R| \cdot \begin{pmatrix}
	|\tilde{R}| \\ \varepsilon \cdot |R|
	\end{pmatrix}
\]
many possibilities.
Hence, we have
\[
	\mathcal{O} \left( n^\frac{1}{\varepsilon} \cdot 2^{|\tilde{R}|} \cdot \varepsilon \cdot |R| \cdot \begin{pmatrix}
	|\tilde{R}| \\ \varepsilon \cdot |R|
	\end{pmatrix} \right) = \mathcal{O} \left( n^\frac{1}{\varepsilon} \cdot 2^{|R|} \cdot \varepsilon \cdot |R| \cdot \begin{pmatrix}
	|R| \\ \varepsilon \cdot |R|
	\end{pmatrix} \right)
\]
Using Sterling's formula,
\[
	\begin{pmatrix}
		k \\ \varepsilon k
	\end{pmatrix} \leq \left( \frac{e}{\varepsilon} \right)^{\varepsilon k}
\]
this yields
\[
	\mathcal{O} \left( n^\frac{1}{\varepsilon} \cdot 2^{|R|} \cdot \varepsilon \cdot |R| \cdot \left( \frac{e}{\varepsilon} \right)^{\varepsilon \cdot |R|} \right).
\]
\end{proof}
\end{document}