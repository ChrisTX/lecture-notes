\documentclass[../skript.tex]{subfiles}

\begin{document}
\addtocounter{chapter}{-1}
\chapter{Motivation}
\begin{problem}[Traveling Salesman Problem]
\begin{tabular}{@{}ll}
\textit{Input:} & A complete (undirected) graph $G = (V, E)$, $c : E(G) \to \R_+$ \\
\textit{Output:} & A Hamiltonian cycle (\emph{TSP-tour}) $T$, \ac{st}
\end{tabular}
\[
	c(T) \coloneqq \sum_{e \in E(T)} c(e) \text{ is minimum.}
\]
\end{problem}
This problem is \emph{\NP-hard}, thus no polynomial time algorithm is known to solve this problem.

What is the \textit{best solution} that we can compute in polynomial time?

Does there exists a polynomial time algorithm for the \textsc{TSP} that always guarantees to find a tour that is at most \textit{twice} (approximation ratio factor) as long as an optimal tour?

\textbf{Sahni, Gonzalez (1976):} no \emph{$r$-approximation} exists for \textsc{TSP} unless $P = \NP$, for $r \in \N$.

\textsc{Metric TSP}:
\[
	c(\{x, y\}) \leq c(\{ x, z \}) + c(\{z, y\}) \quad \forall x, y, z \in V(G).
\]
\begin{description}
\item[Christofides (1976):] A \sfrac{3}{2}-approximation algorithm for \textsc{Metric TSP} exists.
\item[Arora (1996):] For all $\varepsilon > 0$ a $(1 + \varepsilon)$-approximation for \textsc{Euclidean TSP} exists.
\end{description}

Approximation algorithms often are much simpler than exact algorithms. \\
\textsc{Greedy Algorithm}: e.g.\ matching problems.

How to design approximation algorithms?

Literature: 
\begin{itemize}
\item \cite[Chapters 15-18, 20, 21]{KorteVygen}
\item \cite{CormenLeiserson}
\item \cite{GareyJohnson}
\item \cite{Vazirani}
\item \cite{JansenMargraf}
\end{itemize}
Structure:
\begin{enumerate}
\item \NP-completeness
\item Approximation algorithm
\item Knapsack
\item Bin packing
\item Steiner tree
\item \textsc{TSP}
\item \PCP
\end{enumerate}
\chapter{\texorpdfstring{\NP-completeness}{NP-completeness}} % Ch 1
\label{ch:1}
\begin{description}
\item[alphabet] set of symbols, e.g.\ $0,1$.
\item[string] sequences of symbols, e.g.\ $01101$.
\item[length of a string] number of symbols, e.g.\ $|01101| = 5$.
\end{description}
We denote the alphabet as $\Sigma$ and as $\Sigma^n$, $n \in \N$ all strings of length exactly $n$ with symbols from $\Sigma$.
The \emph{strings over $\Sigma$} are denoted as
\[
	\Sigma^* \coloneqq \bigcup_{n \in \N} \Sigma^n
\]
Decision problem: only ``yes'' and ``no'' are possible answers.
Identify all ``yes''-instances as the decision problem, i.e.\ a decision problem is simply a subset of $\Sigma^*$ (\emph{languages}).

Let $X \subseteq \{ 0, 1\}^*$. An algorithm $A$ for the decision problem $X$ is an algorithm that for given $s \in \{0, 1\}^*$ returns the value $A(s)$, which is ``yes'' or ``no'', \ac{st} $s \in X \Leftrightarrow A(s) = \text{``yes''}$. $A$ has \emph{polynomial runtime} if there exists a polynomial $p$, \ac{st} for every input $s$ the algorithm $A$ after at most $\mathcal{O}(p(|S|))$ \emph{``steps''}.

Models of computation: Turing Machine, RAM, C++ program, and etc. Refined Church's Thesis: These are equivalent and can be transformed in each other in polynomial time.

$P$: class of all decision problems that have polynomial time algorithms. \\
\NP: class of all decision problems that have polynomial time algorithms to verify a solution.

More formally: Let $X$ be a decision problem. Then $x$ belongs to \NP{} if there exists a polynomial time algorithm
\[
	A : \{ 0, 1 \}^* \times \{ 0, 1 \}^* \to \{ \text{``yes''}, \text{``no''}\}
\]
and a polynomial $p$:
\[
	s \in X \; \Leftrightarrow \; \exists t \in \{ 0, 1\}^* \text{ \ac{st} } |t| \leq p(|s|) \text{ and } A(s, t) = \text{``yes''}.
\]
$t$ is called \emph{certificate} for $s$.
$P = \NP$? $P \subseteq \NP$. \\
\NP-complete: ``hardest'' problems in \NP.

Let $X$ and $Y$ be decision problems. We say that $Y$ is \emph{polynomial time reducible} to $X$ ($Y$ polynomially reduces to $X$) if there exists an algorithm for $Y$ that is allowed to make polynomially many calls to an algorithm for $X$ and needs in addition a polynomial number of steps (do no count steps used for $X$) ``Turing-reduction''.

Karp-reduction: $X$, $Y$ decision problems. $Y$ is Karp-reducible to $X$ is there exists a function $f : \{ 0, 1\}^* \to \{ 0, 1\}^*$ computable in polynomial time, \ac{st} $f(s) \in X \; \Leftrightarrow \; s \in Y$.
\begin{proposition} % Prop 1
\label{prop:1}
If $Y$ polynomial time reduces to $X$ and there exists a polynomial time algorithm for $X$, then there exists a polynomial time algorithm for $Y$.
\end{proposition}
A problem is called \emph{\NP-hard} if all problems in \NP{} polynomially reduce to $X$.
\begin{proposition} % Prop 2
\label{prop:2}
If there exists a polynomial time algorithm for any \NP-problem, then $P = \NP$.
\end{proposition}
\begin{problem}[Satisfiability \lbrack Cook, 1971\rbrack]
\begin{tabular}{@{}ll}
\textit{Input:} & A boolean formula $F$ in conjunctive normal form. \\
\textit{Question:} & Is $F$ satisfiable?
\end{tabular}
\end{problem}
Boolean formula: Boolean variables $X = \{ x_1, \ldots, x_k \}$ \textit{true}, \textit{false}.
\[
	f : X \to \{ \text{\textit{false}}, \text{\textit{true}} \}^k \text{ truth assignment}
\]
recursive definition of formula:
$x_i$ is a boolean formula.
\begin{itemize}
\item $F$ is a boolean formula implies $\bar{F}$ is a boolean formula.
\item $F_1$, $F_2$ boolean formulas imply $F_1 \wedge F_2$ is a boolean formula.
\item $F_1$, $F_2$ boolean formulas imply $F_1 \vee F_2$ is a boolean formula.
\end{itemize}
\emph{Literal:} $x_i$, $\overline{x_i}$. \\
\emph{conjunctive normal form:} $F = C_1 \wedge C_2 \wedge \ldots \wedge C_m$.
The $C_i$ are called clauses and of the form
\[
	C_i = x_{i_1} \vee x_{i_2} \vee \ldots \vee x_{i_j} \quad x_{i_l} \text{ literals}.
\]
Every boolean formula is equivalent to some boolean formula in conjunctive normal form:
\[
	(x_1 \wedge x_2) \vee (x_3 \wedge x_4) \vee \ldots \vee (x_{2n+1} \wedge x_{2n})
\]
However, with respect to satisfiability equivalence holds and the size of the conjunctive normal form is \emph{linear} in the size of the formula:
\[
	z_1 \wedge z_2 \wedge \ldots \wedge z_n \wedge (\overline{z_1} \vee x_1) \cap (\overline{z_1} \vee x_2) \wedge \ldots \wedge (\overline{z_n} \vee x_{2n}) \cap (\overline{z_n} \vee x_{2n+1})
\]
\begin{theorem} % Theorem 3
\label{thm:3}
Satisfiability is \NP-complete.
\end{theorem}
\begin{proof}
Reformulate computation of a Turing machine in a boolean formula.
\end{proof}
\end{document}