\documentclass[../skript.tex]{subfiles}

\begin{document}
\begin{algorithmbox}[Johnson's Algorithm (Reformulation)]
\begin{tabular}{@{}ll}
\textit{Input:} & Set $\mathcal{C}$ of clauses with variables $X = \{ x_1, \ldots, x_n \}$, $w \; : \; \mathcal{C} \to \R_+$. \\
\textit{Output:} & A truth assignment for $X$.
\end{tabular}
\end{algorithmbox}
\vspace{-7pt}
\begin{algorithm}[H]
Set $w_0(C) \coloneqq \frac{w(C)}{2^{|\mathcal{C}|}}$ for $C \in \mathcal{C}$.\;
$T_0 \coloneqq \emptyset$\;
\For{$i \coloneqq 1$ \KwTo $n$}{
	\leIf{$\sum_{x_i \in C} w_{i-1}(C) > \sum_{\overline{x_i} \in C} w_{i-1}(C)$}{$y \coloneqq x_i$}{$y \coloneqq \overline{x_i}$}
	Assign \textit{true} to $y$.\;
	$w_i(C) \coloneqq \begin{cases}
	0, & \text{if } y \in C \\
	2 \cdot w_{i-1}(C), & \text{if } \bar{y} \in C \\
	w_{i-1}(C), & \text{else}
	\end{cases}$.\;
	$T_i \coloneqq T_{i-1} \cup \{ C \in \mathcal{C} \mid y \in C\}$.\;
}
\end{algorithm}
\vspace{-7pt}
\EndAlgorithmLine
\begin{example}
Consider the following instances:
\begin{center}
\begin{tabular}{ccc}
$x_1$ & $x_1$ & $\overline{x_1} \vee x_2$ \\
$w = 1$ & $w=1$ & $w = 2$
\end{tabular}
\end{center}

The algorithm without scaling may choose to set $x_1$ to \textit{false}, and the total weight would be 2. This gives a \sfrac{1}{2}-approximation, as the optimum is 4.
\begin{center}
\begin{tabular}{cc}
$x_1$ & $\overline{x_1} \vee x_2$ \\
$w = 1$ & $w = 2$
\end{tabular}
\end{center}

The algorithm with scaling finds a solution of weight 2. Thus, we have a \\ \sfrac{2}{3}-approximation.
\end{example}
We define:
\begin{IEEEeqnarray*}{rCl}
T_i &\coloneqq& \text{ set of all clauses that are \textit{true} after iteration } i \\
F_i &\coloneqq& \mathcal{C}\setminus T_i \text{ is the set of all clauses that are possibly \textit{false}} \\
&& \text{at the end of the algorithm.} \\
w_i(F_i) &\leq& w_{i-1}(F_{i-1}) \; \Leftrightarrow \; w_i(F_i) - w_{i-1}(F_{i-1}) \leq 0
\end{IEEEeqnarray*}
\begin{lemma} % Lemma 28
\label{thm:28}
Let $\mathcal{C}$ be a set of clauses with variables $\{ x_1, \ldots, x_n \}$ and weights $w : \mathcal{C} \to \R_+$. Let $\pi : \{ x_1, \ldots, x_n\} \to \{ \textit{true}, \textit{false} \}$ be an arbitrary truth assignment.
After each iteration ($i = 1, \ldots, n$) of \textsc{Johnson's Algorithm} we have
\[
	w_i\left( F_i^\pi \right) - w_{i-1} \left( F_{i-1}^\pi \right) \leq \frac{1}{2} \left( w(T_i) - w(T_{i-1}) \right),
\]
where $F_i^\pi \subseteq F_i$ is the set of all clauses in $F_i$ that are \textit{false} if the assignment of $x_1, \ldots, x_i$ by the algorithm is extended to $x_{i+1}, \ldots, x_n$ by using $\pi$.
\end{lemma}
\begin{theorem} % Thm 29
\label{thm:29}
\textsc{Johnson's Algorithm} for \textsc{Max-Sat} has approximation ratio \sfrac{2}{3}. 
\end{theorem}
\begin{proof}
Let $\mathcal{C}$, $w : \mathcal{C} \to \R_+$ be an instance of \textsc{Max-Sat} with variables $\{ x_1, \ldots, x_n \}$. Let $\pi : \{ x_1, \ldots, x_n\} \to \{ \textit{true}, \textit{false} \}$ be an optimum assignment and $\OPT$ its value.

\underline{Claim:} $w(T_n) \geq \frac{1}{3} \left( w(\mathcal{C}) + \OPT \right)$.
\[
\sum_{i=1}^n \left( w_i \left(F_i^\pi \right) - w_{i-1} \left( F_{i-1}^\pi \right) \right) \leq \frac{1}{2} \sum_{i=1}^n \left( w(T_i) - w(T_i-1) \right)
\]
Thus,
\[
	w_n\left(F_n^\pi\right) - w_0\left(F_0^\pi\right) \leq \frac{1}{2} \left( w(T_n) - w(T_0) \right).
\]
Then, $w(T_0) = 0$ and $F_n^\pi = F_n$. Hence,
\[
	w_n \left( F_n^\pi \right) = w(F_n) = w(\mathcal{C}) - w(T_n).
\]
Lastly,
\[
	w_0 \left( F_0^\pi \right) \leq \frac{1}{2} w \left( F_0^\pi \right) = \frac{1}{2} (w(\mathcal{C}) - \OPT).
\]
By inserting these equalities into above equation:
\[
	w(\mathcal{C}) - w(T_n) \leq \frac{3}{2} w(T_n) + \frac{1}{2} w(\mathcal{C}) - \frac{1}{2} \OPT. 
\]
This gives
\[
	\frac{1}{2} ( w(\mathcal{C}) + \OPT ) \leq \frac{3}{2} w(T_n). 
\]
\end{proof}
\begin{proof}[\cref{thm:28}]
Consider iteration $i$ of \textsc{Johnson's Algorithm}. \Ac{wlog} assume that the algorithm assigns \textit{true} to $x_i$.
Then,
\begin{equation}
\tag{$\star$}
\label{eq:thm-28-star}
\sum_{x_i \in C} w_{i-1}(C) \geq \sum_{\overline{x_i} \in C} w_{i-1} (C).
\end{equation}
We now prove
\begin{equation}
\tag{i}
\label{eq:thm-28-i}
\frac{1}{2} ( w(T_i) - w(T_{i-1}) ) \geq \sum_{x_i \in C} w_{i-1}(C)
\end{equation}
and
\begin{equation}
\tag{ii}
\label{eq:thm-28-ii}
\sum_{\overline{x_i} \in C} w_{i-1}(C) \geq w_i \left( F_i^\pi \right) - w_{i-1} \left( F_{i-1}^\pi \right)
\end{equation}
\begin{proof}[\labelcref{eq:thm-28-i}] \NoEndMark
Let $C \in T_i \setminus T_{i-1}$. Then, $x_i \in C$ and thus
\[
	w_{i-1} (C) \leq \frac{w(C)}{2}.
\]
Hence,
\[
	w(T_i) - w(T_{i-1}) \geq 2 \cdot \sum_{x_i \in C} w_{i-1}(C)
\]
\end{proof}
\begin{proof}[\labelcref{eq:thm-28-ii}] \NoEndMark
\underline{Case 1:} $\pi(x_i) = \textit{true}$. \\
Algorithm also assigns \textit{true} to $x_i$. Thus, $F_i^\pi = F_{i-1}^\pi$.
Each clause in $F_i^\pi$ contains either $\overline{x_i}$ or neither $x_i$ nor $\overline{x_i}$.
Thus we have:
\begin{IEEEeqnarray*}{rCl}
w_i \left( F_i^\pi \right) &=& \sum_{C \in F_i^\pi} w_i(C) \\
&=& \sum_{\substack{C \in F_i^\pi \\ \overline{x_i} \in C}} w_i(C) + \sum_{\substack{C \in F_i^\pi \\ x_i, \overline{x_i} \notin C}} w_i(C) \\
&=& \sum_{\substack{C \in F_i^\pi \\ \overline{x_i} \in C}} 2\cdot w_{i-1}(C) + \sum_{\substack{C \in F_i^\pi \\ x_i, \overline{x_i} \notin C}} w_{i-1}(C) \\
&=& \sum_{\substack{C \in F_i^\pi \\ \overline{x_i} \in C}} w_{i-1}(C) + w_{i-1} \left( F_{i-1}^\pi \right)
\end{IEEEeqnarray*}
Hence,
\[
	w_i \left( F_i^\pi \right) - w_{i-1} \left( F_{i-1}^\pi \right) = \sum_{\substack{C \in F_i^\pi \\ \overline{x_i} \in C }} w_{i-1}(C) \leq \sum_{\overline{x_i} \in C} w_{i-1}(C).
\]

\underline{Case 2:} $\pi(x_i) = \textit{false}$. \\
The algorithm and $\pi$ disagree on $x_i$. No clause in $F_{i-1}^\pi$ contains $\overline{x_i}$. No clause in $F_i^\pi$ contains $x_i$. If a clause contains neither $x_i$ nor $\overline{x_i}$ then it's in both sets $F_{i-1}^\pi$ and $F_i^\pi$ or in none.
Therefore,
\begin{IEEEeqnarray*}{rCl}
	w_i \left( F_i^\pi \right) - w_{i-1} \left( F_{i-1}^\pi \right) &=& \sum_{\substack{C \in F_i^\pi, \\ \overline{x_i} \in C }} w_i(C) - \sum_{\substack{C \in F_{i-1}^\pi, \\ x_i \in C}} w_{i-1}(C) \\
	&=& 2 \cdot \sum_{\substack{C \in F_i^\pi, \\ \overline{x_i} \in C}} w_{i-1}(C) - \sum_{\substack{C \in F_{i-1}^\pi, \\ x_i \in C}} w_{i-1}(C) \\
	&\leq& 2 \cdot \sum_{\overline{x_i} \in C} w_{i-1}(C) - \sum_{x_i \in C} w_{i-1}(C) \\
	&\leq& \sum_{\overline{x_i} \in C} w_{i-1}(C).
\end{IEEEeqnarray*}
\end{proof}
The statements follows from both claims.
\end{proof}
\begin{remark}
\begin{itemize}
\item \textsc{Johnson's Algorithm}: $0.66$ approximation.
\item Avidor, Borkovitch, Zwick (2006): $0.7968$ approximation.
\end{itemize}
\end{remark}
\paragraph{Example.}
Consider the \textsc{Set Cover} problem.
\[
	U = \{ 1, \ldots, n \}, \; S_1, \ldots, S_m \subseteq U, \; c : \{ S_1, \ldots, S_m \} \to \R_+.
\]
\underline{IP}:
\begin{IEEEeqnarray*}{lrCl}
\min \sum_{i=1}^m c(S_i) - x_i \\
\text{\ac{st}} & \sum_{j \in S_i} x_i &\geq& 1 \quad \text{for each } j \in \{ 1, \ldots, n \} \\
& x_i &\in& \{ 0, 1\}
\end{IEEEeqnarray*}
We obtain an LP from this, by replacing the integrality condition with $x_i \geq 0$.
Set 
\[
	f \coloneqq \max_{1 \leq j \leq n} |\{ i \mid j \in S_i \}|
\]
as the maximal occurrence of an element.
\begin{algorithmbox}[LP-Based Algorithm for Set Cover]
\begin{tabular}{@{}ll}
\textit{Input:} & $U = \{ 1, \ldots, n \}$, $S_1, \ldots, S_m \subseteq U$, $c : \{ S_1, \ldots, S_m \} \to \R_+$. \\
\textit{Output:} & A family $\mathcal{F} \subset \{ S_1, \ldots, S_m \}$ \ac{st}~$\bigcup_{S \in \mathcal{F}} S = U$.
\end{tabular}
\end{algorithmbox}
\vspace{-7pt}
\begin{algorithm}[H]
Compute an optimum solution $x \in \R^m$ of the \textsc{Set Cover}-LP.\;
$\mathcal{F} \coloneqq \left\{ S_i \mid x_i \geq \frac{1}{f} \right\}$.\;
\end{algorithm}
\vspace{-7pt}
\EndAlgorithmLine
\begin{theorem} % Thm 30
The \textsc{LP-Based Algorithm for Set Cover} has approximation ratio $f$.
\end{theorem}
\begin{proof}
Let $x$ be an optimum solution of the \textsc{Set Cover}-LP. Set 
\[
	x_i' \coloneqq \begin{cases}
	1, & \text{if } x_i \geq \frac{1}{f} \\
	0, & \text{else}
	\end{cases}.
\]

\underline{Claim:} $x'$ is a feasible solution of the \textsc{Set Cover} problem (IP). Consider an arbitrary $j \in \{ 1, \ldots, n \}$. We have
\[
	\sum_{j \in S_i} x_i \geq 1.
\]
As $j$ appears in at most $f$ different sets $S_i$ there must exist at least $i$ with $j \in S_i$ and $x_i \geq \frac{1}{f}$.
This implies
\[
	x_i' = 1 = \sum_{j \in S_i} x_i' \geq 1.
\]

Furthermore,
\begin{IEEEeqnarray*}{rCl}
	\sum_{i=1}^m c(S_i) \cdot x_i' &\leq& \sum_{i=1}^m c(S_i) \cdot x_i \cdot f \\
	&=& f \cdot \sum_{i=1}^m c(S_i) \cdot x_i \\
	&=& f \cdot \OPT_{LP} \\
	&\leq& f \cdot \OPT_{IP}.
\end{IEEEeqnarray*}
This proves the approximation ratio.
\end{proof}
\begin{remark}
The difference between $\OPT_{LP}$ and $\OPT_{IP}$ is called \emph{integrality gap}.
\end{remark}
\end{document}