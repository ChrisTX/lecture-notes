\documentclass[../skript.tex]{subfiles}

\begin{document}
We want to derive a \ac{PTAS}. For this we want to use dynamic programming with a finite grid overlaying the points.

Fix some $\varepsilon$, $0 < \varepsilon < 1$. Our goal is finding a $(1 + \varepsilon)$-approximation for \textsc{Euclidean TSP}.
We call an instance of \textsc{Euclidean TSP} \emph{well-rounded}, if:
\begin{enumerate}
\item\label[equation]{eq:wellrounded-i} $\forall (v_x, v_y) \in V$: $v_x$ and $v_y$ are odd integers
\item\label[equation]{eq:wellrounded-ii} $\max_{v, w \in V} \| v - w \|_2 \leq \frac{64 \cdot |V|}{\varepsilon} + 16$
\item\label[equation]{eq:wellrounded-iii} $\min_{v, w \in V} \| v - w \|_2 \geq 8$
\end{enumerate}
\begin{theorem} % Thm 93
\label{thm:93}
If there exists a \ac{PTAS} for well-rounded \textsc{Euclidean TSP} instances, then there exists a \ac{PTAS} for \textsc{Euclidean TSP}.
\end{theorem}
\begin{proof}
Let $V \subseteq \R^2$, $|V|$ finite and $|V| \geq 3$.
We define
\[
	L \coloneqq \max_{v, w \in V} \| v - w \|_2.
\]
We define a new instance by
\[
	V' \coloneqq \left\{ \left( 1 +  8 \cdot \left\lfloor \frac{8n}{\varepsilon \cdot L} \cdot v_x \right\rfloor, 1 +  8 \cdot \left\lfloor \frac{8n}{\varepsilon \cdot L} \cdot v_y \right\rfloor  \right) \gmid (v_x, v_y) \in V \right\}.
\]
\textit{Remark:} $|V'| < |V|$ is possible.
With this choice, $v_x', v_y'$ are numbers of the form $8a + 1$, $a \in \mathbb{Z}$. Thus \labelcref{eq:wellrounded-i} and \labelcref{eq:wellrounded-iii} are fulfilled.
We have
\[
	\max_{v', w' \in V'} \| v' - w' \|_2 \leq \frac{64n'}{\varepsilon \cdot L} \cdot \max_{v, w \in V} \| v - w \|_2 + 16 = \frac{64 \cdot |V'|}{\varepsilon} + 16.
\]
Thus, \labelcref{eq:wellrounded-ii} is fulfilled, too and $V'$ is well-rounded.

By assumption, we have a \ac{PTAS} for $V'$. We use it with $\frac{\varepsilon}{4}$ to get a $\left( 1 + \frac{\varepsilon}{4} \right)$-approximation.
Then, we have a tour for $V'$ of length
\[
	\ell' \leq \left(1 + \frac{\varepsilon}{4}\right) \cdot \OPT(V').
\]
This gives us a tour for $V$ of length at most
\[
	\ell \leq \left( \frac{\ell'}{8} + 4n \right) \cdot \frac{\varepsilon \cdot L}{8n}.
\]
For this, we consider the tour that by undoing the rounding, we add for each vertex at most a distance of $2 \cdot \sqrt{2}$ times the grid size $\frac{8n}{\varepsilon \cdot L}$, which gives us the $4n$ term ($2 \sqrt{2} < 4$). The factor of $8$ is due to us undoing the scaling.

By the same argument applied the other way around, we obtain
\[
	\OPT(V') \leq 8 \cdot \left( \frac{8n}{\varepsilon \cdot L} \cdot \OPT(V) + 4n \right).
\]
Hence,
\begin{IEEEeqnarray*}{rCl}
	\ell &\leq& \frac{\varepsilon \cdot L}{8n} \left( 4n + \left( 1 + \frac{\varepsilon}{4} \right) \cdot \frac{1}{8} \cdot 8 \left( \frac{8n}{\varepsilon \cdot L} \cdot \OPT(V) + 4n \right) \right)  \\
	&=& \frac{\varepsilon \cdot L}{2} + \left( 1 + \frac{\varepsilon}{4} \right) \cdot \OPT(V) + \left( 1 + \frac{\varepsilon}{4} \right) \cdot \frac{\varepsilon \cdot L}{2} \\
	&=& \left( 1 + \frac{\varepsilon}{4} \right) \cdot \OPT(V) + \varepsilon \cdot L + \frac{\varepsilon^2}{8} \cdot L.
\end{IEEEeqnarray*}
As $L \leq \frac{1}{2} \OPT(V)$, we obtain
\begin{IEEEeqnarray*}{rCl}
\ell &\leq& \left( 1 + \frac{\varepsilon}{4} + \frac{\varepsilon}{2} + \frac{\varepsilon^2}{16} \right) \cdot \OPT(V) \\
&\overset{\varepsilon < 1}{\leq}& (1 + \varepsilon) \cdot \OPT(V),
\end{IEEEeqnarray*}
which is what we wanted to prove.
\end{proof}
\underline{Assumption:} All coordinates are within $[0, 2^N] \times [0, 2^N]$ with
\begin{IEEEeqnarray*}{rCl}
N &\coloneqq& \lceil \log L \rceil + 1, \\
L &\coloneqq& \max_{v, w \in V} \| v - w \|_2
\end{IEEEeqnarray*}
For $i \coloneqq 1$ to $N-1$ define
\begin{IEEEeqnarray*}{rCl}
	G_i &\coloneqq& X_i \cup Y_i \\
	X_i &\coloneqq& \left\{ \left[ \left( 0, k \cdot 2^{N-i} \right), \left(2^N, k \cdot 2^{N-i}\right) \right] \gmid k = 0, \ldots, 2^i -1 \right\} \\
	Y_i &\coloneqq& \left\{ \left[ \left( j \cdot 2^{N-i}, 0\right) , \left(j \cdot 2^{N-i}, 2^N\right) \right] \gmid j = 0, \ldots, 2^i -1 \right\}
\end{IEEEeqnarray*}
Let $a, b \in \{ 0, 2, \ldots, 2^N - 2 \}$ be even integers. We define the shifted grids
\begin{IEEEeqnarray*}{rCl}
	G_i^{(a, b)} &\coloneqq& X_i^{(b)} \cup Y_i^{(a)} \\
	X_i^{(b)} &\coloneqq& \left\{ \left[ \left( 0, \left( b + k \cdot 2^{N-i} \right) \; \operatorname{mod} 2^N \right), \left( 2^N, \left( b + k \cdot 2^{N-i} \right) \; \operatorname{mod} 2^N \right) \right] \right\} \\
	Y_i^{(b)} &\coloneqq& \left\{ \left[ \left( \left( a + j \cdot 2^{N-i} \right) \; \operatorname{mod} 2^N, 0 \right), \left( \left( a + j \cdot 2^{N-i} \right) \; \operatorname{mod} 2^N, 2^N \right) \right] \right\}
\end{IEEEeqnarray*}
\underline{Observation:} $G_{N-1}^{(a, b)} = G_{N-1}$.

We say a line is in \emph{level $1$}, if $\ell \in G_1^{(a, b)}$ and in \emph{level $i > 1$}, if $\ell \in G_i^{(a, b)} \setminus G_{i-1}^{(a, b)}$.
A region of a grid $G_i^{(a, b)}$ is:
\begin{IEEEeqnarray*}{rCl}
\Big\{ (x, y) \in [0, 2^N] \times [0, 2^N] \Big|&& \left(x-a-j\cdot 2^{N-i}\right) \; \operatorname{mod} 2^N < 2^{N-i}, \\
&& \left(y-b-k\cdot 2^{N-i}\right) \; \operatorname{mod} 2^N < 2^{N-i}, \; \text{for } j, k \in \{ 0, \ldots, 2^i - 1 \} \Big\}
\end{IEEEeqnarray*}

Let $\ell$ be a line in $G_{N-1}$, we define
\[
\crossings(T, \ell) \coloneqq \text{the number of crossing of $\ell$ with a polygon $T$}.
\]
\begin{lemma} % Lemma 94
\label{thm:94}
For an optimum TSP-tour $T$ of a well-rounded \textsc{Euclidean TSP} instance $V$, we have
\[
	\sum_{\ell \in G_{N-1}} \crossings(T, \ell) \leq \OPT(V).
\]
\end{lemma}
\begin{proof}
Consider one segment of $T$ of length $s$. Let $x, y$ be the projections of the line segment to the $x$ and $y$-axises, respectively.
Then the segment crosses at most $\frac{x}{2} + 1 + \frac{y}{2} + 1$ times the grid $G_{N-1}$.
As $s^2 = x^2 + y^2$, we have
\[
	(x+y)^2 \leq (x+y)^2 + (x-y)^2 = 2s^2.
\]
Thus, $x + y \leq \sqrt{2} \cdot s$, and
\[
	\frac{x + y}{2} + 2 \leq \frac{\sqrt{2}}{2} \cdot s + 2 \leq s, \quad \text{if } s \geq 8.
\]
By summing over all line segments, the statement of the Lemma follows.
\end{proof}
Define
\begin{IEEEeqnarray*}{rCl}
	C &\coloneqq& 7 + \left\lceil \frac{36}{\varepsilon} \right\rceil, \\
	P &\coloneqq& N \cdot \left\lceil \frac{6}{\varepsilon} \right\rceil.
\end{IEEEeqnarray*}
We define \emph{portals}:
If
\[
	\left[ \left( 0, \left(b + k \cdot 2^{N-i} \right) \; \operatorname{mod} 2^N  \right), \left( 2^N, \left(b + k \cdot 2^{N-i} \right) \; \operatorname{mod} 2^N \right) \right]
\]
is a line in $G_i^{(a, b)}$ (horizontal) then define its set of portals as
\[
\left\{ \left( \left( a + \frac{h}{P} \cdot 2^{N-i} \right) \; \operatorname{mod} 2^N, \left(b + k \cdot 2^{N-i} \right) \; \operatorname{mod} 2^N \right) \gmid h = 0, \ldots, P \cdot 2^i\right\}.
\]

Let $V \subseteq [0, 2^N] \times [0, 2^N]$ be a well-rounded \textsc{Euclidean TSP}-instance and let $a, b \in \{ 0, 2, \ldots, 2^N - 2 \}$, $C$ and $P$ as above.
A \emph{Steiner tour} is a closed sequence of line segments, that contains $V$ and crosses grid lines only in portals. A Steiner tour is \emph{light}, if for all $i$ and each region in $G_i^{(a, b)}$ the tour crosses each border edge of the region at most $C$ times.
\begin{lemma}[Patching Lemma] % Lemma 95
\label{thm:95}
Let $V \subset \R^2$ be an \textsc{Euclidean TSP} instance and $T$ a tour for $V$. Let $\ell$ be a segment of length $s$ of a line that contains no point of $V$. Then there exists a tour for $V$ that is at most $6s$ longer than $T$ and crosses $\ell$ at most twice.
\end{lemma}
\begin{proof}
\ac{wlog} $\ell$ is a vertical line segment. Suppose $T$ crosses $\ell$ exactly $k$ times in the edges $e_1, e_2, \ldots, e_k$, $k \geq 3$.
Subdivide each $e_i$ by two new vertices $p_i$ and $q_i$ that are close to $\ell$ without increasing the tour length. Hereby let $p_i$ be left of $\ell$ and $q_i$ right of $\ell$ ($i = 1, \ldots, k$).
Define
\[
	t \coloneqq \left\lfloor \frac{k-1}{2} \right\rfloor.
\]
Then, $k - 2 \leq 2 t \leq k - 1$.
Remove the edges $p_i, q_i$ for $i = 1, \ldots, 2t$ (i.e.\ avoid the crossings). Let $P$ be a shortest tour through $p_1, \ldots, p_k$ and add a minimum cost perfect matching on $p_1, \ldots, p_{2t}$. Analogously, we'll proceed on the other side and denote the result by $Q$.
The total length of the edges is at most $3s$ in each of $P$ and $Q$.
By adding $P$ and $Q$ to the graph, we obtain an Eulerian graph in which $\ell$ is being crossed at most $k - 2t \leq 2$ times. By shortcutting, we obtain a new tour with length at most $6s$ and at most $2$ crossings with $s$.
\end{proof}
\end{document}