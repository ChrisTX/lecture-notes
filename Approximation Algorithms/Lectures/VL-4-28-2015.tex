\documentclass[../skript.tex]{subfiles}

\begin{document}
An equivalent formulation of \textsc{$k$-Center} in terms of graph theory is:
\begin{problem}[$k$-Center]
\begin{tabular}{@{}ll}
\textit{Input:} & A complete graph $G = (V, E)$, a metric $\dist : E(G) \to \R$, $k \in \N$. \\
\textit{Output:} & A set $C \subseteq V(G)$ with $|C| = k$, \ac{st} $\max_{x \in V} \min_{c \in C} \{\dist(x, c)\}$ \\
& is minimum.
\end{tabular}
\end{problem}
A dominating set is then defined as
\[
	S \subseteq V(G) \; : \; \forall x \in V \setminus S \; \exists s \in S \; : \; \{ x, s \} \in E(G).
\]
We call
\[
	\dom(G) \coloneqq \min \{ |D| \mid D \text{ is a dominating set for } G \}
\]
the \emph{domination number}.
\begin{theorem} % Thm 23
\label{thm:23}
Computing $\dom(G)$ is \NP-hard.
\end{theorem}
\begin{proof}
Reduction from \textsc{Vertex Cover}: Let $G$ be an instance of \textsc{Vertex Cover}.
Construct a new graph $G'$ as follows: Replace all edges of $G$ by a path of length $2$.
Completely connect all vertices in $G'$ that are in $G$.

If $C$ is a \textsc{Vertex Cover} in $G$, then $C$ is a dominating set in $G'$: This is clear for vertices in $G$, clear for other vertices as they correspond to edges in $G$.

If $C$ is a dominating set in $G'$, there always  exists a dominating set in $G'$ of the same size using only vertices in $G$.
\end{proof}
Consider an $n \times n$-grid. What is its domination number? The restricted problem is in $P$ because sparse \NP-hard sets do not exist unless $P = \NP$.
In 2011, it was found that the domination number for an $n \times n$-grid (if $n \geq 7$) is
\[
	\left\lfloor \frac{(n+2)^2}{5} \right\rfloor - 4.
\]
\begin{theorem} % Thm 24
\label{thm:24}
\textsc{$k$-Center} is \NP-hard.
\end{theorem}
\begin{proof}
Reduction from \textsc{Dominating Set}: Let $G = (V, E)$ be an instance of \textsc{Dominating Set}.
Construct a complete graph $G'$ on $V(G)$ with
\[
\dist(e) \coloneqq \begin{cases}
1, & \text{if } e \in E(G) \\
2, & \text{if } e \notin E(G)
\end{cases}
\]
as metric.
Then
\[
	\dom(G) = \min_{1 \leq k \leq |V|} \left\{ C \subseteq V \mid |C| = k \text{ and } \max_{x \in V} \min_{c \in C} \{ \dist(x, c) = 1 \} \right\}.
\]
This forces a solution to only use edges in the graph, as other choices would increase the maximum to $2$.
\end{proof}
Let $G = (V, E)$ be a graph. A subset $S \subseteq V(G)$ is called \emph{2-stable} if the shortest path between any two different vertices in $S$ has a length of at least $3$.

We can compute \underline{maximal} 2-stable sets in polynomial time.
\begin{algorithm}[$k$-Center-Algorithm]
\begin{tabular}{@{}ll}
\textit{Input:} & A complete graph $G = (V, E)$, metric weights $\dist : E(G) \to \R$, $k \in \N$ \\
\textit{Output:} & A set $S \subseteq V(G)$.
\end{tabular}
\begin{algorithmic}[1]
\State Let $E = \{ e_1, \ldots, e_m \}$ \ac{st} $\dist(e_1) \leq \dist(e_2) \leq \ldots \leq \dist(e_m)$.
\For{$i = 1$ to $m$}
\State $G_i \coloneqq (V, \{ e_1, \ldots, e_i \})$
\State Compute a maximal 2-stable set $S_i$ in $G_i$
\If{$|S_i| \leq k$}
\State return $(S_i)$
\EndIf
\EndFor
\end{algorithmic}
\end{algorithm}
\begin{theorem} % Thm 25
\label{thm:25}
The \textsc{$k$-Center-Algorithm} has an approximation ratio of $2$.
\end{theorem}
\begin{proof}
The algorithm will always return a solution.
Let $S$ be the maximal $2$-stable set returned by the algorithm in iteration $j$.
Each vertex not in $S$ is connected by a path of length at most $2$ with a vertex in $S$.
Thus, $S$ is a solution to the \textsc{$k$-Center} problem with value smaller or equal to $2 \cdot \dist(e_j)$.

\underline{Claim 1:} $\dom(G_i) \geq |S_i|$ for all $1 \leq i \leq m$. \\
Let $D$ be a dominating set in $G_i$. Then, $G_i$ be covered by $|D|$ stars. The set $S_i$ can contain at most $1$ vertex from each star, implying $\dom(G_i) \geq |S_i|$.

\underline{Claim 2:} $\OPT \geq \dist(e_j)$. \\
For $i < j$ the algorithm finds a 2-stable set $S_i$ in $G_i$ with $|S_i| > k$. By Claim 1: $\dom(G_i) \geq |S_i| > k$.
Thus, an optimum solution to the \textsc{$k$-Center} problem has to use at least one edge of at least $\dist(e_j)$.

By Claim 2, we have a 2-approximation.
\end{proof} 
\begin{remark}
The analysis is tight:
Consider a graph consisting out of $K_n$ with distance $2$ on its edges and an additional vertex connected to all vertices with distance $1$. For the \textsc{$k$-Center} problem with $k = 1$, $\OPT = 1$. However, the algorithm may find a solution with value $2$.
\end{remark}
\begin{theorem} % Thm 26
\label{thm:26}
For no $\varepsilon > 0$ there exists a $(2-\varepsilon)$-approximation algorithm for the \textsc{$k$-Center} problem, unless $P = \NP$.
\end{theorem}
\begin{proof}
Reduce \textsc{Dominating Set} to this problem.
Let $G = (V, E)$ be an instance of \textsc{Dominating Set}.
Construct a complete graph $G'$ on $V(G)$ with
\[
	\dist(e) \coloneqq \begin{cases}
	1, & \text{if } e \in E(G) \\
	2, & \text{if } e \notin E(G) \\
	\end{cases}.
\]
If $\dom(G) \leq k$, then the \textsc{$k$-Center} problem has a solution with value $1$.
If $\dom(G) > k$, then the \textsc{$k$-Center} problem has a solution with value $2$.

A $(2-\varepsilon)$-approximation algorithm for the \textsc{$k$-Center} can distinguish between two cases: If $\OPT = 1$, then the algorithm returns a solution with value $2 - \varepsilon$ and if $\OPT = 2$ the algorithm returns $2$.
\end{proof}
\begin{remark}
\begin{itemize}
\item In the Euclidean case a $1.833$-approximation is known.
\item Its runtime is $\mathcal{O}(n\cdot m)$.
\end{itemize}
\end{remark}
\begin{problem}[Maximum Satisfiability (Max-Sat)]
\begin{tabular}{@{}ll}
\textit{Input:} & A set $\mathcal{C}$ of clauses with variables $X = \{ x_1, \ldots, x_n \}$ and \\
& weights $c : \mathcal{C} \to \R_+$ \\
\textit{Output:} & A truth assignment to the $x_i$ \ac{st} the total weight of all \\
& satisfied clauses of $\mathcal{C}$ is maximal.
\end{tabular}
\end{problem}
A simple randomized \sfrac{1}{2}-approximation algorithm: Assign randomly the value ``\textit{true}'' or ``\textit{false}'' to each variable.
What is the approximation ratio of this algorithm?
For a randomized algorithm $A$ we define its approximation ratio to be c, if
\[
	\mathbb{E}(A(I)) \leq c \cdot \OPT(I)
\]
(if $I$ is a minimization problem).
Set
\[
	Y_z \coloneqq \begin{cases}
	1, & \text{if clause } z \text{ is satisfied} \\
	0, & \text{else}
	\end{cases}
\]
and
\[
	|z| \coloneqq \text{\# of literals}.
\]
In a preprocessing step, we ensure each variable occurs one per clause, i.e.~we remove $x_1 \vee \overline{x_1}$.
Then,
\[
	\mathbb{E}(Y_z) = \Pr [Y_z = 1] = \frac{1}{2^{|z|}} \cdot \left( 2^{|z|} - 1 \right) = 1 - 2^{-|z|}.
\]
Hence,
\begin{IEEEeqnarray*}{rCl}
\mathbb{E} \left[ \sum_{z \in Z} Y_z \cdot c(z) \right] &=& \sum_{ z \in Z} c(z) \cdot \mathbb{E}(Y_z) \\
&=& \sum_{z \in Z} c(z) \cdot \left( 1 - 2^{-|z|} \right) \\
&\geq& \left(1 - 2^{-p} \right) \cdot \sum_{z \in Z} c(z),
\end{IEEEeqnarray*}
with $p \coloneqq \min_{z \in Z} |z|$. $p \geq 1$ implies this being a \sfrac{1}{2}-approximation.
\begin{remark}
This algorithm can be derandomized by the method of conditional probabilities.
\end{remark}
\begin{algorithm}[Johnson's Algorithm]
\begin{tabular}{@{}ll}
\textit{Input:} & Set $\mathcal{C}$ of clauses with variables $X = \{ x_1, \ldots, x_n \}$, $w \; : \; \mathcal{C} \to \R_+$ \\
\textit{Output:} & A truth assignment for $X$.
\end{tabular}
\begin{algorithmic}[1]
\State Set $\tilde{w}(C) \coloneqq \frac{w(C)}{2^{|\mathcal{C}|}}$ for $C \in \mathcal{C}$.
\For{$i \coloneqq 1$ to $n$}
\If{$\sum_{x_i \in C} \tilde{w}(C) > \sum_{\overline{x_i} \in C} \tilde{w}(C)$}
\State $y \coloneqq x_i$
\Else
\State $y \coloneqq \overline{x_i}$.
\EndIf
\State Assign \textit{true} to $y$.
\State Set
\[
\tilde{w}(C) \coloneqq \begin{cases}
0, & \text{if } y \in C \\
2 \cdot \tilde{w}(C), & \text{if } \bar{y} \in C
\end{cases}
\]
\EndFor
\end{algorithmic}
\end{algorithm}
\begin{theorem} % Thm 27
\label{thm:27}
\textsc{Johnson's Algorithm} for \textsc{Max-Sat} has approximation ratio $1 - 2^{-p}$ with $p \coloneqq \min_{C \in \mathcal{C}} |C|$.
\end{theorem}
\begin{proof}
\begin{itemize}
\item $\sum_{C \in \mathcal{C}} \tilde{w}(C)$ will never increase during the algorithm.
\item at the start we have
\[
	\sum_{C \in \mathcal{C}} \tilde{w}(C) = \sum_{C \in \mathcal{C}} \frac{w(C)}{2^{|C|}}
\]
\end{itemize}
Let $F \subseteq \mathcal{C}$ be the set of clauses not satisfied by the algorithm at termination. Then
\begin{IEEEeqnarray*}{rCl}
\sum_{C \in \mathcal{C}} \tilde{w}(C) &=& \sum_{F \subseteq \mathcal{C}} \tilde{w}(C) \\
&=& \sum_{F \subseteq \mathcal{C}} w(C) \\
&\leq& \sum_{C \in \mathcal{C}} \frac{w(C)}{2^{|C|}} \\
&\leq& 2^{-p} \sum_{C \in \mathcal{C}} w(C).
\end{IEEEeqnarray*}
Hence, the total weight of satisfied clauses is greater or equal to
\[
	\left(1 - 2^{-p} \right) \sum_{C \in \mathcal{C}} w(C).
\]
\end{proof}
\begin{example}

Is the analysis tight?
\begin{table}[h]
\begin{tabular}{cc}
$x_1$ & $\overline{x_1} \vee x_2$ \\
$w = 1$ & $w = 2$ \\
$\tilde{w} = \sfrac{1}{2}$ & $\tilde{w} = \sfrac{1}{2}$
\end{tabular}
\end{table}

Then $x_1$ would be \textit{false}. The algorithm achieves $2$ but $\OPT = 3$.
\end{example} 
\end{document}