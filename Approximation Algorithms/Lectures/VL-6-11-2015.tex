\documentclass[../skript.tex]{subfiles}

\begin{document}
As a next step, we want to discuss approximation algorithms with a ratio strictly smaller than 2. These algorithms work by computing approximations of $k$-Steiner trees.
We define $\SMT_k$ and $\smt_k$ as an optimum $k$-Steiner tree and its length, respectively.
It cannot be always guaranteed that a $3$-Steiner tree exists in a graph (consider a star with 4 leaves and a Steiner vertex in the middle). Therefore, we make the assumption that we take enough ($\leq n$) copies of $G$ which we identify in the terminals.
Moreover, we will view the \textsc{Steiner Tree Problem} as a problem on hypergraphs.

As we've seen already, the following inequalities hold:
\[
	\smt_k \geq \smt = \smt_n, \quad \smt_2 \leq 2 \cdot \smt.
\]
The \emph{Steiner ratio $r_k$} is defined as
\[
	r_k \coloneqq \sup_{G, R, c} \frac{\smt_k(G, R, c)}{\smt(G, R, c)}.
\]
We can restate $\smt_2 \leq 2 \cdot \smt$ then as $r_2 = 2$. In fact, one can find an explicit formula for $r_k$, which will be \cref{thm:73}. Beforehand, we need the following statement:
\begin{proposition} % Prop 72
\label{thm:72}
Let $T$ be a full binary tree. Then there is an injective mapping $f$ from the interior nodes of $T$ to the leaves of $T$ such that
\begin{enumerate}[(i)]
\item If $x$ is an interior node, then $f(x)$ is a descendant of $x$
\item For any two interior nodes $x$ and $y$ the two paths $p(x)$ from $x$ to $f(x)$ and $p(y)$ from $y$ to $f(y)$ are edge-disjoint
\item\label[equation]{eq:thm-73-iii} There is a leaf in $T$ such that the path from the root to this leaf is edge-disjoint to all paths $p(x)$.
\end{enumerate}
\end{proposition}
\begin{proof}
By induction: The beginning is trivial and for the induction step we use two subtrees of size $n$ and use the edge of the root node to its left child combined with the path guaranteed by \labelcref{eq:thm-73-iii} to guarantee the existence of that path in the larger tree. The edge to the right node with the path from \labelcref{eq:thm-73-iii} induces a leaf that we can assign to $f$ for the root node. By construction, this path is disjoint to all others.
\end{proof}
\begin{theorem}[Borchers, Du \lbrack{}1997\rbrack{}] % Thm 73
\label{thm:73}
Let $k = 2^r + s$ with $0 \leq s < 2^r$.
Then,
\[
	r_k = \frac{(r+1) \cdot 2^r + s}{r \cdot 2^r + s} \approx 1 + \frac{1}{\log k}.
\]
\end{theorem}
\begin{proof}
We first prove
\[
r_k \geq \frac{(r+1) \cdot 2^r + s}{r \cdot 2^r + s}.
\]
For this we construct a family of instances such that
\[
	\frac{\smt_k}{\smt} \xrightarrow{n \to \infty} r_k.
\]
Let $B_n$ be the following instance: We take a complete binary tree of $n$ levels and $2^n$ terminals which we connect to a single, different leaf of the tree each. The weights $c : E(B_n) \to \R_+$ are defined to be $1$ on the edges connecting the terminals with the leaves and inside the tree for level $1 \leq i \leq n$ we have a weight of $2^{n-i}$. This way, each level has a total weight of $2^n$.

As $B_n$ is a tree, an optimum Steiner tree has to contain all edges, i.e.
\[
	\smt = (n+1) 2^n.
\]
Let $T$ be a $k$-Steiner tree for $B_n$. Then, each full component $T_i$ of $T$ is a subtree of $B_n$ containing at most $k$ terminals. Each full component $T_i$ contains a \emph{root vertex} (highest-level Steiner vertex of degree 2) and some Steiner vertices of degree 3.

\underline{Claim 1:} There exists a minimum $k$-Steiner tree $T$ for $B_n$ such that each vertex of $B_n$ is used as root or a Steiner vertex of degree 3 in at most one full component of $T$.
\begin{proof}[Claim 1]
Suppose $T$ is a minimum $k$-Steiner tree but does not satisfy the claim. Let $v$ be the highest vertex in $B_n$ such that there exists two different full components $A$ and $B$ of $T$ such that $v$ is a root or Steiner vertex of degree 3 in $A$ and $B$. We call the subtrees of $A$ and $B$ in the left subtree of $v$ $A_1$ and $B_1$ and the ones in the right subtree $A_2$, $B_2$ respectively.
At least one of $A_1 \cap B_1$ and $A_2 \cap B_2$ contains no terminal (otherwise there was a circuit and we could remove an edge to lower the weight).
\ac{wlog} $A_2 \cap B_2$ contains no terminal.

We remove the left subtree from both $A$ and $B$ and add a new component that arises by connecting $A_2$ to $B_2$ using the path in tree. The choice of weights means this won't change the weight. This might create a new vertex of degree 3, but it has in a lower level than $v$, so we can iterate this argument to prove the claim.
\end{proof}
\underline{Claim 2:} There exists a minimum $k$-Steiner tree $T$ for $B_n$ such that each vertex of $B_n$ which is not a leaf or a neighbor of a leaf is used as the root or a Steiner vertex of degree 3 in exactly one full component of T.
\begin{proof}[Claim 2]
Let $T$ be a minimum $k$-Steiner tree satisfying the condition of Claim 1. A full component with exactly $i$ terminals contains one root and $i-2$ Steiner vertices of degree 3. This means, after adding a component of size $i$, we use $i-1$ many terminals and added $i-1$ vertices that are roots or Steiner vertices of degree 3. 
As there are $2^n$ many terminals, we end up with $2^n-1$ vertices that are roots or Steiner vertices of degree 3. As $B_n$ has only $2^n - 1$ vertices that are not leaves or neighbors thereof, this proves the statement.
\end{proof}
From now on let $T$ be a minimum $k$-Steiner tree that satisfies the condition of Claim 2 and let $T_1, \ldots, T_p$ be its components.
We partition the edges of each full component into two sets:
\begin{itemize}
\item \emph{$P$-edges:} Edges that connect the root vertex of $T_i$ or a Steiner vertex of degree 3 of $T_i$ to one of its children,
\item \emph{$C$-edges:} All the remaining edges that are not $P$-edges.
\end{itemize}
These will be denoted by $P(T_i)$ and $C(T_i)$, respectively.

\underline{Claim 3:} For each full component $T_i$ of $T$ we have
\[
	\sum_{e \in C(T_i)} c(e) \geq \frac{2^r}{r \cdot 2^r + s} \cdot \sum_{e \in P(T_i)} c(e).
\]
\begin{proof}[Claim 3]
It suffices to prove the statement for full components of size $k$ (remember $k = 2^r + s$).
If $k' < k$ with $k' = 2^{r'} + s'$ and $0 \leq s' < 2^{r'}$, then
\[
	\frac{2r'}{r' \cdot 2^{r'} + s'} > \frac{2^r}{r \cdot 2^r + s}
\]
(this can be proven by induction; for $r = r'$ it is obvious, for $k = 2^r$ and $k' = k - 1$ it is easily verified.)

We prove the claim by induction on the total distance of all Steiner vertices of degree 3 to the root of $T_i$ (i.e.\ the number of edges on shortest paths). This distance is minimal if $P(T_i)$ contains $2^r$ edges on some level $j$ and $2s$ edges at level $j + 1$. Let $w$ be the weight of edges starting on the $j+1$st level. Hence,
\[
\sum_{e \in P(T_i)} c(e) = 2^r \cdot 2 w \cdot r + 2 s \cdot w.
\]
For the edges in $C(T_i)$, we have that they form $2s$ paths of length $w$ and $2^r - s$ paths of lengths $2w$:
\[
\sum_{e \in C(T_i)} c(e) = 2s \cdot w + \left( 2^r -s \right) \cdot 2w = 2^r \cdot 2 w.
\]

Now assume $T_i$ does not minimize the total distance of all Steiner vertices of degree 3 in $T_i$ from the root of $T_i$.
Then either of the following two cases holds
\begin{enumerate}[(i)]
\item There is a Steiner vertex $x$ of degree 3 with an incoming edge belonging to $C(T_i)$
\item There exist two Steiner vertices of degree 3 in $T_i$ such that no vertex below these two vertices has degree 3 in $T_i$ and the level of these two vertices differs by at least 2.
\end{enumerate}
In case (i), we can take a subtree of $x$ and attach it to the parent node. That way we ``move up'' the vertex $x$. Then, $c(C(T_i)) = c(C(T'_i))$ and $c(P(T_i)) < c(P(T_i'))$. By induction, this yields
\begin{IEEEeqnarray*}{rCl}
\sum_{e \in C(T_i)} c(e) &=& \sum_{e \in C(T_i')} c(e) \\
&\geq& \frac{2^r}{r \cdot 2^r + s} \cdot \sum_{e \in P(T_i')} c(e) \\
&>& \frac{2^r}{r \cdot 2^r + s} \cdot \sum_{e \in P(T_i)} c(e) 
\end{IEEEeqnarray*}
The other case (ii) can be treated similarly. This proves the claim.
\end{proof}
Summing up the inequality of Claim 3 over all full components of $T_i$ gives:
\begin{IEEEeqnarray*}{rCl}
\sum_{i=1}^p \sum_{e \in C(T_i)} c(e) &\geq& \frac{2^r}{r \cdot 2^r + s} \cdot \sum_{i=1}^p \sum_{e \in P(T_i)} c(e).
\end{IEEEeqnarray*}
By Claim 2, we know that each vertex which is not a leaf or a neighbor thereof appears exactly once as a root or Steiner vertex of degree 3 in some $T_i$. Thus:
\[
	\sum_{i=1}^p \sum_{e \in P(T_i)} c(e) = \mathrm{smt} - 2^n.
\]
Now, combining these two, we get
\begin{IEEEeqnarray*}{rCl}
\smt_k(B_n) &=& \sum_{i=1}^p \sum_{e \in C(T_i)} c(e) + \sum_{i=1}^p \sum_{e \in P(T_i)} c(e) \\
&\geq& \left( 1 + \frac{2^r}{r \cdot 2^r + s} \right) \cdot \sum_{i=1}^p \sum_{e \in P(T_i)} c(e). \\
&=& \left( 1 + \frac{2^r}{r \cdot 2^r + s} \right) \cdot \left( \mathrm{smt} - 2^n \right).
\end{IEEEeqnarray*}
Hence,
\begin{IEEEeqnarray*}{rCl}
	r_k &\geq& \frac{\smt_k}{\smt} \\
	&\geq& \frac{(r+1) \cdot 2^r + s}{r \cdot 2^r + s} - \frac{2^n}{\smt} \\
	&=& \frac{(r+1) \cdot 2^r + s}{r \cdot 2^r +s} - \underbrace{ \frac{1}{n +1} }_{\to 0}
\end{IEEEeqnarray*}
For $n \to \infty$, this shows that
\[
	\frac{(r+1) \cdot 2^r + s}{r \cdot 2^r +s}
\]
is a lower bound for $r_k$.

For the other direction, we only prove the case $k = 3$, the general case is very similar. We have to prove that $r_3 \leq \frac{5}{3}$.
Start with an optimum Steiner tree. We will show that for any instance of the Steiner tree problem, we can define three 3-Steiner trees of a total length at most $5 \cdot \smt$.

We can make the following assumptions without loss of generality: \\
\underline{Assumption 1:} There is an optimum solution that is a full Steiner tree. \\
To see this, we use induction. Given an instance with terminal set $R$ and a minimum Steiner tree $\SMT$ that is not full, we must have a terminal of degree at least two. We can split the tree into two smaller trees on that terminal, so that we obtain trees for smaller terminal sets $R_1$, $R_2$ and
\[
	\smt(R) = \smt(R_1) + \smt(R_2).
\]
For the minimum 3-Steiner tree we have
\[
	\smt_3(R) \leq \smt_3(R_1) + \smt_3(R_2).
\]
This implies by using the induction hypothesis
\begin{IEEEeqnarray*}{rCl}
	\frac{\smt_3(R)}{\smt(R)} &\leq& \frac{\smt_3(R_1) + \smt_3(R_2)}{\smt(R_1) + \smt(R_2)} \\
	&\leq& \max \left\{ \frac{\smt_3(R_1)}{\smt(R_1)}, \frac{\smt_3(R_2)}{\smt(R_2)} \right\} \\
	&\leq& \frac{5}{3}
\end{IEEEeqnarray*}

\underline{Assumption 2:} The Steiner minimum tree is a complete binary tree in which all terminals are leaves.
Split vertices of degree greater 3, using edges of weight 0 and contract Steiner vertices of degree 2. This does not change the length of an optimal solution. By adding a vertex on an arbitrary edge of this tree, we obtain a Steiner minimum tree that is a binary tree where all terminals are leaves. By adding terminals using edges of length 0, we can assume this tree to be complete as well.

We label each vertex of the complete binary tree $\SMT$ by a 2-element subset of $\{ 1, 2, 3 \}$ and assign the root of $\SMT$ the set $\{ 1, 2\}$. For each vertex $x$ there exist two 2-element subsets of $\{ 1, 2, 3 \}$ that are not assigned to $x$. We assign one of these to the first child of $x$ and the other one to the second child of $x$.

From here on, we define 3-Steiner trees $T_1$, $T_2$, $T_3$ as follows. 
All vertices that contain $i$ in their label and the root of $\SMT$ are being chosen as roots of the full components within $T_i$. The edges of $T_i$ are defined as follows: Starting from the root $x$, we descend until a vertex is reached whose label contains $i$. Such vertices are called \emph{intermediate vertices}. For each intermediate vertex, we use function $f$ from \cref{thm:72} to add a path to a terminal.

To prove that each $T_i$ is a 3-Steiner tree, we first note that all full components contain at most $3$ terminals.
In order to prove that the $T_i$ are connected, we apply induction on the depth of the $\SMT$: Clearly this is true if the depth of $\SMT$ is 1. For an $\SMT$ of depth $n + 1$ with root $r$, we observe that by definition some full component in $T_i$ contains $r$ and both incident edges. Since the two children of $r$ are full binary trees of depth  $n$, $T_i$ has to be connected by induction.

Consider all paths $p(v)$ for intermediate vertices $v$ in some $T_i$. As all these paths are disjoint within each $T_i$ and in each vertex of $\SMT$ at most two of such paths start for different $T_i$, we can bound their total length by $2 \cdot \smt$.
On the other hand, all other edges of the $T_i$ up to the intermediate vertices have total length $\smt$. Thus
\[
	c(E(T_1)) + c(E(T_2)) + c(E(T_3)) \leq 2 \cdot \mathrm{smt} + 3 \cdot \mathrm{smt} = 5 \cdot \mathrm{smt}.
\]
Thus, there exists an $i$ such that
\[
	c(E(T_i)) \leq \frac{5}{3} \smt.
\]
\end{proof}
\end{document}