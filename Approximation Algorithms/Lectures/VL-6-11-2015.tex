\documentclass[../skript.tex]{subfiles}

\begin{document}
We define $\SMT_k$ and $\smt_k$ as an optimum $k$-Steiner tree and its length, respectively.
It cannot be always guaranteed that a $3$-Steiner tree exists in a graph (consider a 4-wheel). Therefore, we make the assumption that we take enough ($\leq n$) copies of $G$ which we identify in the terminals.
Moreover, we will view the \textsc{Steiner Tree Problem} as a problem on hypergraphs.

As we've seen already, the following inequalities hold:
\[
	\smt_k \geq \smt = \smt_n, \quad \smt_2 \leq 2 \cdot \smt.
\]
The \emph{Steiner ratio $r_k$} is defined as
\[
	r_k \coloneqq \sup_{G, R, c} \frac{\smt_k(G, R, c)}{\smt(G, R, c)}.
\]
We can restate $\smt_2 \leq 2 \cdot \smt$ then as $r_2 = 2$. In fact, one can find an explicit formula for $r_k$, which will be \cref{thm:73}. Beforehand, we need the following statement:
\begin{proposition} % Prop 72
\label{thm:72}
Let $T$ be a full binary tree. Then there is an injective mapping $f$ from the interior nodes of $T$ to the leaves of $T$ such that
\begin{enumerate}[(i)]
\item If $x$ is an interior node, then $f(x)$ is a descendant of $x$
\item For any two interior nodes $x$ and $y$ the two paths $p(x)$ from $x$ to $f(x)$ and $p(y)$ from $y$ to $f(y)$ are edge-disjoint
\item There is a leaf in $T$ such that the path from the root to this leaf is edge-disjoint to all paths $p(x)$.
\end{enumerate}
\end{proposition}
\begin{proof}
By induction: The beginning is trivial and the induction step can be easily verified.
\end{proof}
\begin{theorem}[Borchers, Du \lbrack{}1997\rbrack{}] % Thm 73
\label{thm:73}
Let $k = 2^r + s$ with $0 \leq s < 2^r$.
Then,
\[
	r_k = \frac{(r+1) \cdot 2^r + s}{r \cdot 2^r + s} \approx 1 + \frac{1}{\log k}.
\]
\end{theorem}
\begin{proof}
We first prove
\[
r_k \geq \frac{(r+1) \cdot 2^r + s}{r \cdot 2^r + s}.
\]
Let $B_n$ be the following instance: We take $2^n$ terminals with a complete binary tree above them. The weights $c : E(B_n) \to \R_+$ are defined to be $1$ on the lowest and second-lowest level, and above that the edge weights are double the weight of the level below. This way, each level has a total weight of $2^n$.
Then, an optimum Steiner tree has to contain all edges, i.e.
\[
	\smt = (n+1) 2^n.
\]
Let $T$ be a $k$-Steiner tree for $B_n$. Then, each full component $T_i$ of $T$ is a subtree of $B_n$ containing at most $k$ terminals. Each full component $T_i$ contains a root (``highest vertex'') and some Steiner vertices of degree 3.

\underline{Claim 1:} There exists a minimum $k$-Steiner tree $T$ for $B_n$ such that each vertex of $B_n$ is used as root or a Steiner vertex of degree 3 in at most one full component of $T$.
\begin{proof}[Claim 1]
Suppose $T$ is a minimum $k$-Steiner tree but does not satisfy the claim. Let $v$ be the highest vertex in $B_n$ such that there exists two different full components $A$ and $B$ of $T$ such that $v$ is a root or Steiner vertex of degree 3 in $A$ and $B$. We call the subtrees of $A$ and $B$ in the left subtree of $v$ $A_1$ and $B_1$ and the ones in the right subtree $A_2$, $B_2$ respectively.
At least one of $A_1 \cap B_1$ and $A_2 \cap B_2$ contains no terminal (otherwise there was a circuit and we could remove an edge to lower the weight).
\ac{wlog} $A_2 \cap B_2$ contains no terminal.

Informally, we have to remove the edge going to $v$ to $B_2$ and connect $A_2$ to $B_2$ using the path in tree. The choice of weights means this won't change the weight.
\end{proof}
\underline{Claim 2:} There exists a minimum $k$-Steiner tree $T$ for $B_n$ such that each vertex of $B_n$ which is not a leaf and has no leaf as a neighbor is used as the root or a Steiner vertex of degree 3 in exactly one full component of T.
\begin{proof}[Claim 2]
$2^n -1$ vertices of $B_n$ are not leaves or a neighbor thereof. We need $2^n - 1$ roots or a Steiner vertices of degree 3 in order to attain connectivity. Thus, all are needed.

Let
\begin{itemize}
\item $P$-edges: 2-edges between roots or a Steiner vertices of degree 3
\item $C$-edges: all others.
\end{itemize}
These will be denoted by $P(T_i)$ and $C(T_i)$, respectively.
\end{proof}
\underline{Claim 3:} For each full component $T_i$ of $T$ we have
\[
	\sum_{e \in C(T_i)} c(e) \geq \frac{2^r}{r \cdot 2^r + s} \cdot \sum_{e \in P(T_i)} c(e).
\]
\begin{proof}[Claim 3]
It suffices to prove the statement for full components of size $k = 2^r + s$.
If $k' < k$, then
\[
	\frac{2r'}{r' \cdot 2^{r'} + s'} > \frac{2^r}{r \cdot 2^r + s}.
\]
Proof by induction on total sum of level distances of all roots or a Steiner vertices of degree 3 to the root of $T_i$. Let $w$ be the weight of edges starting on the $r+1$st level. Hence,
\[
\sum_{e \in P(T_i)} c(e) = 2^r \cdot 2 \cdot w \cdot r + 2 \cdot s \cdot w.
\]
and
\[
\sum_{e \in C(T_i)} c(e) = 2 \cdot s \cdot w + \left( 2^r -s \right) \cdot 2w = 2^r \cdot 2 w.
\]
Now assume $T_i$ does not have this structure.
\begin{enumerate}[(i)]
\item There is a the root or a Steiner vertex of degree 3 whose parent isn't such. By exchanging them, the induction hypothesis can be applied
\item If there is one subtree with only such vertices and others that do not contain any, we can improve the solution by interchanging again.
\end{enumerate}
\end{proof}
Sum inequality of Claim 3 over all full components of $T_i$:
\begin{IEEEeqnarray*}{rCl}
\sum_{i=1}^P \sum_{e \in C(T_i)} c(e) &\geq& \frac{2^r}{r \cdot 2^r + s} \cdot \underbrace{ \sum_{i=1}^P \sum_{e \in P(T_i)} c(e) }_{= \smt - 2^n}.
\end{IEEEeqnarray*}
Now,
\begin{IEEEeqnarray*}{rCl}
\smt_k(B_n) &=& \sum_{i=1}^P \sum_{e \in C(T_i)} c(e) + \sum_{i=1}^P \sum_{e \in P(T_i)} c(e) \\
&\geq& \left( 1 + \frac{2^r}{r \cdot 2^r + s} \right) \cdot \left( \smt - 2^n \right).
\end{IEEEeqnarray*}
Hence,
\[
	r_k \geq \frac{\smt_k}{\smt} \geq \frac{(r+1) \cdot 2^r + s}{r \cdot 2^r + s} - \frac{2^n}{\smt} = \frac{(r+1) \cdot 2^r + s}{r \cdot 2^r +s} - \underbrace{ \frac{1}{n +1} }_{\to 0}
\]

For the other direction: Prove that $r_3 \leq \frac{5}{3}$ (the general case is essentially the same).
Start with an optimum Steiner tree. We will present 3 3-Steiner trees of a total length at most $5 \cdot \smt$.

\underline{Assumption 1:} Optimum solution is a full Steiner tree. If not: use induction.
Given $R_1$ and $R_2$ with $D = R_1 \cap R_2$, we get
\[
\frac{\smt_k(R_1) + \smt_k(R_2)}{\smt(R_1) + \smt(R_2)} \leq \max \left\{ \frac{\smt_k(R_1)}{\smt(R_1)}, \frac{\smt_k(R_2)}{\smt(R_2)} \right\}.
\]
\underline{Assumption 2:} $\SMT$ is a complete binary tree with exactly its leaves being terminals.
Split vertices of degree at least 4, connect them using an edge of weight 0, contract Steiner vertices of degree 2. Add terminals with edges of length 0.
\end{proof}
\end{document}