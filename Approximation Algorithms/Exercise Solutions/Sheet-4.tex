\documentclass[oneside,a4paper]{amsart}

\usepackage{geometry}
\usepackage{parskip}
\usepackage{mathtools}
\usepackage{enumerate}
\usepackage{IEEEtrantools}
\usepackage{relsize}
\usepackage{multirow}

\newcommand{\NP}{\ensuremath{N \! P}}
\newcommand{\tp}{\ensuremath{\mathsmaller T}}

\usepackage{algorithm2e}
\usepackage{tikz}
\newcommand*\circled[1]{\tikz[baseline=(char.base)]{\node[shape=circle,draw,inner sep=1pt] (char) {#1};}}

% Algorithm2e specific
\IncMargin{2.5em}
\SetNlSty{circled}{}{}
\SetNlSkip{1em}
\DontPrintSemicolon
\newcommand{\EndAlgorithmLine}{\vskip\medskipamount % or other desired dimension
\leaders\vrule width \textwidth\vskip0.4pt % or other desired thickness
\vskip\smallskipamount % ditto
\nointerlineskip}

\begin{document}
\title{Approximation Algorithms - Exercise Sheet 4 solutions}
\maketitle{}
\section*{Exercise 4.1}
Not discussed in the tutorial session.
\section*{Exercise 4.2}
Define
\[
	a = \frac{\sqrt{5} - 1}{2}.
\]
Let $I$ be a \textsc{Sat} instance with variables $x_1, \ldots, x_n$ and clauses $C_1, \ldots, C_m$.
Set literal $\alpha$ to \textit{true} with probability
\[
	\begin{cases}
	a, & \text{if $\alpha$ appears as in a one-elemental clause} \\
	\frac{1}{2}, & \text{otherwise}.
	\end{cases}
\]
Now, a clause $C$ is satisfied with probability
\begin{itemize}
\item $a$ if $C$ is a one-elemental clause.
\item \underline{Claim:} greater or equal to $a$: For a clause $C = \{a_1 \vee \ldots \vee a_l \}$, we have
\[
\mathbb{P}[C \text{ is satisfied}] = 1 - \prod_{i=1}^l \mathbb{P}[a_i = \textit{false}].
\]
Additionally:
\[
	\mathbb{P}[a_i = \textit{false}] = \begin{cases}
	\frac{1}{2} \\
	a \\
	1 - a
	\end{cases} \overset{\frac{1}{2} < a < 1}{\leq} a.
\]
(Here $a$ and $1-a$ are due to (negated) one-element clauses).
Then:
\begin{IEEEeqnarray*}{rCl}
\mathbb{P}[C \text{ is satisfied}] &=& 1 - \prod_{i=1}^l \mathbb{P}[a_i = \textit{false}] \\
&\geq& 1 - a^l \\
&\geq& 1 - a^2 \\
&=& 1 - \left( \frac{\sqrt{5} - 1}{2} \right)^2 \\
&=& \frac{4 - 5 - 2 \sqrt{5} - 1}{4} \\
&=& \frac{-2 +2 \sqrt{5}}{4} \\
&=& \frac{\sqrt{5} - 1}{2}.
\end{IEEEeqnarray*}
This gives us
\[
	\mathbb{E}[\text{\# satisfied clauses}] \geq m \cdot a.
\]
\end{itemize}
\underline{Method of conditional expectation:} \\
Consider
\begin{IEEEeqnarray*}{l"ll}
	\IEEEeqnarraymulticol{3}{l}{ \mathbb{E}[\text{\# satisfied clauses} \mid x_1, \ldots, x_{k-1} \text{ already set}] = } \\
	&& \mathbb{E}[\text{\# satisfied clauses} \mid x_1, \ldots, x_{k-1}, x_i = \textit{true}] \cdot P[x_i = \textit{true}] \\
	& {} + {} & \mathbb{E}[\text{\# satisfied clauses} \mid x_1, \ldots, x_{k-1}, x_i = \textit{false}] \cdot P[x_i = \textit{false}]
\end{IEEEeqnarray*}
\EndAlgorithmLine
\begin{algorithm}[H]
\For{$i \coloneqq 1$ \KwTo $m$}{
	Set $x_i$ to $\begin{cases}
		\textit{true} \\
		\textit{false}
	\end{cases}$\;
	Compute expectation values for the number of clauses in both cases, take the assignment with better expectation value\;
}
\end{algorithm}
\EndAlgorithmLine
\section*{Exercise 4.3}
We can partition the set of vertices into 2 disjoint sets $S$, $V \setminus S$, then for each such enumerate we check if the 2 partitions are 2-colorable. If we can find a partition $S$ where $S$ and $V \setminus S$ are each 2-colorable, then we conclude that $G$ is 4-colorable, otherwise it is not 4-colorable.
Deciding if the graph is 2-colorable is equivalent to determining whether it is bipartite (does not contain odd cycles). This can be done in $\mathcal{O}(|V| + |E|)$. Enumeration complexity: $2^{|V|}$. Thus, we attain a runtime $\mathcal{O}(2^{|V|} \cdot |E|)$.
\section*{Exercise 4.4}
As long as there is a vertex $v_i$ with degree greater or equal to $n^\frac{2}{3}$, we color
\[
	H_i = G[\Gamma(v_i)],
\]
which is 3-colorable with $\mathcal{\sqrt{n_i}}$ colors, where $n_i = |\Gamma(v_i)|$. Afterwards, we remove $H_i$ and $v_i$ from the graph and iterate.
When all vertices have degree smaller than $n^\frac{2}{3}$, we color the rest by the \textsc{Greedy-Coloring} with $\mathcal{O}(n^\frac{2}{3})$ many colors.

Let $k$ be the number of $H_i$'s we color. We need $\mathcal{O}(\sqrt{n_i})$ colors each time. In total this gives us
\[
	\mathcal{O} \left( \sum_{i=1}^k \sqrt{n_i} \right)
\] 
many colors.
However, by Jensen's inequality
\[
	\sum_{i=1}^k \sqrt{n_i} = k \cdot \frac{\sum_{i=1}^k \sqrt{n_i}}{k} \leq k \cdot \left( \frac{\sum_{i=1}^k n_i}{k} \right)^\frac{1}{2}.
\]
Thus,
\[
	\sum_{i=1}^k \sqrt{n_i} \leq k \left( \frac{n}{k} \right)^\frac{1}{2} = k^\frac{1}{2} n^\frac{1}{2} \leq \left( n^\frac{1}{3} \right)^\frac{1}{2} \cdot n^\frac{1}{2},
\]
where $k \leq n^\frac{1}{3}$ because we remove at least $n^\frac{2}{3}$ many vertices in each iteration and thus
\[
	k \leq \frac{n}{n^\frac{2}{3}} = n^\frac{1}{3}.
\]
Hence, in total we need $\mathcal{O}(n^\frac{2}{3})$ many colors.
The running time is obviously polynomial.
\section*{Exercise 4.5}
Not discussed in the tutorial session. The second part has been proven in the lecture.
\end{document}