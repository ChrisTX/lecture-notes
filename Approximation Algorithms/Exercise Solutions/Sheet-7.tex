\documentclass[oneside,a4paper]{amsart}

\usepackage{geometry}
\usepackage{parskip}
\usepackage{mathtools}
\usepackage{enumerate}
\usepackage{relsize}
\usepackage{multirow}
\usepackage{xfrac}

\newcommand{\NP}{\ensuremath{N \! P}}
\newcommand{\tp}{\ensuremath{\mathsmaller T}}
\newcommand{\gmid}{\mathrel{}\middle|\mathrel{}}
\newcommand{\midcolon}{\mathrel{}:\mathrel{}}
\newcommand{\smallo}{\ensuremath{\mathit{o}}}

\usepackage[vlined,linesnumbered]{algorithm2e}
\usepackage{cleveref}
\usepackage{IEEEtrantools}
\usepackage{tikz}
\newcommand*\circled[1]{\tikz[baseline=(char.base)]{\node[shape=circle,draw,inner sep=1pt] (char) {#1};}}

\DeclareMathOperator{\OPT}{OPT}

% Algorithm2e specific
\IncMargin{2.5em}
\SetNlSty{circled}{}{}
\SetNlSkip{1em}
\DontPrintSemicolon
\newcommand{\EndAlgorithmLine}{\vskip\medskipamount % or other desired dimension
\leaders\vrule width \textwidth\vskip0.4pt % or other desired thickness
\vskip\smallskipamount % ditto
\nointerlineskip}

\DeclareMathOperator{\dist}{dist}

\begin{document}
\title{Approximation Algorithms - Exercise Sheet 7 solutions}
\maketitle{}
\section*{Exercise 7.1}
\subsection*{(i)}
Let $m$ be the number of different item sizes and let them be ordered, i.e.\ we'd have $c_1 \leq c_2 \leq \ldots \leq c_m$. We can describe an instance of this restricted \textsc{Bin Packing Problem} by a numbers $n_1, \ldots, n_m \in \mathbb{N}$, denoting the multiplicity of each item size, i.e.\ we have $n_1$ many items of size $c_1$ and so on. Given an instance with $m$ different item sizes, we can determine this ordering and the numbers in $\mathcal{O}(n \log n)$.

The idea is now to use dynamic programming where we pack a bin in each step and consider the minimum number of bins required to packing the remaining items. Let $\operatorname{bins}(q_1, q_2, \ldots, q_m)$ be the minimum amount of bins required to pack for $i = 1, \ldots, m$ $q_i$ many items of size $c_i$ together in a single bin. Moreover, $0 \leq q_i \leq n_i$, as we only have $n_i$ many items of size $c_i$ in total. Therefore, only $\prod_{i=1}^m n_i \in \mathcal{O}(n^m)$ many of such vectors exist.
Therefore, we can compute in $\mathcal{O}(n^m)$ time the set $Q$ of such vectors with $\operatorname{bins}(q_1, q_2, \ldots, q_m) = 1$.

From there on, we can use for elements not in $Q$ the recursion
\[
	\operatorname{bins}(l_1, \ldots, l_m) = 1 + \min_{q \in Q} \operatorname{bins}(l_1 - q_1, \ldots, l_m - q_m).
\]
This way, we can calculate $\operatorname{bins}(n_1, \ldots, n_m)$. Since $\sum_{i=1}^m l_i - q_i < \sum_{i=1}^m l_i$, we obtain a recursion depth of at most $\mathcal{O}(n^m)$ and in each step we need $\mathcal{O}(n^m)$ to compute the minimum. Thus, this takes $\mathcal{O}(n^{2m})$ time in total.
\subsection*{(ii)}
Transform the instance $I$ by rounding. Let
\begin{IEEEeqnarray*}{rCl}
I_{< \varepsilon} &=& (a \in I \mid a < \varepsilon) \\
I_{\geq \varepsilon} &=& (a \in I \mid a \geq \varepsilon)
\end{IEEEeqnarray*}
Denote by $(n_1, \ldots, n_m)$:
\[
	n_i \coloneqq |(a \in I \setminus I_{< \varepsilon} \mid a \in [\varepsilon(1+ \varepsilon)^{i-1}, \varepsilon(1+\varepsilon)^i) |.
\]
Then, the rounded instance has
\[
	m = \left\lceil \log_{1+\varepsilon} \frac{1}{\varepsilon} \right\rceil.
\]
Now apply (i) on $I_{\geq \varepsilon}$.
By reverting the rounding, this increases the required bin size by at most $(1 + \varepsilon)$.
Since we only decrease item size and count the solution requires at most $\OPT(I)$ many bins, since the algorithm from (i) is exact.

Now, add the elements from $I_{< \varepsilon}$ using \textsc{Greedy} (we only add a new bin if required). We can only add bins if every other bin is filled with items amounting to weight greater 1. Thus, the optimum (correct) solution has to require an extra bin, too.
\subsection*{(iii)}
If $T \leq \sum t_i$, $\log(T) \leq \log( \sum t_i)$ and this is polynomially bound in the input size.

The idea is to consider the processors as bins of size $T$, where $T$ is the optimal processing time. In this case, we've got to solve bin packing with that unknown time.
We apply the algorithm of the previous part and use binary search to iteratively find a suitable processing time $T$. If the algorithm requires at most $m$ bins, set the right boundary to $(1 + \varepsilon)$ times the current time estimate. Otherwise, let the left bound to the current time estimate.
\section*{Exercise 7.2}
Let's discuss the implementability first: Use Dijkstra (on the directed graph obtained by replacing each edge with a pair of opposing, directed edges) starting from $v_1, v_2$ and $v_3$ and store the result. This can be done in $\mathcal{O}(m + n \log n)$. Using that information, we can determine $P$ and $a$ in linear time. Afterwards, traverse all vertices and if they fulfill the conditions (which can be checked in constant time after this step) and compute $\sum_{i=1}^3 \operatorname{dist}(v_i, z)$ for each one. By sorting the list of vertices obtained this way ($\mathcal{O}(n \log n)$), we can determine a minimizer and use the shortest paths to it computed by Dijkstra earlier. Overall, that's $\mathcal{O}(m + n \log n)$.

As for correctness, first note that the algorithm will always output a Steiner tree: For this, we just have to show that the union of the paths does not intersect. Let w.l.o.g. the $v_1$-$z$ and $v_2$-$z$ paths intersect, and let $y$ be their last common vertex. By taking $y$ and the segment of the $v_1$-$z$-path from $z$ to $y$ (which is as long as the segment of the $v_2$-$z$ path, since both were shortest paths), we obtain $\dist(v_3, y) \leq \dist(z, y) + \dist(z, v_3)$. Furthermore, $\dist(v_i, y) \leq \dist(v_i, z) - \dist(z, y)$ for $i \in \{ 1, 2\}$. Thus:
\[
	\sum_{i=1}^3 \dist(v_i, y) \leq \sum_{i=1}^3 \dist(v_i, z) + \dist(z, y) - 2 \cdot \dist(z, y) < \sum_{i=1}^3 \dist(v_i, z).
\]
Moreover, $\dist(v_i, y) < \dist(v_i, z) \leq \dist(v_1, v_2)$ for $i \in \{ 1, 2\}$. Assume $\dist(v_3, y) = \dist(v_3, z) + \dist(z, y) > a$. As the new shortest $v_1, y$ and $v_2, y$ paths concatenated yield a $v_1$-$v_2$ path, we have that $\dist(v_1, y) + \dist(v_2, y) \geq |P|$. Thus, $\dist(v_3, y) > a$ or else $z$ is not a minimum. Then, $P$ together with the $v_3$-$P$ path yields a solution smaller or equal to this and the endpoint of the $v_3$-$P$ path fulfills both conditions by definition. The shortests paths from that endpoint to the 3 vertices cannot intersect each other either, by definition. Hence, $z$ cannot be a minimizer.

On the other hand, consider an optimum Steiner tree for these terminals in the graph: We can decompose it into a path $P'$ from $v_1$ to $v_2$ and a path $Q$ from $v_3$ to some vertex $v^*$ in $P$. The weight of the tree is then $\sum_{i=1}^3 \dist(v_i, v^*)$. Note that because $P'$ is a $v_1$-$v_2$-path, $\dist(v_1, v^*) + \dist(v_2, v^*) = |P'| \geq |P| = \dist(v_1, v_2)$. Therefore, $\dist(v_3, v^*) \leq a$, since otherwise the Steiner tree found by the algorithm would be of smaller weight. As of such, $v^*$ induces a Steiner tree for the algorithm.
\end{document}