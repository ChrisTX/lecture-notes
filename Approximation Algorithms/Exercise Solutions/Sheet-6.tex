\documentclass[oneside,a4paper]{amsart}

\usepackage{geometry}
\usepackage{parskip}
\usepackage{mathtools}
\usepackage{enumerate}
\usepackage{relsize}

\newcommand{\NP}{\ensuremath{N \! P}}
\newcommand{\tp}{\ensuremath{\mathsmaller T}}
\newcommand{\gmid}{\mathrel{}\middle|\mathrel{}}
\newcommand{\midcolon}{\mathrel{}:\mathrel{}}
\newcommand{\smallo}{\ensuremath{\mathit{o}}}

\usepackage[vlined,linesnumbered]{algorithm2e}
\usepackage{cleveref}
\usepackage{tikz}
\usepackage{IEEEtrantools}
\newcommand*\circled[1]{\tikz[baseline=(char.base)]{\node[shape=circle,draw,inner sep=1pt] (char) {#1};}}

\DeclareMathOperator{\OPT}{OPT}

% Algorithm2e specific
\IncMargin{2.5em}
\SetNlSty{circled}{}{}
\SetNlSkip{1em}
\DontPrintSemicolon
\newcommand{\EndAlgorithmLine}{\vskip\medskipamount % or other desired dimension
\leaders\vrule width \textwidth\vskip0.4pt % or other desired thickness
\vskip\smallskipamount % ditto
\nointerlineskip}

\begin{document}
\title{Approximation Algorithms - Exercise Sheet 6 solutions}
\maketitle{}
\section*{Exercise 6.1}
\EndAlgorithmLine
\vspace{-7pt}
\begin{algorithm}[H]
Sort the items such that $a_1 \geq a_2 \geq \ldots \geq a_n$\label{ex61-s1}\;
$i=1$, $j = n$\;
\While{$i < j$}{
	\eIf{$a_i + a_i \leq 1$}{
		Put $a_i$ and $a_j$ into new bin $i = i + 1$, $j = j - 1$\;
	}{
		Put $a_i$ is new bin $i = i +1$\;
	}
}
\end{algorithm}
\vspace{-7pt}
\EndAlgorithmLine
Runtime: \cref{ex61-s1} $\mathcal{O}(n \log n)$. The number of iterations is at most $n$ and we have cost $\mathcal{O}(1)$ per iteration.

Correctness: Consider the $j$ at the end of the algorithm. We have $n - j$ bins with two items. Then, every item $a_k$, $k > j$ is matched with another item. Need to show that an optimum solution does not have more matches.
\begin{itemize}
\item Case 1: $a_j > \frac{1}{2}$. Clear by ordering.
\item Case 2: W.l.o.g.\ assume all $a_j$ during the algorithm were also at most $\frac{1}{2}$ and every unmatched $a_i > \frac{1}{2}$. Let $a_i$ be unmatched. If there is no $a_j$ to match to put it in a new bin.
\end{itemize}
\section*{Exercise 6.2}
\subsection*{(i)}
\textsc{Next Fit} is monotone.
Let $S$ be the instance $a_1, \ldots, a_n$ and let $T$ be the instance
\[
a_1, \ldots, a_{i_1}, b_1, a_{i_1+1}, \ldots, a_{i_2}, b_2, a_{i_2+1}, \ldots, a_n \quad (b_1, \ldots, b_k).
\]
Let $f_S : \{ 1, \ldots, n \} \to \{ 1, \ldots, m_S\}$, $f_T : \{ 1, \ldots, n + k \} \to \{ 1, \ldots, m_T\}$ be the outputs of \textsc{Next Fit}.

Claim: $m_T \geq m_S$ \\
Observation: $f_S^{-1}(l) = \{ \lambda_l, \ldots, \lambda_{l+1} - 1\}$ for some $\lambda_1, \ldots, \lambda_{m_S + 1}$.
Analogously: $f_T^{-1}(l) = \{ \mu_l, \ldots, \mu_{l+1} - 1 \}$ for $\mu_1, \ldots, \mu_{m_T + 1}$.
We have:
\begin{enumerate}[(i)]
\item\label{ex62-i} $\lambda_1 = \mu_1 = 1$
\item\label{ex62-ii} $\lambda_{m_S + 1} = n +1$, $\mu_{m_T + 1} = n + m + 1$.
\item\label{ex62-iii} $a_{\lambda_l} + a_{\lambda_l + 1}, \ldots, a_{\lambda_{l+1}} > 1 \quad \forall l$ (otherwise we could put item $a_{\lambda_{l+1}}$ into the bin $l$). Similar for $T$.
\end{enumerate}
Prove by induction on $l$: $\lambda_l \leq \mu_l$.
For $l = 1$, this is \cref{ex62-i}. For $l > 1$ we have by induction $\lambda_{l-1} \leq \mu_{l-1}$. Assume $\lambda_l > \mu_l$. Then
\[
	\mu_{l-1} \leq \lambda_{l-1} < \lambda_l < \mu_l.
\]
But
\[
	\sum_{i=\mu_{l-1}}^{\mu_l} b_i \geq \sum_{i = \lambda_{l-1}}^{\lambda_l} a_i > 1
\]
by \cref{ex62-iii} for $S$. Contradiction by \cref{ex62-iii} for $T$.
In particular,
\[
	\mu_{m_S + 1} \leq \lambda_{m_S + 1} \overset{\text{\cref{ex62-ii}}}{=} n + 1 < n +k + 1 = \mu_{m_T + 1}.
\]
Thus,
\[
	m_S + 1 < m_T + 1.
\]
\subsection{(ii)}
\textsc{First Fit} is not monotone.
Counterexample:
\begin{IEEEeqnarray*}{rCl}
	S &=& \left( \frac{1}{2}, \frac{3}{4}, \frac{1}{2}, \frac{1}{2}, \frac{1}{8}, \frac{3}{8} - \varepsilon, \frac{1}{4} \right) \\
	T &=& \left( \frac{1}{2}, \frac{3}{4}, \varepsilon, \frac{1}{2}, \frac{1}{2}, \frac{1}{8}, \frac{3}{8} - \varepsilon, \frac{1}{4} \right)
\end{IEEEeqnarray*}
\textsc{First Fit} requires 4 bins on $S$ and 3 bins on $T$.
\section*{Exercise 6.3}
\subsection*{(i)}
$a_1, \ldots, a_n$, $k \in \mathbb{N}$ instance of $k$-Partition. $A \coloneqq \{ a_1', \ldots, a_n' \}$. $t(a_i') \coloneqq a_i \;\; \forall i = 1, \ldots, n$, $m = k$.
If
\[
	\max_{i=1}^m \left\{ \sum_{a \in A_i} t(a) \right\} \text{ is minimum}  \quad \Leftrightarrow \quad \forall i \sum_{a \in A_i} t(a) = K, \;\; K \text{ constant}
\]
As $k$-Partition is strongly \NP-hard, \textsc{Multiprocessor Scheduling} is \NP-hard.
\[
	(K = ) \OPT(I) \leq q(\operatorname{size}(I), \operatorname{largest}(I)).
\]
By Exercise 5.4, there is no FPAS.
\subsection*{(ii)}
Let $A = \bigcup_{i=1}^m A_i'$ be the solution of \textsc{Greedy} and $a_{n'}$ be the last element considered in the algorithm in the set $A_j'$ such that
\[
	\sum_{a \in A_j} t(a)
\]
is maximum in the end.
\[
	\underbrace{ \sum_{a \in A_j \setminus \{ a_{n_i} \}} }_{\leq \OPT} t(a) \leq \sum_{a \in A_i'} t(a) \quad \forall i \in \{ 1, \ldots, m \} \setminus \{ j \}.
\]
This with $t(a_{n'}) \leq \OPT$ gives $2 \OPT \geq \sum_{a \in A_i'} t(a) = \operatorname{Greedy}(I)$. Thus we have a 2-factor approximation.
\subsection*{(iii)}
Let $M = \sum_{a \in A} t(a)$. Let $k$ be the optimum value, then we have $k \geq \frac{M}{m}$.
Now let, $\varepsilon > 0$. There are at most $\frac{m}{\varepsilon}$ elements with size at least $\frac{M}{m} \varepsilon$.

We can find the optimum solution for these elements in $m^{\frac{m}{\varepsilon}} = m^m \cdot m^{\frac{1}{\varepsilon}}$ (constant time). Now we use \textsc{Greedy} (polynomial time). This algorithm is at most $\frac{M}{m} \varepsilon \leq k \cdot \varepsilon$ higher than $k$ ($1 + \varepsilon$)-approximation. In each step we have that the \textsc{Greedy} assigns $b$ to $i$ then we have
\[
\sum_{a \in A} (ta) \leq \sum_{a \in A_j} t(a) \quad \forall j \neq i.
\]
Thus for the last element we have
\[
\sum_{a \in A_i, a \neq b} t(a) + t(b) \leq k + t(b) \leq k (1+\varepsilon).
\]
\end{document}