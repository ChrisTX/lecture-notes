\documentclass[oneside,a4paper]{amsart}

\usepackage{geometry}
\usepackage{parskip}
\usepackage{mathtools}
\usepackage{enumerate}
\usepackage{IEEEtrantools}
\usepackage{relsize}
\usepackage{multirow}
\usepackage{xfrac}

\newcommand{\NP}{\ensuremath{N \! P}}
\newcommand{\tp}{\ensuremath{\mathsmaller T}}
\newcommand{\gmid}{\mathrel{}\middle|\mathrel{}}
\newcommand{\midcolon}{\mathrel{}:\mathrel{}}
\newcommand{\smallo}{\ensuremath{\mathit{o}}}

\usepackage[vlined,linesnumbered]{algorithm2e}
\usepackage{cleveref}
\usepackage{tikz}
\newcommand*\circled[1]{\tikz[baseline=(char.base)]{\node[shape=circle,draw,inner sep=1pt] (char) {#1};}}

\DeclareMathOperator{\OPT}{OPT}

% Algorithm2e specific
\IncMargin{2.5em}
\SetNlSty{circled}{}{}
\SetNlSkip{1em}
\DontPrintSemicolon
\newcommand{\EndAlgorithmLine}{\vskip\medskipamount % or other desired dimension
\leaders\vrule width \textwidth\vskip0.4pt % or other desired thickness
\vskip\smallskipamount % ditto
\nointerlineskip}

\DeclareMathOperator{\dist}{dist}
\DeclareMathOperator{\SMT}{SMT}
\DeclareMathOperator{\smt}{smt}

\begin{document}
\title{Approximation Algorithms - Exercise Sheet 8 solutions}
\maketitle{}
\section*{Exercise 8.1}
Consider
\[
	[t_i, t_j] \coloneqq \{ t_i, t_{i+1}, \ldots, t_k \}
\]
(indices are taken modulo here).

We use here
\[
	\smt_k ([t_i, t_j] + v) = \min_{i \leq k < j - 1} \{ \smt([t_i, t_k] + v) + \smt([t_{k+1}, t_j] + v)\}
\]
and
\[
	\smt([t_i, t_j] + v) = \min \left\{ \min_{i \leq k \leq j} \{ \dist(t_k, v) + \smt([t_i, t_j]) \}, \min_{w \notin [t_i, t_j]} \{ \dist(v, w) + \smt_w([t_i, t_j] + w) \} \right\}.
\]
Consider a Steiner Minimum Tree $T$ for $[t_i, t_j] + v$ with $\delta_T(v) \geq 2$.
Take the unique $t_i$-$v$-path $P_i$ and denote its last edge as $e = \{ w, v \}$
Take subtree $T'$ of $T$ rooted at $w$ and add $v$.
$T'$ is a Steiner Minimum Tree for a subset of $([t_i, t_j]) + v$.
Claim: Subtree has form $[t_i, t_k]$. $t_k$ as ``maximum'' vertex contained.
Suppose $t_l$ not part of the subset with $i \leq l \leq k$.
$t_k$-$v$-path $P_k$, $t_l$-$v$-path $P_l$. Consider area bounded by outer face, $P_i$, $P_l$. $P_l$ starts within but its last edge is outside, implying we have a cycle and $T$ was not a tree.

The rest is as in Dreyfus-Wagner.
\section*{Exercise 8.2}
Use the same proof as in the lecture but take a cycle through these ordered terminals. Then $c(C) \leq 2 \cdot \smt(G)$. We can delete the longest path between two terminals and obtain a path $P$ from that.
Then $c(T) \leq c(P) \leq c(C) (1 - \frac{1}{t}) \leq 2 \smt(G) (1 - \frac{1}{t})$.

For the second part, consider $K_{n+1}$, $v_0 \in V(K_{n+1})$, $v \in V(K_{n+1}) \setminus \{ v_0 \}$ terminals.
\[
	c(e) = \begin{cases}
	1 & v_0 \in e\\
	2 & \text{else}
	\end{cases}
\]
Then, $\smt = n$, but $T = 2(n-1)$, and $2 \cdot \frac{2}{n}$, thus we are tight.
\section*{Exercise 8.3}
\subsection*{(i)}
Add a vertex $v$ with zero-cost edges $\{ v, s \} \quad \forall s \in S$ to obtain $G'$. Compute a MST in $G'$.
Return solution restricted to the original graph.

Correctness: Let $T$ be some optimum solution. Adding the zero-cost edges transforms it into a spanning tree in $G'$. The algorithm computes a minimum spanning tree in $G'$, thus the returned solution is optimum.
\subsection*{(ii)}
If $|S| = 1$, this is equivalent to the Steiner tree problem. To obtain a 2-factor approximation: We proceed as in the first part but use a 2-factor approximation for SMT instead of MST (e.g.\ Kou-Markowsky-Berman) for $T = \{v\} \cup R$.
\section*{Exercise 8.4}
$G = (V, E)$, add vertex of degree at most $B$.
$G' = (V', E')$, $R \subseteq V'$ terminals, $c \equiv 1$
\begin{itemize}
\item $V' = S \cup R = \{ x_v \mid v \in V\} \cup \{ x_{v,w} = x_{w,v} \mid \{v,w\} \in E\}$.
\item connect all $x_v \in S$.
\item connect $x_v$ to $x_{v,w}$ if $v \in \{ v, w \}$
\end{itemize}
If you have a vertex cover $C$ in $G$ of size $K$. Take a path in $G'$ through all vertices corresponding to $C$ and connect each $x \in R$ to it (once). This is a Steiner tree of size $m + k - 1$. Other direction works the same.
\end{document}