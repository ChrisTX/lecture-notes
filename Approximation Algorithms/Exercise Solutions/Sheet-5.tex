\documentclass[oneside,a4paper]{amsart}

\usepackage{geometry}
\usepackage{parskip}
\usepackage{mathtools}
\usepackage{enumerate}
\usepackage{IEEEtrantools}
\usepackage{relsize}
\usepackage{multirow}
\usepackage{xfrac}

\newcommand{\NP}{\ensuremath{N \! P}}
\newcommand{\tp}{\ensuremath{\mathsmaller T}}
\newcommand{\gmid}{\mathrel{}\middle|\mathrel{}}
\newcommand{\midcolon}{\mathrel{}:\mathrel{}}
\newcommand{\smallo}{\ensuremath{\mathit{o}}}

\usepackage{algorithm2e}
\usepackage{tikz}
\newcommand*\circled[1]{\tikz[baseline=(char.base)]{\node[shape=circle,draw,inner sep=1pt] (char) {#1};}}

\DeclareMathOperator{\OPT}{OPT}

% Algorithm2e specific
\IncMargin{2.5em}
\SetNlSty{circled}{}{}
\SetNlSkip{1em}
\DontPrintSemicolon
\newcommand{\EndAlgorithmLine}{\vskip\medskipamount % or other desired dimension
\leaders\vrule width \textwidth\vskip0.4pt % or other desired thickness
\vskip\smallskipamount % ditto
\nointerlineskip}

\begin{document}
\title{Approximation Algorithms - Exercise Sheet 5 solutions}
\maketitle{}
\section*{Exercise 5.1}
Let $W = 1$, $n = 2$ and
\begin{IEEEeqnarray*}{rCl"rCl}
c_1 &=& k, & w_1 &=& 2 \\
c_2 &=& 1, & w_2 &=& 1
\end{IEEEeqnarray*}
Then $\OPT(I_k) = 1$, but $LP(I_k) = k \cdot \frac{1}{2} + 0 \cdot 1 = \frac{k}{2}$. Thus
\[
	\sup_I \left\{ \frac{LP(I)}{\OPT(I)} \midcolon \OPT(I) \neq 0 \right\} \geq \sup_k \left\{ \frac{LP(I_k)}{\OPT(I_k)} \midcolon k \geq 2 \right\} = \infty.
\]

Now $w_1 \leq W$, $W$ arbitrary. Let $\frac{c_1}{w_1} \geq \ldots \geq \frac{c_n}{w_n}$ and $\sum_{i} w_i \geq W$.
By Theorem 39 we have
\[
	x_j = \begin{cases}
	1, & j < k \\
	0, & j > k \\
	\frac{W - \sum_{i=1}^{k-1} w_i}{w_k}, & j = k
	\end{cases},
\]
with $k \coloneqq \min \{ \ldots \}$.
Then,
\[
LP(I) = \underbrace{ \sum_{i=1}^{k-1} c_j }_{\leq \OPT(I)} + \underbrace{ \left( \frac{W - \sum_{i=1}^{k-1} w_i}{w_k} \right) \cdot c_k }_{\leq \OPT(I)}.
\]
Thus:
\begin{IEEEeqnarray*}{rCl}
2 &\geq& \sup \left\{ \frac{LP(I)}{\OPT(I)} \midcolon \OPT(I) \neq 0 \right\} \\
&\geq& \sup \left\{ \frac{LP(I_\varepsilon)}{\OPT(I_\varepsilon)} \right\}.
\end{IEEEeqnarray*}
For the instance $c_1 = c_2 = 1$ and
\[
	w_1 = \frac{1}{2}, \;\; w_2 = \left( \frac{1}{2} + \varepsilon \right), \;\; W = 1
\]
we have
\[
	LP(I_\varepsilon) = 1 + \frac{\frac{1}{2}}{\frac{1}{2} + \varepsilon}.
\]
Hence,
\begin{IEEEeqnarray*}{rCl}
\sup \left\{ \frac{LP(I_\varepsilon)}{\OPT(I_\varepsilon)} \right\} &=& \lim_{\varepsilon \to 0} \left\{ \frac{LP(I_\varepsilon)}{\OPT(I_\varepsilon)} \right\} \\
&=& \lim_{\varepsilon \to 0} 1 + \frac{\frac{1}{2}}{\frac{1}{2} + \varepsilon} \\
&\geq& 2.
\end{IEEEeqnarray*}
\section*{Exercise 5.2}
\EndAlgorithmLine
\begin{algorithm}[H]
Set $c \coloneqq \max\{ c_i \mid w_i \leq W\}$ (value of the best solution with only one chosen item)\;
\ForAll{pairs $(i, j) \in \{ 1, \ldots, n \}^2$, $i \neq j$}{
	\label{alg:3}\If{$w_i + w_j \leq W$}{
		Apply the \sfrac{1}{2}-approximation algorithm on $\{ k \mid c_k \leq \min \{ c_i, c_j\}, k \neq i, j\}$ with $W - (w_i + w_j)$ with cost $c'$.\;
		\If{$c_i + c_j + c' > c$}{
			$c = c_i + c_j + c'$ (this is the best solution so far).\;
		}
	}
}
\end{algorithm}
\EndAlgorithmLine
\underline{Claim:} This is a \sfrac{3}{4}-approximation. \\
\underline{Proof:} Let $c$ be the solution from the algorithm and $c^*$ be an optimum solution.
We can assume that there is no optimum solution with only one item.
Let $i^*, j^* \in \{ 1, \ldots, n \}$ in an optimum solution with $c_{i^*} + c_{j^*}$ is maximum.
An optimum for instance in \ref{alg:3} is $c^{'*} = c^* - c_{i^*} - c_{j^*}$.

\underline{Case 1:} $c_{i^*} + c_{j^*} \geq \frac{1}{2} c^*$.
Then
\[
	c \geq c_{i^*} + c_{j^*} \geq c_{i^*} + c_{j^*} + \frac{c'^*}{2} \geq \frac{1}{2} c_{i^*} + \frac{1}{2} c_{j^*} + \frac{1}{2} c^* \geq \frac{3}{4} c^*.
\]

\underline{Case 2:} $c_{i^*} + c_{j^*} < \frac{1}{2} c^*$.
By \ref{alg:3} in the solution corresponding to $c'^*$ only the cost smaller or equal to $\frac{1}{4} c^*$ can be chosen.
Let $S$ be the solution for the \sfrac{1}{2} approximation in \ref{alg:3}.
Hence,
\[
	c'^* \leq c' + c_S \leq c' + \frac{1}{4} c^*
\]
and thus
\begin{IEEEeqnarray*}{rCl}
c^* &=& c_{i^*} + c_{j^*} + c'^* \\
&\leq& c_{i^*} + c_{j^*} + c' + \frac{1}{4} c^* \\
&\leq& \frac{1}{2} c^* + \frac{1}{4} c^* \\
&\leq& \frac{3}{4} c^*
\end{IEEEeqnarray*}
Runtime $\mathcal{O}(n^3)$.
\end{document}