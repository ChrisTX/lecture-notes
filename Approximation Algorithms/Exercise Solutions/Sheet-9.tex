\documentclass[oneside,a4paper]{amsart}

\usepackage{geometry}
\usepackage{parskip}
\usepackage{mathtools}
\usepackage{enumerate}
\usepackage{relsize}
\usepackage{multirow}
\usepackage{xfrac}

\newcommand{\NP}{\ensuremath{N \! P}}
\newcommand{\tp}{\ensuremath{\mathsmaller T}}
\newcommand{\gmid}{\mathrel{}\middle|\mathrel{}}
\newcommand{\midcolon}{\mathrel{}:\mathrel{}}
\newcommand{\smallo}{\ensuremath{\mathit{o}}}

\usepackage[vlined,linesnumbered]{algorithm2e}
\usepackage{cleveref}
\usepackage{IEEEtrantools}
\usepackage{tikz}
\newcommand*\circled[1]{\tikz[baseline=(char.base)]{\node[shape=circle,draw,inner sep=1pt] (char) {#1};}}

\DeclareMathOperator{\OPT}{OPT}

% Algorithm2e specific
\IncMargin{2.5em}
\SetNlSty{circled}{}{}
\SetNlSkip{1em}
\DontPrintSemicolon
\newcommand{\EndAlgorithmLine}{\vskip\medskipamount % or other desired dimension
\leaders\vrule width \textwidth\vskip0.4pt % or other desired thickness
\vskip\smallskipamount % ditto
\nointerlineskip}

\DeclareMathOperator{\dist}{dist}
\DeclareMathOperator{\SMT}{SMT}
\DeclareMathOperator{\smt}{smt}
\DeclareMathOperator{\loss}{loss}
\DeclareMathOperator{\mst}{mst}

\begin{document}
\title{Approximation Algorithms - Exercise Sheet 9 solutions}
\maketitle{}
\section*{Exercise 9.4}
\subsection*{(i)}
Let $T^*$ be a full component. Then $T^*$ has to be a star, since $V \setminus R$ is a stable set.
Obtain the MST by taking an Eulertour, except for a single edge with cost at least $2 \cdot \loss(T^*)$.
Then,
\[
	\mst(R \cap V(T^*)) \leq 2 \cdot c(T^*) - 2 \cdot \loss(T^*).
\]
Adding all full components yields the result.
\subsection*{(ii)}
Let $T^*$ be a full component. Assume that $|V(T^*)| > 2$, then every path between two terminals has at least two edges. Thus,
\[
	c(T^*) \geq |R \cap V(T^*)|.
\]
On the other hand, we get get
\[
	(|R \cap V(T^*)| - 1) \cdot 2 \geq \mst(R \cap V(T^*)).
\]
Therefore:
\begin{IEEEeqnarray*}{rCl}
	\mst(R \cap V(T^*)) &\leq& (|R \cap V(T^*)| - 1) \cdot 2 \\
	&=& (|V(T^*)| - |V(T^*) \setminus R| - 1) \cdot 2 \\
	&\leq& (\smt(R) - \loss(T_1^* \ldots T_k^*) \cdot 2 \\
	&\leq& 2 \cdot ( c(T) - \loss (T^*) )
\end{IEEEeqnarray*}
\subsection*{(iii)}
Proceed as in the proof done in the lecture.
Then,
\begin{IEEEeqnarray*}{rCl}
\mst(R/\loss(T_1 \ldots T_{i_{\max}})) + \sum_{i=0}^{i_{\max}} \loss(T_i) &=& \ldots \\
&=& \smt_k + \loss_k \cdot \ln \frac{\mst(R) - \smt_k + \loss_k}{\loss_k} \\
&\overset{\text{(i),(ii)}}{\leq}& \smt_k + \loss_k \cdot \frac{\smt_k - \loss_k}{\loss_k} \\
&=& \smt_k \left( 1 + \frac{\loss_k}{\smt_k} \ln \left( \frac{\smt_k}{\loss_k} - 1 \right) \right) \\
&\leq& \smt_k \left( 1 + \max_{0 \leq x \leq \frac{1}{2}} x \ln \left( \frac{1}{x} - 1 \right) \right) \\
&\leq& \smt_k (1 + 0.279).
\end{IEEEeqnarray*}
This gives an approximation ratio of $1.279 \cdot r_k$.
\end{document}