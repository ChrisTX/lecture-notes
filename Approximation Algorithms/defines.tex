% Differentiale
\newcommand{\dx}{\:\mathrm{d}x}
\newcommand{\dy}{\:\mathrm{d}y}
\newcommand{\dz}{\:\mathrm{d}z}
\newcommand{\ds}{\:\mathrm{d}s}
\newcommand{\dt}{\:\mathrm{d}t}
\newcommand{\du}{\:\mathrm{d}u}
\newcommand{\dxi}{\:\mathrm{d}\xi}
\newcommand{\dxy}{\:\mathrm{d}(x,y)}
\newcommand{\dxyz}{\:\mathrm{d}(x,y,z)}

% Differentialoperatoren
\DeclareMathOperator{\grad}{grad}
\DeclareMathOperator{\dive}{div}
\DeclareMathOperator{\curl}{curl}

% ess sup, ess inf
\DeclareMathOperator*{\esssup}{ess\,sup}
\DeclareMathOperator*{\essinf}{ess\,inf}

% arg min/max
\DeclareMathOperator*{\argmin}{arg\;min}
\DeclareMathOperator*{\argmax}{arg\;max}

% Maß
\DeclareMathOperator{\meas}{meas}

% Im
\DeclareMathOperator{\im}{Im}

% Zahlen
\newcommand{\Q}{\ensuremath{\mathbb{Q}}}
\newcommand{\R}{\ensuremath{\mathbb{R}}}
\newcommand{\N}{\ensuremath{\mathbb{N}}}
\newcommand{\Z}{\ensuremath{\mathbb{Z}}}
\newcommand{\C}{\ensuremath{\mathbb{C}}}

% Weitere Operatoren
\DeclareMathOperator*{\spn}{span}
\DeclareMathOperator*{\supp}{supp} 
\DeclareMathOperator*{\dist}{dist}
\DeclareMathOperator*{\diam}{diam}
\DeclareMathOperator{\conv}{conv}
\DeclareMathOperator{\Int}{int}

\DeclareMathOperator{\sgn}{sgn}
\DeclareMathOperator{\diag}{diag}

% Transpose symbol
\newcommand{\tp}{\ensuremath{\mathsmaller T}}
\newcommand{\gmid}{\mathrel{}\middle|\mathrel{}}
\newcommand{\midcolon}{\mathrel{}:\mathrel{}}

% Acronyms
\DeclareAcronym{wlog}{short={w.l.o.g.}, long={without loss of generality}}
\DeclareAcronym{st}{short={s.t.}, long={such that}}
\DeclareAcronym{iff}{short={iff}, long={if and only if}}
\DeclareAcronym{wrt}{short={w.r.t.}, long={with respect to}}
\DeclareAcronym{resp}{short={resp.}, long={respectively}}
\DeclareAcronym{PTAS}{short={PTAS}, long={Polynomial Time Approximation Scheme}}
\DeclareAcronym{FPTAS}{short={FPTAS}, long={Fully Polynomial Time Approximation Scheme}}
\DeclareAcronym{APTAS}{short={APTAS}, long={Asymptotic Polynomial Time Approximation Scheme}}

% Approx. Algorithms specifics
\newcommand{\NP}{\ensuremath{N \! P}}
\newcommand{\coNP}{\ensuremath{co-N \! P}}
\newcommand{\PCP}{\ensuremath{PC \! P}}
\DeclareMathOperator{\OPT}{OPT}
\DeclareMathOperator{\goal}{goal}
\DeclareMathOperator{\dom}{dom}
\DeclareMathOperator{\NF}{NF}
\DeclareMathOperator{\FF}{FF}
\DeclareMathOperator{\FFD}{FFD}
\newcommand{\BP}{\ensuremath{B \! P}}