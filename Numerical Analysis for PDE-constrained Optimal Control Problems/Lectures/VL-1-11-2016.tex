\documentclass[../skript.tex]{subfiles}

\begin{document}
\chapter{Elliptic optimal control problems with semilinear state equation}
Literature: \cite[Chapter 4]{Troeltzsch}.
\begin{example}
\begin{IEEEeqnarray*}{c}
\frac{1}{2} \| y - y_d \|_{L^2(\Omega)}^2 + \frac{\lambda}{2} \| u \|_{L^2(\Omega)}^2 \\
\begin{IEEEeqnarraybox}{rCl"l}
- \lapl y + c_0(x) y + y^3 &=& u & \text{in $\Omega$} \\
\partial_n y + \alpha(x) y &=& 0 & \text{on $\Gamma$}
\end{IEEEeqnarraybox}
u_a \leq u(x) \leq u_b
\end{IEEEeqnarray*}
\end{example}
This is not a convex problem anymore, so we cannot expect unique solutions, and we might have additional local solutions that aren't global ones.

We have to discuss:
\begin{itemize}
\item existence of unique solutions to the PDE
\item existence of solutions to the optimal control problem
\item First Order Necessary conditions
\item Second order Sufficient Conditions
\item Finite Element error estimates
\end{itemize}

First, let us consider the simplified case
\begin{example}
\begin{IEEEeqnarray*}{c}
\frac{1}{2} \| y - y_d \|_{L^2(\Omega)}^2 + \frac{\lambda}{2} \| u \|_{L^2(\Omega)}^2 \\
\begin{IEEEeqnarraybox}{rCl"l}
- \lapl y + y^3 &=& u & \text{in $\Omega$} \\
\partial_n y &=& 0 & \text{on $\Gamma$}
\end{IEEEeqnarraybox}
u_a \leq u(x) \leq u_b
\end{IEEEeqnarray*}
\end{example}
What do we expect? What are the first order optimality conditions (formally)?
Were this a linear problem, we'd first formulate the reduced $f$:
\[
	f(u) = \frac{1}{2} \| S(u) - y_d \|_{L^2}^2 + \frac{\lambda}{2} \| u \|_{L^2}^2.
\]
As necessary condition we'd assert
\[
	f'(\bar{u}) [u - \bar{u}] \geq 0 \quad \forall u \in U_{ad}.
\]
We haven't proven this for the non-linear case yet, but intuitively it will hold.
As a next step, we calculate:
\begin{IEEEeqnarray*}{rCl}
	f'(\bar{u}) [u - \bar{u}] &=& (\lambda \bar{u}, u - \bar{u})_{L^2} + (S(\bar{u}) - y_d, S'(\bar{u})(u - \bar{u}))_{L^2} \\
	&=& (\lambda \bar{u}, u - \bar{u})_{L^2} + (\underbrace{[S'(\bar{u})]^* (S(\bar{u}) - y_d)}_{\bar{p}}, u - \bar{u} )_{L^2} \\
	&\overset{!}{\geq}& 0 \quad \forall u \in U_{ad}
\end{IEEEeqnarray*}
Thus we obtain the condition:
\[
	(\lambda \bar{u} + \bar{p}, u - \bar{u} ) \geq 0 \quad \forall u \in U_{ad}.
\]
Therefore, we'll obtain the projection formula:
\begin{IEEEeqnarray*}{c}
\begin{IEEEeqnarraybox}{rCl}
- \lapl \bar{y} + \bar{y}^3 &=& \bar{u} \\
\partial_n \bar{y} &=& 0
\end{IEEEeqnarraybox} \\
\bar{u} = \PP_{[u_a, u_b]} \left\{ - \frac{1}{\lambda} \bar{p} \right\}
\end{IEEEeqnarray*}
We can formally state the Lagrange function for this problem:
\[
	L(y, u, p) = \frac{1}{2} \int_\Omega (y - y_d)^2 \dx + \frac{\lambda}{2} \int_\Omega u^2 \dx - \int_\Omega (- \lapl y + y^3 - u) p \dx - \int_\Gamma \partial_n y p \ds
\]
Thus:
\[
	\frac{\partial}{\partial y} L(\bar{y}, \bar{u}, \bar{p}) v = \int_\Omega (\bar{y} - y_d) v \dx - \int_\Omega - \lapl v \bar{p} \dx - \int_\Omega 3 \bar{y}^2 v \bar{p} \dx - \int_\Gamma \partial_n v \bar{p} \ds \overset{!}{=} 0
\]
Using integration by parts:
\[
	- \int_\Omega - \lapl v \bar{p} \dx = \int_\Omega - v \lapl \bar{p} \dx + \int_\Gamma \partial_n v \bar{p} \ds - \int_\Gamma v \partial_n \bar{p}
\]
This yields:
\begin{IEEEeqnarray*}{rCl}
	\int_\Omega (- \lapl \bar{p} v - (\bar{y} - y_d) v + 3 \bar{y}^2 \bar{p} v) \dx &=& 0 \\
	\int_\Gamma (- \partial_n v \bar{p} + \partial_n v \bar{p} + \partial_n \bar{p} v) \ds &=& 0
\end{IEEEeqnarray*}
Therefore, we have the full system:
\begin{IEEEeqnarray*}{c}
\begin{IEEEeqnarraybox}{rCl}
- \lapl \bar{y} + \bar{y}^3 &=& \bar{u} \\
\partial_n \bar{y} &=& 0
\end{IEEEeqnarraybox} \qquad \begin{IEEEeqnarraybox}{rCl}
- \lapl \bar{p} + 3 \bar{y}^2 \bar{p} &=& \bar{y} - y_d \\
\partial_n \bar{p} &=& 0
\end{IEEEeqnarraybox} \\
\bar{u} = \PP_{[u_a, u_b]} \left\{ - \frac{1}{\lambda} \bar{p} \right\}
\end{IEEEeqnarray*}
\section{The semilinear state equation}
The most general situation we can discuss is
\begin{IEEEeqnarray*}{rCl"l}
A y + c_0(x) y + d(x, y) &=& f & \text{on $\Omega$} \\
\partial_{n_A} y + \alpha(x) y + b(x, y) &=& g & \text{on $\Gamma$}
\end{IEEEeqnarray*}
\begin{theorem}
Let $V$ be a separable Hilbert space and $A : V \to V^*$ a monotone, coercive and continuous operator. Then, $Ay = f$ admits for every $f \in V^*$ a solution $y$. The set of solutions is bounded, convex and closed. If $A$ is strictly monotone, then $y$ is unique. If $A$ is strongly monotone, then $A^{-1}: V^* \to V$ is Lipschitz continuous.
\end{theorem}
\begin{itemize}
\item An operator $A: V \to V^*$ is monotone if
\[
	(Ay_1 - Ay_2, y_1 - y_2)_{V^*, V} \geq 0 \quad \forall y_1, y_2 \in V
\]
holds. It is called strictly monotone if equality to zero holds only for $y_1 = y_2$.
\item $A$ is called coercive if
\[
	\frac{(Ay, y)_{V^*, V}}{\| y\|_V} \to \infty \quad \text{for } \| y \|_V \to \infty.
\]
\item $A$ is called hemi-continuous, if
\[
	\varphi : [0, 1] \to \R \quad : \quad t \mapsto (A(y + tv), w)_{V^*, V}
\]
is continuous for all fixed $y, v, w \in V$.
\item If there exists $\beta_0 > 0$ such that
\[
	(Ay_1 - Ay_2, y_1 - y_2)_{V^*, V} \geq \beta_0 \| y_1 - y_2 \|_{V}^2 \quad \forall y_1, y_2 \in V,
\]
then $A$ is called strongly monotone.
\end{itemize}
% Let's first consider the case with simplified boundary conditions:
% \begin{IEEEeqnarray*}{rCl"l}
% A y + c_0(x) y + d(x, y) &=& f & \text{on $\Omega$} \\
% \partial_n y &=& 0 & \text{on $\Gamma$}
% \end{IEEEeqnarray*}
\end{document}