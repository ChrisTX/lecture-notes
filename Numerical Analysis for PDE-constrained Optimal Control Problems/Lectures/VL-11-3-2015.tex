\documentclass[../skript.tex]{subfiles}

\begin{document}
\begin{definition}
\emph{Adjoint operator:} Let real Hilbert spaces $\{ U, (\cdot, \cdot)_U \}$ and $\{ V, (\cdot, \cdot)_V \}$ be given and let $A: U \to V$ be a bounded and continuous operator 
\begin{IEEEeqnarray*}{rCl}
A^* : V^* &\to& U^* \\
(Au, v)_V &=& (u, A^* v)_U
\end{IEEEeqnarray*}
(in Hilbert spaces $V \to U$).
We call $A^*$ the (Hilbert space) \emph{adjoint operator}.
\end{definition}
\section{First order necessary optimal conditions}
Let $U$ be a real Banach space $U_{ad} \subset U$ convex set and $f$ a real functional that is Gâteaux differentiable on a set containing $U_{ad}$.
Let $\bar{u} \in U_{ad}$ denote the solution $g$ of
\[
	\min_{u \in U_{ad}} f(u)
\]
Then, the following variational inequality
\[
	f'(\bar{u})(u - \bar{u}) \geq 0
\]
holds for all $u \in U_{ad}$.
In in addition $f$ is convex and $\bar{u}$ solves the variational inequality, then $\bar{u}$ solves the minimzation problem.
Let $U$ and $H$ be real Hilbert spaces, $C \subset U$ be a convex set, an element $y_d \in H$ and $\lambda \geq 0$ be given.
Moreover, $S : U \to H$ is a linear and continuous operator. Then $\bar{u} \in U_{ad}$ solves \cref{prb:ElP} \ac{iff} the variantional ineuqality
\[
	(S\bar{u} - y_d, S(u - \bar{u}))_V + \lambda(\bar{u}, u - \bar{u})_U \geq 0 \quad \forall u \in U_{ad}
\]
is satisfied.

The state equation
\begin{IEEEeqnarray*}{rCl"l}
- \lapl y &=& \beta u & \text{in } \Omega \\
y &=& 0 & \text{on } \Gamma
\end{IEEEeqnarray*}
with $S : L^2 \to L^2$ has the adjoint equation
\begin{equation}
\label{eq:SupposedAdjP}
\tag{*}
\begin{IEEEeqnarraybox}[][c]{rCl"l}
- \lapl p &=& y - y_d & \text{in } \Omega \\
p &=& 0
\end{IEEEeqnarraybox}
\end{equation}
with $S^* : L^2 \to L^2$.
\begin{theorem}
The solution $p \in H_0^1(\Omega) \hookrightarrow L^2(\Omega)$ of the adjoint equation \cref{eq:SupposedAdjP} is called the adjoint state associated with $\bar{y}$.
\end{theorem}
\begin{lemma}
In case of the state equation
\begin{IEEEeqnarray*}{rCl"l}
- \lapl y &=& \beta u & \text{in } \Omega \\
y &=& 0 & \text{on } \Gamma
\end{IEEEeqnarray*}
with $y = Su$, the adjoint operator $S^* : L^2 \to L^2$ is given by $S^* z = \beta p$, where $p$ solves
\begin{IEEEeqnarray*}{rCl"l}
- \lapl p &=& z & \text{in } \Omega \\
p &=& 0 & \text{on } \Gamma
\end{IEEEeqnarray*}
and $\beta p = S^* z$.
\end{lemma}
\begin{proof}
To show
\[
	(\underbrace{Su}_y, z) = (u, \underbrace{S^* z}_{\beta p}).
\]
We have
\begin{itemize}
\item $(\nabla y, \nabla p) = (\beta u, p)$
\item $(\nabla p, \nabla y) = (z, y)$
\end{itemize}
Now the left hand sides are equal and thus
\[
	(\beta u, p) = (z, y).
\]
\end{proof}
We have the optimality system:
\begin{equation}
\label{eq:ElPOptimalitySystem}
\tag{**}
\begin{IEEEeqnarraybox}[][c]{c}
\begin{IEEEeqnarraybox}[][c]{rCl"l}
- \lapl \bar{y} &=& \beta \bar{u} & \text{in } \Omega \\
\bar{y} &=& 0 & \text{on } \Gamma
\end{IEEEeqnarraybox} \qquad
\begin{IEEEeqnarraybox}[][c]{rCl"l}
- \lapl \bar{p} &=& \bar{y} - y_d & \text{in } \Omega \\
\bar{p} &=& 0 & \text{on } \Gamma
\end{IEEEeqnarraybox} \\
(\beta \bar{p} + \lambda \bar{u}, u - \bar{u}) \geq 0 \quad \forall u \in U_{ad}.
\end{IEEEeqnarraybox}
\end{equation}
The variational inequality is derived as follows, we have the condition
\[
	f'(\bar{u})(u - \bar{u}) \geq 0 \quad \forall u \in U_{ad},
\]
or more specifically for this problem
\[
	(S \bar{u} - y_d, S(u - \bar{u})) + \lambda (\bar{u}, u - \bar{u}) \geq 0,
\]
which can be transformed by using $S^*$ into
\[
	(S^* (S \bar{u} - y_d), u - \bar{u}) + \lambda (\bar{u}, u - \bar{u}) \geq 0.
\]
By letting $z = \bar{y} - y_d = S\bar{u} - y_d$, we obtain that $S^* z = \beta \bar{p}$, as we've just proved. This is the exact inequality we claimed.

If $\bar{u}$ is the optimal control of \cref{prb:ElP} with associated optimal state $\bar{y}$, then there exists a unique weak solution $\bar{p}$ of the adjoint equation \cref{eq:ElPOptimalitySystem} such that
\[
	\int_\Omega (\beta \bar{p} + \lambda \bar{u})(u - \bar{u}) \geq 0 \quad \forall u \in U_{ad}
\]
is satisfied (and vice versa).

One can show that
\[
	\bar{u} = \PP_{[u_a(x), u_b(x)]} \left\{ - \frac{\beta}{\lambda} \bar{p}(x) \right\}.
\]
\section{The formal Lagrange technique}
We want to discuss how to derive a possible form for the adjoint equation.
Consider the model problem
\begin{IEEEeqnarray*}{c}
\min J(y, u) = \frac{1}{2} \| y - y_d \|_{L^2}^2 + \frac{\lambda}{2} \| u \|^2\\
\begin{IEEEeqnarraybox}[][c]{rCl"l}
- \lapl y &=& u & \text{in } \Omega \\
y &=& 0 & \text{on } \Gamma
\end{IEEEeqnarraybox}
\end{IEEEeqnarray*}
We make the approach
\[
	L(y, u, p) = \frac{1}{2} \int_\Omega (y - y_d)^2 \dx + \frac{\lambda}{2} \int_\Omega u^2 \dx - \int_\Omega (-\lapl y - u) p_1 \dx - \int_\Gamma y p_2 \ds.
\]
Now we want to investigate what happens if we assert $L_y(\bar{y}, \bar{u}, \bar{p}) = 0$ and $L_u(\bar{y}, \bar{u}, \bar{p}) = 0$:
\begin{IEEEeqnarray*}{rClCl"l}
0 & \overset{!}{=} & L_y (\bar{y}, \bar{u}, \bar{p}) v &=& \int_\Omega (\bar{y} - y_d) v \dx - \int_\Omega - \lapl v p_1 \dx - \int_\Gamma v p_2 \ds & \forall v
\end{IEEEeqnarray*}
By integrating partially twice (we operate formally here, disregarding whether we are allowed to do so for now)
\begin{IEEEeqnarray*}{rCl}
\int_\Omega - \lapl v p_1 \dx &=& \int_\Omega \nabla v \nabla p_1 \dx - \int_\Gamma \partial_n v \partial_n p_1 \dx \\
&=& \int_\Omega - v \lapl p_1 \dx + \int_\Gamma \partial_n v p_1 \dx - \int_\Gamma \partial_n v \partial_n p_1 \dx - \int_\Gamma v p_2 \ds
\end{IEEEeqnarray*}
By assuming that we'll have a zero boundary condition, this is
\[
	\int_\Omega - \lapl v p_1 \dx = \int_\Omega - v \lapl p_1.
\]
This gives us
\[
	0 = \int_\Omega (\bar{y} - y_d + \lapl p_1) v \dx + \int_\Gamma v p_2 \ds
\]
and therefore
\begin{IEEEeqnarray*}{rCl}
- \lapl p &=& \bar{y} - y_d \\
p &=& 0
\end{IEEEeqnarray*}

Consider a convex polygonal or polyhedral domain $\Omega \subset \R^{2,3}$ (or a domain with $C^2$-boundary $\Gamma$).
For every $f \in L^2(\Omega)$ the PDE
\begin{IEEEeqnarray*}{rCl"l}
- \lapl y + c_0 y &=& f & \text{in } \Omega \\
y &=& 0 & \text{on } \Gamma
\end{IEEEeqnarray*}
admits a unique solution in $H^2(\Omega) \cap H_0^1(\Omega)$.
There exists a constant $C$ independent of $y$ with
\[
	\| y \|_{H^2} \leq C \| f \|_{L^2}.
\]
The existence of an optimal control $\bar{u} \in L^2$ (if $u_a, u_b \in L^\infty$, then $\bar{u} \in L^\infty(\Omega)$) implies
\[
	\bar{y} \in H^2(\Omega) \cap H_0^1(\Omega)
\]
and since $y_d \in L^2$,
\[
	\bar{p} \in H^2(\Omega) \cap H_0^1(\Omega).
\]
A result by Kinderlehrer and Stampacchia gives if $u_a, u_b \in H^1$ that $\bar{u} \in H^1(\Omega)$.
If $U_{ad} = L^2$, then we even have $\bar{u} \in H^2(\Omega)$, of course.
\end{document}