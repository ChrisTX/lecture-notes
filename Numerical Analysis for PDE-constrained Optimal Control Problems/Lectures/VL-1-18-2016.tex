\documentclass[../skript.tex]{subfiles}

\begin{document}
\section{Nemytskii-Operators (superposition operators)}
\begin{example}
An example for a Nemytskii operator would be $y(\cdot) \mapsto z(\cdot)$ with $z(x) = y(x)^3$ or $z(x) = \sin(y(x))$
\end{example}
\begin{itemize}
\item Let $E \subset \R^m$, $m \in \N$ be a bounded and measurable set and $\varphi : \varphi(x, y) : E \times \R \to \R$ a function. The mapping $\Phi$ defined by
\[
	\Phi(y) = \varphi(\cdot, y(\cdot)),
\]
that maps a function $y : E \to \R$ to a function $z : E \to \R$ defined by
\[
	z(x) = \varphi(x, y(x))
\]
is called a Nemytskii operator (or superposition operator)
\item A function $\varphi : \varphi(x, y) : E \times \R \to \R$ satisfies the \emph{Carathèodory condition}, if $\varphi$ is measurable in $x$ for every fixed $y \in \R$, and continuous in $y$ for almost all $x \in E$.
\item $H$ satisfies the boundedness condition if there exists a constant $K > 0$ such that
\[
	|\varphi(x, 0)| \leq K
\]
for almost all $E$.
\item $H$ is called locally Lipschitz continuous in $y$ if for every constant $M > 0$ there exists $L(M) > 0$, such that for almost all $x \in E$, and all $y, z \in [-M, M]$ the estimate
\[
	| \varphi(x, y) - \varphi(x, z) | \leq L(M) | y - z|
\]
holds.

\textit{Note:} Boundedness condition and local Lipschitz condition yield boundedness for all $y \in [-M, M]$ and almost all $x \in E$.
\end{itemize}
\begin{proposition}
Let the function $\varphi = \varphi(x, y)$ be measurable in $x \in E$ for all fixed $y \in \R$, and locally Lipschitz continuous in $y$.
Moreover, let the boundedness condition be satisfied.
Then the associated Nemytskii operator $\Phi$ is continuous in $L^\infty(E)$. Moreover, for $r \in [1, \infty]$
\[
	\| \Phi(y) - \Phi(z) \|_{L^r(E)} \leq L(M) \| y - z \|_{L^r(E)}
\]
holds for all functions $y, z \in L^\infty(E)$ that satisfy $\| y \|_{L^\infty(E)}, \| z \|_{L^\infty(E)} \leq M$.
\end{proposition}
\begin{proof}
For $y(x) \in L^\infty(E)$, we have $|y(x)| \leq M$ for almost all $x$ and sufficiently large $M$.
\begin{IEEEeqnarray*}{rCl}
| \varphi(x, y(x)) | &\leq& | \varphi(x, 0) | + | \varphi(x, y(x)) - \varphi(x, 0) | \\
&\leq& M + L(M) M.
\end{IEEEeqnarray*}
This proves the first claim that $\Phi$ is continuous in $L^\infty(E)$.

Next, we consider
\begin{IEEEeqnarray*}{rCl}
\int_E |\varphi(x, y(x)) - \varphi(x, z(x))|^r \dx &\leq& L(M)^r \underbrace{ \int_E |y(x) - z(x)|^r \dx }_{\| y - z \|^r_{L^r(E)}}.
\end{IEEEeqnarray*}
By taking the $r$-th root the claim follows.
\end{proof}
\subsection{Differentiability of Nemytskii operators}
\begin{itemize}
\item Let $E \subset \R^m$, $m \in \N$ be a bounded and measurable set and $\varphi = \varphi(x, y) : E \times \R \to \R$ a function of the spatial variable $x$ and the function variable $y$. Let $\varphi$ be $k$-times partially differentiable with respect to $y$ for almost all $x \in E$.
\item We say that $\varphi$ satisfies the boundedness condition of order $k$, if there exists a constant $K > 0$ such that
\[
	|D_y^\ell \varphi(x, 0)| \leq K
\]
for almost all $x \in E$, and $\ell = 0, \ldots, k$.
\item $\varphi$ satisfies the local Lipschitz condition of order $k$ if there exists an $M$-dependent Lipschitz constant $L(M)$ such that
\[
	|D_y^k \varphi(x, y_1) - D_y^k(x, y_2)| \leq L(M) | y_1 - y_2 |
\]
for all $|y_i| \leq M, i = 1, 2$.
\end{itemize}
\paragraph{First order derivatives in $L^\infty$}
If $\varphi$ is continuous differentiable, is $\Phi$ continuously differentiable as well?
We calculate
\begin{IEEEeqnarray*}{rCl}
(\Phi'(y) h)(x) &=& \lim_{t \downarrow 0} \frac{1}{t} \left[ \varphi(x, y(x) + th(x)) - \varphi(x, y(x)) \right] \\
&=& \varphi_y(x, y(x)) h(x)
\end{IEEEeqnarray*}
This exists for every fixed $x$, but does it hold in $L^r$ spaces?
\begin{example}
We consider $\sin(\cdot)$. Obviously, $\sin(x) \in L^\infty$ and all derivatives are globally bounded.
Let's test differentiability in $L^p(0,1)$, $p < \infty$.
We assume differentiability: The candidate for the derivative would be
\[
	(\Phi'(y) h)(x) = \cos(y(x))h(x)
\]
but
\begin{IEEEeqnarray*}{rCl}
	\sin (0 + h(x)) &=& \underbrace{\sin(0)}_{=0} + \underbrace{\cos(0)}_{=1} h(x) + \underbrace{ \int_{s=0}^1 (\cos(0 + sh(x)) - \cos(0)) h(x) \ds}_{r(x)} \\
	&=& h(x) + r(x)
\end{IEEEeqnarray*}
Therefore, let's consider the choice
\[
	h(x) = \begin{cases}
	1 & \text{on } [0, \varepsilon] \\
	0 & \text{on } (\varepsilon, 1]
	\end{cases}
\]
and consider $\varepsilon \downarrow 0$.
Now we have $r(x) = \sin(h(x)) - h(x)$ and thus
\[
	r(x) = \begin{cases}
	c & \text{on } [0, \varepsilon] \\
	0 & \text{on } (\varepsilon, 1]
	\end{cases}
\]
For differentiability we have to check whether the norm of the remainder term $r$ divided by the norm of $h$ goes to zero if $h$ goes to zero:
\begin{IEEEeqnarray*}{rCl}
\frac{\| r \|_{L^p(0, 1)}}{\| h \|_{L^p(0, 1)}} &=& \frac{ \left( \int_0^\varepsilon |r(x)|^p \dx \right)^{\frac{1}{p}}}{ \left( \int_0^\varepsilon |h(x)|^p \dx \right)^{\frac{1}{p}}} \\
&=& \frac{c \varepsilon^{\frac{1}{p}}}{\varepsilon^{\frac{1}{p}}} \\
&=& c
\end{IEEEeqnarray*}
But a fixed constant $c$ does not go to zero for $\| h \|_{L^p} \to 0$. Therefore, we cannot have differentiability in zero.
\end{example}
\begin{proposition}
Let the function $\varphi$ be measurable in $x$ for every $y \in \R$, and differentiable with respect to $y$ for almost all $x \in E$.
Let the boundedness condition and local Lipschitz condition of order $k = 1$ be satisfied.
Then, the Nemytskii operator $\Phi$ associated with $\varphi$ is Fréchet-differentiable in $L^\infty(E)$ and for all $h \in L^\infty(E)$ we have
\[
	(\Phi'(y)h)(x) = \varphi_y(x, y(x))h(x)
\]
for almost all $x \in E$.
\end{proposition}
\begin{proof}
We start by considering
\[
	\varphi(x, y(x) + h(x)) - \varphi(x, y(x))
\]
and estimating it using Taylor-type arguments, so that if $h$ goes to zero, so does this term.
\begin{IEEEeqnarray*}{rCl}
\varphi(x, y(x) + h(x)) - \varphi(x, y(x)) &=& \varphi_y(x, y(x))h(x) \\
&& {} + \underbrace{ \int_{s=0}^1 \varphi_y(x, y(x) + sh(x)) - \varphi_y(x, y(x)) \ds \cdot h(x) }_{r(x, y, h)}
\end{IEEEeqnarray*}
For $r(x, y, h)$ we have
\[
	|r(x, y, h)| \leq L(M) \int_{s=0}^1 sh(x)^2 \ds = \frac{L(M)}{2} |h(x)^2|.
\]
Therefore,
\[
	\frac{\| r(x) \|_{L^\infty}}{\| h(x) \|_{L^\infty}} = c \| h(x) \|_{L^\infty} \xrightarrow{\| h(x) \|_{L^\infty} \to 0} 0,
\]
which proves differentiability.
\end{proof}
\end{document}