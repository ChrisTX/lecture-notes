\documentclass[../skript.tex]{subfiles}

\begin{document}
\begin{theorem}
Let $\bar{u}$ be the optimal control for \cref{prb:P-est-wo-constraint} (withount inequality constraints) with associated state $\bar{y}$, and $\bar{u}_h$ with associated discrete state $\bar{y}_h$ be the solution of \cref{prb:Ph-est-wo-constraint} with variational control discretization, i.e.\ $U_h = L^2(\Omega)$. Then there exists $c > 0$ independent of $h$, such that
\[
\| \bar{y} - \bar{y}_h \| + \| \bar{u} - \bar{u}_h \|_{L^2} \leq c h^2 \left( \| \bar{u} \| + \| y_d \| \right).
\]
\end{theorem}
\begin{proof}
The first order necessary conditions are
\[
	f'(\bar{u}) v = 0 = f'_h(\bar{u}_h) v \quad \forall v \in L^2(\Omega).
\]
Thus:
\begin{IEEEeqnarray*}{rCl}
0 &=& f'(\bar{u}) v - f'_h(\bar{u}_h) v \\
&=& \left( f'(\bar{u}) v - f_h'(\bar{u}) v \right) + \left( f_h'(\bar{u}) v - f_h'(\bar{u}_h) v \right)
\end{IEEEeqnarray*}
With the choice $v = \bar{u} - \bar{u}_h$, the first term can be estimated as
\[
	 \left( f'(\bar{u}) v - f_h'(\bar{u}) v \right) \leq ch^2 \left( \| \bar{u} \| + \| y_d \| \right) \| \bar{u} - \bar{u}_h \|.
\]
For the second term, we can estimate by a Taylor-type estimation it as
\[
\left( f_h'(\bar{u}) v - f_h'(\bar{u}_h) v \right) = f_h''(u_\xi) (\bar{u} - \bar{u}_h, v) \overset{v = \bar{u} - \bar{u}_h}{\geq} \lambda \| \bar{u} - \bar{u}_h \|^2.
\]
Thus, we obtain
\[
	0 \geq \left( f'(\bar{u}) - f_h'(\bar{u}), \bar{u} - \bar{u}_h \right) + \lambda \| \bar{u} - \bar{u}_h \|^2.
\]
By bringing the first term on the other side:
\begin{IEEEeqnarray*}{rCl}
\lambda \| \bar{u} - \bar{u}_h \|^2 &\leq& \left( f'_h(\bar{u}) - f'(\bar{u}), \bar{u} - \bar{u}_h \right) \\
&\leq& ch^2 \left( \| \bar{u} \| + \| y_d \| \right) \| \bar{u} - \bar{u}_h \|.
\end{IEEEeqnarray*}
Therefore, we obtain
\[
	\| \bar{u} - \bar{u}_h \| \leq \frac{c}{\lambda} h^2 \left( \| \bar{u} \| + \| y_d \| \right).
\]
% Recall that our objective function is:
% \[
% J(y, u) = \frac{1}{2} \| y - y_d \|^2 + \frac{\lambda}{2} \| u \|^2.
% \]
Thus, the other term we need to estimate is $\| \bar{y} - \bar{y}_h \|$:
\begin{IEEEeqnarray*}{rCl}
\| \bar{y} - \bar{y}_h \| &=& \| S \bar{u} - S_h \bar{u}_h \| \\
&\leq& \| S \bar{u} - S_h \bar{u} \| + \| S_h \bar{u} - S_h \bar{u}_h \| \\
&\leq& \| \bar{y} - y_h(\bar{u}) \| + \| y_h(\bar{u}) - \bar{y}_h \| \\
&\leq& ch^2 \| \nabla^2 \bar{y} \| + c \| \bar{u} - \bar{u}_h \| \\
&\leq& ch^2 \| \bar{u} \| + ch^2 \left( \| \bar{u} \| + \| y_d \| \right).
\end{IEEEeqnarray*}
This proves the claim.
\end{proof}
\section{Problems with control constraints}
We use variational discretization as previously and let
\[
	U_{ad} = \left\{ u \in L^2(\Omega) \midcolon u_a \leq u \leq u_b \; \text{a.e.} \right\}, \quad u_a < u_b \in L^\infty.
\]
The problems we have are then
\begin{problem}
\begin{equation}
\tag{$P$}
\label{prb:P-est-w-constraints}
\min_{u \in U_{ad}} f(u)
\end{equation}
\end{problem}
and
\begin{problem}
\begin{equation}
\tag{$P_h$}
\label{prb:Ph-est-w-constraints}
\min_{u \in U_{ad}^h} f(u)
\end{equation}
with the variational discretization $U_{ad}^h = U_{ad} \neq \emptyset$. Therefore, $\bar{u}$ and $\bar{u}_h$ exist.
\end{problem}

For \cref{prb:P-est-w-constraints}, we have the first order necessary conditions
\[
	f'(\bar{u})(u - \bar{u}) \geq 0 \quad \forall u \in U_{ad}.
\]
As we have seen this is equivalent to
\[
	\left( \lambda \bar{u} + \bar{p}, u - \bar{u} \right) \geq 0 \quad \forall u \in U_{ad}
\]
and thus we have the projection formula
\[
	\bar{u} = \PP_{[u_a, u_b]} \left\{ - \frac{1}{\lambda} \bar{p} \right\}.
\]

On the other hand, for \cref{prb:Ph-est-w-constraints}, we obtain
\[
	f'_h(\bar{u}_h)(u_h - \bar{u}_h) \geq 0 \quad \forall u_h \in U_{ad}^h.
\]
This is in turn equivalent to
\[
	\left( \lambda \bar{u}_h + \bar{p}_h, u_h - \bar{u}_h \right) \geq 0 \quad \forall u_h \in U_{ad}^h
\]
Because we are testing with $L^2$, this implies $\lambda \bar{u}_h + \bar{p}_h = 0$, and thus
\[
	\bar{u}_h = \PP_{[u_a, u_b]} \left\{ - \frac{1}{\lambda} \bar{p}_h \right\}.
\]

While this is a projection of a function linear on the elements, the projection will still yield a piecewise linear function for $\bar{u}_h$, but the projection can cut off a linear functional on an element and therefore it does not have to be a finite-element function on $\mathcal{T}_h$. This leads to involved to numerics. (There's no equivalence to a control discretization with linear finite elements).

\begin{theorem}
Let $\bar{u}$ be the optimal control of \cref{prb:P-est-w-constraints} with associated state $\bar{y}$, and $\bar{u}_h$ is the optimal control of \cref{prb:Ph-est-w-constraints} with associated state $\bar{y}_h$ (variational discretization, $U_{ad}^h = U_{ad}$).
Then there exists a constant $c > 0$ independent of $h$, such that
\[
\| \bar{y} - \bar{y}_h \| + \| \bar{u} - \bar{u}_h \| \leq ch^2 \left( \| \bar{u} \| + \| y_d \| \right).
\]
\end{theorem}
\begin{proof}
As usual, the first order necessary conditions hold:
\begin{IEEEeqnarray*}{rCl"l}
f'(\bar{u})(u - \bar{u}) &\geq& 0 & \forall u \in U_{ad} \\
f'(\bar{u}_h)(u_h - \bar{u}_h) &\geq& 0 & \forall u_h \in U_{ad}^h = U_{ad}
\end{IEEEeqnarray*}
Because $U_{ad}^h = U_{ad}$, we have $\bar{u}_h \in U_{ad}$ and $\bar{u} \in U_{ad}^h$. Thus, we also have
\begin{IEEEeqnarray*}{rCl"l}
f'(\bar{u})(\bar{u}_h - \bar{u}) &\geq& 0 \\
f'(\bar{u}_h)(\bar{u} - \bar{u}_h) &\geq& 0
\end{IEEEeqnarray*}

With similar arguments to what we've seen in the last proof, we estimate:
\begin{IEEEeqnarray*}{rCl}
\lambda \left\| \bar{u} - \bar{u}_h \right\|^2 &\leq& f_h'' (u_\xi)\left(\bar{u} - \bar{u}_h\right)^2 \\
&=& f_h'(\bar{u})\left(\bar{u} - \bar{u}_h\right) \underbrace{ - f_h'(\bar{u}_h)\left(\bar{u} - \bar{u}_h\right) }_{\leq 0} \\
&\leq& f_h'(\bar{u})\left(\bar{u} - \bar{u}_h \right) + \underbrace{ f'(\bar{u})(\bar{u}_h - \bar{u}) }_{\geq 0} \\
&=& f_h'(\bar{u})\left(\bar{u} - \bar{u}_h \right) - f'(\bar{u})\left(\bar{u} - \bar{u}_h \right) \\
&\leq& ch^2 \| \bar{u} - \bar{u}_h \| \left( \| \bar{u} \| + \| y_d \| \right).
\end{IEEEeqnarray*}
Thus we obtain
\[
	\| \bar{u} - \bar{u}_h \| \leq \frac{c}{\lambda} h^2 \left( \| \bar{u} \| + \| y_d \| \right).
\]

This proves the claim, but one can prove this with a proof more closely related to the previous one, starting from
\[
	\left( f'(\bar{u}) - f_h'(\bar{u}_h) \right) \left( \bar{u}_h - \bar{u} \right) \geq 0.
\]
\end{proof}
A quick note on how to generate test examples: Consider the problem
\begin{IEEEeqnarray*}{c}
\min \frac{1}{2} \| y - y_d \|^2 + \frac{\lambda}{2} \| u - u_d \|^2 \\
\begin{IEEEeqnarraybox}{rCl"rCl}
- \lapl \bar{y} &=& \bar{u} & - \lapl \bar{p} &=& \bar{y} - y_d \\
\bar{y} &=& 0 & \bar{p} &=& 0
\end{IEEEeqnarraybox} \\
\lambda (\bar{u} - u_d) + \bar{p} = 0
\end{IEEEeqnarray*}
Choose $\bar{y}, \bar{p} \in H^2 \cap H_0^1$ and fix $\lambda$.
By considering the state and adjoint equation, you can derive $\bar{u}$, $y_d$ and using the gradient method,  one can obtain $u_d$.
In MATLAB, this can be solved for example using the routine ``quadprog''.
\end{document}