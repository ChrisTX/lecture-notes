\documentclass[../skript.tex]{subfiles}

\begin{document}
\section{Finite element error estimates for state-constrained optimal control problems}
As before, we consider the problem
\begin{equation}
\tag{$P$}
\label{prb:P-StateConstrained}
\begin{IEEEeqnarraybox}[][c]{c}
\min_{u \in L^2} \frac{1}{2} \| y - y_d \|_{L^2}^2 + \frac{\lambda}{2} \| u \|_{L^2}^2 \\
\begin{IEEEeqnarraybox}{rCl}
\begin{IEEEeqnarraybox}{rCl}
- \lapl \bar{y} &=& \bar{u} \\
\bar{y} &=& 0
\end{IEEEeqnarraybox} \quad&&\quad
\begin{IEEEeqnarraybox}{rCl}
- \lapl \bar{p} &=& \bar{y} - y_d + \bar{\mu}_b - \bar{\mu}_a \\
\bar{p} &=& 0
\end{IEEEeqnarraybox} \\
\lambda \bar{u} &+& \bar{p} = 0 
\end{IEEEeqnarraybox} \\
y_a \leq y \leq y_b \quad \bar{\mu}_b, \bar{\mu}_a \geq 0 \quad \langle \bar{\mu}_a, y_a - \bar{y} \rangle = 0 = \langle \bar{\mu}_b, \bar{y} - y_b \rangle
\end{IEEEeqnarraybox}
\end{equation}
Here, $\bar{u} \in L^2(\Omega)$ and therefore, $\bar{y} \in H_0^1(\Omega) \cap H^2(\Omega) \hookrightarrow C(\bar{\Omega})$.
Furthermore $\bar{\mu}_b - \bar{\mu}_a \in M(\Omega)$ and this implies
\[
	\bar{p} \in W^{1, s}(\Omega), \quad s < \frac{n}{n-1},\, \text{2D: } s < 2,
\]
in other words ``$\bar{p}$ is just not in $H^1$''. This however implies $\bar{u} \in W^{1, s}$.

For some situations, one has
\[
	\bar{\mu}_a, \bar{\mu}_b \in M(\Omega) \cap H^{-1}(\Omega)
\]
and this gives $\bar{p}, \bar{u} \in L^\infty(\Omega)$, see Pieper/Vexler. % TODO: cite

\subsection{The discrete problem}
The discrete version of the previous problem would be
\begin{equation}
\tag{$P_h$}
\label{prb:Ph-StateConstrained}
\begin{IEEEeqnarraybox}[][c]{c}
\min_{u \in U_{h}} \frac{1}{2} \| y_h - y_d \|^2 + \frac{\lambda}{2} \| u_h \|^2 \\
\begin{IEEEeqnarraybox}[][c]{rCl"l}
(\nabla y_h, \nabla \varphi_h) &=& (u_h, \varphi_h) & \forall \varphi_h \in V_h \\
\IEEEeqnarraymulticol{3}{c}{ \mathcal{I}_h y_a(x) \leq y_h(x) \leq \mathcal{I}_h y_b } & \forall x \in \bar{\Omega}
\end{IEEEeqnarraybox}
\end{IEEEeqnarraybox}
\end{equation}
Here, $U_h$ could be either a variational or a full discretization and $\mathcal{I}_h : C(\bar{\Omega}) \to V_h$ refers to the nodal interpolant. In this setting, constraints have to be checked in the nodes only.
Let
\[
	M_h(\mathcal{T}_h) = \left\{ \partial x_i \mid x_i \in N(\mathcal{T}_h) \right\} \subset M(\Omega)
\]
and $N(\mathcal{T}_h)$ is the set of all (interior) nodes of $\mathcal{T}_h$.
For additional insight on this topic, see
\begin{itemize}
\item Meyer 2006 (also additional control constraints and cellwise constant discretization) % TODO: cite
\item Deckelnick 2007 (variational discretization) % TODO: cite
\end{itemize}

\subsection{Discretization of the discrete problem}
\begin{itemize}
\item Assume that $y_a, y_b \in \R$
\item Existence of a unique solution $y_h$ of the discrete state equation for every $u_h \in U_h$. This we have already proven.
\item Solvability (uniqueness) of \cref{prb:Ph-StateConstrained}?
\item Optimality conditions?
\item error estimates
\end{itemize}
Let's discuss the solvability first. On the continuous level, we have already proven that for constant $y_b, y_a$ we have a feasible solution $u_0, y_0$. The question is, is there a $u_{0, h} \in U_h, y_{0, h}$ that is feasible for \cref{prb:Ph-StateConstrained}. Consider the case of a variational discretization for the control, i.e.\ $U_h = L^2(\Omega)$. On the discrete level we have $y_a \leq G(u_0) \leq y_b$. What we would need would be $y_a \leq G_h(u_0) \leq y_b$.
We have the characterizations
\begin{IEEEeqnarray*}{rCl"l}
(y_0, \varphi) &=& (u_0, \varphi) & \forall \varphi \in V_0 \\
(y_{0, h}, \varphi_h) &=& (u_0, \varphi_h) & \forall \varphi_h \in V_h
\end{IEEEeqnarray*}
Given that we are introducing a discretization error, $y_a \leq G_h(u_0) \leq y_b$ could not hold if we test with the \textit{same} $u_0$ as in the continuous case.

However, the stronger characterization
\[
	y_a + \delta \leq G(u_0) \leq y_b - \delta
\]
has been proven already. Thus we need to estimate the $L^\infty$-error of the discretization. Could we even prove a result like
\[
	y_a + \frac{\delta}{2} \leq G_h(u_0) \leq y_b - \frac{\delta}{2},
\]
then we'd also have found a Slater point that we need for the optimality conditions.

In summary, we need an $L^\infty$-error estimate for the state equation.
One can show
\[
	\| G(u) - G_h(u) \|_{L^\infty(\Omega)} \leq ch^{2 - \frac{n}{2}} \| \nabla^2 G(u) \|_{L^2(\Omega)} \leq ch^{2 - \frac{n}{2}} \| u \|_{L^2(\Omega)}.
\]
For our setting of $n = 2$, we obtain
\[
	\| G(u) - G_h(u) \|_{L^\infty(\Omega)} \leq ch \| u \|_{L^2(\Omega)}.
\]
Variational discretization: If $u_0$ is a Slater point for \cref{prb:P-StateConstrained}, then it is a slater point for \cref{prb:Ph-StateConstrained} (for $h$ small enough) (and thus feasible).
\begin{proof}
We have
\begin{IEEEeqnarray*}{rCl"l}
y_h(u_0) &=& \underbrace{ y_h(u_0) - y(u_0) }_{\leq \| y_h(u_0) - y(y_0) \|_{L^\infty} } + \underbrace{ y(u_0) }_{= y_0 \leq y_b - \delta} \\
&\leq& ch \| u_0 \|_{L^2} + y_b - \delta \\
&\leq& y_b - \frac{\delta}{2} & \text{for $h$ small enough} 
\end{IEEEeqnarray*}
\end{proof}
Thus we have if $u_0$ is feasible that $P_h$ admits a unique solution $\bar{u}_h$ with $\bar{y}_h$.
If $u_0$ is a Slater point, we obtain an optimality system:
\begin{theorem}
Let $(\bar{u}_h, \bar{y}_h) \in U_h \times V_h$ be the solution of \cref{prb:Ph-StateConstrained}. Then there exist Lagrange multipliers $\bar{\mu}_{a,h}, \bar{\mu}_{b,h} \in M_h(\Omega)$, $\bar{\mu}_{a, h}, \bar{\mu}_{b, h} \geq 0$ and an adjoint state $\bar{p}_h \in V_{h, 0}$, such that
\begin{IEEEeqnarray*}{rCl"l}
(\nabla \bar{p}_h, \nabla \varphi_h) &=& (\bar{y}_h - y_d, \varphi_h) + \langle \bar{\mu}_{b, h} - \bar{\mu}_{a, h}, \varphi_h \rangle & \forall \varphi_h \in V_{h, 0} \\
\lambda \bar{u}_h + \bar{p}_h &=& 0 \\
\langle \bar{\mu}_{a, h}, y_a - \bar{y}_h \rangle &=& 0 = \langle \bar{\mu}_{b, h}, \bar{y}_h - y_b \rangle
\end{IEEEeqnarray*}
\end{theorem}
\begin{remark}
The equation $\lambda \bar{u}_h + \bar{p}_h = 0$ gives us
\[
\bar{u}_h = - \frac{1}{\lambda} \bar{p}_h \in V_{h, 0}
\]
\end{remark}
\paragraph{Auxiliary Result} There exists $c > 0$ independent of $h$ such that
\[
	\| \bar{u}_h \| + \| \bar{y}_h \| + \| \bar{\mu}_{a, h} \|_{M(\Omega)} + \| \bar{\mu}_{b, h} \|_{M(\Omega)} \leq c.
\]
\begin{proof}
Let $u_0$ be the Slater point of \cref{prb:P-StateConstrained}.
For the variational discretization we find
\begin{IEEEeqnarray*}{rCl}
\frac{\lambda}{2} \| \bar{u}_h \|^2 &\leq& J(\bar{u}_h, \bar{y}_h) \\
&\leq& J(u_0, y_h(u_0)) \\
&=& \frac{1}{2} \| y_{0, h} - y_a \|^2 + \frac{\lambda}{2} \| u_0 \|^2
\end{IEEEeqnarray*}
This implies
\begin{IEEEeqnarray*}{rCl}
\lambda \| \bar{u}_h \|^2 &\leq& \| y_{0, h} - y_a \|^2 + \frac{\lambda} \| u_0 \|^2.
\end{IEEEeqnarray*}
Given that $\| y_{0, h} \| \leq c \| u_0 \|$ with $c$ independent of $h$, we have $\| \bar{u}_h \| \leq c$ and $\| \bar{y}_h \| \leq c$.

Furthermore,
\begin{IEEEeqnarray*}{rCl}
0 &=& \langle \bar{\mu}_{b, h}, \bar{y}_h - y_b \rangle \\
&=& \langle \bar{\mu}_{b, h}, \bar{y}_h - \underbrace{ y_{0, h} }_{=y_h(u_0)} \rangle + \langle \bar{\mu}_{b, h}, \underbrace{ y_{0, h} - y_b }_{\leq - \frac{\delta}{2} } \rangle
\end{IEEEeqnarray*}
This implies
\begin{IEEEeqnarray*}{rCl}
- \frac{\delta}{2} &\leq& \langle \bar{\mu}_{b, h}, \bar{y}_h - y_{0, h} \rangle \\
&=& \langle \bar{\mu}_{b, h},  G_h( \bar{u}_h - u_{0} ) \rangle \\
&=& ( \nabla \bar{p}_h, \nabla (\bar{y}_h - y_{0, h})) - ( \bar{y}_h - y_d, \bar{y}_h - y_{0, h}) \\
&=& (\bar{u}_h - u_0, \bar{p}_h) - (\bar{y}_h - y_d, \bar{y}_h - y_{a, h}) \\
&=& - \frac{1}{\lambda} (\bar{u}_h - u_0, \bar{u}_h) - (\bar{y}_h - y_d, \bar{y}_h - y_{a, h}) \\
&\leq& c.
\end{IEEEeqnarray*}
Here we assumed that there was no lower bound present, for the general case, one exploits the separation of the supports of $\bar{\mu}_{b, h}$ and $\bar{\mu}_{a, h}$.
\end{proof}
\end{document}