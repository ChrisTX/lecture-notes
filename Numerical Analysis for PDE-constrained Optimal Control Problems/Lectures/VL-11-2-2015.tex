\documentclass[../skript.tex]{subfiles}

\begin{document}
\begin{itemize}
\item Every convex and continuous functional $f$ on a Banach space $U$ is weakly lower semi-continuous, i.e.
\[
u_n \rightharpoonup u \implies \liminf f(u_n) \geq f(u).
\]
Example:
\[
	f(u) = \frac{1}{2} \| Su - y_d \|_{L^2(\Omega)}^2 + \frac{\lambda}{2} \| u \|_{L^2(\Omega)}^2.
\]
\end{itemize}
\section{Existence of optimal controls}
\begin{theorem}
Consider Hilbert spaces $\{ U, \| \cdot \|_U \}$ and $\{ H, \| \cdot \|_H \}$, a nonempty, bounded, closed and convex set $U_{ad} \subset U$, an element $y_d \in H$ and a constant $\lambda \geq 0$.
Moreover, $S : U \to H$ is a linear and continuous operator.
Then, the problem
\[
	\min_{u \in U_{ad}} f(u) = \frac{1}{2} \| Su - y_d \|^2_{H} + \frac{\lambda}{2} \| u \|_U^2
\]
admits an optimal solution $\bar{u} \in U_{ad}$. If $\lambda > 0$, then the solution is unique.
\end{theorem}
\begin{proof}
Since $f(u) \geq 0$, $j \coloneqq \inf_{u \in U_{ad}} f(u) \in \R$ exists.
There exists a minimizing sequence $\{ u_n \} \subset U_{ad}$ with $f(u_n) \to j$.
$U_{ad}$ is nonempty, convex, closed, bounded, which implies $U_{ad}$ is weakly sequentially compact (see the results listed above).
Thus, there exists a subsequence $\{ u_{n_k} \} \subset U_{ad}$ which converges weakly to $\bar{u} \in U_{ad}$, i.e.
\[
	u_{n_k} \rightharpoonup \bar{u} \in U_{ad}.
\]
Since $f$ is continuous and convex, $f$ is \ac{wlsc} (again, see the results above).
Then,
\[
	f(\bar{u}) \leq \liminf_{k \to \infty} f\left(u_{n_k}\right) = j,
\]
implying that $\bar{u}$ is optimal.

Uniqueness: If $\lambda > 0$, then $f$ is strictly convex. Assume $\bar{u}_1, \bar{u}_2$ are optimal, then
\[
	f\left( \frac{1}{2} \bar{u}_1 + \frac{1}{2} \bar{u}_2 \right) < \frac{1}{2} f\left(\bar{u}_1\right) + \frac{1}{2} f \left(\bar{u}_2\right) = j.
\]
This is a contradiction to $\bar{u}_1, \bar{u}_2$ being optimal.
\end{proof}
We will consider the unbounded case now:
Let $U = U_{ad} = L^2(\Omega)$ and $u_0 \in U_{ad}$.
Then we define
\[
	U_{ad}^{aux} \coloneqq U_{ad} \cap \left\{ u \in U \coloneqq \| u \|^2 \leq \frac{2}{\lambda} f(u_0) \right\},
\]
which is closed, bounded, convex and nonempty.
Then $\bar{u}$ is solving
\[
	\min_{U \in U_{ad}^{aux}} f(u).
\]
For all elements $u \notin U_{ad}$, we find
\[
	f(u) = \frac{1}{2} \| Su - y_d \| ^2 + \frac{\lambda}{2} \| u \|^2 \geq \frac{\lambda}{2} \| u \|^2 > f(u_0).
\] 
``Any element outside $U_{ad}^{aux}$ is not a candidate for the optimal solution.''
\begin{theorem}
If $\lambda > 0$ and $U_{ad}$ is nonempty, convex and closed, then \cref{prb:ElP} admits a unique solution $\bar{u}$.
\end{theorem}
\section{Differentiability in Banach spaces}
Let $U, V$ be Banach spaces, an open subset $U_0 \subset U$ and $F : U \supset U_0 \to V$ and $u,v \in U$.

\begin{itemize}
\item If the limit
\[
	\partial F(u, v) \coloneqq \lim_{t \downarrow 0} \frac{1}{t} \left( F(u + tv) - F(u) \right)
\]
exists, it is called directional derivative of $F$ in $u$ in direction $v$. If this limit exists for all $v \in U$, then the mapping $v \mapsto \partial F(u, v)$ is called first variation of $F$ in $u$.
\item If $\partial F(u, v)$ exists in $u$ and there exists a linear and continuous operator $A : U \to V$ with
\[
	\partial F(u, v) = Av \quad \forall v \in U,
\]
then $F$ is called \emph{Gâteaux differentiable} and $A$ is the Gâteaux derivative of $F$ in $u$, $A = F'(u) \in U^*$.
\item A mapping $F : U \supset U_0 \to V$ is called \emph{Fréchet differentiable} in $U \in U_0$ if there exists $A \in \mathcal{L}(U, V)$ and a mapping $r(u, \cdot) : U \to V$ such that for all $v \in U$ with $u + v \in U_0$
\[
	F(u + v) = F(u) + Av + r(u, v), \quad \frac{\| r(u, v) \|_V}{\| v \|_U} \xrightarrow{\| v \|_U \to 0} 0.
\]
\end{itemize}
\begin{example}
\begin{itemize}
\item Consider the function
\[
f(x_1, x_2) = r \cos \varphi
\]
(polar coordinates). Compute $\partial f(0, v)$:
\[
	\partial f(0, v) = \lim_{t \downarrow 0} \frac{f(tv)}{t} = \frac{t}{t} \underbrace{ \sqrt{v_1^2 + v_2^2} }_{\text{not linear}} \cos \varphi_v.
\]
Therefore, it is neither Gâteaux nor Fréchet differentiable.
\item Consider the function
\[
f(x, y) = \begin{cases}
1 & y = x^2, \; x \neq 0 \\
0 & \text{else}
\end{cases}
\]
It is not Fréchet differentiable in $(0, 0)$ (not even continuous), but Gâteaux differentiable.
\item Consider the function
\[
	f(u) = \| u \|_H^2.
\]
It is Fréchet differentiable.
\end{itemize}
\end{example}
\begin{theorem}[Chain rule]
Let $U, V, Z$ be Banach spaces, $U_0 \subset U$, $V_0 \subset V$ be open sets and $F : U_0 \to V_0$ and $G : V_0 \to Z$ Fréchet differentiable in $u \in U_0$ and $F(u) \in V_0$.
Then $E = G \circ F$ is Fréchet differentiable in $u$ with $E'(u) = G'(F(u)) F'(u)$.
\end{theorem}
\begin{example}
\[
E(u) = \frac{1}{2} \| Su - y_d \|_H^2
\]
Then,
\[
E'(u) = (Su - y_d, Sv) = (S^*(Su - y_d), v).
\]
\end{example}

\end{document}