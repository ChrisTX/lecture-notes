\documentclass[../skript.tex]{subfiles}

\begin{document}
\textit{Sometimes} the two-norm discrepancy does not play a role.
\begin{itemize}
\item control enters the PDE only linearly.
\item $G$ continuous from $L^2 \to C(\bar{\Omega})$.
\item $J$ ``linear-quadratic'' structure with respect to $u$.
\end{itemize}
Then, we can work with the $L^2$-norm.
\begin{itemize}
\item Let $\Omega \subset \R^N$ be a bounded Lipschitz domain with $N \leq 3$, and let our general assumptions hold.
Then the control-to-state mapping for
\begin{IEEEeqnarray*}{rCl}
- \lapl y + d(y) &=& u \\
\partial_n y &=& 0
\end{IEEEeqnarray*}
is Lipschitz continuous from $L^2(\Omega) \to C(\bar{\Omega})$.
\item Let the objective function
\[
	J(y, u) = \int_\Omega \varphi(x, y) \dx + \int_\Omega \psi(x, u) \dx 
\]
fulfill
\[
	\psi(x, u) = \gamma_1(x) u + \gamma_2(x) u^2,
\]
where $\gamma_1, \gamma_2 \in L^\infty(\Omega)$ and $\gamma_2 \geq 0$ almost everywhere.
Then $J$ is twice continuously Fréchet differentiable in $L^2(\Omega)$.
\end{itemize}
Consider
\[
	\min J(y, u) \coloneqq \int_\Omega \psi(x, y) + \int_\Omega (\gamma_1 u + \gamma_2 u^2) \dx,
\]
with $\gamma_1, \gamma_2 \in L^\infty$, $\gamma_2 \geq 0$, where
\begin{IEEEeqnarray*}{c}
\begin{IEEEeqnarraybox}{rCl"l}
- \lapl y + d(y) &=& u & \text{in $\Omega$} \\
\partial_n y &=& 0 & \text{on $\partial \Omega$}
\end{IEEEeqnarraybox} \\
u_a \leq u \leq u_b,
\end{IEEEeqnarray*}
where $\Omega \subset \R^N$, $N \leq 3$ is a bounded Lipschitz domain, $u_a, u_b \in L^\infty$, $u_a \leq u_b$ almost everywhere under our general assumptions.

Let $\bar{u} \in U_{ad} \subset L^2(\Omega)$ fulfill
\[
	f'(\bar{u})(u - \bar{u}) \geq 0 \quad \forall u \in U_{ad}.
\]
If $\bar{u}$ with associated state $\bar{y}$ and adjoint state $\bar{p}$ fulfills
\[
	f''(\bar{u}) u^2 = \int_\Omega \varphi_{yy}(x, \bar{y}) - p d_{yy} (x, \bar{y}) \tilde{y}^2 + \gamma_2 u^2 \dx \geq \delta \| u \|_{L^2}^2, \quad \forall u = \begin{cases}
	u \geq 0 & \bar{u}(x) = u_a(x) \\
	u \leq 0 & \bar{u}(x) = u_b(x)
	\end{cases}
\]
with $\tilde{y}$ as the solution of
\begin{IEEEeqnarray*}{rCl}
	- \lapl \tilde{y} + d_y(x, \bar{y}) \tilde{y} &=& u \\
	\partial_n \tilde{y} &=& 0
\end{IEEEeqnarray*}
then there exist constants $\varepsilon > 0, \sigma > 0$ such that the quadratic growth condition
\[
	J(y, u) \geq J(\bar{y}, \bar{u}) + \sigma \| u - \bar{u} \|_{L^2(\Omega)}^2
\]
with $y = G(u)$ for all $u \in U_{ad}$ with $\| u - \bar{u} \|_{L^2(\Omega)} \leq \varepsilon$.
Thus, $\bar{u}$ is a strict local minimum in the sense of $L^2$.

Exercise:
We consider
\[
	f(u) = \frac{1}{2} \| G(u) - y_d \|_{L^2}^2 + \frac{\lambda}{2} \| u \|_{L^2}^2,
\]
then
\[
	f'(\bar{u}) u = (G(\bar{u}) - y_d, G'(\bar{u}) u) + \lambda(\bar{u}, u).
\]
One could write this as
\[
	(\underbrace{G'(\bar{u})^* (G(\bar{u}) - y_d)}_{\bar{p}} + \lambda (\bar{u}, u),
\]
but we'll work of the first formulation here
\[
	f''(\bar{u}) u^2 = (G(\bar{u}) - y_d, G''(\bar{u}) u^2) + \underbrace{( G'(\bar{u}) u, G'(\bar{u}) u)}_{(\tilde{y}, \tilde{y})} + \lambda (u, u)
\]
with $\tilde{y} = G'(\bar{u}) u$.
Furthermore, let us write
\[
	G''(\bar{u}) u^2 = z,
\]
so that
\[
	- \lapl z + d_y(x, \bar{y}) z = -d_{yy}(x, \bar{y}) (\underbrace{ G'(\bar{u}) u}_{\tilde{y}})^2,
\]
meaning we can write
\[
	z = G'(\bar{u}) ( -d_{yy}(x, \bar{y}) \tilde{y}^2).
\]
Putting this into the above equation:
\begin{IEEEeqnarray*}{rCl}
f''(\bar{u}) u^2 &=& (G(\bar{u}) - y_d, - G'(\bar{u}) (d_{yy}(x, \bar{y}) \tilde{y}^2)) + (\tilde{y}, \tilde{y}) + \lambda(u, u) \\
&=& (\underbrace{ G'(\bar{u})^* (G(\bar{u}) - y_d) }_{\bar{p}}, -d_{yy}(x, \bar{y}) \tilde{y}^2 ) + (\tilde{y}, \tilde{y}) + \lambda (u, u) \\
&=& \int_\Omega -d_{yy}(x, \bar{y}) \bar{p} \tilde{y}^2 + \tilde{y}^2 + \lambda u^2 \dx
\end{IEEEeqnarray*}

\section{Numerical analysis of the Finite Element-discretization of \texorpdfstring{\cref{eq:Nemytskii-FON-P}}{(P)}}
\paragraph{Main ideas}
\cref{eq:Nemytskii-FON-P} can have multiple local solutions $\bar{u}_i$. By discretizing the controls via some means we obtain a problem $P_h$ that - being non-convex as well - might have multiple local solutions $\bar{u}_{h, i}$ (and not even necessarily the same number as $P$ had). Because of this, we need to ensure that our local solutions on the discrete level correspond to a single local solution of the original problem, otherwise we cannot expect convergence.

For this, we use a localization technique due to Casas\slash{}Tröltzsch 2002. We pick $\bar{u}$ fulfilling first order necessary and (strong) second order sufficient conditions:
\begin{IEEEeqnarray*}{rCl"l}
f'(\bar{u})(u - \bar{u}) &\geq& 0 & \forall u \in U_{ad} \\
f''(\bar{u})v^2 &\geq& \delta \| v \|_{L^2(\Omega)}^2 & \forall v \in L^2(\Omega)
\end{IEEEeqnarray*}
and obtain the quadratic growth condition:
\[
	f(u) \geq f(\bar{u}) + \sigma \| u - \bar{u} \|_{L^2(\Omega)}^2 \quad \forall u \in U_{ad}, \| u - \bar{u} \|_{L^2(\Omega)} \leq \varepsilon.
\]
Then we consider an $L^2$ neighborhood in which the quadratic growth condition holds with maximal distance around $\bar{u}$ being $r$.
Then, we introduce the auxiliary problem $(P_h^r)$:
\begin{equation}
\label{eq:Phr}
\tag{$P_h^r$}
\begin{IEEEeqnarraybox}{c}
\min_{u_h \in U_{ad, h}} f_h(u_h) \\
\| u_h - \bar{u} \|_{L^2} \leq r,
\end{IEEEeqnarraybox}
\end{equation}
with $r$ small enough that the quadratic growth condition holds ($\varepsilon/2 > r$). This problem then has a global solution $\bar{u}_h^r$.
\begin{itemize}
\item $\bar{u}_h^r$ could either fulfill $\| \bar{u} - \bar{u}_h \|_{L^2} < r$ (nice situation $\bar{u}_h^r$ is a local minimum of $(P_h)$) or
$\| \bar{u} - \bar{u}_h \|_{L^2} = r$ (we would like to exclude this, because $\bar{u}_h^r$ may not be a local minimum of $(P_h)$).
\item Show (for both cases)
\[
	\bar{u}_h^r \xrightarrow[h\to 0]{L^2} \bar{u}_h.
\]
The rates of convergence depending on the particular discretization. This excludes $\| \bar{u}_h^r - \bar{u} \|_{L^2} = r \; \forall h$.
We will have shown that $\bar{u}$ can be approximated by local solutions of $(P_h)$.
\end{itemize}
\end{document}