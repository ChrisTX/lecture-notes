\documentclass[../skript.tex]{subfiles}

\begin{document}
\subsection{Existence of optimal controls}
\begin{problem}
\begin{equation}
\tag{$P$}
\label{prb:NemytskiiP}
\begin{IEEEeqnarraybox}[][c
$U_{ad}$ is bounded in $L^\infty$]{c}
\min J(y, u) = \int_\Omega \varphi(x, y(x)) \dx + \int_\Omega \psi(x, u(x)) \dx \\
\begin{IEEEeqnarraybox}{rCl"l}
- \lapl y + d(x, y) &=& u & \text{in $\Omega$} \\
\partial_\gamma y &=& 0 & \text{on $\Gamma$}
\end{IEEEeqnarraybox} \\
u_a \leq u \leq u_b
\end{IEEEeqnarraybox}
\end{equation}
\end{problem}
\begin{assumption}[A]
\label{as:NemytskiiA}
\begin{itemize}
\item $\Omega \subset \R^N$ bounded Lipschitz domain
\item $u_a, u_b \in L^\infty(\Omega)$, $u_a \leq u_b$ almost everywhere
\item $d$, $\varphi$, $\psi$ are measurable with respect to $x$ for all real $y$\slash{}$u$ and twice partially differentiable with respect to $y$\slash{}$u$ for almost all $x$
\item They fulfill boundedness and local Lipschitz conditions of order $k = 2$
\item it holds $d_y(x, y) \geq 0$ for almost all $x \in \Omega$, $\forall y \in \R$.
There exists a set $E_d \subset \Omega$ with positive measure and a constant $\lambda_a > 0$ such that
\[
	d_y(x, y) \geq \lambda_d \quad \forall x \in E_d, \forall y \in \R
\]
\end{itemize}
\end{assumption}
A control $\bar{u} \in U_{ad}$ is called optimal for \cref{prb:NemytskiiP} with associated optimal state $\bar{y}$ if
\[
	J(y(\bar{u}), \bar{u}) \leq J(y(u), u) \quad \forall u \in U_{ad}
\]
holds.
It is called locally optimal in the sense of $L^r(\Omega)$, if there exists $\varepsilon > 0$ such that
\[
	J(y(\bar{u}), \bar{u}) \leq J(y(u), u) \quad \forall u \in U_{ad} : \| u - \bar{u} \|_{L^r(\Omega)} \leq \varphi
\]
\begin{theorem}
Let \cref{as:NemytskiiA} hold and let $\psi$ be convex with respect to $u$ for almost all $x \in \Omega$.
Then \cref{prb:NemytskiiP} admits at least one optimal control $\bar{u}$ with associated state $\bar{y}$.
\end{theorem}
\begin{proof}
\begin{itemize}
\item We already know: The state equation admits for every $u \in U_{ad}$ exactly one solution $y = y(u) \in H^1(\Omega) \cap C(\bar{\Omega})$.
$U_{ad}$ is bounded in $L^\infty{\Omega}$, thus bounded in any $L^r(\Omega)$, $r > N/2$ (r > 2).
This implies:
\[
	\| y(u) \|_{C(\bar{\Omega})} \leq M \quad \text{for all $u \in U_{ad}$}
\]
\item $J(y, u)$ is bounded from below (by assumption). Hence, there exists
\[
	j = \inf_{u \in U_{ad}} J(y(u), u).
\]
\item Choose a minimizing sequence 
\[
	(y_n, u_n) : u_n \in U_{ad}, y_n = y(u_n), J(y_n, u_n) \xrightarrow{n \to \infty} j.
\]
\item $U_{ad} \subset L^r(\Omega)$ is nonempty, closed, convex, bounded in $L^r$ (reflexive Banach space).
Therefore, $U_{ad}$ is weakly sequentially compact, implying that
\[
	\exists \{ u_{n_k} \} \xrightharpoonup{L^r} \bar{u} \in U_{ad}.
\]
\ac{wlog} $u_n \xrightharpoonup{L^r} \bar{u} \in U_{ad}$ (candidate for optimal control).
\end{itemize}
\paragraph{Convergence of $y_n$} Consider
\[
	z_n(\cdot) = d(\cdot, y_n(\cdot)) - y_n
\]
with $y_n$ is bounded in $L^r$.
There exists a weakly convergent subsequence $z_n \xrightharpoonup{L^r} \bar{z}$.
Thus, we consider
\begin{IEEEeqnarray*}{rCl}
- \lapl y_n + y_n &=& \underbrace{-d(x, y_n) + y_n}_{= -z_n \rightharpoonup -\bar{z}} + \underbrace{u_n}_{\rightharpoonup \bar{u}} = R_n \\
\partial_\gamma y_n &=& 0
\end{IEEEeqnarray*}
$R_n \mapsto y_n$ is linear and continuous from $L^r$ in $H^1$. In consequence, the mapping is weakly continuous and we have
\[
	- z_n + u_n \rightharpoonup - \bar{z} + \bar{u},
\]
which in turn implies $y_n \xrightharpoonup{H^1(\Omega)} \bar{y}$.
Furthermore, because of the compact embedding of $H^1$ in $L^2$ we have $y_n \xrightharpoonup{L^2} \bar{y}$.
The set
\[
	\{ y \in L^\infty(\Omega) : \| y \|_{L^\infty} \leq M \}
\]
is \emph{convex} and closed, so $|\bar{y}(x)| \leq M$.
As of such,
\[
	\| d(\cdot, y_n) - d(\cdot, \bar{y}) \|_{L^2} \leq L(M) \| y_n - \bar{y} \|_{L^2(\Omega)},
\]
which in turn implies
\[
	d(\cdot, y_n) \xrightarrow{L^2} d(\cdot, \bar{y}).
\]
\paragraph{``$\bar{u}$ and $\bar{y}$ belong together''} We calculate
\begin{IEEEeqnarray*}{rCl"l}
\underbrace{ \int_\Omega \nabla y_n \nabla v \dx }_{\to \int_\Omega \nabla \bar{y} \nabla v \dx} + \underbrace{ \int_\Omega d(\cdot, y_n) v \dx }_{\to \int_\Omega d(\cdot, \bar{y}) v \dx} &=& \underbrace{ \int_\Omega u_n v \dx }_{\to \int_\Omega \bar{u} v \dx} & \forall v \in H^1(\Omega)
\end{IEEEeqnarray*}
\paragraph{$\bar{u}$ is optimal}
We know
\begin{IEEEeqnarray*}{rCl}
	j = \lim_{n \to \infty} J(y_n, u_n) &=& \lim_{n \to \infty} F(y_n) + \lim_{n \to \infty} Q(u_n) \\
	&=& F(\bar{y}) + \liminf Q(\bar{u}) \\
	&\geq& F(\bar{y}) + Q(\bar{u}) \\
	&=& J(\bar{y}, \bar{u}).
\end{IEEEeqnarray*}
because $Q$ is continuous and convex and therefore weakly lower semi-continuous.
By the definition of $j$, equality has to hold and
\[
	J(\bar{y}, \bar{u}) = j,
\]
implying that $\bar{u}$ is optimal.
\end{proof}
Let us consider a non-convex example:
\begin{example}
Consider
\begin{IEEEeqnarray*}{c}
\min f(u) = - \int_0^1 \cos(u(x)) \dx \\
0 \leq u(x) \leq 2 \pi \\
u \in L^\infty(0, 1)
\end{IEEEeqnarray*}
The minimum of this is $-1$, and thus both $\bar{u} \equiv 0$ and $\bar{u} \equiv 2 \pi$ are solutions. More specifically any function such that $\bar{u}(x) \in \{ 0, 2 \pi \}$ is an optimal solution. Therefore, solutions can be arbitrarily close in the $L^2$ norm.
\end{example}
\paragraph{The control-to-state operator}
\begin{proposition}
Under our assumptions the control to state operator $G : L^r(\Omega) \to H^1(\Omega) \cap C(\bar{\Omega})$ is Lipschitz-continuous, i.e.\ there exists a constant $L$ such that
\[
	\| y_1 - y_2 \|_{H^1} + \| y_1 - y_2 \|_{C(\bar{\Omega})} \leq L \| u_1 - u_2 \|_{L^r}.
\]
\end{proposition}
\begin{proof}
One has
\begin{IEEEeqnarray*}{rCl}
- \lapl y_1 + d(x, y_1(x)) &=& u_1 \\
- \lapl y_2 + d(x, y_2(x)) &=& u_2
\end{IEEEeqnarray*}
Subtracting one from the other yields
\[
	- \lapl (y_1 - y_2) + d(x, y_1) - d(x, y_2) = u_1 - u_2.
\]
By expanding using Taylor expansion we obtain
\[
	- d(x, y_2) = \Bigg( d(x, y_1) + \underbrace{ \left( \int_{s=0}^1 d_y(x, y_1 + s(y_2 - y_1)) \ds \right) }_{c_0(x)} (y_2 - y_1) \Bigg).
\]
This gives us:
\begin{IEEEeqnarray*}{rCl}
- \lapl (y_1 - y_2) + c_0(x) (y_1 - y_2) &=& u_1 - u_2 \\
\partial_\gamma(y_1 - y_2) &=& 0
\end{IEEEeqnarray*}
For $c_0(x)$ we have that it's nonnegative and that it is positive on a set of positive measure $E_d$. Therefore, $c_0(x)(y_1 - y_2)$ is monotone.
In consequence:
\[
	\| y_1 - y_2 \|_{H^1} + \| y_1 - y_2 \|_{C(\bar{\Omega})} \leq c \| u_1 - u_2 \|_{L^r}.
\]
\end{proof}
\begin{proposition}
Under the same assumptions as before, $G$ is Fréchet-differentiable from $L^r(\Omega)$ to $H^1(\Omega) \cap C(\bar{\Omega})$, $r > N/2$.
The derivative is given by
\[
	G'(\bar{u}) u = y
\]
where $y$ solves
\begin{IEEEeqnarray*}{rCl}
- \lapl y + d_y(x, \bar{y}) y &=& u \\
\partial_\gamma y &=& 0
\end{IEEEeqnarray*}
in a weak sense.
\end{proposition}
\end{document}