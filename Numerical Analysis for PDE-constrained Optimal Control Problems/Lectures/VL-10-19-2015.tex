\documentclass[../skript.tex]{subfiles}

\begin{document}
\section*{Literature recommendations}
For more general background knowledge, the following books are recommended:
\begin{itemize}
\item \cite{AltNO} % Nichtlineare Optimierung - W. Alt (Viewig)
\item Nichtlineare Optimierung - M/S. Ulbrich (Birkhäuser)
\item \cite{Evans} % Partial differential equations - Evans
\item \cite{Ciarlet} % Ciarlet - Theory of finite elements
\end{itemize}
The followings books are recommended specifically for this course:
\begin{itemize}
\item F. Tröltzsch - Optimale Steuerung partieller Differentiagleichungen (*)
\item \cite{HinzePinnauUlbrich} % Hinze, Pinnau, M/S. Ulbrich - Optimization with PDE constraints
\end{itemize}
\chapter{Optimal control - An introduction}
% Ch. 1
\label{sec:c1}
\begin{example}[``Rocket car'' \lbrack{}Mackie-Strauss\rbrack{}]
Imagine we have a rocket car that we intend to move from point $A$ to point $B$ as fast as possible without it speeding over point $B$, i.e.\ fully stopping at $B$. For this purpose, we can accelerate or decelerate the car in both directions with the same force.

In a mathematical model, let
\begin{itemize}
\item $y(t) \in \R$ position of car at time $t$
\item $m$ mass of the car
\item $u(t)$ thrust (``Schubkraft'')
\item $-1 \leq u(t) \leq 1$. In this setting $-1$ would be ``break'' and $1$ equals ``full acceleration''
\end{itemize}
The objective is then:
\begin{IEEEeqnarray*}{u"l}
Objective: & \text{minimize time $T$} \\
Contraints (``Nebenbedingungen''):  & \begin{IEEEeqnarraybox}[][t]{rCl"l}
my'(t) &=& u(t) & \text{in } (0, T) \\
y(0) &=& y_0 \\
y(T) &=& y_T \\
y'(0) &=& 0 \\
y'(T) &=& 0 \\
|u(t)| &\leq& 1
\end{IEEEeqnarraybox}
\end{IEEEeqnarray*}
Thus, we have a differential equation plus some inequality constraints.
\begin{itemize}
\item \emph{control $u$} can be chosen freely between given bounds
\item \emph{state $y$} is uniquely determined by the differential equation (and boundary constraints)
\end{itemize}
Goal: Find a an optimal control $\bar{u}$ (such that the objective function takes the smallest possible value)
\end{example}
\paragraph{Applications:}
robotics, motion patterns, control processes (these are mainly described by ODEs)

Often, processes are described by PDEs:
heat transfer, diffusion, electromagnetism
\begin{examplenumb}[``hot potato''] % Example 1
\label{ex:1}
Let $\Omega \subset \R^3$, a time $T > 0$ and $y(x, t)$ be the temperature at $(x, t) \in \Omega \times (0, T)$. \\
Goal: ``well baked potato'' \\
The initial temperature at time $t = 0$ is given as $y_0 = y_0(x)$. Find a control such that $y(x, T) \approx y_\Omega(x)$ ``fit to eat''.

We obtain the following mathematical model for the problem:
\begin{IEEEeqnarray*}{u"l}
& \min J(y, u) = \frac{1}{2} \int_\Omega \left( y(x, T) - y_\Omega(x) \right)^2 \dx + \frac{\lambda}{2} \int_0^T \int_\Omega u(x, t)^2 \dx \dt \\
such that & \begin{IEEEeqnarraybox}[][t]{rCl"l}
y_t - \lapl y &=& 0 & \text{in } Q = \Omega \times (0, T) \\
\frac{\partial y}{\partial n} &=& \alpha(u - y) & \text{in } \Sigma = \partial \Omega \times (0, T) \\
y(x, 0) &=& y_0(x) & \text{in } \Omega \\
u_a(x, t) &\leq& u(x, t) \leq u_b(x, t)
\end{IEEEeqnarraybox}
\end{IEEEeqnarray*}
\end{examplenumb}
\begin{examplenumb}
% Example 2
\label{ex:2}
Let $\Omega \subset \R^3$ and
\begin{IEEEeqnarray*}{u"l}
& \min J(y, u) = \frac{1}{2} \int_\Omega \left( y(x) - y_\Omega(x) \right)^2 \dx + \frac{\lambda}{2} \int_\Omega u(x)^2 \dx \\
such that & \begin{IEEEeqnarraybox}[][t]{rCl"l}
- \lapl y + y &=& \beta u & \text{in } \Omega \\
y &=& 0 & \text{in } \Gamma \coloneqq \partial \Omega \\
u_a(x) &\leq& u(x) \leq u_b(x)
\end{IEEEeqnarraybox}
\end{IEEEeqnarray*}
For example, if we'd only want control to live within $\Omega_C \subset \Omega$, we could set $\beta = \chi_{\Omega_C}$. More realistic would be to use $- \dive (a \grad y)$ instead.
\end{examplenumb}
The model problem for the next weeks will be:
\begin{IEEEeqnarray*}{u"l}
& \min J(y, u) = \frac{1}{2} \int_\Omega \left( y(x) - y_\Omega(x) \right)^2 \dx + \frac{\lambda}{2} \int_\Omega u(x)^2 \dx \\
\circled{1} & \begin{IEEEeqnarraybox}[][c]{rCl"l}
- \lapl y + y &=& u & \text{in } \Omega \\
y &=& 0 & \text{in } \Gamma
\end{IEEEeqnarraybox} \\
\circled{2} & {} + u_a(x) \leq u(x) \leq u_b(x) \\
\circled{3} & {} + y_a(x) \leq y(x) \leq y_b(x)
\end{IEEEeqnarray*}

\begin{samepage}
Numerical Analysis of optimal control problems:
\begin{enumerate}
\renewcommand{\labelenumi}{\circled{\theenumi}}
\item ``Understand'' the PDE
\begin{itemize}
\item existence, uniqueness, regularity for the PDE.
\item ``control-to-state operator'' $S : U \mapsto Y$
\end{itemize}
\item Solvability of the optimal control problem
\begin{itemize}
\item existence (and uniqueness?) of global or possibly local solutions $\bar{u}$
\end{itemize}
\item Characterization of solutions via optimality conditions
\begin{itemize}
\item first order necessary conditions (FON)
\item (in non-convex situations: second order sufficient conditions (SSC))
\end{itemize}
\item Numerical solution:
\begin{itemize}
\item Discretization of the PDE (here: linear finite elements)
\item Discretization of the control (different possibilities)
\end{itemize}
\item a priori discretization error estimates
\begin{itemize}
\item The difference between the optimal controls for the continuous and discrete problems: $\| \bar{u} - \bar{u}_h \|_{L^2} \leq ?$ depending on $h$ (discretization parameter). Ideally we'd like to have a right hand side of $ch^p$.
\end{itemize}
\end{enumerate}
\end{samepage}
\begin{center}
\begin{tabular}{c c}
Continuous Problem & Discrete Problem \\
\begin{IEEEeqnarraybox}{c}
\min f(u) = J(S(u), u) \\
u_a \leq u \leq u(b)
\end{IEEEeqnarraybox} &
\begin{IEEEeqnarraybox}{c}
\min f_h(u_h) = J(S_h(u_h), u_h) \\
I_h u_a \leq u_h \leq I_h u(b)
\end{IEEEeqnarraybox} \\
optimal solution $(\bar{u}, \bar{y}), \bar{y} = y(\bar{u})$ &
optimal solution $(\bar{u}_h, \bar{y}_h), \bar{y}_h = y_h(\bar{u}_h)$
\end{tabular}
\end{center}
Note that
\[
	\| \bar{y} - \bar{y}_h \| \leq \| \underbrace{ \bar{y} }_{ y(\bar{u}) } \| + \| y_h(\bar{u}) - y_h(\bar{u}_h) \|.
\]
\end{document}