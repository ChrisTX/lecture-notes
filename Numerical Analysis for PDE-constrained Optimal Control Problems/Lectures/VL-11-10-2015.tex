\documentclass[../skript.tex]{subfiles}

\begin{document}
\section{Finite-Element-Discretization of the state equation (and the adjoint equation)}
The continuous equation is
\begin{IEEEeqnarray*}{rCl"l}
- \Delta y &=& f & \text{in } \Omega \subset \R^2 \\
y &=& 0 & \text{on } \Gamma
\end{IEEEeqnarray*}
The weak formulation thereof is:

Find $y \in H_0^1(\Omega)$ such that
\begin{equation}
\label{eq:FEM-StateEqWeak}
\tag{E}
	(\nabla y, \nabla v) = (f, v) \quad \forall v \in H_0^1(\Omega).
\end{equation}

Discretization:
\begin{itemize}
\item $\Omega$ is convex, and polygonally bounded
\item A partitioning $\mathcal{T} = \{ K_1, \ldots, K_M \}$ of $\Omega$ into triangles (or quadrilaterals) is called feasible if:
\[
\bar{\Omega} = \bigcup \bar{K}_i
\]
\item If we consider $K_i \cap K_j$ consists of exactly one point this has to be the corner of both $K_i$ and $K_j$; if it contains more than one point it has to be an edge of both cells
\end{itemize}
A family of partitionings $\{ \mathcal{T}_h \}$ is called uniform, if there exists a real number $\varkappa > 0$, such that every element $K$ of $\mathcal{T}_n$ contains a circle with radius $\varrho_{\mathcal{T}} \geq \frac{h}{\varkappa}$.

We consider the space $V_h$:
\[
	V_h = V_h(\mathcal{T}_h) = \left\{ v_h \in C(\bar{\Omega}) \midcolon \left. v_h \right|_K \in \mathcal{P}_1(K) \;\; \forall K \in \mathcal{T}_h, \quad \left. v_h \right|_{\partial \Omega} = 0 \right\}.
\]
Then $V_h \subset V = H_0^1(\Omega)$. As a basis we choose
\[
	\left\{ \phi_1^h, \ldots, \phi_{N_h}^h \right\} \subset V_h, \quad \phi_i^h(x_j) = \delta_{ij}.
\]

The discrete formulation of \cref{eq:FEM-StateEqWeak} reads:
Find $y_h \in V_h$ such that
\begin{equation}
\tag{$E_D$}
\label{eq:FEM-StateWeakDiscrete}
(\nabla y_h, \nabla v_h) = (f, v_h) \quad \forall v_h \in V_h.
\end{equation}
For every $f \in L^2(\Omega)$ there exists a unique weak solution $y_h$ of \cref{eq:FEM-StateWeakDiscrete}. This can be proved by using Lax-Milgram.

The solution $y_h$ can be determined by a system of linear equations.
We have a representation
\[
	y_h = \sum_j y_{h_j} \phi_j^h
\]
and test with $v_h = \phi_i^h$ for all $i$. This gives us a system
\[
	A_h y_h = B_h,
\]
where
\[
	A_{h_{ij}} = \left( \nabla \phi_i^h, \nabla \phi_j^h \right), \quad B_{h_j} = \left( f, \phi_j^h \right).
\]
$A_h$ is positive definite:
\begin{IEEEeqnarray*}{rCl"l}
	y^\tp A_h y &=& \sum_{jk} y_j A_{jk} y_k \\
	&=& a \left( \sum_k y_k \nabla \phi_k^h, \sum_j y_j \nabla \phi_j^h \right) \\
	&=& \left\| y_h \right\|_E^2 \\
	&\geq& \alpha \left\| y_h \right\|_{H^1}^2 \\
	&>& 0, & \text{if $y_h \neq 0$.}
\end{IEEEeqnarray*}

Stability:
\[
	\left\| \nabla y_h \right\| \leq \frac{1}{\tilde{\alpha}} \| f \|_{L^2}.
\]
To show this:
\begin{IEEEeqnarray*}{rCl}
\alpha \left\| \nabla y_h \right\|^2 & \leq& a(y_h, y_h) \\
&\leq& \| f \| \left\| y_h \right\| \\
&\leq& c \| f \| \left\| \nabla y_h \right\|.
\end{IEEEeqnarray*}
This implies 
\[
	\left\| \nabla y_h \right\| \leq \frac{1}{\tilde{\alpha}} \| f \|_{L^2}.
\]

We obtain the Galerkin orthogonality again:
\begin{IEEEeqnarray*}{rCl"l}
a(y, v) &=& (f, v) & \forall v \in V \\
a(y_h, v_h) &=& (f, v_h) & \forall v_h \in V_h \subset V.
\end{IEEEeqnarray*}
Combining these, we obtain:
\[
	a(y - y_h, v_h) = 0 \quad \forall v_h \in V_h.
\]
\begin{lemma}[Céa]
Let the bilinear form $a$ be elliptic ($H_0^1(\Omega)$) and let $y$ and $y_h$ be the weak solutions of \cref{eq:FEM-StateEqWeak} and \cref{eq:FEM-StateWeakDiscrete}, respectively. Then:
\[
	\left\| \nabla (y -y_h) \right\| \leq \frac{c}{\alpha} \inf_{v_h \in V_h} \left\| \nabla (y - v_h) \right\|.
\]
\end{lemma}
The nodal interpolant $i_h : C(\Omega^\tp) \to V_h$ is defined by
\[
	i_h v = \sum_{x_i \in N_h(\mathcal{T}_h)} v(x_i) \phi_i^h.
\]
The following estimates hold for $v \in H^2(\Omega) \cap H_0^1(\Omega)$:
\begin{IEEEeqnarray*}{rCl}
	\left\| v - i_h v \right\|_{L^2(K)} &\leq& c h_K^2 \left\| \nabla^2 v \right\|_{L^2(K)} \\
	\left\| \nabla ( v - i_h v ) \right\|_{L^2(K)} &\leq& c h_K^2 \left\| \nabla^2 v \right\|_{L^2(K)} \\
	\left\| v - i_h v \right\|_{L^2(\Omega)} &\leq& c h^2 \left\| \nabla^2 v \right\|_{L^2(\Omega)} \\
	\left\| \nabla( v - i_h v ) \right\|_{L^2(\Omega)} &\leq& c h \left\| \nabla^2 v \right\|_{L^2(\Omega)}
\end{IEEEeqnarray*}
The solutions $y$ and $y_h$ fulfill
\begin{IEEEeqnarray*}{rCl}
\left\| \nabla (y - y_h) \right\|_{L^2(\Omega)} &\leq& ch \left\| \nabla^2 y \right\|_{L^2(\Omega)} \\
\left\| y - y_h \right\|_{L^2(\Omega)} &\leq& ch^2 \underbrace{ \left\| \nabla^2 y \right\|_{L^2(\Omega)} }_{\leq c \| f \|_{L^2(\Omega)}}
\end{IEEEeqnarray*}
\begin{proof}
We define $w$ by the following equation:
\begin{IEEEeqnarray*}{rCl}
- \Delta w &=& y - y_h \\
w &=& 0
\end{IEEEeqnarray*}
Then,
\[
	\left( \nabla w, \nabla (y - y_h) \right) = \| y - y_h \|_{L^2}^2.
\]
Furthermore, we have
\[
	( \nabla \underbrace{ i_h w }_{\in v_h}, \nabla (y - y_h) ) = 0.
\]
By using these facts, we obtain
\begin{IEEEeqnarray*}{rCl}
\| y - y_h \|_{L^2}^2 &=& \left( \nabla w, \nabla (y - y_h) \right) \\
&=& ( \nabla (w - \underbrace{ i_h w }_{\in v_h}), \nabla (y - y_h) ) \\
&\leq& \underbrace{ \left\| \nabla (w - i_h w) \right\| }_{ \leq ch \underbrace{ \| \nabla^2 w \| }_{\leq c \| y - y _h \| } } \underbrace{ \| \nabla ( y - y_h ) \| }_{\leq c h \| \nabla^2 y \|_{L^2} } \\
&\leq& ch^2 \| y - y_h \| \| \nabla^2 y \|.
\end{IEEEeqnarray*}
By dividing by $\| y - y_h \| $ we obtain
\[
	\| y - y_h \| \leq c h^2 \underbrace{ \| \nabla^2 y \| }_{\leq c \| f \|}
\]
\end{proof}
\chapter{A priori error estimates for optimal control problems without inequality constraints}
\begin{problem}
Let $\Omega \subset \R^2$ be polygonally bounded and convex.
\begin{equation}
\tag{$P$}
\label{prb:P-est-wo-constraint}
\begin{IEEEeqnarraybox}[][c]{c}
\min_{u \in L^2(\Omega)} \frac{1}{2} \| y - y_d \|^2 + \frac{\lambda}{2} \| u \|^2 \\
\begin{IEEEeqnarraybox}[][c]{rCl"l}
- \Delta y &=& u & \text{in } \Omega \\
y &=& 0 & \text{on } \Gamma
\end{IEEEeqnarraybox}
\end{IEEEeqnarraybox}
\end{equation}
\end{problem}
The constraints have the weak formulation: Find $y \in V$ such that
\[
	(\nabla y, \nabla v) = (u, v) \quad \forall v \in V.
\]
As a reduced formulation we obtain
\[
	\min_{u \in L^2(\Omega)} f(u) \coloneqq \frac{1}{2} \| Su - y_d \|^2 + \frac{\lambda}{2} \| u \|^2.
\]
The optimality conditions are (because $u - \bar{u}$ spans the whole space due to a lack of constraints)
\[
f'(\bar{u}) v \geq 0.
\]
With the adjoint equation
\begin{IEEEeqnarray*}{rCl}
- \Delta \bar{p} &=& \bar{y} - y_d \\
\bar{p}&=& 0
\end{IEEEeqnarray*}
this yields
\[
	\lambda \bar{u} + \bar{p} = 0.
\]
\begin{problem}
The discrete analogous is
\begin{equation}
\tag{$P_h$}
\label{prb:Ph-est-wo-constraint}
\begin{IEEEeqnarraybox}[][c]{c}
\min_{u_h \in U_h} \frac{1}{2} \| y_h - y_d \|^2 + \frac{\lambda}{2} \| u_h \|^2 \\
y_h \in V_h \; : \; (\nabla y_h, \nabla v_h) = (u_h, v_h) \quad \forall v_h \in V_h
\end{IEEEeqnarraybox}
\end{equation}
\end{problem}
By using the discrete control-to-state operator $S_h : U_h \to V_h \hookrightarrow L^2(\Omega)$ we obtain
\[
	\min_{u_h \in U_h} f_h(u_h) \coloneqq \frac{1}{2} \| S_h u_h - y_d \|^2 + \frac{\lambda}{2} \| u_h \|^2.
\]
For optimality conditions, we obtain with
\[
	p_h \in V_h \; : \; (\nabla p_h, \nabla v_h) = (y_h - y_d, v_h) \quad \forall v_h \in V_h
\]
the condition
\[
	\lambda \bar{u}_h + \bar{p}_h = 0.
\]
For $U_h$, we will consider the choices:
\begin{itemize}
\item piecewise constant
\item piecewise linear
\item no control discretization (variational discretization)
\end{itemize}
\end{document}