\documentclass[../skript.tex]{subfiles}

\begin{document}
\chapter{Optimal control problems with pointwise state constraints}
\begin{problem}
Let
\[
	U = L^2(\Omega)
\]
and
\[
	U_{ad} = \left\{ u \in L^2(\Omega) \midcolon u_a \leq u \leq u_b \quad \text{a.e.} \right\}.
\]
Then the problem is
\begin{equation}
\tag{$P$}
\begin{IEEEeqnarraybox}[][c]{c}
\min_{u \in U_{ad}} \frac{1}{2} \| y - y_d \|_{L^2}^2 + \frac{\lambda}{2} \| u \|_{L^2}^2 \\
\begin{IEEEeqnarraybox}[][c]{rCl"l}
- \Delta y + y &=& u & \text{in } \Omega \\
y &=& 0 & \text{on } \Gamma \\
\IEEEeqnarraymulticol{3}{r}{ y_a(x) \leq y(x) \leq y_b(x)} & \forall x \in \bar{\Omega}
\end{IEEEeqnarraybox}
\end{IEEEeqnarraybox}
\end{equation}
Furthermore, let $\Omega \subset \R^2$ be polygonally bounded, convex (or smooth boundary), $y_d \in L^2(\Omega)$, $\lambda > 0$.
Let $y_a, y_b \in C(\bar{\Omega})$. Additionally, for reasons apparent later on, we assume
\begin{IEEEeqnarray*}{rCl"l}
y_a(x) &<& 0 < y_b & \forall x \in \partial \Omega \\
y_a(x) &<& y_b(x) & \forall x \in \bar{\Omega}
\end{IEEEeqnarray*}
\end{problem}
Let us first see why we demand pointwise constraints and do not work in an $L^2$ sense:
With $L^2$ constraints, we'd formally we have a problem
\begin{IEEEeqnarray*}{c}
\min f(u) \\
y_a \leq Su \leq y_b
\end{IEEEeqnarray*}
For sake of simplicity, let $y_a \equiv -\infty$ for now.
In analogy to the case without constraints on the state, we apply the formal Lagrange technique: The Lagrange function would be
\[
	L(u, \mu) = f(u) + \int (Su - y_d) \mu.
\]
Thus we obtain a set of conditions as follows
\begin{IEEEeqnarray*}{rCl}
Lu (\bar{u}, \bar{\mu}) &=& 0, \\
S \bar{u} - y_b &\leq& 0, \\
\mu &\geq& 0, \\
\int (S\bar{u} - y_b) \bar{\mu} &=& 0.
\end{IEEEeqnarray*}
The integrand is a product of $S\bar{u} - y_b$, which is smaller than zero outside of a single point, where it is zero. As $\bar{\mu}$ is greater or equal than zero, it can only be non-zero in a single point, namely where $S \bar{u} = y_b$. Therefore, we have a Dirac measure and not an $L^2$ function.
Thus, $y_a \leq y \leq y_b$ a.e.\ would be a problem when showing first order optimality conditions.

\paragraph{Solvability of the PDE (Regularity)}
As $u \in L^2(\Omega)$, we have 
\[
	y(u) \in H^2 \cap H_0^1(\Omega) \xhookrightarrow{n \leq 3} C(\bar{\Omega}).
\]
The control-to-state mapping is then operating on the following spaces:
\[
	G : L^2(\Omega) \to H^2(\Omega) \cap H_0^1(\Omega) \hookrightarrow C(\bar{\Omega}).
\]
As before, we define
\[
	S \coloneqq EG : L^2(\Omega) \to L^2(\Omega).
\]
This gives us a reduced problem form of
\[
\begin{IEEEeqnarraybox}[][c]{c}
\min_{u \in U_{ad}} f(u) \coloneqq \frac{1}{2} \| Su - y_d \|_{L^2}^2 + \frac{\lambda}{2} \| u \|_{L^2}^2 \\
y_a(x) \leq Gu(x) \leq y_b(x) \quad \forall x \in \bar{\Omega}
\end{IEEEeqnarraybox}
\]
\paragraph{Solvability of \texorpdfstring{$P$}{P}}
We have two constraints:
\begin{itemize}
\item $u_a \leq u \leq u_b$
\item $y_a(x) \leq Gu(x) \leq y_b(x)$
\end{itemize}
There are examples where no control $u$ exists such that both equations are fulfilled
We can always find a control fulfilling the first condition while ignoring the second, but can we also find a state while ignoring the first condition?
Assume $U_{ad} = L^2(\Omega)$.
Then the conditions on the state are
\begin{IEEEeqnarray*}{rCl"l}
y_a \leq Gu &\leq& y_b & \text{in } \Omega \\
- \Delta y + y &=& u & \text{in } \Omega \\
y &=& 0 & \text{on } \Gamma
\end{IEEEeqnarray*}
By assumption ($0 < y_b$ on $\Gamma$, but $y = 0$ on $\Gamma$),
\[
	d = \min \left( y_b(x) - y(x) \right) > 0
\]
Our starting point fulfilling the first condition would be
\[
	v = \frac{1}{2} \left( y_a + y_b \right)
\]
Next, we construct, using a mollifier, a $w \in C^2 \hookrightarrow C^\infty$ with
\[
	\| v - w \|_{L^\infty} < \frac{d}{2}
\]
Then $w \in H^2$ and
\[
	y_a(x) \leq w(x) \leq y_b(x).
\]
Therefore, the second condition can be fulfilled by the choice of an appropriate $u_0$.
To fulfill the zero boundary conditions: Choose
\[
	\xi(x) = \begin{cases}
	0 & \dist(x, \partial \Omega) < \frac{r}{4} \\
	1 & \dist(x, \partial \Omega) > \frac{r}{2} \\
	\in [0, 1] & \text{else}
	\end{cases}.
\]
Then $\xi \in C^\infty$ and
\[
	y_0 = \xi w \in H^2 \cap H_0^1(\Omega)
\]
and
\[
	y_a(x) \leq y_0(x) \leq y_b(x) \quad \forall x \in \bar{\Omega}.
\]
Therefore,
\[
	u_0 = - \Delta y_0 + y_0 \in L^2(\Omega).
\]

We have proven:
Let $U_{ad} = L^2(\Omega)$. Then there exists a feasible pair $(u_0, y_0)$ such that
\[
	y_a(x) \leq y_0(x) \leq y_b(x) \quad \forall x \in \bar{\Omega},
\]
and the state equation and boundary conditions are fulfilled.

If $U_{ad} \subsetneq L^2(\Omega)$, the existence of a feasible pair $(u_0, y_0)$ needs to be assumed.

If the set of feasible controls ($u \in U_{ad}$, and $y_a \leq y(u) \leq y_b$) is nonempty, existence of a unique optimal control $\bar{u}$ with optimal state $\bar{y}$ follows by standard arguments.
\section{The Lagrange-Multiplier-Rule}
For additional information on this topic, see \cite{Troeltzsch,TroeltzschEN}.

Let $U, Z$ be Banach spaces and $C \subset U$ be convex and non-empty.
\begin{itemize}
\item A convex set $K \subset Z$ is called a convex cone if for all $z \in K$ there holds $\lambda z \in K$ for all $\lambda \geq 0$.
We write $z \geq_K 0$ if and only if $z \in K$.
\item Let $K \subset Z$ be a convex cone. We call
\[
	K^* = \left\{ z^* \in Z^* \midcolon \langle z^*, z \rangle_{Z^*, Z} \geq 0 \quad \forall z \in K \right\}
\]
is the dual cone to $K$.
\end{itemize}
\begin{example}
Let
\[
	K = \left\{ z \in L^2(\Omega) \midcolon z(x) \geq 0 \quad \text{a.e. in } \Omega \right\}.
\]
Then $Z^* = Z = L^2(\Omega)$ and
\[
	K^* = \left\{ z^* \in L^2 \midcolon \int z^* z \dx \geq 0 \quad \forall z \in K \right\} = K.
\]
\end{example}
\begin{problem}
We consider the problem
\begin{equation}
\tag{$*$}
\label{prb:ConeLgrStar}
\begin{IEEEeqnarraybox}[][c]{c}
\min f(u) \\
G(u) \leq_K 0 \quad u \in U.
\end{IEEEeqnarraybox}
\end{equation}
with $G : U \to Z$.
\end{problem}

Note that this is not the same $G$ as before, but it has rather been chosen for consistency with \cite{Troeltzsch}.
To recast our previous problem into this notation, consider that we have - with the control-to-state operator $G$:
\begin{IEEEeqnarray*}{c}
\min_{u \in U_{ad}} \frac{1}{2} \| Su - y_d \|_{L^2}^2 + \frac{\lambda}{2} \| u \|_{L^2(\Omega)}^2 \\
G(u)(x) \leq y_b.
\end{IEEEeqnarray*}
Therefore we can recast this into the cone formulation by using the function $g(u)$ defined as
\[
	g(u) = G(u)(x) - y_b \leq 0.
\]
\end{document}