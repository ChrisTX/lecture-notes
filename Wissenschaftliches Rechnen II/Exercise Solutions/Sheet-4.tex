\documentclass[oneside,a4paper]{amsart}

\usepackage{geometry}
\usepackage{parskip}
\usepackage{mathtools}
\usepackage{enumerate}
\usepackage[dvipsnames]{xcolor}
\usepackage{IEEEtrantools}

\DeclareMathOperator{\dive}{div}

\begin{document}
\title{Wissenschaftliches Rechnen II - Exercise Sheet 2 solutions}
\maketitle{}
\section*{Exercise 18}
\subsection*{(a)}
Let
\[
	E(u) \coloneqq \int_\Omega |\nabla u|^2 + \int_{\partial \Omega} |u|^2 - 2  \int_\Omega f u.
\]
Obviously:
\[
	E(u) \geq \| u \| - F(u),
\]
thus, we're coercive. Furthermore, $E(u)$ is strictly convex, giving us the uniqueness of solutions.
By Lax-Milgram, we have
\[
	\| u \| \leq \underbrace{ \frac{1}{\alpha} }_{= 1} \| f \|.
\]
\section*{Exercise 19}
\subsection*{(a)}
Fix $\varphi \in C_C^\infty(T^+ \cup T^-, \mathbb{R}^d)$.
\[
	\int_{T^+\cup T^-} \varphi \cdot \nabla u + \dive(\varphi) u = \int_{T^+ \cup T^-} \dive (pu) = \int_{\partial T^+ \cup T^-} (\varphi \cdot \nu) u = 0, \quad \text{(because $\varphi \in C_C^\infty$)}.
\]
Furthermore,
\[
\int_{T^+ \cup T^-} \varphi \cdot \nabla u = -\int_{T^+ \cup T^-} \dive(\varphi) u.
\]
However,
\[
\int_{T^+ \cup T^-} \dive (pu) = \int_{T^+} \dive (pu) + \int_{T^-} \dive (pu) = \int_{\partial T^+} (\varphi \cdot \nu) u_+ - u_-.
\]
If we pick $\varphi = \eta \nu$ (if $\eta \in C_c^\infty(T^+ \cup T^-)$) we get
\[
\int_{\partial T^+} (\varphi \cdot \nu) u_+ - u_- = \int_F \eta(u_+ - u_-),
\]
implying
\[
	u_+ = u_- \quad \text{in } F.
\]

See also Ern/Di Pietro, Lemma 1.23.
\subsection*{(b)}
Ern/Di Pietro, Lemma 1.24.
\section*{Exercise 20}
Let
\[
	x = \sum_{i=1}^m x_i \xi_i, \quad y = \sum_{k = 1}^n y_k \eta_k
\]
Then,
\begin{IEEEeqnarray*}{rCl}
b(x, y) &=& b\left(\sum x_i \xi_i, \sum y_k \eta_k\right) \\
&=& \sum x_i y_i b(\xi_i, \eta_k).
\end{IEEEeqnarray*}
We have $B \in \mathbb{R}^m \otimes \mathbb{R}^n$ by the exercise. Furthermore there exists (singular value decomposition) $U \in O(n)$, $V \in O(n)$, and $\Sigma$ is diagonal, s.t.
\[
	B = U \Sigma V.
\]
Observation:
\begin{IEEEeqnarray*}{rCl}
\beta &\coloneqq& \inf_{X^*} \sup_{Y^*} \frac{\overbrace{ b(x, y) }^{= x \cdot B \cdot y}}{\underbrace{\| x \|}_{= Vx} \underbrace{\| y \|}_{= Uy}}.
\end{IEEEeqnarray*}
W.l.o.g.: $m \leq n$, $\lambda_1 \leq \lambda_2 \leq \ldots \leq \lambda_n$.
Then, any $y$ attaining a supremum will be also attained by $y = (y_1, \ldots, y_n, 0, \ldots, 0)$.
Let
\[
	y = \left( \frac{\lambda x}{|\lambda x|}, 0 \right).
\]
Taking $\sup$: $|\lambda \cdot x$, $x > l_1$, $\beta = \lambda_1$.
\section*{Exercise 17}
Let $H = (H, (\cdot, \cdot))$, $P \in \mathcal{L}(H, H)$ with $P^2 = P$, $\ker P \neq \{ 0\}$, $\operatorname{im} P \neq H$.

Both $P$ and $1 - P$ are oblique projections. Furthermore, $\| P \|$, $\| 1 - P \| \geq 1$, as
\begin{IEEEeqnarray*}{rCl}
	\| P \| &=& \sup_{\| v \|_H \leq 1} \frac{\| P v \|_H}{\| v \|} \\
	&=& \sup_{\| v \|_H = 1} \| P^2 v \| \\
	&\leq& \sup_{\| v \|_H = 1} \|P \| \|Pv \|. 
\end{IEEEeqnarray*}
Thus, $1 \leq \|P \|$.
Define
\[
	T \coloneqq 1 - P.
\]
Then, $T + P = \operatorname{id}$, thus $T(H) + P(H) = H$, implying $\ker (P) + \ker(T) = H$.
This is because $T(H) \subseteq \ker(P)$, as $P(1 - P) = 0$ and $(1 - P)(P) = 0$. Furthermore, the sum is direct:
\begin{IEEEeqnarray*}{r"rCl}
	& 0 &=& \overbrace{ T(a) }^{= (1-P)} + P(b) \\
	\Longrightarrow & P(a) &=& a + P(b) \\
	& P(a) &=& P(a) + P(b) \\
	& P(b) &=& 0
\end{IEEEeqnarray*}
\end{document}