\documentclass[a4paper]{amsart}

\usepackage{geometry}
\usepackage{parskip}
\usepackage{mathtools}
\usepackage{enumerate}
\usepackage{IEEEtrantools}

\DeclareMathOperator{\im}{Im}

\begin{document}
\title{Wissenschaftliches Rechnen II - Exercise Sheet 2 solutions}
\maketitle{}
\section*{Exercise 6}
\begin{enumerate}[(a)]
\item $\im A_1$ is closed in W'. Let $f \in \im A \subseteq W'$. Then, $\exists v_f \in V$ with $A_1 (v_f) = b$.
\begin{IEEEeqnarray*}{rCl}
	0 < \alpha &=& \inf_{v \in V_0} \sup_{w \in W_0} \frac{(A_1(v), w)}{\| v \|_V \| w \|_W} \\
	&\leq& \sup_{w \in W_0} \frac{(A_1(v_f), W)}{\| v_f \|_V \| w \|_W} \\
	&=& \sup_{w \in W_0} \frac{f(w)}{\| v_f \|_V \| w \|_W}.
\end{IEEEeqnarray*}
Thus, $\alpha \| v_f \|_V \leq \| f \|_{W'}$.
Since $A_1 \in \mathcal{L}(V; W')$ and $\exists \alpha > 0 \; \forall f \in W', \; v_f \in V$ such that $A_1 v_f = f$ and $\alpha \| v_f \|_V \leq \| f \|_{W'}$. By Lemma 1.3.5 $\mathcal{R}(A_1)$ is closed in W'.
\item Show $\im(A_1) = W$.
From (BNB2), $A_2 = A_1'$, since $w_1, w_2 \in W$ with $A_2(w_1) = A_2(w_2)$ is equivalent to $a(v, w_1) = a(v, w_2) \; \forall v \in V$, which gives $a(v, w_1 - w_2) = 0 \; \forall v \in V$. By BNB, $w_1 = w_2$.

\textit{Theorem:} $a$ closed image implies $W' = 0^\perp = (\ker A_2)^\perp = \im A_1$.
\item $A_1$ is an isomorphism. $\im(A_1)$ is closed from (a). By Lemma 1.3.5,
\[
	\exists \alpha > 0 \, \forall w' \in W' \, \exists v_{w'} \in V \, \forall A_1 v_{w'} = w' \text{ and } \alpha \| v_{w'} \|_V \leq \| w' \|_{W'}.
\]
By Theorem 1.3.8, $A_1$ is an isomorphism.
Then, $\| A_1^{-1} F \|_V \leq \alpha^{-1} \| F \|_{W'}$.
\item $A_2$ is an isomorphism. $A_2 : W \to V'$. We know $A_2$ can be identified with $A_1'$ (see (b)), and $\im (A_2) = (\ker A_1)^\perp = 0^\perp = V'$.
To show: $\| A_1^{-1} \|_{\mathcal{L}(W'; V)} = \alpha^{-1} = \| A_2^{-1} \|_{\mathcal{L}(V'; W)}$.
\[
\sup_{w' \in W'_0} \frac{\| A_1^{-1}, W' \|_V}{\| F \|_{W'}} \leq \alpha^{-1} \leq \| A_1^{-1} \|_{\mathcal{L}(W', V)}.
\]
This implies the left inequality. The right one can be proven analogously.
\item Follows directly from Lemma 1.3.7.
\end{enumerate}

\section*{Exercise 7}
\textbf{$\Rightarrow\quad$} Let $u\in H$ solution of $a(u,v) = b(v),\,\forall v\in H$. Then we have
	\begin{eqnarray*}
		\forall v\in V:&J(v) &= J(u) + a(u,v-u) - l(v-u) + \frac{1}{2} a(v-u,v-u)\\	
						    &&= J(u) + \frac{1}{2} a(v-u,v-u)\\
						    &&\geq J(u)
	\end{eqnarray*}

	and we see that $u$ minimizes the functional $J$ on $H$.\newline\newline\noindent
	\textbf{$\Leftarrow\quad$ :} Let now $v\in V,\,\lambda \in (0,1]$ and set $w = u + \lambda v$. Calculation gives:
	\[
		J(w) - J(u) = \lambda\left[a(u,v) - l(v)\right] + \frac{\lambda^2}{2}a(v,v) \geq 0\text{ because }u\text{ minimizes }J 
	\]
	Dividing by $\lambda$ gives 
	\[
		a(u,v)-l(v) + \frac{\lambda}{2} a(v,v) \geq 0
	\]
	Let now $\lambda \to 0$:
	\[
		a(u,v)\geq l(v),\quad\forall v\in V
	\]
	This can also be written as
	\begin{eqnarray*}
		a(u,-v) \geq l(-v)\\
		-a(u,v) \geq -l(v) \Leftrightarrow a(u,v) \leq l(v)
	\end{eqnarray*}
	So in the end
	\[
		a(u,v) = l(v),\quad\forall v\in H
	\]
	which means that $u$ is a solution of the problem.


\section*{Exercise 8}
To show: $\| \nabla (u - u_h)_{L^2(\Omega)}^2 =\| \nabla u \|_{L^2(\Omega)}^2 - \| \nabla u_h \|_{L^2(\Omega)}^2$.
Use the Galerkin-orthogonality.
\[
	E(u - u_h, v_h) = 0, \quad \forall v_h \in V_h.
\]
Then,
\begin{IEEEeqnarray*}{rCl}
\| \nabla u - u_h \|_{L^2}^2 &=& a(u - u_h, u - u_h) \\
&=& a(u - u_h, u) \\
&=& \| \nabla u \|_{L^2}^2 - a(u_h, u) \\
&=& \| \nabla u \|_{L^2}^2 - \| \nabla u_h \|_{L^2}^2.
\end{IEEEeqnarray*}

\section*{Exercise 9}
We can calculate the nodal basis-functions on the triangle given through the nodes $v_0=(0,0), v_1=(1,0), v_2=(0,1)$ by using the condition $\delta_i(v_j) = \delta_{i,j}$ on all 3 nodes. We obtain:
	\begin{eqnarray*}
		\delta_0(x,y) = -x -y +1\\
		\delta_1(x,y) = x\\
		\delta_2(x,y) = y
	\end{eqnarray*}
	We can directly compute the derivatives:
	\begin{eqnarray*}
		\nabla\delta_0(x,y) = \begin{bmatrix} -1&-1\end{bmatrix}\\
		\nabla\delta_1(x,y) = \begin{bmatrix} 1&0\end{bmatrix}\\
		\nabla\delta_2(x,y) = \begin{bmatrix} 0&1\end{bmatrix}
	\end{eqnarray*}
	Now we can just compute the matrices $S_T,T_T,C_T$ by using those expressions, together with a $\beta=[\beta_0,\beta_1]$. Therefore we use the integration rule
	\[
		\int_T f(x,y)\,d(x,y) = \int_0^1\int_0^{1-x} f(x,y)\,dy\,dx
	\]In the end we obtain
	\begin{eqnarray*}
		M_T = 	\begin{bmatrix}
					0.0833 & 0.0417 & 0.0417\\
					0.0417 & 0.0833 & 0.0417\\
					0.0417 & 0.0417 & 0.0833 
				\end{bmatrix}
				\\
		C_T = 	\frac{1}{6}\begin{bmatrix}
					-(\beta_0+\beta_1) & \beta_1 & \beta_1 \\
					-(\beta_0+\beta_1) & \beta_1 & \beta_2 \\
					-(\beta_0+\beta_1) & \beta_1 & \beta_2 \\
				\end{bmatrix}
				\\
		T_T = 	\begin{bmatrix}
					1 & -0.5 & -0.5\\
					-0.5 & 0.5 & 0\\
					-0.5 & 0 & 0.5
				\end{bmatrix}
	\end{eqnarray*}

\end{document}