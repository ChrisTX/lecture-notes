\documentclass[a4paper]{amsart}

\usepackage{geometry}
\usepackage{parskip}
\usepackage{mathtools}
\usepackage{enumerate}
\usepackage{IEEEtrantools}

\DeclareMathOperator{\im}{Im}

\begin{document}
\section*{Einfache MATLAB-Implementierung der $P_1$-FEM}

\emph{Modellproblem:}
\[
	-\Delta u = f,\quad u_{|\partial\Omega} = 0
\]
mit $f\in L^2(\Omega)$.\par
\emph{Schwache Formulierung:} Gesucht ist $u\in V \coloneqq H^1_0(\Omega)$, sodass
\[
	\forall v\in V:\quad a(u,v) = \int_\Omega fv\,dx
\]
wobei
\[
	a(v,w) \coloneqq \int_\Omega\nabla v\cdot\nabla w\,dx
\]
\emph{Diskretisierung:} Tirangulierung $\mathcal{T}$ von $\Omega$, und definiere $P_1(\mathcal{T}) = \{\text{stückweise affine Funktionen}\}$. Dann
\[
	V_h \coloneqq = P_1(\mathcal{T})\cap V = \{\text{stetige stückweise affine FUnktionen mit Nebenbedinungen}\}
\]
\emph{Diskretisiertes Problem (Galerkin-Approximation):} Suche $u_h\in\mathcal{V}_h$ mit 
\[
	\forall u_h\in V_h:\,a(u_h,v_h) = \int_\Omega f v_h\, dx
\]
\emph{Basis von $V_h$:} Nodale Basisfunktionen, definiert durchc
\[
	\mathcal{I}_y(x) = \delta_{x,y}\quad\text{für innere Knoten }(y,z)\in\left[\mathcal{N}(\Omega)\right]^2
\]
Dann gilt
\[
	u_h = \sum_{z\in\mathcal{N}(\Omega)} = x_z\lambda_z
\]
\emph{Diskretes Gleichungssystem}: $Sx = b$ wobei $S = \left(\int_\Omega \lambda_y\lambda_z\, dx\right)_{y,z\in\mathcal{N}(\Omega)}$ \emph{`Steifigkeitsmatrix'}, $b=\left(\int_\Omega f\lambda_z\right)_{z\in\mathcal{N}(\Omega)}$ (bspw. Approximation durch Quadriken)\par
Dann lässt sich $b$ darstellen durch die \emph{'Massenmatrix'} 
\[
	M\coloneqq \left(\int_\Omega \lambda_y\lambda_z\,dx\right)_{y,z\in\mathcal{N}}
\]
mit 
\[
	\tilde{b} = \left( M(f(x)) \right)_{x\in\mathcal{N}(\Omega)}
\]
\emph{Assemblierung der Matrizen:} Diskretisierung $\mathcal{T}$ ist beschrieben durch
\begin{eqnarray*}
	coords\in\mathbb{R}^{\# \mathcal{N}\times 2}\\
	elems\in\mathbb{R}^{\#\mathcal{T}\times 3}\\
	dirichlet\in\mathbb{R}^{\xi_D\times 2}
\end{eqnarray*}
Hierbei sind $\mathcal{N} = \text{Menge der Knoten},\,\mathcal{N}(\Omega) = \text{Menge der inneren knoten} = \mathcal{N}\setminus\Gamma_D$

Beispiel: Würfel mit Ecken $(0,0),(1,0),(1,1),(0,1)$ ergibt folgende Datenstruktur:
\begin{eqnarray*}
	coords &=& \begin{bmatrix} 0 & 0\\ 1 & 0 \\ 1 & 1 \\ 0 & 1\end{bmatrix} \\
	elems &=& \begin{bmatrix}3&1&2\\1&3&4\end{bmatrix}\\
	dirichlet &=& \begin{bmatrix}..\end{bmatrix} %FEHLT!
\end{eqnarray*}


Assemblierung von $S$ (Steifigkeitsmatrix) und $M$ durch lokale Matrizen. Zu $T\in\mathcal{T}$ definiere
\begin{eqnarray*}
		S_T &=& \left(\int_T\nabla\lambda_j\cdot\nabla\lambda_k\,dx\right)_{j,k = 1,2,3}\\
		M_T &=& \left(\int_T \lambda_j\lambda_k\,dx\right)_{j,k = 1,2,3}
\end{eqnarray*}
Wobei $T = conv\{P_1,P_2,P_3\}$ mit Notation $\lambda_j = \lambda_{P_j}$. Bekannt ist aus der Übung, dass
\[
	M_T = \frac{\vert T\vert}{12}\begin{bmatrix}2&1&1\\1&2&1\\1&1&2\end{bmatrix}
\]
Bekannt ist auch, dass $\lambda_1(x) + \lambda_2(x) + \lambda_3(x) = 1,\quad\forall x\in T$, sowie $\lambda_j(P_k) = \delta_{j,k}$, woraus folgt dass
\[
	\forall x\in T:\,\lambda_1(x)P_1 + \lambda_2(x)P_2 + \lambda_3(x)P_3 = x\text{ in baryzentrischen Koordinaten}
\]
Das führt auf die Gleichung
\[
	\underbrace{\begin{bmatrix} 1&1&1\\P_1&P_2&P_3\end{bmatrix}}{\coloneqq R}\begin{bmatrix}\lambda_1\\ \lambda_2\\ \lambda_3\end{bmatrix} = \begin{bmatrix} 1 \\ x_1 \\ x_2\end{bmatrix}
\]
Ableiten ergibt
\[
	R\,\begin{bmatrix}\nabla\lambda_1^T\\ \nabla\lambda_2^T\\ \nabla\lambda_3^T\end{bmatrix} = \begin{bmatrix} 0 & 0 \\ 1 & 0\\ 0 & 1 \end{bmatrix}
\]
c.f. Zeile 11 im Code:
\[
	grads\coloneqq\begin{bmatrix} \nabla\lambda_1^T\\ \nabla\lambda_2^T\\ \nabla\lambda_3^T\end{bmatrix} = R^{-1} \begin{bmatrix}0 & 0 \\ 0 & 1 \\ 0 & 1\end{bmatrix}
\]
und zudem
\[
	S_T = \vert T \vert\, grads \otimes grads\text{ dyadisches Produkt: }x\otimes y = xy^T
\]

\emph{Kommentare zum Matlab-Code:}\par
Input: $\mathcal{T}, f, u_D$ (Dirichlet-Daten), $N = \#\mathcal{N}$ (Anzahl der Knoten (auch Rand))\par
Schleife über alle $T\in\mathcal{T}$ (Zeilen 9-16)\par
Zeile 10: $\vert T\vert = \frac{1}{2} \det \begin{bmatrix}1 & 1 & 1\\ P_1 & P_2 & P_3\end{bmatrix}$ (Übung)\par
Zeile 11: s.o., Berechnung lokaler Gradienten\par
Zeile 12/13: $(P_1,P_2,P_3)$ haben globale Nummern $(\alpha_1,\alpha_2,\alpha_3)$\par
Udate: 
\[
	\underbrace{S([\alpha_1,\alpha_2,\alpha_3],[\alpha_1,\alpha_2,\alpha_3])}{3\times 3\text{ Matrix}} = S([\alpha_1,\alpha_2,\alpha_3],[\alpha_1,\alpha_2,\alpha_3]) + S_T
\]
Zeile 14/15: Analog für $M$\par
Zeile 19: $x = \text{Nullvektor der Länge }\#\mathcal{N}$ \par
Zeile 20: $x(\mathcal{N}_D) = \text{Auswertung der Randdaten}$\par
Aufspaltung $x = x_0 + x_D$ wobei $x_0 = 0$ für alle Dirichlet-Knoten, $x_D$ erfüllt Dirichlet-Bedingung\par
\[
	\Rightarrow Sx = S(x_0+x_D) \overbrace{!}{=} b \Leftrightarrow Sx_0 = (b-\varrho x_D)
\]
Zeile 22: Dies ergibt neue rechte Seite 
\[
	b = M \left[f(z)\right]_{z\in\mathcal{N}} - Sx_D
\]
Es ergibt sich also ein lineares Gleichungssystem zu  Nullrandbedingungen!\par
Zeile 24: $freeNodes \equiv \text{Vektor der Indices von den Innenknoten } \mathcal{N}(\Omega) \text{ enthält}$. Insbesondere ist die Untermatrix $S(freeNodes,freeNodes)$ invertierbar! Lösung des Systems erfolgt in Zeile 26.\par
Lösen von $Ax = b$ in MATLAB durch $x=A\,b$.\newline\newline\noindent
\emph{Bemerkungen: } 
Effizientere Implementierungen sind möglich durch
\begin{enumerate}
	\item Parallele Berechnung der lokalen Matrizen
	\item Effiziente Assemblierung der dünnbesetzten Matrizen
\end{enumerate}

Übungsblatt 3 befasst sich mit der folgenden Gleichung:
\[
	-\varepsilon\Delta u + \beta \nabla u = f
\]
Schwache Formulierung
\[
	\forall v\in V:\, \varepsilon(\nabla u,\nabla v)_L^2 + (\beta\varepsilon u,v)_{{L^2}} = (f,v)_{L^2}
\]
Wir benötigen also noch die ``Konvektionsmatrix''
\[
	C = \left(\int_\Omega (\beta\nabla\lambda_y)\lambda_z\, dx\right)_{y,z}
\]
Die lokale Konvektionsmatrix (auf einem Element) ist gegeben durch
\[
	C_T = \int_T(\underbrace{\beta\nabla\lambda_j}{konstant})\lambda_k\, dx)_{j,k = 1,2,3}
\]

\textbf{Bemerkung: } $\forall j\in\{1,2,3\}$ gilt $\int_T \lambda_j\,dx = \frac{1}{3}\vert T\vert$
\end{document}