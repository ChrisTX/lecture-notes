\documentclass[../skript.tex]{subfiles}

\begin{document}

% VL: 4/9/2015
\section{Charakterisierung bijektiver Operatoren} % 1.3
\label{sec:c1e3}
Seien $V$, $W$ (reelle) Banachräume, $A \in \mathcal{L}(V, W)$. Bezeichne $\ker(A)$ den Kern\slash{}Nullraum von $A$ und $\im(A)$ das Bild von $A$.
\begin{theoremact} % Thm 1.3.1
\label{thm:c1e3s1}
Der Quotientenraum $\sfrac{V}{\ker(A)}$ ist ein Banachraum, wenn er mit der Norm
\[
	\| \cdot \|_{\sfrac{V}{\ker(A)}} : \bar{v} \mapsto \inf_{v \in \bar{v}} \| v \|_{V}
\]
ausgestattet wird.
Der Operator
\begin{IEEEeqnarray*}{rCl}
	\bar{A} : \sfrac{V}{\ker(A)} &\to& \im A \\
	\bar{v} &\mapsto& Av \text{ für ein beliebiges } v \in \bar{v}
\end{IEEEeqnarray*}
ist ein Isomorphismus.
\end{theoremact}
\begin{proof}
Siehe \cite[S.~60]{Yosida}.
\end{proof}
Sei $M \subseteq V$. Definiere den \emph{Annihilator} von $M$ bzgl. $V'$ durch
\[
	M^\perp \coloneqq \{ v' \in V' \mid \forall m \in M \; \langle v', m \rangle_{V' \times V} = 0\}.
\]
Sei $N \subseteq V'$. Definiere den Annihilator von $N$ bzgl. $V$ durch
\[
	N^\perp \coloneqq \{ v \in V \mid \forall n \in N \; \langle n, v \rangle_{V' \times V} = 0\}.
\]
\begin{lemma}[Charakterisierung von \unboldmath$\ker A$ und $\im A$] % Lemma 1.3.2
\label{thm:c1e3s2}
Für $A \in \mathcal{L}(V, W)$ gilt
\begin{enumerate}[(a)]
\item $\ker A = (\im A')^\perp$
\item $\ker A' = (\im A)^\perp$
\item $\overline{\im A} = (\ker A')^\perp$
\item $\overline{\im A'} = (\ker A)^\perp$
\end{enumerate}
\end{lemma}
\begin{proof}
Siehe \cite[S.~202]{Yosida}.
\end{proof}
\begin{theoremact}[Satz vom abgeschlossenen Bild] % Thm 1.3.3
\label{thm:c1e3s3}
Sei $A \in \mathcal{L}(V, W)$. Die folgenden Aussagen sind äquivalent:
\begin{enumerate}[(a)]
\item $\im A$ abgeschlossen
\item $\im A'$ abgeschlossen
\item $\im A = (\ker A')^\perp$
\item $\im A' = (\ker A)^\perp$ 
\end{enumerate}
\end{theoremact}
\begin{proof}
Siehe \cite[S.~205]{Yosida}.
\end{proof}
\begin{theoremact}[Satz von der offenen Abbildung] % Thm 1.3.4
\label{thm:c1e3s4}
Falls $A \in \mathcal{L}(V, W)$ surjektiv und $U \subseteq V$ offen, dann ist $A(U)$ offen in $W$.
\end{theoremact}
\begin{proof}
Siehe \cite[S.~75]{Yosida}.
\end{proof}
\begin{lemma} % Lemma 1.3.5
\label{thm:c1e3s5}
Sei $A  \in \mathcal{L}(V, W)$. Die folgenden Aussagen sind äquivalent:
\begin{enumerate}[(a)]
\item \label[equation]{thm:c1e3s5-a} $\im A$ abgeschlossen
\item \label[equation]{thm:c1e3s5-b} $\exists \alpha > 0 \; \forall w \in \im A \: \exists v_w \in V \; : \; A v_w = w$ und $\alpha \| v_w \|_V \leq \| w \|_W$.
\end{enumerate}
\end{lemma}
\begin{proof}
\labelcref{thm:c1e3s5-a} $\Rightarrow$ \labelcref{thm:c1e3s5-b}: $\im A$ abgeschlossen impliziert, dass $\im A$ ein Banachraum ist.
Mit \cref{thm:c1e3s4} folgt: $A(B_V(0, 1))$ ist offen in $W$ ($B_V(0, 1)$ bezeichnet den offenen Einheitsball in $V$).
D.h., da $0 \in A(B_V(0, 1))$, dass
\[
	\exists \gamma > 0 \quad B_W(0, \gamma) \subseteq A(B_V(0, 1)).
\]
Sei $w \in \im A$. Dann ist
\[
	\frac{\gamma}{2} \frac{w}{\| w \|_W} \; \in B_W(0, \gamma) \subseteq A(B_V(0, 1))
\]
und
\[
	\exists z \in B_V(0, 1) \; : \; Az = \frac{\gamma}{2} \frac{w}{\| w \|_W}. 
\]
Setze
\[
	v_W \coloneqq \frac{2 \| w \|_W}{\gamma} z,
\]
denn damit ist
\begin{IEEEeqnarray*}{rCl}
	A v_W &=& \frac{2 \| w \|}{\gamma} A z \\
	&=& \frac{2 \| w \|}{\gamma} \frac{\gamma}{2} \frac{w}{\| w \|} = w.
\end{IEEEeqnarray*}
Dann gilt:
\[
	\| v_W \| \leq \frac{2}{\gamma} \| w \|_W \underbrace{ \| z \|_V }_{\leq 1}.
\]
Somit gilt \labelcref{thm:c1e3s5-b} mit der Wahl
\[
	\alpha = \frac{\gamma}{2}.
\]

\labelcref{thm:c1e3s5-b} $\Rightarrow$ \labelcref{thm:c1e3s5-a}: Sei $(w_n) \subset \im A$ mit
\[
	\lim_{n \to \infty} w_n = w \; \in W.
\]
\labelcref{thm:c1e3s5-b} impliziert dann
\[
	\exists (v_n) \subseteq V \; : \; A v_n = w_n \text{ und } \alpha \| v_n \| \leq \| w_n \|_W.
\]
Da $(v_n)$ eine Cauchy-Folge ist, gilt
\[
	v_n \xrightarrow{n \to \infty} v \; \in V
\]
und aufgrund der Stetigkeit folgt $A v = w$, und somit $w \in \im A$.
\end{proof}
\begin{remark} % Bem 1.3.6
\label{bem:c1e3s6}
\cref{thm:c1e3s5} impliziert, dass
\[
	A \in \mathcal{L}(V, W) \text{ bijektiv} \; \Rightarrow \; A^{-1} \in \mathcal{L}(W, V). 
\]
\end{remark}
\begin{proof}
Setze $v_w = A^{-1} w$ in \cref{thm:c1e3s5}. Dann gilt
\[
	\sup_{w \in W \setminus \{ 0 \}} \frac{\| A^{-1} w \|_V}{\| w \|_W} \leq \frac{1}{\alpha} \| w \|_W.
\]
\end{proof}
\begin{lemma}[Charakterisierung surjektiver Operatoren] % Lemma 1.3.7
\label{thm:c1e3s7}
Sei $A \in \mathcal{L}(V, W)$. Die folgenden Aussagen sind äquivalent:
\begin{enumerate}
\item $A'$ ist surjektiv
\item $A$ ist injektiv und $\im A$ abgeschlossen in $W$
\item $\exists \alpha > 0 \; \forall v \in V \; : \; \| A v \| \geq \alpha \| v \|_V$
\item $\exists \alpha > 0 \; : \; \inf_{v \in V} \sup_{w' \in W'} \frac{\langle w', Av \rangle_{W' \times W}}{\| w' \|_{W'} \| v \|_{V}} \geq \alpha$
\end{enumerate}
Ferner sind folgenden Aussagen äquivalent
\begin{enumerate}
\item $A$ ist surjektiv
\item $A'$ ist injektiv und $\im A'$ abgeschlossen in $V'$
\item $\exists \alpha > 0 \; \forall w' \in W' \; : \; \| A' w' \| \geq \alpha \| w' \|_{W'}$
\item $\exists \alpha > 0 \; : \; \inf_{w' \in W'} \sup_{v \in V} \frac{\langle A' w, v \rangle_{V' \times V}}{\| w' \|_{W'} \| v \|_V} \geq \alpha$
\end{enumerate}
\end{lemma}
\begin{proof}
Folgt aus \cref{thm:c1e3s3,thm:c1e3s5}. Übung!
\end{proof}
\begin{theoremact}[Charakterisierung bijektiver Operatoren] % Thm 1.3.8
\label{thm:c1e3s8}
Sei $A \in \mathcal{L}(V, W)$. Dann sind äquivalent:
\begin{enumerate}[(a)]
\item $A$ ist bijektiv
\item $A' \in \mathcal{L}(W', V')$ ist injektiv und $\exists \alpha > 0 \; \forall v \in V \; : \; \| A v \|_W \geq \alpha \| v \|_V$.
\end{enumerate}
\end{theoremact}
\begin{proof}
``$\Rightarrow$'': $A$ surjektiv impliziert nach \cref{thm:c1e3s2}. Damit
\[
	\ker(A') \overset{\text{\cref{thm:c1e3s2}}}{=} (\im A)^\perp \overset{A \text{ surjektiv}}{=} \{ 0 \}.
\]
Damit ist $A'$ injektiv.
Da $\im A = W$ und somit abgeschlossen in $W$ is und $A$ injektiv ist folgt mit \cref{thm:c1e3s7}:
\[
	\exists \alpha > 0 \; \forall v \in V \; : \; \| A v \| \geq \alpha \| v \|.
\]
``$\Leftarrow$'': $A'$ injektiv impliziert
\[
	\overline{\im A} = (\ker A')^\perp = W,
\]
d.h. $\im A$ liegt dicht in $W$.

\underline{Zu zeigen:} $\im A$ ist abgeschlossen.
Sei $(v_n) \subseteq V \; : \; (A v_n) \subseteq W$ eine Cauchy-Folge.
Man beachte nun, dass
{
\crefformat{eqstar}{#2($\star$)#3}
\addtocounter{equation}{-1}
\def\theequation{$\star$}
\begin{IEEEeqnarray*}{r"rCl}
& \forall w' \in W' \; \left(\forall v \in V \; : \; \langle w', A v \rangle_{W' \times W} = 0 \right) &\Rightarrow& w' = 0 \IEEEyesnumber \label[eqstar]{eq:c1e3s8-star} \\
\Leftrightarrow & \forall w' \in W' \setminus \{ 0 \} \; \exists v \in V \; : \; \langle w', Av \rangle_{W' \times W} &>& 0 \\
\Leftrightarrow & \inf_{w' \in W' \setminus \{ 0 \}} \sup_{v \in V} \frac{\langle w', Av \rangle_{W' \times W}}{\| w' \| \| v \|} &>& 0.
\end{IEEEeqnarray*}
} \par
Aus \cref{eq:c1e3s8-star} folgt dass $(v_n)$ auch eine Cauchy-Folge in $V$ ist, mit einem Grenzwert $v \in V$. Aufgrund der Stetigkeit von $A$ folgt daraus:
\[
	Av_n \to Av \; \in  A.
\]
Damit ist $A$ surjektiv. Injektivität folgt aus der Voraussetzung.
\end{proof}
\begin{corollary} % Kor 1.3.9
\label{thm:c1e3s9}
Sei $A \in \mathcal{L}(V, W)$. Die folgenden Aussagen sind äquivalent:
\begin{enumerate}[(a)]
\item $A$ ist bijektiv
\item $\exists \alpha > 0 \; \forall v \in V \; : \; \| A v \|_W \geq \alpha \| v \|$ und $\forall w' \in W' \; : \; (A' w' = 0 \Rightarrow w' = 0)$
\item $\exists \alpha > 0 \; : \; \inf_{v \in V \setminus \{ 0\}} \sup_{w' \in W' \setminus \{ 0\}} \frac{\langle w', Av \rangle_{W' \times W}}{\| w' \|_{W'} \| v \|_V} \geq \alpha$
\end{enumerate}
\end{corollary}
\end{document}