\documentclass[../skript.tex]{subfiles}

\begin{document}

% VL: 4/7/2015
\chapter{Einige funktionalanalytische Grundlagen} % 1.
\label{sec:c1}
\textbf{Literatur:} \cref{sec:c1} basiert im Wesentlichen auf \cite{ErnGuermond}.
\begin{remarknonumb}
Wir beschränken uns in diesem Kapitel auf reelle Vektorräume.
\end{remarknonumb}
\section{Banachräume\slash{}lineare Operatoren} % 1.1
\label{sec:c1e1}
\begin{definition}[Banachraum] % Def 1.1.1
\label{def:c1e1s1}
Ein normierter Vektorraum $(V, \| \cdot \|_V)$ heißt \\ \emph{Banachraum}, falls er vollständig ist.
\end{definition}
\begin{definition}[Lineare Operatoren und der Raum \unboldmath$\mathcal{L}(V, W)$] % Def 1.1.2
\label{def:c1e1s2}
Eine \\ stetige, lineare Abbildung $A : V \to W$ zwischen normierten Vektorräumen $V$ und $W$ heißt \emph{linearer Operator}.
$\mathcal{L}(V, W)$ ist der \emph{Raum aller linearen Operatoren von $V$ nach $W$}.
\end{definition}
\begin{proposition} % Prop 1.1.3
\label{prop:c1e1s3}
Sei $(V, \| \cdot \|_V)$ normierter Vektorraum und $(W, \| \cdot \|_W)$ ein Banachraum. Dann ist $(\mathcal{L}(V, W), \| \cdot \|_{\mathcal{L}(V, W)})$ ein Banachraum, wobei
\begin{IEEEeqnarray*}{rCl}
\| \cdot \|_{\mathcal{L}(V, W)} : \mathcal{L}(V, W) &\to& \R_{\geq 0}, \\
A &\mapsto& \sup_{v \in V \setminus \{ 0 \}} \frac{\| A v \|_{W}}{\| v \|}.
\end{IEEEeqnarray*}
\end{proposition}
\begin{proof}
Übung.
\end{proof}
\begin{definition}[kompakter Operator] % Def 1.1.4
\label{def:c1e1s4}
Seien $V$, $W$ Banachräume. Ein Operator $A \in \mathcal{L}(V, W)$ heißt \emph{kompakter Operator}, falls es zu jeder beschränkten Folge $(v_n)_n \subseteq V$ eine Teilfolge $(v_{n_k})_k \subseteq (v_n)_n$ gibt, sodass $(A v_{n_k})_k \subseteq W$ konvergiert.
\end{definition}
\section{Lineare Funktionale und Dualität} % 1.2
\label{sec:c1e2}
\begin{definition}[lineares Funktional, Dualraum] % Def 1.2.1
\label{def:c1e2s1}
Der \emph{Dualraum} eines normierten Vektorraums $V$ ist $\mathcal{L}(V, \R)$ und wird mit $V'$ bezeichnet. Die Elemente von $V'$ heißen lineare Funktionale oder Linearformen. Die Auswertung eines $f \in V'$ in einem Element $v \in V$ wird durch Dualitätsklammern $\langle \cdot, \cdot \rangle_{V' \times V}$ ausgedrückt, d.h. $\langle f, v \rangle_{V' \times V} \coloneqq f(v)$.
\end{definition}
\begin{remark} % Bem 1.2.2
\label{bem:c1e2s2}
\cref{prop:c1e1s3} zeigt, dass $\mathcal{L}(V, \R)$ ein Banachraum ist, wenn er mit der Norm
\begin{IEEEeqnarray*}{rCl}
\| \cdot \|_{\mathcal{L}(V, \R)} : \mathcal{L}(V, \R) &\to& \R_{\geq 0}, \\
f &\mapsto& \sup_{v \in V \setminus \{ 0 \}} \frac{|\langle f, v \rangle_{V' \times V}|}{\| v \|_V}.
\end{IEEEeqnarray*}
ausgestattet ist.
Das Hahn-Banach Theorem impliziert, dass
\[
	\| v \|_V = \sup_{\substack{f \in V' \\ \| f \|_{V'} = 1}} | \langle f, v \rangle_{V' \times V} | = \max_{\substack{f \in V' \\ \| f \|_{V'} = 1}} | \langle f, v \rangle_{V' \times V} |.
\]
\end{remark}
\begin{definition}[dualer Operator] % Def 1.2.3
\label{def:c1e2s3}
Seien $V$, $W$ normierte Vektorräume, \\ $A \in \mathcal{L}(V, W)$. Der \emph{duale (adjungierte) Operator} $A' : W' \to V'$ ist definiert durch
\[
	\forall w' \in W', \; \forall v \in V \; : \; \langle A' w', v \rangle_{V' \times V} = \langle w', A v \rangle_{W' \times W}.
\]
\end{definition}
\begin{definition}[Raum der stetigen Bilinearformen] % Def 1.2.4
\label{def:c1e2s4}
Es seinen $V_1$, $V_2$ normierte Vektorräume. Der \emph{Vektorraum der stetigen Bilinearformen} wird mit $\mathcal{L}(V_1 \times V_2, \R)$ bezeichnet. $\mathcal{L}(V_1 \times V_2, \R)$ ist ein Banachraum, wenn er mit der Norm
\begin{IEEEeqnarray*}{rCl}
\| \cdot \|_{V_1 \times V_2} : \mathcal{L}(V_1 \times V_2, \R) &\to& \R_{\geq 0}, \\
a &\mapsto& \sup_{v, w \neq 0} \frac{a(v, w)}{\| v \|_{V_1} \| w \|_{V_2}}
\end{IEEEeqnarray*}
ausgestattet ist. \emph{Stetigkeit} bedeutet hier, dass
\[
	\exists c > 0 \forall v_1 \in V_1 \; v_2 \in V_2 \; : \; a(v_1, v_2) \leq c \| v_1 \|_{V_1} \| v_2 \|_{V_2}.
\]
\end{definition}
\begin{proposition} % Prop 1.2.5
\label{prop:c1e2s5}
Seien $V_1$, $V_2$ Banachräume, $a \in \mathcal{L}(V_1 \times V_2, \R)$. Definiere (``zugehörigen'') Operator $A : V_1 \to V_2$ durch
\[
	\forall v_1 \in V_1 \; \forall v_2 \in V_2 \quad \langle A v_1, v_2 \rangle_{V_2' \times V_2} = a(v_1, v_2)
\]
($A : V_1 \ni v_1 \mapsto a(v_1, \cdot)$).
Dann gilt $A \in \mathcal{L}(V_1, V_2')$ und $\| A \|_{\mathcal{L}(V_1, V_2')} = \| a \|_{V_1 \times V_2}$.
\end{proposition}
\begin{definition}[Bidualraum] % Def 1.2.6
\label{def:c1e2s6}
Der \emph{Bidualraum} eines Banachraums $V$ ist der Dualraum von $V'$ und wird mit $V''$ bezeichnet.
\end{definition}
\begin{remark} % Bem 1.2.7
\label{bem:c1e2s7}
\cref{prop:c1e1s3} zeigt, dass $V''$ ein Bidualraum ist.
\end{remark}
\begin{proposition} % Prop 1.2.8
\label{prop:c1e2s8}
Sei $V$ ein Banachraum. Definiere $J_V : V \to V''$ durch
\[
	\forall u \in V, \; \forall v' \in V' \quad \langle J_V u, v' \rangle_{V'' \times V'} = \langle v', u \rangle_{V' \times V} = v'(u).
\]
$J_V$ ist eine Isometrie, d.h.
\[
	\| J_V u \|_{V''} = \| u \|_{V} \quad \text{für alle } u \in V.
\]
\end{proposition}
\begin{remark} % Bem 1.2.9
\label{bem:c1e2s9}
\begin{enumerate}
\item $J_V$ Isometrie impliziert $J_V$ injektiv. Damit kann $V$ mit dem Unterraum $J_V(V) \subseteq V''$ identifiziert werden.
\item $J_V$ ist im Allgemeinen nicht surjektiv. \par 
\underline{Gegenbeispiel:} $V = L^1(\Omega)$, $V' = L^\infty(\Omega)$, aber $L^1(\Omega) \subsetneq L^\infty(\Omega)'$.
\end{enumerate}
\end{remark}
\begin{definition}[reflexiver Banachraum] % Def 1.2.10
\label{def:c1e2s10}
Ein Banachraum $V$ heißt \emph{reflexiv}, falls $J_V$ ein Isomorphismus ist.
\end{definition}
\end{document}