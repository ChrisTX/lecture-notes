\documentclass[../skript.tex]{subfiles}

\begin{document}

\begin{proof}[Skizze]
\begin{itemize}
\item $d=1$: Benutze die Lagrange-Basis-Darstellung
	\[
		p(\hat{x}) = \sum_{z\in\hat{\Sigma}_k}u_z\prod_{y\in\hat{\Sigma}_k\setminus\{z\}}\left(\frac{\hat{x}-y}{z-y}\right).
	\]
\item $d=2$: 
	\begin{description}
		\item[Fall $k=1$:] Seien $\hat{z}_j$ die Knoten von $\hat{T}$ (Referenzelement), $0\leq 1\leq d$. Definiere \emph{baryzentrische Koordinaten} über
		\begin{IEEEeqnarray*}{rCl"l}
			\hat{\lambda}_0(\hat{x})&\coloneqq& 1-\sum_{j=1}^d\hat{x}_j,\\
			\hat{\lambda}_k(\hat{x})&\coloneqq& \hat{x}_k, & k=1, \ldots ,d.
		\end{IEEEeqnarray*}
		Mit dieser Definition gilt  die Lagrange-Eigenschaft:
		\[
			\hat{\lambda}_k(\hat{z}_j) = \delta_{k,j},\quad 0\leq k,j\leq d.
		\]
		Dann
		\[
			p(\hat{x}) = \hat{u}_{\hat{z}_0}\hat{\lambda}_0 + \hat{u}_{\hat{z}_1}\hat{\lambda}_1 + \hat{u}_{\hat{z}_2}\hat{\lambda}_2.
		\]
		\item[Fall $k=2,3$:] Die Monome $b_\alpha(\hat{x}) = x^\alpha,\;\alpha\in\mathcal{M}_k$ bilden eine Basis von $\PP_k$. Wir müssen zeigen, dass es Koeffizienten $c_\alpha,\;\alpha\in\mathcal{M}_k$ gibt, so dass 
		\[
			\forall\alpha\in\mathcal{M}_k\;:\;\sum_{\beta\in\mathcal{M}_k} P_{\alpha,\beta}c_\beta = u_{z_\alpha}.
		\] 
		wobei $P_{\alpha,\beta}\coloneqq (z_\alpha)^\beta$. D.h.~wir müssen zeigen, dass $(P_{\alpha,\beta})_{\alpha,\beta\in\mathcal{M}_k}\in\R^{|\mathcal{M}_k|\times|\mathcal{M}_k|}$. Für $k=2,3$ kann das explizit nachgeprüft werden.
		\item[Fall $k=4$:] Wir zeigen, dass der Kern der Matrix $( P_{\alpha,\beta})_{\alpha,\beta\in\mathcal{M}_k}$ invertierbar ist. Angenommen es sei $p\in\PP_k$ mit $p(z) = 0$ für alle $z\in\hat{\Sigma}_k$, also im Kern der Matrix. Dann gilt insbesondere
		\[
			p|_{\partial\hat{T}} = 0,
		\]
		d.h.~wir haben auf den Kanten des Referenzelements $\hat{T}$ ein 1-dimensionales Polynom $(k-1)$-ten Grades. Daher gibt es $\Psi\in\PP_{k-3}$:
		\[
			p = \hat{\lambda}_0\hat{\lambda}_1\hat{\lambda}_2\Psi.
		\]
		Wobei $\hat{\lambda}_i$ per Definition auf der $i$-ten Kante des Referenzelements verschwindet. Induktionsannahme zeigt, dass $\Psi\equiv 0$.
	\end{description}
\end{itemize}
\end{proof}

\begin{corollary}\label{cor:c2e3s7}
	Sei $k\in\mathbb{N}$ und $\Phi$ eine affine Transformation. Definiere 
	\[
		\Sigma_k \coloneqq \Phi\left(\hat{\Sigma}_k\right).
	\]
	Dann gilt
	\[
		\forall u=(u_z)_{z\in\Sigma_k} \; \exists ! p \in \PP_k\;:\;\forall z\in\Sigma_k\,:\,p(z)=u_z.
	\]
\end{corollary}
\begin{proof}
	Sei $q\in\PP_k$ das eindeutige Polynom, dass auf $\hat{\Sigma}_k$ interpoliert. Dann erfüllt
	\[
		p = q\circ\Phi^{-1}
	\] 
	die Behauptung. 
\end{proof}

\begin{definition}[Finite-Elemente-Räume]\label{def:c2e3s8}
	Sei $\tau$ eine reguläre Triangulierung von $\Omega$ und $k\in\mathbb{N}_0$. Definiere die Funktionenräume
	\begin{IEEEeqnarray*}{rClClCl}
		P^{k,-1}(\tau) &\coloneqq& \{p:\bar{\Omega}\to\R &\mid& \forall T\in\tau: p|_T\in\PP_k\},\\
		P^{k,0}(\tau) &\coloneqq& \{p\in P^{k,-1} &\mid& \, p \text{ stetig auf } \bar{\Omega}\} &=& P^{k,-1}\cap C^0(\bar{\Omega}),\\
		P^{k,0}_D(\tau) &\coloneqq& \{p\in P^{k,0}(\tau) &\mid& \, p|_{\Gamma_D} = 0\}.
	\end{IEEEeqnarray*}
	$P^{k,-1}$ sind dabei unstetige FE-Räume (auch discontinuous Galerkin (dG)-Räume genannt).
	$P^{k,0}$ und $P^{k,0}_D$ sind \emph{stetige FE-Räume}, und $\Gamma_D\subseteq\partial\Omega$ bezeichnet den Dirichlet-Teil des Randes. Die stetigen FE-Räume werden auch als \emph{$H^1(\Omega)$-konforme} Räume genannt. Unstetige FE-Räume werden als \emph{nicht-$H^1(\Omega)$-konforme} Räume bezeichnet.
\end{definition}

\begin{theorem}\label{thm:c2e3s9}
	Sei $\tau$ eine reguläre Triangulierung, $k\in\mathbb{N}, T_1,T_2\in\tau$ mit $T_1\cap T_2\not=\emptyset$. Seien ferner $p_1,p_2\in\PP_k$  zwei Polynome. Definiere 
	\begin{IEEEeqnarray*}{rCl}
		\varphi: T_1\cup T_2&\to&\R\\
		\varphi|_{T_1} &=& p_1\\
		\varphi|_{T_2} &=& p_2
	\end{IEEEeqnarray*}
	d.h.~$\varphi$ ist ein Element aus $P^{k,-1}(\{T_1,T_2\})$. Seien $\Phi_{T_1},\Phi_{T_2}$ affine Diffeomorphismen, die das Referenzelement $\hat{T}$ auf $T_1$ bzw. $T_2$ abbilden. Seien 
	\begin{IEEEeqnarray*}{rCl}
		\Sigma_k(T_1)&\coloneqq&\Phi_{T_1}\left(\hat{\Sigma}_k\right),\\
		\Sigma_k(T_2)&\coloneqq&\Phi_{T_2}\left(\hat{\Sigma}_k\right).		
	\end{IEEEeqnarray*} 
	Dann gilt
	\[
		\varphi\text{ ist stetig, also } \varphi\in P^{k,0}(\{T_1,T_2\}) \quad
			\Longleftrightarrow \quad
		\forall z\in\Sigma_k(T_1)\cap\Sigma_k(T_2):\,p_1(z) = p_2(z),
	\]
	(d.h.~$\varphi$ ist genau dann stetig ($d=2$: auf der Schnittkante), wenn die Funktionswerte von $p_1$ und $p_2$ in allen Schnittpunkten der Elemente ($d=2$: auf der Kante) übereinstimmen).  
\end{theorem}
\begin{proof}
	Folgt aus \cref{thm:c2e3s6}.
\end{proof}

\begin{corollary}
\label{cor:c2e3s10}
	Sei
	\[
		\Sigma_k(\tau) \coloneqq \bigcup_{T\in\tau} \Sigma_k(T) = \bigcup_{T\in\tau}\Phi_T(\hat{\Sigma}_k)
	\]
	die Menge der Knoten von $\tau$ bzgl.~des Polynomgrades $k$ ($k$-ter Ordnung). Eine FE-Funktion $u_\varphi\in P^{k,0}(\tau)$ ist durch ihre Werte in $\Sigma_k(\tau)$ eindeutig bestimmt. 
\end{corollary}
\begin{remarknonumb}
	Wir haben folgende Äquivalenz: 
	\[
		P^{k,0}(\tau) \cong \R^{|\Sigma_k(\tau)|}.
	\]
\end{remarknonumb}
\begin{definition}[Nodale Basis\slash{}Lagrange-Basis von $P^{k,0}$]\label{def:c2e3s11}
	Das eindeutig \\ bestimmte Element $b_z\in P^{k,0}(\tau)$ mit $b_z(y) = \delta_{z,y}$ heißt \emph{nodale Basisfunktion} zum Knoten $z\in\Sigma_k(\tau)$. Die Menge $\{b_z \mid z\in\Sigma_k(\tau)\}$ ist eine Basis von $P^{k,0}(\tau)$, die so genannte \emph{nodale Basis}.
\end{definition} 
\begin{remark}\label{rem:c2e3s12}
	\begin{enumerate}[(a)]
		\item Jedes $u\in P^{k,0}(\tau)$ hat eine eindeutige Basisdarstellung $u(x) = \sum_{z\in\Sigma_k(\tau)} u_z b_z$ wobei $u_z = u(z)$.
		\item Der Träger von $b_z$ ist lokal: $\supp b_z\subseteq \bigcup\{T\in\tau \mid z\in T\}$.
		\item Die Wahl der Basis ist nicht eindeutig! Die richtige Wahl der Basis entscheidet (!) über Erfolg und Misserfolg der Methode.
	\end{enumerate}
 \end{remark}


 \section{Galerkin-Finite-Elemente-Methode}\label{sec:c2e4}
Gemäß \cref{sec:c1e6} lautet die Galerkin-FEM für die Approximation des Variationsproblems \cref{prb:P} wie folgt:
\begin{problem}\label{prb:c2e4s1}
	Finde $u_h\in P^{k,0}_D(\tau_h)$, sodass
	\[
		\forall v_h\in P^{k,0}_D\;:\;a(u_h,v_h) = F(v_h),
	\]
	wobei
	\begin{itemize}
		\item $\Omega$ mit
		\begin{IEEEeqnarray*}{rCl}
			\Gamma_N &\subseteq& \partial \Omega, \\
			\Gamma_D &\subseteq& \partial\Omega, \\
			\Gamma_D \cup \Gamma_N &=& \partial\Omega, \\
			\Gamma_D\cap\Gamma_N &=& \emptyset, \\
			|\Gamma_D| &>& 0.
		\end{IEEEeqnarray*}
		\item $A,b,c,f,g$ wie in \cref{sec:c2e1} seien.
		\item $\tau_h$ eine reguläre Triangulierung von $\Omega$ ist, sodass $\Gamma_D$ eine Vereinigung von Kanten\slash{}Seiten von $\tau_h$ ist. 
		\item $h$ die globale Gitterweite ist.
		\item $P^{k,0}_D(\tau_h)$ der $H^1_D(\tau_h)$-konforme FE-Raum aus \cref{def:c2e3s8} ist.
	\end{itemize}
 \end{problem}

 \begin{theorem}[Stabilität und Quasi-Optimalität der Galerkin-FE]\label{thm:c2e4s1}
 	Unter den generellen Annahmen von \cref{sec:c2e1} an die Daten und den Annahmen von \cref{thm:c2e2s7} hat \cref{prb:c2e4s1} eine \emph{eindeutige} Lösung $u_h\in P^{k,0}_D(\tau_h)$, sodass
 	\[
 		\forall v_h\in P^{k,0}_D(\mathcal{T}_h)\;:\;a(u_h,v_h) = F(v_h) = a(u,v_h)
 	\]
 	und es gilt
 	\[
 		\|u_h\|_{H^1(\Omega)}\leq C(A,b,c,\Omega)\|u\|_{H^1(\Omega)} 
 	\]
 	mit 
 	\[
 		C(A,b,c,\Omega) = \lambda^{-1}\left( 1+C_F\frac{\diam \Omega}{\pi} \right) \left(\Lambda + \|b\|_{L^\infty(\Omega;\R^d)}+\|c\|_L^\infty\right).
 	\]
 	Dann gilt die Fehlerabschätzung
 	\[
 		\|u-u_h\|_{H^1}\leq C(A,b,c,\Omega)\min_{v_k\in P^{k,0}_D(\tau_h)}\|u-v_h\|_{H^1}.
 	\]

 \end{theorem}

\end{document}