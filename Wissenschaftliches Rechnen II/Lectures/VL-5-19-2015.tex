\documentclass[../skript.tex]{subfiles}

\begin{document}

\begin{example}\label{ex:c2e7s4}
	Betrachte das Eigenwert-Problem
	\begin{IEEEeqnarray*}{rCl"l}
		-\varepsilon\lapl u - b\cdot\nabla u &=& \lambda u, & \text{in }\Omega\coloneqq(0,1)^2 \\
		u &=& 0, & \text{auf }\partial\Omega
	\end{IEEEeqnarray*}
	mit $b = (1,0)^\tp$ (es gilt also $b\cdot\nabla u = \partial_x u$). Separation der Variablen $u(x,y) = v(x)w(y)$ führt auf ein System gewöhnlicher DGL
	\begin{IEEEeqnarray*}{rCl}
		-\varepsilon v'' + v' - \delta^2 v &=&0\\
		-\varepsilon w'' - (\delta^2+\lambda)w &=&0,
	\end{IEEEeqnarray*}
	für ein $\delta\in\R$. Die Lösungen sind gegeben durch
	\[
		u(x,y) \coloneqq e^{\frac{x}{2\varepsilon}}\left( e^{\frac{x}{2\varepsilon}\sqrt{1-r\delta^2\varepsilon}} - e^{-\frac{x}{2\varepsilon}\sqrt{1-4\delta^2\varepsilon}} \right)\left( C_1e^{iy\sqrt{\frac{\lambda+\delta^2}{\varepsilon}}} + C_2 e^{-iy\sqrt{\frac{\lambda+\delta^2}{\varepsilon}}} \right).
	\]
	Die Randbedingung ergibt dann
	\[
		u(x,y) \coloneqq e^{\frac{2}{2\varepsilon}} \sin{(k_1\pi x)}\sin{(k_2\pi y)},\quad k_1,k_2\in\N
	\]
	zum Eigenwert
	\[
		\lambda = \frac{1}{4\varepsilon} + \varepsilon(k_1^2+k_2^2)\pi^2.
	\]
	Für moderate $k_1,k_2$ ist die Ableitung von $u$ in $x$-Richtung um ungefähr Faktor $\frac{1}{2\varepsilon}$ größer als in $y$-Richtung. Es scheint also, dass die Probleme in der Numerik nur in Stromrichtung ($x$) auftreten, und orthogonal dazu alles in Ordnung ist.  
\end{example}
\textbf{Idee: Stabilisiere FEM in Stromrichtung (in Richtung von $b$) wie im 1D-Modell.}
\begin{definition}[Stromlinien-Diffusions-FEM (SDFEM)]\label{def:c2e7s5}
	Sei $\mathcal{T}_h$ eine reguläre Triangulierung von $\Omega$. \\
	Definiere die gitterabhängige Bilinearform $a_{\varepsilon,\delta} : P^{1,0}_0(\mathcal{T}_h)\to\R$ und Linearform $F_\delta:P^{1,0}_0(\mathcal{T}_h)\to\R$ durch
	\begin{IEEEeqnarray*}{rCll}
		a_{\varepsilon,\delta}(u_h,v_h) &\coloneqq& a_\varepsilon(u_h,v_h) & {} + \sum_{T\in\mathcal{T}_h} \delta_T\int_T b\cdot\nabla u_h b\cdot\nabla v_h \dx, \\
		F_\delta(v_h) &\coloneqq& F(v_h) & {} + \sum_{T\in\mathcal{T}_h} \delta_T\int_T f(b\cdot\nabla v_h).
	\end{IEEEeqnarray*}
	für gegebene (also wählbare) Parameter $\delta_T\geq 0, T\in\mathcal{T}_h$.
	Wir korrigieren also sowohl die Bilinearform $a_\varepsilon$ als auch die rechte Seite $F$. Oftmals genügt jedoch eine Korrektur ausschließlich der Bilinearform.\par
	Dann sucht die SDFEM ein $u_h\in P^{1,0}_0(\mathcal{T}_h)$, so dass
	\begin{equation}\label{eq:c2e7s4}
		a_{\varepsilon,\delta}(u_h,v_h) = F_\delta(v_h),\quad\forall v_h\in P^{1,0}_0(\mathcal{T}_h).
	\end{equation}
	Die Methode wird auch als \emph{Streamline-Upwind-Petrov-Galerkin} Methode (SUPG) bezeichnet.

	Hintergrund: \cref{eq:c2e7s4} kann auch als \emph{Petrov-Galerkin} Methode mit Ansatzraum $P^{1,0}_0(\mathcal{T}_h)$ und Testfunktionen
	\[
		\left\{v_h + \sum_{T\in\mathcal{T}_h} b\cdot\nabla v_h \gmid v_h\in P^{1,0}_0(\mathcal{T}_h)\right\}
	\]
	betrachtet werden. 
\end{definition}

Die Wohlgestelltheit von \cref{eq:c2e7s4} wird mittels der gitterabhängigen Norm $|\cdot|_{1,\varepsilon,\delta}$ definiert durch
\[
	|v|_{1,\varepsilon,\delta} \coloneqq \sqrt{ \varepsilon\|\nabla v\|_{L^2(\Omega)}^2 + \sum_{T\in\mathcal{T}_h} \delta_T \smash[b]{ \underbrace{\int_T b\cdot\nabla v\,b\cdot\nabla v \dx}_{\|b\nabla v\|_{L^2(T)}^2} } } \vphantom{\underbrace{\int_T b\cdot\nabla v\,b\cdot\nabla v \dx}_{\|b\nabla v\|_{L^2(T)}^2} }
\]
ausgedrückt. Offensichtlich ist $|\cdot|_{1,\varepsilon,\delta}$ wohldefiniert in $H^1_0(\Omega)$.
\begin{theorem}[Koerzivität von $a_{\varepsilon,\delta}$]\label{thm:c2e7s6}
	Für alle $v_h\in P^{1,0}_0(\mathcal{T}_h)$ gilt $a_{\varepsilon,\delta}(v_h,v_h)\geq |v_h|_{1,\varepsilon,\delta}^2$.
\end{theorem} 
\begin{proof}
	Die Aussage ergibt sich aus der Beziehung 
	\[
		\int_\Omega b\cdot\nabla v_h\,v_h\dx = 0.
	\]
\end{proof}
\begin{remark}\label{rem:c2e7s7}
	\begin{enumerate}
		\item \cref{thm:c2e7s6} liefert zusammen mit Lax-Milgram die Wohlgestelltheit von SDFEM.
		\item Die Wohlgestelltheit gilt für alle $\delta_T\geq 0$.
	\end{enumerate}
\end{remark}
\begin{theorem}[Fehler der SDFEM]\label{thm:c2e7s8}
	Sei $u_\varepsilon\in H^1_0(\Omega)$ die Lösung des Modellproblems \cref{prb:c2e7s1} und $u_{\varepsilon,h,\delta}$ ($\delta$ repräsentiert den Vektor der $\delta_T$) die SDFEM-Lösung von \cref{eq:c2e7s4}. Falls $u_\varepsilon\in H^2(\Omega)$ (gilt z.B.\ wenn $\Omega$ konvex und $f\in L^2(\Omega)$), dann gilt 
	\[
		|u_\varepsilon-u_{\varepsilon,h,\delta}|_{1,\varepsilon,\delta} \leq C\left( \sum_{T\in\mathcal{T}_h} (\varepsilon^2\delta_T + \varepsilon h_T^2+\delta_T h_T^2 + \delta_T^{-1}h_T^4 ) \|D^2u\|_{L^2(\Omega)}^2\right)^{\frac{1}{2}},
	\]
	wobei $C$ nur von der Formregularität des Gitters abhängt. 
\end{theorem}
\begin{proof}
	Sei $I_h \coloneqq I_h^1$ der nodale Inteprolationsoperator aus \cref{def:c2e5s1}. Die Dreiecksungleichung zeigt
	\[
		|u_\varepsilon-u_{\varepsilon,h,\delta}|_{1,\varepsilon,\delta} \leq |u_\varepsilon-I_h u_\varepsilon|_{1,\varepsilon,\delta} + |I_h u_\varepsilon - u_{\varepsilon,h,\delta}|_{1,\varepsilon,\delta}.
	\]
	\cref{thm:c2e5s2} (bzw.\ \cref{thm:c2e5s8}) zeigt
	\begin{IEEEeqnarray*}{rCl}
		|u_\varepsilon-I_hu_\varepsilon|_{1,\varepsilon,\delta}^2 &=& \varepsilon\|\nabla(u_\varepsilon-I_h u_\varepsilon)\|_{L^2(\Omega)}^2 + \sum_{T\in\mathcal{T}_h}\delta_T\|b\cdot\nabla(u_\varepsilon-I_h u_\varepsilon)\|_{L^2(T)}^2 \\
		&\leq& \sum_{T\in\mathcal{T}_h} (\varepsilon+\delta_T)C_{int}^2\delta_T^{-2}h_T^2\|D^2u_\varepsilon\|_{L^2(T)}^2\\
		&\leq& C_{int}^2\delta_{\mathcal{T}_h}^{-2}\sum_{T\in\mathcal{T}_h} (\varepsilon+\delta_T)h_T^2\|D^2u\|_{L^2(T)}.
	\end{IEEEeqnarray*}
	Für die Untersuchung des anderen Terms $|I_hu_\varepsilon - u_{\varepsilon,h,\delta}|_{1,\varepsilon,\delta}$ setzen wir $v_h \coloneqq I_hu_\varepsilon - u_{\varepsilon,h,\delta}$.
	\cref{thm:c2e7s6} impliziert
	\[
		|v_h|_{1,\varepsilon,\delta}\leq a_{\varepsilon,\delta}(v_h,v_h) = a_{\varepsilon,\delta}(I_hu_\varepsilon-u_\varepsilon,v_h) + a_{\varepsilon,\delta}(u_\varepsilon-u_{\varepsilon,h,\delta},v_h).
	\]
	Da $u\in H^2(\Omega)$ angenommen wurde, gilt
	\begin{IEEEeqnarray*}{rCl}
		a_{\varepsilon,\delta}(u_\varepsilon-u_{\varepsilon,h,\delta},v_h) &=& \underbrace{a_{\varepsilon}(u_\varepsilon,v_h)}_{=\int_\Omega f v_h} + \sum_{T\in\mathcal{T}_h}\delta_T\int_T b\cdot\nabla u_\varepsilon\,b\cdot\nabla v_h \dx \\
		&& \quad {}- \int_\Omega fv_h \dx - \sum_{T\in\mathcal{T}_h}\delta_T \int_T f(b\cdot\nabla v_h) \dx \\
		&=& \sum_{T\in\mathcal{T}_h}\delta_T\int_T\varepsilon\lapl u_\varepsilon b\cdot\nabla v_h \dx\\
		&\leq& \sum_{T\in\mathcal{T}_h}\left(\sqrt{\delta_T}\varepsilon\|\lapl u_\varepsilon\|_{L^2(T)}\right)\left(\|b\cdot\nabla v_h\|_{L^2(T)}\sqrt{\delta_T}\right) \\
		&\overset{\text{C.S.}}\leq& \left(\sum_{T\in\mathcal{T}_h}\delta_T\varepsilon^2 \smash[b]{ \underbrace{\|\lapl u\|_{L^2(T)}^2}_{\leq\|D^2u\|_{L^2(T)}^2} }\right)^{\frac{1}{2}}\underbrace{\left( \sum_{T\in\mathcal{T}_h}\delta_T\|b\cdot\nabla v_h\|_{L^2(T)}^2 \right)^{\frac{1}{2}} }_{\leq|v_h|_{1,\varepsilon,\delta}}.
	\end{IEEEeqnarray*}
	Im klassischen Galerkin Verfahren war $a_{\varepsilon,\delta}(u_\varepsilon-u_{\varepsilon,h,\delta},v_h)$ die \emph{Galerkin-Orthogonalität}, also $0$, was in diesem Fall jedoch nicht gilt. Ferner gilt aber
	\begin{IEEEeqnarray*}{rCl}
		a_{\varepsilon,\delta}(I_hu_\varepsilon -u_\varepsilon,v_h) &=& \varepsilon\int_\Omega\nabla(I_hu_\varepsilon-u_\varepsilon)\cdot\nabla v_h \dx + \underbrace{\int_\Omega b\cdot\nabla(I_hu_\varepsilon-u_\varepsilon)v_h}_{= -\int_\Omega b\cdot\nabla v_h(I_hu_\varepsilon-u_\varepsilon) \dx} + \ldots\\
		&& \quad {}+\sum_{T\in\mathcal{T}_h}\delta_T\int_T b\cdot\nabla (I_hu_\varepsilon-u_\varepsilon)b\cdot\nabla v_h\\
		&\overset{(\star)}\leq& \ldots\\
		&\leq& 2|v_h|_{1,\varepsilon,\delta} C(\delta_{\mathcal{T}_h})\left( \sum_{T\in\mathcal{T}_h} h_T^2(\varepsilon+\delta_T+\delta_T^{-1}h_T^2)\|D^2u\|_{L^2(T)}^2 \right)^{\frac{1}{2}}.
	\end{IEEEeqnarray*}
	Im ausgelassenen Schritt $(\star)$ wendet man eine Interpolationsfehlerabschätzung und die Cauchy-Schwarz Ungleichung an.\par
	Mit Hilfe der gemachten Abschätzungen ergibt sich die Aussage des Theorems durch deren Kombination.
\end{proof}
\begin{remark}[Wahl von $\delta_T$]\label{rem:c2e7s9}
	Die Wahl 
	\[
		\delta_T\coloneqq\frac{h_T^2}{\sqrt{\varepsilon^2+h_T^2}}
	\]
	minimiert die Fehlerschranke in \cref{thm:c2e7s8}. Im Regime $\varepsilon\ll h$ heißt das $\delta_T\equiv h_T$ und \cref{thm:c2e7s8} führt zu 
	\[
		\|b\cdot\nabla(u_\varepsilon-u_{\varepsilon,h,\delta}) \|_{L^2(\Omega)}\leq Ch\|D^2u_\varepsilon\|_{L^2(\Omega)}.
	\]
	Die hintere Schranke ist jedoch nicht unbedingt absolut klein, eher relativ klein. Sie verhält sich etwa $nh\varepsilon^{-1}$.
\end{remark}
\end{document}