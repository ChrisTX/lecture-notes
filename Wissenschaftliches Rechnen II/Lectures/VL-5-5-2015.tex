\documentclass[../skript.tex]{subfiles}
\begin{document}

\begin{proof} %Proof of theorem 2.4.1 ( Stability and quasi-optimality )
	Koerzivität von $a$, Stetigkeit von $a$ und $F$ bleiben auch auf Unterräumen $P^{k,0}_D(\mathcal{T}_h)\subseteq H^1_D(\Omega)$ erhalten. \emph{Lax-Milgram} liefert dann die Wohlgestelltheit des Problems. \emph{Céa's Lemma} gibt uns die Fehlerabschätzung.
\end{proof}

\section*{Überführung von \cref{prb:c2e4s1} in ein lineares Gleichungssystem}

Wir erinnern uns
\[
	P^{k,0}_D(\mathcal{T}_k) \overset{\text{\cref{def:c2e3s11}}}{=} \spn\{b_z \mid z\in\Sigma_k(\mathcal{T}_k)\setminus\Gamma_D\},
\]
wobei $\Sigma_k(\mathcal{T}_k)$ die Menge der Knoten (siehe \cref{cor:c2e3s10}), und $b_j$ die nodalen Basisfunktionen sind. Nummeriere die Knoten $z_1,\ldots,z_N$, wobei $N \coloneqq |\Sigma_k(\mathcal{T}_k)\setminus\Gamma_D| = \dim P^{k,0}_D(\mathcal{T}_k)$. Jedes $u_h\in P^{k,0}_D(\mathcal{T}_k)$ ist eine Linearkombination der $b_j$:
\[
	u_h(x) = \sum_{j=1}^N u_h(z_j)b_j(x),\quad x\in\bar{\Omega}.
\]
\pagebreak
Somit ist \cref{prb:c2e4s1} äquivalent zu folgendem linearen Gleichungssystem:
\begin{problem}
	Finde $c\in\R^N$, sodass 
	\[
		Sc = r,
	\]
	wobei die \emph{Systemmatrix} $S\in\R^{N\times N}$ durch
	\[
		S_{i,j} = a(b_i,b_j),\quad 1\leq i,j\leq N
	\]
	und die rechte Seite $r\in\R^N$ durch
	\[
		f_i\coloneqq F(b_i),\quad 1\leq i\leq N
	\]
	gegeben sind.
	Das ergibt das Gleichungssystem
	\[
		\left[\begin{IEEEeqnarraybox}[][c]{r'rCl}
				\forall b_j & a\left(\sum_{i=1}^N u_h(z_i)b_i,b_j\right) &=& F(b_j)\\
				& \sum_{i=1}^N u_h(z_i) a(b_i,b_j) &=& F(b_j)
			\end{IEEEeqnarraybox}\right]
	\]
\end{problem}

\begin{remark}\label{rem:c2e4s2}
	Die Matrix $S$ ist positiv definit (insbesondere regulär), also invertierbar. $S$ ist dünn besetzt, d.h.
	\[
		 | \{ S_{i,j} \not=0 \mid 1\leq i,j\leq N\} | \sim N
	\]
	(``$\sim$'' heißt ``proportional zu'').
\end{remark}



\section{Approximationseigenschaften finiter Elemente}\label{sec:c2e5}
Erinnerung: Die allgemeinen Lösungen der in diesem Kapitel betrachteten Probleme liegen in $H^1_D(\Omega)$. Es gibt jedoch Funktionen aus $H^1_D(\Omega)$ die \emph{nicht} stetig sind.
\begin{definition}[Nodale Interpolation]\label{def:c2e5s1}
	Sei $\mathcal{T}_h$ eine reguläre Triangulierung von $\Omega\subseteq\R^d$ und $k\in\mathbb{N}$. Sei $\Sigma_k(\mathcal{T}_h)$ die zugehörige Knotenmenge. Definiere den \emph{nodalen Interpolationsoperator} $I_h^k:C^0(\bar{\Omega})\to P^{k,0}(\mathcal{T}_k)$ durch 
	\[
		\underbrace{(I_h^k v)}_{\mathclap{\substack{\text{Interpolierende} \\ \text{von } v}}}(x) \coloneqq \sum_{z\in \Sigma_k(\mathcal{T}_h)} \hspace*{-10pt} v(z)b_z(x),\quad x\in\bar{\Omega}.
	\]  
	Wegen $H^2(\Omega)\hookrightarrow C^0(\Omega)$, ist $I_h^k$ auch für hinreichend glatte Lösungen $u\in H^2(\Omega)\cap H^1_0(\Omega)$ wohldefiniert. \cref{sec:c2e6} wird sich mit hinreichenden Bedingungen für $u\in H^2(\Omega)\cap H^1_D(\Omega)$ beschäftigen.
\end{definition}
Wir nehmen jetzt zunächst an, dass $d=2$ und $k=1$, d.h.\ wir betrachten die $P1$-FE in $2D$.
\begin{theorem}[Interpolationsfehlerabschätzung]\label{thm:c2e5s2}
	Sei $T$ ein Dreieck mit Durchmesser $h_T$ (Länge der längsten Seite) und größtem Innenwinkel $0 < \omega < \pi$. Dann existieren Konstanten 
	\[
		c_0,\;c_1(\omega)\coloneqq\frac{\sqrt{\frac{1}{4}+\frac{2}{\pi^2}}}{\sqrt{1-|\cos{\omega}|}},\; c_2(\omega) = 1,
	\]
	sodass für alle $v\in H^2(\Omega)$ gilt
	\[
		\| D^m(v-I_h^1 v)\|_{L^2(T)} \leq c_m(\omega)h_T^{2-m}\| D^2 v\|_{L^2(T)},\quad m=0,1,2.
	\]
\end{theorem}

\begin{proof}
	\underline{$m=2$:} $D^2 I^1_v = 0 \implies \| D^2(v-I_h^1 v)\| = \|D^2 v\|$.
% Bemerkung: Die Tangentialvektoren tau_1 und tau_2 müssen so orientiert werden, so dass omega der eingeschlossene Winkel ist. Es wird verwendet a*b= |a||b| cos w.
	\underline{$m=1$:} Sei $T \coloneqq \conv(\{z_1,z_2,z_3\})$, $\tau_j$ die Einheitstangentialvektoren der Kanten $E_j$ von $T$ (orientiert gegen den Uhrzeigersinn), sodass $E_j$ die Knoten $z_i$ und $z_k$ verbindet, mit $j\not\in\{i,k\}$, und sei $\omega$ \ac{obda}\ der Winkel bei $z_3$. Für $a\in\R^2$ gilt mit der euklidischen Norm $|\cdot|$ und einer Basis $(\mu_1,\mu_2)$ des $\R^2$ aus linear unabhängigen Einheitsvektoren:
	\[
		| a|^2 \leq \frac{(a\cdot\mu_1)^2 + (a\cdot\mu_2)^2}{1- |\mu_1\cdot\mu_2|}
	\]
	(Übung: Nachrechnen!). 
	Für $a = \nabla(v-I_h^1 v)(x)$ und $\mu_i = \tau_i$ gilt dann
	\[
		|\nabla(v-I_h^1 v)(x)|^2\leq (| 1-\underbrace{(\tau_1\cdot\tau_2)}_{\cos{\omega}}| )^{-1}\big( [\underbrace{(\nabla(v-I_h^1 v))(x)\cdot\tau_1}_{\eqqcolon f_1(x)}]^2 + [\underbrace{(\nabla(v-I_h^1 v))(x)\cdot\tau_2}_{\eqqcolon f_2(x)}]^2 \big).
	\] 
	Es gilt
	\[
		\int_{E_j} f_j \ds = \int_{E_j}\frac{\partial}{\partial_s}(v-I_h^1 v) \ds = 0,\quad j=1,2,
	\]
	% Das wichtigere Argument ist eher, dass  I_hv und v an den Endpunkten übereinstimmen.
	da $E_j$ $z_{j-1}$ und $z_{j+1}$ verbindet.

	\emph{Erinnerung:} Aufgabe 2 auf Übungsblatt 1 hat gezeigt: \\
	Für $f\in W^{1,1}(T)$ mit $T=\conv(\{P\}\cup E)$ (wobei $E$ eine Kante und $P$ der Punkt gegenüber von $E$ ist), gilt
	\[
		\fint_E f(s) \ds = \fint_T f(x) \dx  + \frac{1}{2}\fint_T\nabla f(x)\cdot(x-P)\dx.
	\]
	Zusammen mit der Spurgleichung folgt
	\begin{IEEEeqnarray*}{rCl}
		\int_T f_j(x) \dx  &\leq& \frac{1}{2}\left\vert \int_T\nabla f_j(x)\cdot(x-z_j) \dx  \right\vert\\  &\leq& \frac{1}{2}\|\nabla f_j\|_{L^2(T)}\|\cdot-z_j\|_{L^2(T)}.
	\end{IEEEeqnarray*}
	Eine einfache Rechnung (mittels Polarkoordinaten) ergibt:
	\[
		\| \cdot - z_j\|_{L^2(T)}\leq\frac{h_T}{\sqrt{2}}
	\]
	Die \emph{Poincaré-Ungleichung} zeigt dann, wegen
	\[
		\int_\Omega \left(f_j - \fint f_j\dx\right)\left( \fint f_j \dx  \right) \dx  = 0
	\]
	und
	\begin{IEEEeqnarray*}{rCl}
		\left\| \fint_\Omega f_j \dx \right\|_{L^2(T)}^2 &=& \left| \frac{1}{| T|}\int_T f_j \dx \right|^2 \|1\|_{L^2(T)}^2 \\
		&=& \left|\int_T f_j \dx \right|^2 \\
		&\leq& \frac{1}{8}h_T^2 \|\nabla f_j\|_{L^2(T)},
	\end{IEEEeqnarray*}
	dass
	\begin{IEEEeqnarray*}{rCl}
			\| f_j\|_{L^2(T)}^2 
		&=& 
			\left\| f_j - \fint_T f_j \dx  \right\|_{L^2(T)}^2 + \left\|\fint_T f_j \dx \right\|_{L^2(T)}^2\\
		&\leq&
			\frac{h_T^2}{\pi^2} \|\nabla f_j\|_{L^2(T)}^2 + \frac{\pi}{8} h_T^2 \|\nabla f_j\|_{L^2(T)}^2 \\
		&\leq&
			\left( \frac{1}{\pi^2}+\frac{1}{8} \right)h_T^2 \|\nabla f_j\|_{L^2(T)}^2.
	\end{IEEEeqnarray*}
	\emph{Zusammenfassung:} Wir haben gezeigt
	\begin{IEEEeqnarray*}{rCl}
			\|\nabla (v-I_h^1 v)\|_{L^2}^2 
		&\leq& 
			\frac{1}{| 1-\cos{\omega}|}\left( \|f_1\|_{L^2(T)}^2 + \| f_2\|_{L^2(T)}^2 \right)\\
		&\leq& 
			\frac{1}{| 1-\cos{\omega}|}\underbrace{\left(\frac{2}{\pi^2} + \frac{1}{4}\right)}_{c_1(\omega)} \frac{1}{2} h_T^2 \underbrace{\left( \|\nabla f_1\|_{L^2(T)}^2 + \|\nabla f_2\|_{L^2(T)}^2 \right)}_{\leq 2\|D^2v\|_{L^2(T)}^2}
	\end{IEEEeqnarray*}
	Die Abschätzung für die $L^2$-Norm von $f_1$ und $f_2$ folgt, da die partiellen Ableitungen zweiter Ordnung des Interpolators $I_h^1$ in alle Richtungen verschwinden.

	\underline{$m=0$:} Übung.
\end{proof}
\begin{remarknonumb}
	Poincaré: Sei $\Omega$ einfach zusammenhängend, beschränkt, Lipschitz, und sei $f\in H^1(\Omega)$. Dann gilt
	\[
		\left\| f-\fint_\Omega f \dx \right\|_{L^2(\Omega)}\leq\frac{\diam \Omega}{\pi}\|\nabla f\|_{L^2(\Omega)}.
	\]
\end{remarknonumb}
\end{document}