\documentclass[../skript.tex]{subfiles}
\begin{document}

\begin{proof} %Proof of theorem 2.4.1 ( Stability and quasi-optimality )
	Koerzivität von $a$, Stetigkeit von $a$ und $F$ bleiben auch auf Unterräumen \newline\noindent $P^{k,0}_D(\mathcal{T}_h)\subseteq H^1_D(\Omega)$ erhalten. \emph{Lax-Milgram} liefert dann die Wohlgestelltheit des Problems. \emph{Céa's Lemma} gibt uns die Fehlerabschätzung.
\end{proof}

\section*{Überführung von \cref{prb:c2e4s1} in ein lineares Gleichungssystem}

Wir erinnern uns
\[
	P^{k,0}_D(\mathcal{T}_k) \overset{\cref{def:c2e3s11}}= span\{b_z\vert\,z\in\Sigma_k(\mathcal{T}_k)\setminus\Gamma_D\}
\]
wobei $\Sigma_k(\mathcal{T}_k)$ die Menge der Knoten (siehe \cref{cor:c2e3s10}), und $b_j$ die nodalen Basisfunktionen sind. Nummeriere die Knoten $z_1,...,z_N$ wobei $N \coloneqq\vert\Sigma_k(\mathcal{T}_k)\setminus\Gamma_D\vert = dim\,P^{k,0}_D(\mathcal{T}_k)$. Jedes $u_h\in P^{k,0}_D(\mathcal{T}_k)$ ist eine Linearkombination der $b_j$:
\[
	u_h(x) = \sum_{j=1}^k u_h(z_j)b_j(x),\quad x\in\bar{\Omega}
\]
Somit ist \cref{prb:c2e4s1} äquivalent zu folgendem linearen Gleichungssystem:
\begin{problem}
	Finde $c\in\R^N$, sodass 
	\[
		Sc = r
	\]
	wobei die \emph{Systemmatrix} $S\in\R^{N\times N}$ durch
	\[
		S_{i,j} = a(b_i,b_j),\quad 1\leq i,j\leq N
	\]
	und die rechte Seite $r\in\R^N$ durch
	\[
		f_i\coloneqq F(b_i),\quad 1\leq i\leq N
	\]
	gegeben sind.
	\begin{equation*}
		\left[ 
			\begin{aligned}
				\forall b_j:\,a\left(\sum_{i=1}^N u_h(z_i)b_i,b_j\right) &=& F(b_j)\\
				\sum_{i=1}^N u_h(z_i) a(b_i,b_j) &=& F(b_j)
			\end{aligned}
		\right]
	\end{equation*}
\end{problem}

\begin{remark}\label{rem:c2e4s2}
	Die Matrix $S$ ist positiv definit (insbesondere regulär), also invertierbar. $S$ ist dünn besetzt, d.h.
	\[
		\vert \{ S_{i,j} \not=0\vert\, 1\leq i,j\leq N\}\vert \approx N
	\]
	(`$\approx$' heißt `proportional zu').
\end{remark}



\section{Approximationseigenschaften finiter Elemente}\label{sec:c2e5}

Erinnerung: Die allgemeinen Lösungen der in diesem Kapitel betrachteten Probleme liegen in $H^1_D(\Omega)$. Es gibt jedoch Funktionen aus $H^1_D(\Omega)$ die \emph{nicht} stetig sind.

\begin{definition}[Nodale Interpolation]\label{def:c2e5s1}
	Sei $\mathcal{T}_h$ eine reguläre Triangulierung von $\Omega\subseteq\R^d$ und $k\in\mathbb{N}$. Sei $\Sigma_k(\mathcal{T}_h)$ die zugehörige Knotenmenge. Definiere den \emph{nodalen Interpolationsoperator} $I_h^k:C^0(\bar{\Omega})\to P^{k,0}(\mathcal{T}_k)$ durch 
	\[
		\underbrace{(I_h^k v)}_{\text{Interpolierende von} v}(x) \coloneqq \sum_{z\in \Sigma_k(\mathcal{T}_h)} v(z)b_z(x),\quad x\in\bar{\Omega}
	\]  
	Wegen $H^2(\Omega)\hookrightarrow C^0(\Omega)$, ist $I_h^k$ auch für hinreichend glatte Lösungen $u\in H^2(\Omega)\cap H^1_0(\Omega)$ wohldefiniert. Abschnitt\cref{sec:c2e6} wird sich mit hinreichenden Bedingungen für $u\in H^2(\Omega)\cap H^1_D(\Omega)$ beschäftigen.
\end{definition}
Wir nehmen jetzt zunächst an, dass $d=2$ und $k=1$, d.h. wir betrachten die $P1$-FE in $2D$.
\begin{theorem}[Interpolationsfehlerabschätzung]\label{thm:c2e5s2}
	Sei $T$ ein Dreieck mit Durchmesser $h_T$ (Länge der längsten Seite) und größtem Innenwinkel $0 < \omega < \pi$. Dann existieren Konstanten 
	\[
		c_0,\,c_1(\omega)\coloneqq\frac{\sqrt{{1}{4}+\frac{2}{\pi^2}}}{\sqrt{1-\vert\cos{\omega}\vert}},\, c_2(\omega) = 1
	\]
	so dass für alle $v\in H^2(\Omega)$ gilt
	\[
		\| D^m(v-I_h^1 v)\|_L^2(\Omega) \leq c_m(\omega)h_T^{2-m}\| D^2 v\|_{L^2(T)},\quad m=0,1,2
	\]
\end{theorem}

\begin{proof}
	\underline{$m=2$: } $D^2 I^1_v = 0 \Rightarrow \| D^2(v-I_h^1 v)\| = \|D^2 v\|$.\newline\newline\noindent
	\underline{$m=1$: } Sei $T \coloneqq conv(\{z_1,z_2,z_3\})$, $\tau_j$ die Einheitstangentialvektoren der Kanten $E_j$ von $T$ (orientiert gegen den Uhrzeigersinn), sodass $E_j$ die Knoten $z_i$ und $z_k$ verbindet, mit $j\not\in\{i,k\}$, und sei $\omega$ o.B.d.A der Winkel bei $z_3$. Für $a\in\R^2$ gilt mit der euklidischen Norm $\vert\cdot\vert$ und einer Basis $(\mu_1,\mu_2)$ des $\R^2$ aus linear unabhängigen Einheitsvektoren:
	\[
		\vert a\vert^2 \leq \frac{(a\cdot\nu_1)^2 + (a\cdot\mu_2)^2}{1-\vert \mu_1\cdot\mu_2}
	\]
	(Übung: nachrechnen!)\newline\noindent 
	Für $a = \nabla(v-I_h^1 v)(x)$ und $\mu_i = \tau_i$ gilt dann
	\[
		\vert\nabla(v-I_h^1 v)(x)\vert^2\leq (\vert 1-\underbrace{(\tau_1\cdot\tau_2)}_{\cos{\omega}}\vert )^{-1}\left( ( \underbrace{(\nabla(v-I_h^1 v)(x))\cdot\tau_1 )}_{\eqqcolon f_1(x)}^2 + ( \underbrace{(\nabla(v-I_h^1 v))(x)\cdot\tau_2)}_{\eqqcolon f_2(x)}^2 \right)
	\] 
	Es gilt
	\[
		\int_{E_j} f_j\,ds = \int_{E_j}\frac{\partial}{\partial_s}(v-I_h^1 v)\,ds = 0,\quad j=1,2 
	\]
	da $E_j$ $z_{j-1}$ und $z_{j+1}$ verbindet.\newline\newline\noindent
	\emph{Erinnerung:} Aufgabe 2 auf Übungsblatt 1 hat gezeigt: \newline Für $f\in W^{1,1}(T)$ mit $T=conv(\{P\}\cup E)$ (wobei $E$ eine Kante und $P$ der Punkt gegenüber von $E$ ist), gilt
	\[
		\strokedint_E f(s)\,ds = \strokedint_T f(x)\,dx + \frac{1}{2}\strokedint_T\nabla f(x)\cdot(x-P)\,dx
	\]
	Zusammen mit der Spurgleichung folgt
	\begin{eqnarray*}
		\int f_j(x)\,dx &\leq& \frac{1}{2}\left\vert \int_T\nabla f_j(x)\cdot(x-z_j)\,dx \right\vert\\  &\leq& \frac{1}{2}\|\nabla f_j\|_{L^2(T)}\|\cdot-z_j\|_{L^2(T)}
	\end{eqnarray*}
	Eine einfache Rechnung (mittels Polarkoordinaten) ergibt:
	\[
		\| \cdot - z_j\|_{L^2(T)}\leq\frac{\sqrt{\pi}h_T}{\sqrt{2}}
	\]
	Die \emph{Poincaré-Ungleichung} zeigt dann wegen
	\[
		\int_\Omega \left(f_j - \strokedint f_j\, dx\right)\left( \strokedint f_j\,dx \right)\,dx = 0
	\]
	und
	\[
		\left\| \strokedint\Omega f_j\,dx\right\|_{L^2(T)}^2 = \left\vert \frac{1}{\vert T\vert}\int_T f_j\,dx\right\vert^2 \|1\|_{L^2(T)}^2 = \frac{\vert T\vert}{\vert T\vert^2}\left\vert\int_T f_j\,dx\right\vert^2 \leq \frac{1}{2}h_T^2
	\]
	\begin{eqnarray*}
			\| f_j\|_{L^2(T)}^2 
		&=& 
			\left\| f_j - \strokedint_T f_j\,dx \right\|_{L^2(T)}^2 + \left\|\strokedint_T f_j\,dx\right\|_{L^2(T)}^2\\
		&\le q& 
			\frac{h_T^2}{\pi^2} \|\nabla f_j\|_{L^2(T)}^2 + \frac{1}{4}\|\nabla f_j\|_{L^2(T)}^2 \frac{h_T^2}{2}
		\leq
			\left( \frac{1}{\pi^2}+\frac{1}{8} \right)h_T^2 \|\nabla f_j\|_{L^2(T)}^2
	\end{eqnarray*}
	\emph{Zusammenfassung:} Haben gezeigt
	\begin{eqnarray*}
			\|\nabla (v-I_h^1 v)\|_{L^2}^2 
		&\leq& 
			\frac{1}{\vert 1-\cos{\omega}\vert}\left( \|f_1\|_{L^2(T)}^2 + \| f_2\|_{L^2(T)}^2 \right)\\
		&\leq& 
			\frac{1}{\vert 1-\cos{\omega}\vert}\underbrace{\left(\frac{2}{\pi^2} + \frac{1}{4}\right)}_{c_1(\omega)} \frac{1}{2} h_T^2 \underbrace{\left( \|\nabla f_1\|_{L^2(T)}^2 + \|\nabla f_2\|_{L^2(T)}^2 \right)}_{\leq 2\|D^2v\|_{L^2(T)}^2}
	\end{eqnarray*}
	Die Abschätzung für die $L^2$-Norm von $f_1$ und $f_2$ folgt, da die partiellen Ableitungen zweiter Ordnung des Interpolators $I_h^1$ in alle Richtungen verschwinden.\newline\newline\noindent
	\underline{$m=0$: } Übung.
\end{proof}
\ \\

\begin{remark*}
	Poincaré: Sei $\Omega$ einfach zusammenhängend, beschränkt, Lipschitz, und sei $f\in H^1(\Omega)$. Dann gilt
	\[
		\left\| f-\strokedint_\Omega f\,dx\right\|_{L^2(\Omega)}\leq\frac{dia\,\Omega}{\pi}\|\nabla f\|_{L^2(\Omega)}
	\]
\end{remark*}
\end{document}