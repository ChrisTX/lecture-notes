 \documentclass[../skript.tex]{subfiles}

\begin{document}

\begin{remark}[Regularität und Konvergenzrate]\label{rem:c3e4s5}
	Auf konvexen Gebieten gilt für die schwache Lösung $(u,p)$ von \cref{prb:c3e2_p}:
	\[
		(u,p) \in H^2(\Omega)\times H^1(\Omega).
	\]
	Ferner erfüllt die MINI-FEM die Fehlerabschätzung
	\begin{IEEEeqnarray*}{rCl}
		\|D(u-u_h)\|_{L^2(\Omega)} + \|p-p_h\|_{L^2(\Omega)} &\leq& Ch\left(\|D^2u\|_{L^2(\Omega)}+\|\nabla p\|_{L^2(\Omega)}\right) \\
		&\leq& \tilde{C}h\|f\|_{L^2(\Omega)}
	\end{IEEEeqnarray*}
\end{remark}

\begin{IEEEeqnarray*}{rCl}
	\int \nabla u_h\::\:\nabla v_h - \int p_hdiv\:v_h &=& \int f v_h\\
	\int q_h div\:u_h &=& 0
\end{IEEEeqnarray*}
Für alle $v_h\in\left[ P^{1,0)}_0(\mathcal{T}_h) \right]^d, \forall q_h\in P^{k,0}(\mathcal{T}_h)\cap L^2_0(\mathcal{T}_h)$.

\section{Das Crouzieux-Raviart-Element}\label{sec:c3e5}
Manchmal ist es nützlich Konformität aufzugeben (um Stabilität zu verbessern)!

\begin{definition}[Crouzeix-Raviart-Element]\label{def:c3e5s1}
	Gegeben sei eine reguäre Triangulierung $\mathcal{T}_h$. Definiere
	\begin{IEEEeqnarray*}{rCl}
		CR^1(\mathcal{T}_h) &\coloneqq& \left\{ v\in P^{1,-1}(\mathcal{T}_h)|\,v\text{ ist stetig  in den Kanten-/ Seitenmittelpunkten} \right\}\\
		CR^1_0(\mathcal{T}_h) &\coloneqq& \left\{ v\in CR^1(\mathcal{T}_h)|\,v\text{ ist Null im Mittelpunkt von Randkanten} \right\}
	\end{IEEEeqnarray*}
\end{definition}

\begin{remark}\label{rem:c3e5s2}
	$CR^1(\mathcal{T}_h)$ ist nicht konform im Sinne von
	\[
		CR^1(\mathcal{T}_h) \not\subseteq H^1(\Omega).
	\]
\end{remark}

Wir definieren \emph{stückweise Ableitungen} $D_{nc}$ (\emph{NC} = `non-conformal'), $div_{nc}$ für $CR$-Funktionen, durch
\begin{IEEEeqnarray*}{rCl}
	D_{nc}v_{CR} &\coloneqq& \sum_{T\in\mathcal{T}_h} Dv_{CR}|_T\\
	div_{nc} v_{CR} &\coloneqq& \sum_{T\in\mathcal{T}_h} (div\:v_{CR})|_T
\end{IEEEeqnarray*}
und zwar für alle $v_{CR}\in CR^1(\mathcal{T}_h)^d$.\newline\noindent
Die \emph{Crouzeix-Raviart-Methode} beruht auf der Paarung
\[
	V_h \coloneqq \left[ CR_0^1(\mathcal{T}_h) \right]^d,\quad M_h\coloneqq P^{0,-1}(\mathcal{T}_h)\cap L^2_0(\Omega).
\]
Da der Druck nur im Gradienten vorkommt, kann er um eine Konstante geändert werden, ohne dass sich der Wert des Gradienten verändert. Daher muss der Druck irgendwie fixiert werden. Eine Möglichkeit: Mittelwert des Integrals ist $0$, deshalb $p\in L^2_0(\Omega)$.\newline\newline\noindent

\emph{CR-FEM} für Stokes sucht $(u,p)\in V_h\times M_h$ sodass für alle $v_h\in V_h, q_h\in M_h$ gilt:

%DAS HAT LABEL P_{CR}
\begin{equation}
	\begin{aligned}
		\int_\Omega D_{nc}\::\: D_{nc}v_h\,dx + \int_\Omega p_h\,div_{nc}\:v_h\,dx &=& \int_\Omega f\cdot v_h\,dx\\
		\int_\Omega p_h\:div_{nc}\:u_h\,dx &=& 0. 
	\end{aligned}\tag{P-CR}\label{eqn:c3e5s_P_CR}
\end{equation}

Die Stabilitäts- und Fehleranalysis der vorangegangenen Abschnitte ist hier \underline{nicht} anwendbar, da $V_h\not\subseteq V$.

\begin{theorem}\label{thm:c3e5s2}
	Die CR-FEM (\ref{eqn:c3e5s_P_CR}) ist wohlgestellt und erfüllt
	\[ 
		\|Du-D_{nc}u_h\|_{L^2(\Omega)} + \|p-p_h\|_{L^2(\Omega)} \leq 
		C\left( \begin{aligned}&\min_{v_h\in V_h} \|Du-D_{nc}v_h\|_{L^2(\Omega)} \\+& \min_{q_h\in M_h}\|p-q_h\| + h\|f\|_{L^2(\Omega)} \end{aligned}\right)
	\]
\end{theorem}

\begin{proof}
	Übung, Crouzeix-Raviart (1973).
\end{proof}

\begin{remark}\label{rem:c3e5s3}
	Die CR-LÖsung $u_h$ von (\ref{eqn:c3e5s_P_CR}) erfüllt
	\[
		div_{nc}\:u_h = 0.
	\]
	D.h. die Bedingung, die für die tatsächliche Lösung gilt (Massenerhaltung), gilt auch auf dieskretem Level. Dies ist in etwa das, was das Ladyzhenskaya-Lemma besagte: Die Divergenz ist im Falle der Crouzeix-Raviart FEM surjektiv und bildet auf alle stückweisen Konstanten ab. Dies ist insbesondere relevant für zeitabhängige Probleme. Auf dem Mini-Element kann diese Bedingung nicht garantiert werden!
\end{remark}

\text{Wir beschränken uns von hier an auf den Fall $d=2$!}

\begin{remark}[Basis]\label{rem:c3e5s4}
	Der Druck besteht aus Stückweise Konstanten Funktionen, d.h. die Basis des Druckraumes ist einfach die Menge der charakteristischen Funktionen auf den Elementen.\newline\noindent
	Die Freiheitsgrade der CR-FEM für die geschwindigkeit sind den Kantenmittelpunkten zugeordnet. Für eine Kante $E\in\mathcal{E}(\mathcal{T}_h)= E_h$ definieren wir mit $\Psi_E$ die Kantenbasisfunktion die durch
	\[
		\Psi_E(mid(F)) \coloneqq \begin{cases}1,&E=F\\0,&\text{sonst}\end{cases}
	\]
	für $F\in\mathcal{E}_h$ definiert ist.
\end{remark}

\begin{definition}[Nichtkonforme Interpolation]\label{def:c3e5s5}
	$\mathcal{T}_h$ sei eine reguäre Triangulierung von $\Omega\subseteq\mathbb{R}^2$ und $\mathcal{E}_h$ bezeichnet die Menge der Kanten von $\mathcal{T}_h$. Für $v\in H^1(\Omega)$, ist die nichtkonforme Interpolation $I_{CR}v$ durch 
	\[
		I_{CR}v \coloneqq \sum_{E\in\mathcal{E}_h} \left( \frac{1}{|E|} \int_E v\,dS \right)\underbrace{\Psi_E}_{\text{Kantenbasis}}.
	\]
	definiert.
\end{definition}

Die Schlüsseleigenschaft von $I_{CR}$ ist die Folgende:

\begin{lemma}\label{thn:c3e5s6}
	Für $v\in H^1(\Omega)$ und $T\in\mathcal{T}_h$ gilt:
	\[
		\underbrace{\frac{1}{|T|}\int_T \nabla v\,dx}_{\text{Mittelwert von }v\text{ über }T} = \frac{1}{|T|}\int_T\nabla I_{CR} v\,dx
	\]
\end{lemma}
\begin{proof}
	Per Konstruktion (Definition des Interpolators) gilt für jede Kante $E\in\mathcal{E}_h$
	\[
		\int_E v\,dS = \int_E I_{CR} v\,dS.
	\]
	Durch partielle Integration erhalten wir
	\begin{IEEEeqnarray*}{rCl}
		\int_T ( \nabla v) 1 \,dx &=& \int_{\partial T} v\cdot\nu_T\,dS \\
		&=& \underbrace{\sum_{E\in\mathcal{E}_h(T)}}_{\text{Kanten von }T} \int_E v\cdot\underbrace{\nu_E}_{=const\text{ auf }E}\,dS\\
		&=& \sum_{E\in\mathcal{E}_h(T)} \left( \int_E I_{CR}v\,dS \right)\cdot\nu_E\\
		&=& \int_T\nabla I_{CR}v\,dx
	\end{IEEEeqnarray*}
\end{proof}

\end{document}