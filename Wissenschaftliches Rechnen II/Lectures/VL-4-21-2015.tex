\documentclass[../skript.tex]{subfiles}

\begin{document}
\chapter{Skalare elliptische \acs{PDE}s zweiter Ordnung} % Chapter 1
\label{sec:c2}
\section{Modellproblem} % 2.1
\label{sec:c2e1}
\begin{problem} % Problem (1)
\label{prb:c2e1}
Seien $\Omega \subseteq \R^d$, $d = \{ 1, 2, 3\}$, beschränktes Gebiet mit Lipschitz-Rand $\Gamma \coloneqq \partial \Omega$ und äußerer Einheitsnormalen $\nu$.
\end{problem}
In diesem Kapitel betrachten wir die skalare, lineare elliptische \ac{PDE} 2.~Ordnung:
\begin{equation} % (2.1.1)
\label{eq:c2e1s1}
- \dive (A \nabla u) + \beta \cdot \nabla u + cu = f \; \text{in } \Omega,
\end{equation}
wobei
\begin{itemize}
\item $A : \Omega \to \R^{d \times d}$, $A \in L^\infty(\Omega; \R^{d \times d})$ mit uniformen Schranken:
\begin{equation} % (2.1.2)
\label{eq:c2e2s2}
	0 < \lambda \coloneqq \essinf_{x \in \Omega} \inf_{z \in \R^d \setminus \{ 0 \}} \frac{z^\tp A(x) z}{z \cdot z} \leq \esssup_{x \in \Omega} \sup_{z \in \R^d \setminus \{ 0 \}} \frac{z^\tp A(x) z}{z \cdot z} \eqqcolon \Lambda < \infty
\end{equation}
\item $\beta : \Omega \to \R^d$, $\beta \in L^\infty(\Omega; \R^d)$, $\dive \beta \in L^\infty(\Omega)$.
\item $c : \Omega \to \R$, $c \in L^\infty(\Omega)$.
\item $f \in L^2(\Omega)$.
\end{itemize}
Eigenschaft \labelcref{eq:c2e2s2} charakterisiert die Gleichung als elliptisch.
Die eindeutige Lösbarkeit der PDE \labelcref{eq:c2e1s1} erfordert Randbedingungen, z.B.
\begin{itemize}
\item \emph{(homogene) Dirichlet-Randbedingung:}
\[
	u = 0 \quad \text{auf } \Gamma \text{ [im Sinne einer Spurbildung]}.
\]
\item \emph{(inhomogene) Dirichlet-Randbedingung:}
\[
	u = u_D \quad \text{auf } \Gamma,
\]
wobei $u_D$ gegeben ist.
\item \emph{(inhomogene) Neumann-Randbedingung:}
\[
	(A \nabla u) \cdot \nu = g \quad \text{auf } \Gamma,
\]
wobei $g \in L^2(\Gamma)$ gegeben ist.
\item \emph{gemischte Dirichlet\slash{}Neumann-Randbedingungen}
\begin{IEEEeqnarray*}{rCl'l}
u &=& 0 & \text{auf } \Gamma_D, \IEEEyesnumber \label[equation]{eq:c2e1s3} \\ % (2.1.3)
(A \nabla u) \cdot \nu &=& g & \text{auf } \Gamma_N,
\end{IEEEeqnarray*}
wobei $\Gamma = \Gamma_D \cup \Gamma_N$, $\Gamma_D \cap \Gamma_N = \emptyset$, $\Gamma_D$ abgeschlossen in $\Gamma$, $|\Gamma_D| > 0$ und $g \in L^2(\Gamma_N)$.
\end{itemize}
\section{Schwache Lösungen und Wohlgestelltheit}
\begin{example}[Nichtexistenz starker Lösungen] % Beispiel 2.2.1
\label{ex:c2e2s1}
Sei $0 < \alpha < 2 \pi$,
\[
	\Omega_\alpha \coloneqq \left\{ x = r \begin{pmatrix}
	\cos \varphi \\ \sin \varphi
	\end{pmatrix} \in \R^2 \mid 0 < r < 1, 0 < \varphi < \alpha \right\}.
\]
Definiere die Funktion
\[
	v(x) \coloneqq r^{\frac{\pi}{\alpha}} \sin \left( \frac{\pi \varphi}{\alpha} \right),
\]
für
\[
	x = r\left( \frac{\cos \varphi}{\sin \varphi} \right) \in \Omega_\alpha.
\]
Übungsaufgabe: $\lapl v(\alpha) = 0$.
Wähle $w \in C^\infty(\R^2; \R)$ mit $\supp w \subseteq B_{\frac{2}{3}}(0)$ und $w \equiv 1$ in $B_{\frac{1}{3}}(0)$.
Setze $u \coloneqq w v$ und $f = \lapl(v (1 - w))$.
Dann gilt
\begin{IEEEeqnarray*}{rCl'l}
- \lapl u &=& f & \text{in } \Omega_\alpha, \\
u &=& 0 & \text{auf } \partial \Omega_\alpha.
\end{IEEEeqnarray*}
Offensichtlich gilt $(1 - w)v \in C^\infty(\Omega_\alpha)$, somit $f \in C^\infty(\Omega_\alpha)$ und $u \notin C^\infty \left( \bar{\Omega}_\alpha \right)$!
Eine ausführliche Rechnung zeigt:
\[
	u \in C^k(\bar{\Omega}_\alpha) \; \Leftrightarrow \; 0 < \alpha \leq \frac{\pi}{k}, \; (k \geq 1).
\]
\end{example}
Der Sobolev-Raum
\[
	\underbrace{ H^1(\Omega) }_{= W^{1, 2}(\Omega)} \coloneqq \left\{ v \in L^2(\Omega) \mid v \text{ ist 1-mal schwach differenzierbar und } \nabla v \in L^2(\Omega ; \R^1) \right\}
\]
spielt die zentrale Rolle im Konzept der schwachen Lösung.
Wir erinnern uns: $H^1(\Omega)$ ist ein Hilbert-Raum mit dem Skalarprodukt
\begin{IEEEeqnarray*}{rCl}
	(\cdot, \cdot)_{H^1(\Omega)} \; : \; H^1(\Omega) \times H^1(\Omega) &\to& \R \\
	(u, v) &\mapsto& \int_\Omega \nabla u \cdot \nabla v + u \cdot v \dx.
\end{IEEEeqnarray*}
Die induzierte Norm ist
\[
\| v \|_{H^1(\Omega)} \coloneqq \sqrt{\| \nabla v \|_{L^2}^2 + \| v \|_{L^2(\Omega)}^2}.
\]
\begin{remark}[Spuren von Sobolev Funktionen] % Bem 2.2.1
\label{bem:c2e2s1}
Sei $u \in H^1(\Omega)$. Definiere die Spur $\gamma_0 u : \partial \Omega \to \R$ durch
\[
(\gamma_0 u)(x) \coloneqq \begin{cases}
\lim_{r \to 0} \frac{1}{B_r(x) \cap \Omega} \int_{B_r(x) \cap} u(y) \dy, & \text{falls der Grenzwert existiert} \\
0, & \text{sonst}.
\end{cases}
\]
Es gilt: Der Spuroperator $\gamma_0$ ist ein beschränkter, linearer Operator von $H^1(\Omega)$ nach $L^2(\partial \Omega)$.
\[
	\exists C_{trace} \, \forall u \in H^1(\Omega) \; : \; \| \gamma_0 u \|_{L^2(\partial \Omega)} \leq C_{trace} \| u \|_{H^1(\Omega)}.
\]
Das Bild von $\gamma_0$ ist eine abgeschlossener Unterraum von $L^2(\partial \Omega)$ und wird mit
\[
	H^{\frac{1}{2}}(\partial \Omega) \coloneqq \lim \gamma_0.
\]
$H^\frac{1}{2}(\partial \Omega)$ ist ein Hilbert-Raum.
\end{remark}
\begin{definition}[Der Sobolev-Raum \unboldmath $H^1_D(\Omega)$] % Def 2.2.2
\label{def:c2e2s2}
Sei $\Gamma_D \subseteq \partial \Omega$ abgeschlossen mit positivem Oberflächenmaß $|\Gamma_D| > 0$.
Definiere
\[
	H^1_D(\Omega) \coloneqq \big\{ v \in H^1(\Omega) \mid \underbrace{ \gamma_0 v = 0 \quad \text{auf } \Gamma_D }_{v|_{\Gamma_D} = 0} \big\}.
\]
Falls $\Gamma_D = \partial \Omega$, dann schreiben wir auf $H_0^1(\Omega) = H_D^1(\Omega)$.
\end{definition}
\begin{lemma}[Friedrichs'sche Ungleichung] % Lemma 2.2.3
\label{thm:c2e2s3}
Es gibt $C_F > 0$, sodass
\[
	\forall u \in H_D^1(\Omega) \; : \; \| u \|_{L^2(\Omega)} \leq C_F \frac{\diam(\Omega)}{\pi} \|  \nabla u \|_{L^2(\Omega; \R^d)}.
\]
Die Konstante $C_F$ hängt nur von Lage und Größe von $\Gamma_D$ relativ zu $\Gamma$ ab, aber nicht von $\diam \Omega$. Falls $\Gamma_D = \partial \Omega$, dann $C_F \leq 1$.
\end{lemma}
\begin{remark} % Bem 2.2.4
\label{bem:c2e2s4}
$H_D^1(\Omega)$ ist ein Hilbertraum.
Die Friedrischs'sche Ungleichung zeigt, dass die $H^1$-Seminorm $|v|_{H^1} \coloneqq \| \nabla v \|_{L^2}$ eine Norm auf $H_D^1(\Omega)$ definiert, die äquivalent zur vollen $H^1$-Norm ist.
\begin{IEEEeqnarray*}{rCl}
	|v|_{H^1} \leq \| v \|_{H^1} &=& \sqrt{\underbrace{ \| v \|_{L^2}^2 }_{C_F \frac{\diam \Omega}{\pi} |v|_{H^1}^2} + |v|^2_{H^1}} \\
	&\leq& \sqrt{1 + C_F \frac{\diam \Omega}{\pi}} |v|_{H^1}.
\end{IEEEeqnarray*}
\end{remark}
\begin{definition}[schwache Lösung] % Def 2.2.5
Eine Funktion $u \in H_D^1(\Omega)$ heißt schwache Lösung von \labelcref{eq:c2e1s1} mit gemischten Randbedingungen \labelcref{eq:c2e1s3}, falls gilt
\[
	\forall v \in H_D^1(\Omega) \; : \; \underbrace{ \int_\Omega (A \nabla u) \cdot \nabla v + (\beta \cdot \nabla u) + cuv \dx }_{a(u, v)} = \underbrace{ \int_\Omega fv \dx + \int_{\Gamma_N} gv \ds }_{F(v)}.
\]
\end{definition}
\end{document}