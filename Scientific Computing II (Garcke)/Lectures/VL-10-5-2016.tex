\documentclass[../lecture-notes.tex]{subfiles}

\begin{document}
\begin{remark}
One can build knowledge into this setup by considering
\[
\bar{f} = f + h, \quad f \in \spH, h \in \spn\{ \psi_p \}
\]
where $\psi_p$ can be specific functions having interpretability.
Then one obtains a semiparametric representation
\[
\bar{f}(x) = \sum_{i=1}^N \alpha_i K(x, x_i) + \sum_{p=1}^M \beta_p \psi_p(x).
\]
\end{remark}
\begin{remark}
If both the loss function and the regularization $s$ are convex, one has a unique minimum.
\end{remark}
On $L_2(\Omega)$, we have
\[
	\langle f, g \rangle_{L_2(\Omega)} = \int_\Omega fg \dx.
\]
We are aiming for
\[
	\langle f, g \rangle_{\spH} = \langle Sf, Sg \rangle_{L_2} = \int_\Omega Sf(x) Sg(x) \dx.
\]
The transformation $S$ picks those parts of $f$ which should be regularized. Now think of $S$ as extracting derivations and we see that it is $S$ for smoothness.
\begin{definition} % Definition 42
\label{thm:42}
A \keydef{regularization operator} $S$ is defined as a linear map from the space of functions $\{ f \mid f : \Omega \to \R \}$ into a space $D$ equipped with a scalar product.
The regularization term $s(f)$ takes the form
\[
	s(f) \coloneqq \langle Sf, Sf \rangle_D
\]
or sometimes
\[
	s(f) \coloneqq \frac{1}{2} \langle Sf, Sf \rangle_D.
\]
\end{definition}
\begin{remark}
Since we can always define $\tilde{S} \coloneqq (S^\dualst S)^{\frac{1}{2}}$ and
\[
	\langle f, S^\dualst S f \rangle_{\spL} = \langle Sf, Sf \rangle_D,
\]
we can assume $S$ is a positive semidefinite regularization operator.
\end{remark}
\begin{theorem}[Mercer's theorem] % Theorem 43
\label{thm:43}
\index[keydefinitions]{Mercer's theorem}
Let $\Omega$ be a compact domain, $\nu$ a Borel measure on $\Omega$, and $K : \Omega \times \Omega$ symmetric and continuous.
The corresponding integral operator $T_K$ shall be positive semidefinite, i.e.
\[
	\int_{\Omega \times \Omega} K(x, y) f(x) f(y) \geq 0 \quad \text{for all } f \in L_2^\nu(\Omega).
\]
Let $\lambda_i$ be the $i$-th eigenvalue of $T_K$ and $\{ \psi_i \}_{i \geq 1}$ the corresponding and normalized eigenfunctions.
It then holds
\[
	K(x, y) = \sum_{i=1}^\infty \lambda_i \cdot \phi_i(x) \phi_i(y) \quad \text{for all } x, y \in \Omega,
\]
where the convergence is absolute (for each $x, y \in \Omega \times \Omega$) and uniform (on $\Omega \times \Omega$).
\end{theorem}
\begin{proof}
A proof is given in \cite{Hochstadt} for $\Omega = [0, 1]$ and the Lebesgue measure, but the proof is valid for this more general situation.
\end{proof}
\begin{remark}
More generally, Hilbert-Schmidt theory ensures the existence of $L_2$-convergent eigenfunction expansions of compact, self-adjoint operators.
\end{remark}
We go over to weighted inner products with a weight function $\sigma$:
\[
	\langle f, g \rangle = \int_\Omega f(x) g(x) \sigma(x) \dx.
\]
The eigenvalue problem of the integral operator $T_K$ is
\[
	\int_\Omega K(x, z) \phi(x) \sigma(x) \dx = \langle K(x, z), \phi(x) \rangle = \lambda \phi(z).
\]

We consider the (non-homogeneous) linear (ordinary or partial) differential equation
\[
	(Lu)(x) = f(x) \quad \text{on } \Omega \subset \R^d
\]
with a linear and elliptic operator $L$ and some suitable boundary conditions.
The solution can be written with Green's function $g$
\[
	u(x) = \int_\Omega f(z) g(x, z) \dd z,
\]
where for $g$ it holds
\[
	(Lg)(x, z) = \delta(x - z),
\]
where $z$ denotes a fixed and arbitrary ``source''.
\begin{example}[Brownian bridge kernel]
Let $\Omega = [0, 1]$ and
\begin{IEEEeqnarray*}{rCl}
	K(x, z) &=& \min(x, z) - xz \\
	&=& \begin{cases}
	x(1-z) & x \leq z \\
	z(1-x) & x \geq z
	\end{cases}.
\end{IEEEeqnarray*}
This kernel may be obtained by observing properties of the Green's function for
\begin{IEEEeqnarray*}{rCl}
	- \frac{\diffd^2}{\diffd x^2} K(x, z) &=& \delta(x - z) \\
\noalign{\noindent with boundary conditions \vspace{\jot}}
	K(0, z) &=& K(1, z) = 0.
\end{IEEEeqnarray*}
For example 
\begin{itemize}
\item $y$ is continuous along the diagonal $x=z$
\item $\frac{\diffd y}{\diffd x}$ has for any fixed $z \in (0, 1)$ a jump discontinuity at $x = z$
\item $y$ is piecewise linearly defined on $[0, 1]^2$.
\end{itemize}
Therefore $K$ is the Green's function for $L = \frac{\diffd^2}{\diffd x^2}$.
\end{example}
\begin{remark}
It can be seen that the Green's function, and therefore the corresponding kernel, are symmetric for self-adjoint differential operators $L$.
\end{remark}
\begin{theorem} % Theorem 44
\label{thm:44}
For every \ac{RKHS} $\spH$ with reproducing kernel $K$, there exists a corresponding regularization operator $S : \spH \to D$, such that for all $f \in \spH$,
\begin{equation}
\opttag{$\star$}
\label{eq:RegRKf}
	f(x) = \langle SK(x, \cdot), Sf \rangle_D 
\end{equation}
and in particular
\[
	\langle SK(x, \cdot), SK(z, \cdot) \rangle_D = K(x, z).
\]
Likewise, for every regularization operator $S : \spL \to D$, where $\spL$ is some function space equipped with a scalar product and with a corresponding Green's function for $S^\dualst S$, there exists a corresponding \ac{RKHS} $\spH$, with reproducing kernel $K$, such that both equations are satisfied.
\end{theorem}
\begin{proof}
For the first direction, we consider $S = \text{Id}$ and $D = \spH$. This construction fulfills all wanted properties.

Now we start with a function $G_x$ which fulfills
\[
	f(x) = \langle S^\dualst S G_x, f \rangle_\spL \quad \text{for all } f \in S^\dualst S \spL.
\]
This exists as the Green's function for the operator $S^\dualst S$.
\begin{IEEEeqnarray*}{rCl}
	f(x) &=& \langle \underbrace{ S^\dualst S }_{L} G_x, f \rangle_\spL \\
	&=& \langle L G_x, f \rangle \\
	&=& \langle f, \delta(z -x) \\
	&=& f(x).
\end{IEEEeqnarray*}
We have the reproduction property \cref{eq:RegRKf} on the set $S^\dualst S \spL$ using the properties of the adjoint:
\[
	\langle S^\dualst S G_x, f \rangle_{\spL} = \langle S G_x, S f \rangle_D.
\]
With $f = G_z$ it follows 
\begin{IEEEeqnarray*}{rCl}
G_z(x) &=& \langle SG_x, SG_z \rangle_D \\
&=& \langle SG_z, SG_x \rangle_D \\
&=& G_x(z),
\end{IEEEeqnarray*}
i.e.\ $G$ is symmetric in this sense and we write
\[
	g(x, y) = G_z(x),
\]
with
\begin{IEEEeqnarray*}{rCl}
	g(x, z) &=& \langle S G_x, S G_z \rangle_D
\end{IEEEeqnarray*}
we notice that $x \mapsto SG_x$ is actually a feature map.
With \cref{thm:28}, we know that this is a positive semidefinite kernel and we define
\[
	K(x, z) = g(x, z).
\]
It can be seen that the corresponding \ac{RKHS} is the closure of 
\[
	\left\{ f \in S^\dualst S \spL \gmid \| Sf \|_D^2 \leq \infty \right\}.
\]
\end{proof}
$D$ is a \ac{RKHS} with inner product $\langle S \cdot, S \cdot \rangle_D = \langle \cdot, \cdot \rangle_\spH$.

We want to connect
\begin{itemize}
\item integral operator eigenvalue problem
\item related eigenvalue problem for the differential operator
\end{itemize}
To simplify the situation, we consider the free-space\slash{}fullspace Green's function with boundary conditions.
We use $g$ as a kernel $K$, i.e.\ $(LK)(x, z) = \delta(x - z)$.
Now applying $L$ to the integral equation gives
\begin{IEEEeqnarray*}{c"rCl}
	& \underbrace{ L \int_\Omega K(x, z) \phi(x) \sigma(x) \dx }_{\int_\Omega L K(x, z) \phi(x) \sigma(x) \dx} &=& L \gamma \phi(z) \\
\Longleftrightarrow	& \underbrace{ \int_\Omega \delta(x - z) \phi(x) \dx }_{\phi(z) \sigma(z)} &=& \gamma L \phi(z) \\
\implies & L \phi(z) &=& \frac{1}{\gamma} \phi(z) \sigma(z)
\end{IEEEeqnarray*}
where for simplicity we assume that $L$ has no eigenvalue $0$.
This shows that $L$ has eigenvalues which are the reciprocals of the eigenvalues of $T_K$, while the corresponding eigenfunctions are the same, taking the weight function into account.
\end{document}