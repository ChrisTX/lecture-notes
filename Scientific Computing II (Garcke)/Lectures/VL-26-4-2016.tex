\documentclass[../lecture-notes.tex]{subfiles}

\begin{document}
\subsection*{Power function and stability}
A kind of uncertainty principle holds:
It is possible to make the power function and the condition of the kernel matrix small at the same time.
We express now the power function via the kernel matrix.
Besides the set $X = \{ x_1, \ldots, x_N \}$ we now take another point $x_0 \coloneqq x$ and we define $u_0(\cdot) \coloneqq - 1$.
We define
\[
	A = K(x_i, x_k)_{0 \leq i, j \leq N}
\]
and
\[
	u \coloneqq \left( u_0(x), u_1(x), \ldots, u_N(x)\right)^\tp.
\]
Now look at the quadratic form (and remember $K(x_0, x_0) \coloneqq K(x, x)$):
\begin{IEEEeqnarray*}{rCl}
	u^\tp A u &=& \sum_{j=0}^N \sum_{k=0}^N u_j(x) u_k(x) K(x_j, x_k) \\
	&=& K(x, x) - 2 \sum_{j=1}^N u_j(x) K(x, x_j) + \sum_{j=1}^N \sum_{k=1}^N u_j(x) u_k(x) K(x_j, x_k) \\
	&\overset{\text{\cref{thm:25}}}{=}& P_X^2(x).
\end{IEEEeqnarray*}
$A$ is symmetric and positive semidefinite, therefore has $N + 1$ non-negative real eigenvalues $\lambda_0 \geq \lambda_1 \geq \ldots \geq \lambda_N \geq 0$ and we obtain
\[
	P_X^2(x) \geq \lambda_N \left( 1 + \sum_{j=1}^N u_j(x)^2 \right) \geq \lambda_N,
\]
where we can use
\[
	\lambda_N \| u \|_2^2 \leq u^\tp A u \leq \lambda_0 \| u \|_2^2. 
\]
We eliminate the special role of the point $x$ and obtain the following:
\begin{theorem} % Theorem 27
\label{thm:27}
The kernel matrix for $N$ points $x_1, \ldots, x_N$ forming a set $X$ has a smallest eigenvalue $\lambda$ bounded from above by
\[
	\lambda \leq \min_{1 \leq j \leq N} P_{X \setminus \{ x_j \}} (x_j).
\]
\end{theorem}
This gives us information about the condition of the kernel matrix.
In situations where the power function is still small after one point is left out, the kernel matrix must be ill-conditioned.
\section{Hilbert spaces from kernels} % Section 1.3
\begin{theorem} % Theorem 28
\label{thm:28}
Kernels arising from feature maps, i.e.
\[
	K(x, y) = \langle \Phi(x), \Phi(y) \rangle_{\spF}
\]
are positive semidefinite.
\end{theorem}
\begin{proof}
Such kernels result in kernel matrices which are Gram matrices, and these are always positive semidefinite.
\end{proof}
\begin{remark}
A large class of kernels have the form
\[
	K(x, y) = \sum_{i \in I} \lambda_i \varphi_i(x) \varphi_i(y),
\]
where $I$ is a countable index set and $\varphi_i : \Omega \to (\R, i \in I)$, with $\lambda_i \geq 0$, $i \in I$, and the summability condition holds for all $x \in \Omega$:
\[
	K(x, x) = \sum_{i \in I} \lambda_i |\varphi_i(x)|^2 < \infty.
\]
We call these \emph{expansion kernels}.
\end{remark}
\begin{theorem} % Theorem 29
\label{thm:29}
Let $K$ be a symmetric positive semidefinite kernel on $\Omega$. Then
\begin{enumerate}
\item\label{thm:29-1}$K(x, x) \geq 0$ for all $x \in \Omega$.
\item\label{thm:29-2} $K(y, x) = K(x, y)$ for all $x, y \in \Omega$.
\item\label{thm:29-3} $|K(x, y)|^2 \leq K(x, x) K(y, y)$ for all $x, y \in \Omega$.
\item\label{thm:29-4} $2|K(x, y)|^2 \leq K(x, x)^2 + K(y, y)$ for all $x, y \in \Omega$.
\item\label{thm:29-5} Any finite linear combination of positive semidefinite kernels with nonnegative coefficients gives a positive semidefinite kernel. If one of these kernels is positive definite, and its coefficient is positive, then the combination of kernels is positive definite.
\item\label{thm:29-6} Additionally the product of two positive semidefinite kernels is positive semidefinite.
\item\label{thm:29-7} The product of two positive definite kernels is positive definite.
\end{enumerate}
\end{theorem}
\begin{proof}
\begin{enumerate}
\item This follows with the point set $\{ x \} \subset \Omega$ in \cref{thm:22}.
\item One can conclude this statement directly from the definition of a symmetric kernel.
\item The determinant of a positive semidefinite symmetric matrix is nonnegative, therefore this follows from the kernel matrix for the set $\{ x, y\}$.
\item With the inequality
\[
	2 ab \leq a^2 + b^2 \quad \text{for } a, b \in \R_0^+
\]
this follows from \cref{thm:29-3}.
\item This property is easy to see, just expand the sum $x^\tp K x$.
\item The property follows from \cref{thm:30}.
\item The property follows from the proof of \cref{thm:30}. One can repeat the same proof with strict inequalities due to linear algebra.
\end{enumerate}
\end{proof}
\begin{lemma}[Schur's Lemma] % Lemma 30
\label{thm:30}
For two matrices $A$, $B$ the matrix $C$ with elements
\[
	C_{ij} = A_{ij} B_{ij}
\]
is called \emph{Schur product} or \emph{Hadamard product}.
The Schur product of two positive semidefinite matrices is positive semidefinite.
\end{lemma}
\begin{proof}
We can decompose a positive semidefinite matrix into
\[
	A = S^\tp D S
\]
with $S$ an orthogonal matrix and $D = \diag(\lambda_1, \ldots, \lambda_N)$ a diagonal matrix with $\lambda_i \geq 0$ the eigenvalues of $A$.
For all $q \in \R^N$, we look at
\begin{IEEEeqnarray*}{rCl}
q^\tp C q = \sum_{j, k =1}^N q_j q_k \overbrace{a_{jk} b_{jk}}^{c_{jk}} &=& \sum_{j, k =1}^N q_j q_k b_{jk} \sum_{m=1}^N \lambda_m S_{jm} S_{km} \\
&=& \sum_{m=1}^N \lambda_m \sum_{j,k=1}^N \underbrace{ q_j S_{jm} }_{p_{jm}} \underbrace{ q_k S_{km} }_{p_{km}} b_{jk} \\
&=& \sum_{m=1}^N \lambda_m \underbrace{ \sum_{j, k = 1}^N p_{jm} p_{km} b_{jk} }_{\geq 0, \text{ since $B$ is positive semidefinite}} \geq 0.
\end{IEEEeqnarray*}
This however is exactly what we had to prove.
\end{proof}
Let $K$ be a symmetric positive semidefinite kernel on $\Omega$.
We define the linear space
\[
	H \coloneqq \spn \left\{ K(x, \cdot ) \mid x \in \Omega \right\}
\]
of all finite linear combinations of translates of the kernel.
In the same way, we define the linear space
\[
	L \coloneqq \spn \left\{ \delta_x \mid x \in \Omega, \; \delta_x : H \to \R \right\}
\]
of all finite linear combinations of point evaluation functionals acting on functions in $H$.
In particular, we explicitly restrict the action to functions in $H$.
We can write all elements from $L$ and $H$ as
\begin{IEEEeqnarray*}{rCl}
	\lambda_{a, X} &\coloneqq& \sum_{j=1}^N a_j \delta_{x_j} \\
	f_{a, X} (x) &\coloneqq& \lambda_{a, X}^y K(x, y) = \sum_{j=1}^N a_j K(x, x_j)
\end{IEEEeqnarray*}
with some $a \in \R^N$, $X = \{ x_1, \ldots, x_N \} \subset \Omega$, where $X$ is any arbitrary finite subset of $\Omega$.
Unfortunately, from $f_{a, X} (\cdot) = 0$ on $\lambda_{a, X}(\cdot) = 0$ it does not follow that $a = 0$, we head to observe that, but it will pose no problem.
We define a bilinear form on $L$:
\begin{equation}
\label{eq:LBilinearForm}
\opttag{$\star$}
\begin{IEEEeqnarraybox}[][c]{rCl}
	\langle \lambda_{a, X}, \lambda_{b, Y} \rangle &\coloneqq& \sum_{j = 1}^M \sum_{k = 1}^N a_j b_k K(x_j, y_k) \\
	&=& \lambda_{a, X}^x \lambda_{b, Y}^y K(x, y) \\
	&=& \lambda_{a, X} \left( f_{b, Y} \right).
\end{IEEEeqnarraybox}
\end{equation}
This is well-defined, since it is independent of the specific representation, i.e.\ $a$, $b$, due to the functionals and their actions on the second line.
We observe that the form is positive semidefinite since the kernel matrix has this property.
Further, we have
\begin{equation}
\label{eq:LBilinearEstimate}
\opttag{$\star\star$}
\begin{IEEEeqnarraybox}[][c]{rCl}
	\left| \lambda_{a, X} \left( f_{b, Y} \right) \right| &=& \left| \left\langle \lambda_{a, X}, \lambda_{b, Y} \right\rangle_L \right| \\
	&\leq& \left\| \lambda_{a, X} \right\|_L \left\| \lambda_{a, Y} \right\|_L
\end{IEEEeqnarraybox}
\end{equation}
where it may be just a seminorm, not a norm.	
\end{document}