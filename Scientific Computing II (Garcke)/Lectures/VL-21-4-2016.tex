\documentclass[../lecture-notes.tex]{subfiles}

\begin{document}
Consider $f(\cdot) = K(x, \cdot)$ in \cref{thm:20}.
We therefore know that for every $x \in \Omega$
\begin{equation}
\opttag{$\star\star$}
\label{eq:two-star-finite}
	K(x, x_k) = \sum_{j=1}^N u_j(x) K(x_j, x_k), \quad 1 \leq k \leq N
\end{equation}
has a solution $u_j(x)$ as a function on $\Omega$.
Additionally, it holds for all $x, z \in \Omega$
\[
	K(x, \cdot)_{X} (z) = \sum_{j=1}^N u_j(x) K(x_j, z),
\]
i.e. the interpolant of $K(x, \cdot)$ on $X$.
\begin{theorem} % Theorem 24
\label{thm:24}
The $u_j$ in \cref{eq:two-star-finite} are in $\spH_X$ and a Lagrange basis, i.e.
\[
	u_j(x_k) = \delta_{j, k} \quad 1 \leq j, k \leq N
\]
if the kernel matrix is nonsingular. In general, we still have
\[
	f_X = \sum_{j=1}^N u_j(x) f(x_i)
\]
\end{theorem}
\begin{proof}
The first statement follows from \cref{eq:two-star-finite}.
For the second one
\begin{IEEEeqnarray*}{rCl}
f_X(\cdot) &=& \sum_{k = 1}^N \alpha_k K(x_k, \cdot) \\
&\overset{\text{\cref{eq:two-star-finite}}}{=}& \sum_{k=1}^N \alpha_K \sum_{j=1} u_j(\cdot) K(x_j, x_k) \\
&=& \sum_{j=1}^N u_j(\cdot) \sum_{k=1}^N \alpha_k K(x_i, x_k) \\
&=& \sum_{j=1}^N u_j(\cdot) f(x_j).
\end{IEEEeqnarray*}
This proves everything
\end{proof}
We consider now only $f = K(x, \cdot)$ for a fixed $x \in \Omega$.
\addtocounter{dummythm}{-7}
\begin{definition} % Definition 18
\label{thm:18}
The function
\[
	P_X(x) \coloneqq \| K(x, \cdot) - K(x, \cdot)_X \|_{\spH} \quad x \in \Omega
\]
is called \emph{power function} with respect to the set $X$ and the kernel $K$.
\end{definition}
A different definition goes with the error functional
\[
	\epsilon_{x, X} f \mapsto f(x) - (\Pi_X(f))(x).
\]
which is in $\spHstar$ and the power function is defined as
\[
	P_X(x) \coloneqq \| \epsilon_{x, X} \|_{\spHstar} \quad \text{for all } x \in \Omega.
\]
\begin{theorem} % Theorem 19
\label{thm:19}
The error definitions are equivalent. Furthermore, the power function has the properties
\begin{enumerate}
\item $P_X(x) = 0$ for all $x \in X$.
\item $P_0(X)^2 = K(x, x)$ for all $x \in \Omega$
\item $P_\Omega(x) = 0$ for all $x \in \Omega$
\item $0 = P_{\Omega} \leq P_Y(x) \leq P_X(x) \leq P_0(x)$ for all $x \in \Omega$, $X \subseteq Y \subseteq \Omega$
\item $P_X(x) = \inf_{g \in \spH_{X}} \| K(x, \cdot) - g \|_{\spH}$
\item $P_X(x) = \sup_{\substack{f \in \spH, \| f \|_{\spH} \leq 1 \\ f(X) = \{ 0 \}}} f(x)$ for all $x \in \Omega$
\end{enumerate}
and finally th important error bound
\begin{IEEEeqnarray*}{rCl}
|f(x) - f_X(x) | &=& \left| f_X^\perp (x) \right| \\
&\leq& P_X(x) \left\| f_X^\perp \right\|_{\spH} \\
&=& P_X(x) \| f - f_X \|_{\spH} \\
&\leq& P_X(x) \| f \|_{\spH}
\end{IEEEeqnarray*}
for all $x \in \Omega, f \in \spH$.
\end{theorem}
\addtocounter{dummythm}{5}
\begin{proof}
For the equivalence, we show that the Riesz representer of $\delta_x \circ \Pi_X$ is $K(x, \cdot)_X$.
\begin{IEEEeqnarray*}{rCl}
\langle f, R(\delta_x \circ \Pi_X) \rangle_{\spH} &=& \delta_X \circ \Pi_X(f)) \\
&=& f_X(x) \\
&=& \langle f_X, K(x, \cdot) \rangle_{\spH} \\
&=& \langle f_X, K(x, \cdot)_X + K(x, \cdot)_{X^\perp} \rangle_{\spH} \\
&=& \langle f_X, K(x, \cdot)_X \rangle_{\spH} \\
&=& \langle f - f_{X^\perp}, K(x, \cdot)_X \rangle_{\spH} \\
&=& \langle f, K(x, \cdot)_X \rangle_{\spH}
\end{IEEEeqnarray*}
The listed properties are easily derived from \cref{thm:18} and the previous results.
With the error representation
\begin{IEEEeqnarray*}{rCl}
f(X) - f_X(x) &=& f_{X^\perp} (x) \\
&=& \langle f_{X^\perp}, K(x, \cdot) \rangle_{\spH} \\
&=& \langle f_{X^\perp}, K(x, \cdot) - K(x, \cdot)_X \rangle_{\spH}
\end{IEEEeqnarray*}
The error bound follows.
For the sixth property, we see from the first inequality of the error bound that
\[
	P_X(x) \geq \sup_{\left\| f_{X^\perp} \right\| \leq 1} f_{X^\perp} (x)
\]
and equality must hold for the representation of $\epsilon_{x, X}$.
\end{proof}
\begin{remark}
Consider the subspace
\[
	\spHstar_{X} \coloneqq \overline{ \spn \left\{ \delta_x \mid x \in X \right\} }
\]
of the dual space. Then the fifth property can be equivalently be given as
\[
	P_X(x) = \inf_{\lambda \in \spHstar_X} \| \delta_x - \lambda \|_{\spHstar} \quad \text{for all } x \in \Omega.
\]
It indicates how well the point evaluation functional $\delta_x$ can be approximated by arbitrary linear combinations of the $\delta_{x'}$ for $x' \in X$.
\end{remark}
Using again
\[
% (***) in he lecture, but already defined above, see two-star-finite.
K(x, \cdot)_X (z) = \sum_{j=1}^N u_j(x) K(x_j, z)
\]
and plug this into the power function. We set
\begin{theorem} % Theorem 25
\label{thm:25}
The power function has the explicit representation
\begin{IEEEeqnarray*}{rCl}
	P_X^2(x) &=& K(x, x) - 2 \sum_{j=1}^N u_j(x) K(x, x_j) + \sum_{j=1}^N \sum_{k=1}^N u_j(x) u_k(x) K(x_j, x_k) \\
	&=& K(x, x) - K(x, \cdot)_X (x)
\end{IEEEeqnarray*}
\end{theorem}
\begin{proof}
\begin{IEEEeqnarray*}{rCl}
P_X^2(x) &=& \langle K(x, \cdot) - K(x, \cdot)_X, K(x, \cdot) - K(x, \cdot)_X \rangle_{\spH} \\
&\overset{\cref{eq:two-star-finite}}{=}& K(x, x) - 2 \langle K(x, \cdot), \sum_{j=1} u_j(x) K(x_j, \cdot) \rangle_{\spH} + \sum_{j=1}^N \sum_{k=1}^N u_j(x) u_k (x) K(x_j, x_k) \\
&=& K(x, x) - 2 \sum_{j=1}^N u_j(x) K(x, x_j) + \sum_{j=1}^N \sum_{k=1}^N u_j(x) u_k (x) K(x_j, x_k)
\end{IEEEeqnarray*}
Now as we can use
\[
	K(x, x_k) = \sum_{j=1}^N u_j(x) K(x_j, x_k)
\]
in second term, it cancels out once with the last term, and identify the remainder with 
\[
K(x, \cdot)_X (x).
\]
\end{proof}
\subsection*{Best linear estimation}
Consider
\[
	f_X = \sum_{j=1}^N u_j(\cdot) f(x_j)
\]
from \cref{thm:24} in the interpolant on the set $X$. We will see in what kind of sense it is the best linear predictor\slash{}estimator. This is important when the kernel comes from a covariance, i.e.
\[
	K(s, t) \coloneqq \Cov (X_s, X_t),
\]
where for every $t \in \Omega$ we have a random variable $X_t$ with existing second moments.

Now let us consider completely arbitrary estimation formulas
\[
	(x, f) \mapsto \sum_{j=1}^N v_j(x) f(x_j)
\]
with no assumption on $v_j(x)$.
These representations are linear in $f$.
For fixed $x$ we get the error functional
\[
	f \mapsto f(x) - \sum_{j=1}^N v_j(x) f(x_j) = \left( \delta_x - \sum_{j=1}^N v_j(x) \delta_{x_j} \right)(f).
\]
We want to have the estimation to be optimal over all $f \in \spH$, therefore we chose the $v_j(x)$ to minimize
\[
	V_{X, v} (x) \coloneqq \left\| \delta_x - \sum_{j=1}^N v_j(x) \delta_{x_j} \right\|_{\spHstar}.
\]
% Original: By (*) from the remark after \cref{thm:19} ... (*) wasn't defined?
By the equivalent formulation of error function that we listed in the remark after \cref{thm:19} we know the solution, namely the functions $u_j$ and the optimal error in the worst case sense is described by the power function.
\begin{theorem} % Theorem 26
\label{thm:26}
In the above worse case sense, kernel based interpolation yields the best linear predictor of unknown function values $f(x)$ from known functions values $f(x_j)$ at points $x_j$, $1 \leq j \leq N$.
\end{theorem}
Consider the kernel via covariance, with $X_t$ with mean zero and bounded variance. In this case the numerical estimation technique is called Kriging, and $V_{X, v}^2$ is the variance of the prediction error.
We define the error of the general linear predictor at $x$ by 
\[
	\epsilon_{x, X, v} \coloneqq X_{x} - \sum_{j=1}^N v_j(x) X_{x_j}.
\]
It has zero mean, and the variance
\[
	\mathbb{E} \left(\epsilon_{x, X, v}^2 \right) = \Cov (X_x, X_x) - 2\sum_{j=1}^N v_j(x) \Cov(X_x, X_{x_j}) + \sum_{j=1}^N \sum_{k=1}^N v_j(x) v_k(x) \Cov(X_{x_j}, X_{x_k}) \\
\]
Because
\[
	\Cov(X, Y) = K(x, x)
\]
and
\[
	K(x, y) = \langle \delta_x, \delta_y \rangle_{\spH}
\]
this yields:
\[
	\mathbb{E} \left(\epsilon_{x, X, v}^2 \right) = V_{X, v}^2.
\]
\end{document}