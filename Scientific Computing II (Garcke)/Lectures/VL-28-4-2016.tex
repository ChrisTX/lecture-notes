\documentclass[../lecture-notes.tex]{subfiles}

\begin{document}
\addtocounter{dummythm}{1} % Theorem 31 does apparently not exist?
\begin{theorem} % Theorem 32
\label{thm:32}
If $K$ is a positive semidefinite kernel on $\Omega$, then the bilinear form $\langle \cdot, \cdot \rangle_L$ from \cref{eq:LBilinearForm} is positive definite on the space $L$ of functionals defined on function on $\Omega$. Thus $L$ is a pre-Hilbert space of functions on $\Omega$.
\end{theorem}
\begin{proof}
Assume that
\begin{IEEEeqnarray*}{rCl"l}
\langle \lambda_{a, X}, \lambda_{a, X} \rangle_L &=& \lambda_{a, X} \left( f_{a, X} \right) = 0 & \text{for } a \in \R^N \text{, and } X \subset \Omega.
\end{IEEEeqnarray*}
Then by the equality \cref{eq:LBilinearEstimate} we have $\lambda_{a, X} = 0$ as a functional on $H$.
Here we use that the functionals in $L$ are restricted to functions in $H$. Note that we do not get $a = 0$, neither will we need that.
\end{proof}
\begin{theorem} % Theorem 33
\label{thm:33}
The mapping $R : \lambda_{a, X} \mapsto f_{a, X} \coloneqq \lambda_{a, X} \left(K(\cdot, y) \right)$ is linear and bijective from $L$ onto $H$.
Thus
\begin{IEEEeqnarray*}{rCl}
\langle f_{a, X}, f_{b, Y} \rangle_H &\coloneqq& \langle \lambda_{a, X}, \lambda_{b, Y} \rangle_{L} \\
&=& \left\langle R\left(\lambda_{a, X}\right), R\left(\lambda_{b, Y}\right) \right\rangle_H
\end{IEEEeqnarray*}
is an inner product on $H$.
$R$ acts as the Riesz map.
\end{theorem}
\begin{proof}
Linearity is obvious.
If a $f_{b, Y} = R(\lambda_{b, Y}) \in H$ vanishes, the definition of $\langle \cdot, \cdot \rangle_L$ implies that $\lambda_{b, Y}$ is orthogonal to all of $L$, thus zero due to the \cref{thm:32}.
Thus we have bijectivity.
The Riesz property is present in the definition of $\langle \cdot, \cdot \rangle_L$, since
\begin{IEEEeqnarray*}{rCl}
\lambda_{a, X}(f_{b, Y}) &=& \langle \lambda_{a, X}, \lambda_{b, Y} \rangle_L \\
&=& \langle f_{a, X}, f_{b, Y} \rangle_H \\
&=& \langle R(\lambda_{a, X}), f_{b, Y} \rangle_H,
\end{IEEEeqnarray*}
which showed everything.
\end{proof}
We get \cref{eq:starRE} in $H$ by inserting $\lambda_{a, X}$ in the bilinear form, i.e.\ one point $x \in \Omega$, with $a = 1$:
\begin{IEEEeqnarray*}{rCl}
f_{b, Y} (x) &=& \lambda_{1, X} (f_{b, Y}) \\
&=& \langle \lambda_{1, X}, \lambda_{b, Y} \rangle_L \\
&=& \langle R(\lambda_{1, X}), f_{b, Y} \rangle_H \\
&=& \langle K(x, \cdot), f_{b, Y} \rangle_H
\end{IEEEeqnarray*}
with $\lambda_{b, Y} = \lambda_{1, Y}$ it follows
\[
	\langle K(x, \cdot), K(y, \cdot) \rangle_H = \langle \delta_{x}, \delta_{y} \rangle_L.
\]
With the properties of $H$, we now can follow it being a classic Hilbert space and so to the closure $\spH$ of $H$ under $\langle \cdot, \cdot \rangle_H$.
This is an abstract space defined by equivalence classes of Cauchy sequences in $H$.
The closure is complete, therefore a Hilbert space.
Furthermore, each continuous map from $H$ to a Banach space $Y$ extends uniquely to the closure.
\begin{theorem} % Theorem 34
\label{thm:34}
Each symmetric positive definite kernel $K$ on a set $\Omega$ is the reproducing kernel of a Hilbert space called the \keydef{native space} $\spH \eqqcolon \spN_K$ of the kernel.
This Hilbert space is unique and it is a space of functions on $\Omega$.
The kernel $K$ is the reproducing kernel of $\spN_K$, in the sense of
\[
	\langle f, K(y, \cdot) \rangle_H = f(y) \quad \text{for all } y \in \Omega, f \in \spN_K.
\]
\end{theorem}
\begin{proof}
The existence of the native space follows from standard Hilbert space arguments.
In the reproduction equation
\[
	f_{b, Y}(x) = \langle f_{b, Y}, K(x, \cdot) \rangle_H
\]
both sides continuously depend on $f \in H$, therefore the equation carries over to the closure, i.e.\ the native space, which proves the reproduction formula in the theorem.
The equation explains how an abstract element $f$ of the native space can be interpreted as a function, namely the left hand side $\langle f, K(y, \cdot) \rangle_H$ defines the right hand side $f(y)$.

If $K$ is a reproducing kernel in a possibly different Hilbert space $T$ with an analogous reproduction equation, we see 
\[
	\langle K(x, \cdot), K(y, \cdot) \rangle_H = K(x, y) = \langle K(x, \cdot), K(y, \cdot) \rangle_T.
\]
Therefore the inner products of $T$ and $\spN_K$ coincide on $H$.
Since $T$ is a Hilbert space, it must contain the closure of $H$ which is $\spN_K$, as a closed subspace.
If $T$ were larger than $\spN_K$, there must be a nonzero element $f \in T$ which is orthogonal to $\spN_K$ and in particular orthogonal to $H$.
We observe
\[
	f(y) = \langle f, K(y, \cdot) \rangle_T = 0 \quad \text{for all } y \in \Omega,
\] 
which is a contradiction.
\end{proof}
\needspace{5\baselineskip}
\subsection*{Mercer kernels}
\begin{definition} % Definition 35
\label{thm:35}
A \keydef{Mercer kernel} is a continuous, symmetric and positive semidefinite kernel.
\end{definition}
\begin{definition} % Definition 36
\label{thm:36}
Let $K : \Omega \times \Omega \to \R$ be continuous, $\Omega$ be a compact domain, $\nu$ by a Borel measure and $L_2^\nu(\Omega)$ be the Hilbert space of square integrable functions on $\Omega$.
We define the integral operator $T_K : L_2^\nu(\Omega) \to L_2^\nu(\Omega)$ by
\[
	\left( T_K f \right) (\cdot) = \int_\Omega K(x, \cdot) f(x) \dd \nu
\]
We call $K$ the kernel of $T_K$.
\end{definition}
One can show that
\[
	K(x, y) = \sum_{i=1}^\infty \lambda_i \phi_i(x) \phi_i(y)
\]
for positive semidefinite operators $T_K$.
\begin{theorem} % Theorem 37
\label{thm:37}
Let $\Omega \subset \R^d$ be compact and $K$ be a continuous kernel.
Let the corresponding integral operator $T_K$ be positive semidefinite:
\[
	\int_{\Omega \times \Omega} K(x, y) f(x) f(y) \dd x \dd y \geq 0 \quad \text{for all } f \in L_2(\Omega).
\]
Then we can expand $K(x, y)$ in a uniformly convergent series (on $\Omega \times \Omega$) in terms of functions $\phi_j$, satisfying $\langle \phi_i, \phi_j \rangle = \delta_{ij}$, as
\[
	K(x, y) = \sum_{i=1}^\infty \phi_i(x) \phi_i(y).
\]
Furthermore, the series
\[
	\sum_{i=1}^\infty \| \phi_i \|_{L_2}^2
\]
is convergent.
\end{theorem}
\begin{proof}
Assume there is a finite set $\{ x_1, \ldots, x_N \}$ so that the corresponding kernel matrix is not positive semidefinite.
Let $q$ such that
\[
	\sum_{i,j=1}^N q_i q_j K(x_i, x_j) = \varepsilon < 0.
\]
Now let 
\[
	f_{\sigma}(x) = \sum_{i=1}^d q_i \frac{1}{(2\pi \sigma)^{\frac{d}{2}}} \exp \left( - \frac{\| x - x_i \|^2}{2\sigma} \right).
\]
We have $f_\sigma \in L_2$, and additionally
\[
	\lim_{\sigma \to 0} \int_{\Omega \times \Omega} K(x, y) f_\sigma(x) f_\sigma(y) \dd x \dd y = \varepsilon.
\]
But then for some $\sigma > 0$ the integral will be less than $0$, which contradicts the positivity of the integral operator $T_K$.
Therefore, we have a Mercer kernel.
Now consider the native space $\spN_K$ of $K$.
For continuous $K$ and $\Omega \subset \R^d$, one can show that $\spN_K$ is separable and in particular there exists a countable orthonormal basis of $\spN_K$ called $\phi_i$, $i = 1, \ldots$.
The technical proof of the separability is being omitted here.
Then we have an expansion in the orthonormal basis for $K(x, \cdot)$, namely
\begin{IEEEeqnarray*}{rCl}
	K(x, y) &=& \sum_{i=1}^\infty \langle K(x_i, \cdot), \phi_i(\cdot) \rangle \phi_i(\cdot) \\
	&=& \sum_{i=1}^\infty \phi_i(x) \phi_i(y)
\end{IEEEeqnarray*}
with the required properties.
\begin{remark}
Continuity of the kernel is relevant as the eigenvalue expansion
\[
	K(x, y) = \sum_{i=1}^\infty \lambda_i \phi_i(x) \phi_i(y)
\]
would not work for discontinuous kernels.
\end{remark}
Finally, we observe
\begin{IEEEeqnarray*}{rCl}
	\lim_{n \to \infty} \sum_{i=1}^n \| \phi_i \|_{L_2}^2 &=& \lim_{n \to \infty} \sum_{i=1}^n \int_\Omega \phi_i(x) \phi_i(x) \dd x \\
	&=& \lim_{n\to \infty} \int_\Omega \sum_{i=1}^n \phi_i(x) \phi_i(x) \dd x \\
	&=& \int_\Omega \lim_{n \to \infty} \sum_{i=1}^n \phi_i(x) \phi_i(x) \dd x < \infty,
\end{IEEEeqnarray*}
where we use the compactness of $\Omega$.
\end{proof}
\end{document}