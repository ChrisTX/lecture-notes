\documentclass[../lecture-notes.tex]{subfiles}

\begin{document}
In general if we have a countable basis we can write
\[
	K(x, y) = \sum_{i=1}^\infty \phi_i(x) \phi_i(y) = \langle \Phi(x), \Phi(y) \rangle_{\ell_2}.
\]

For example, in the exercises we had $W_0^m(\Pi)$ with the kernel
\[
	R^1(x, y) = 2 \sum_{k \in \Z \setminus \{ 0 \}} k^{-2m} \cos \left( k(x-y) \right).
\]
Here, weights could be added to the different frequencies in order to bias the result in the way desired.

\section{Kernel methods for prediction} % Section 1.4
\label{sec:1.4}
We will now stray away from pure interpolation because we so far assumed data in the form of $f(x) = y$, but with data stemming from measurements, we'll have errors of the form $f(x) = y + \varepsilon$. If our interpolation matrix was ill-conditioned, this would pose a problem.

One might want to value errors differently than just using the plain $L_2$-error. Therefore, we define
\begin{definition} % Definition 38
\label{thm:38}
Let $(\Omega, \Sigma)$ be a measurable space and $Y \subset \R$ be a closed subset.
Denote by $(x, y, f(x)) \in \Omega \times Y \times \R$ the triplet consisting of a pattern $x$, an observation $y$, and a prediction $f(x)$.
A function $\ell : \Omega \times Y \times \R \to [0, \infty)$ is called a \keydef{loss function} if it measurable and $\ell(x, y, y) = 0$ holds for all $x \in \Omega$, $y \in Y$.
\end{definition}
We consider at first real values $y$ and we want $y - f(x)$ to be small.
The most popular choice for a loss function is the squared loss:
\[
	\ell(x, y, f(x)) = \left( f(x) - y \right)^2 = \tilde{\ell}( \underbrace{f(x) - y}_{\zeta} ).
\]
% Plot: $\tilde{\ell}$ plotted for $\zeta$ on $[-5, 5]$
For robust estimation the so called Huber's loss can be useful:
\[
	\tilde{\ell}(\zeta) = \begin{cases}
	\frac{1}{\sigma} (\zeta)^2 & \text{if } |\zeta| \leq \sigma \\
	|\zeta| & \text{otherwise}
	\end{cases}.
\]
% Plot: Huber's loss for some range
Furthermore, one can use the $\ell_1$-loss:
\[
	\tilde{\ell}(\zeta) = |\zeta|.
\]
% Plot: $\ell_1$-loss.
Another possibility is to use the $\varepsilon$-sensitive loss. Here one tolerates errors inside the interval $[\zeta - \varepsilon, \zeta - \varepsilon]$. Formally:
\[
	\tilde{\ell}(\zeta) = \max \left( |\zeta| - \varepsilon, 0 \right) \eqqcolon |\zeta|_{\varepsilon}.
\]
% Plot: $\varepsilon$-sensitive loss.

Let's consider $Y = \{ -1, 1 \}$ or $Y = \{ 0, 1 \}$.
The simplest loss is the misclassification error:
\[
	\ell(x, y, f(x)) = \begin{cases}
	0 & \text{if } y = f(x) \\
	1 & \text{otherwise}
	\end{cases}
\]
or for $Y = \{ -1, 1 \}$:
\[
	\ell(x, y, f(x)) = \begin{cases}
	0 & \text{if } y = \sgn f(x) \\
	1 & \text{otherwise}
	\end{cases}.
\]
A plausible choice, for example for advertisement campaigns, would be the weighted misclassification loss:
\[
	\ell(x, y, f(x)) = \begin{cases}
	0 & \text{if } y = \sgn f(x) \\
	\tilde{\ell}(y) & \text{otherwise}
	\end{cases}.
\]
There's also the soft margin loss:
\[
	\ell(x, y, f(x)) = \max(1 - yf(x), 0) = \begin{cases}
	0 & \text{if } y = f(x) \geq 1 \\
	1 - y f(x) & \text{otherwise}
	\end{cases}.
\]
% Plot: soft margin loss for x-axis of $y (x)$.
This loss is used for support vector machines, which is one of the most common classification procedure.

Finally, the logistic loss is often used, it ``assigns'' a probabilistic meaning to $f(x)$:
\[
	\ell(x, y, f(x)) = \ell_1\left(1 + \exp\left( -y f(x) \right)\right).
\]
% Plot: logistic loss
\begin{definition} % Definition 39
\label{thm:39}
Let $\ell$ be a loss function and $\probP$ be a probability measure on $\Omega \times Y$.
Then, for a measurable function $f : \Omega \to \R$, the \keydef{expected $\ell$-risk} is defined by
\begin{IEEEeqnarray*}{rCl}
	\riskR_{\ell, \probP}(f) &\coloneqq& \int_{\Omega \times Y} \ell(x, y) f(x) \dd \probP(x, y) \\
	&=& \int_\Omega \int_Y \ell(x, y, f(x)) \dd \probP\left(y \mid x\right) \dd \probP_x(x).
\end{IEEEeqnarray*}
\end{definition}
For given $D \coloneqq \left\{ \left( x_i, y_i \right)\right\}_{i=1}^N$, $x_i \in \Omega$, $y_i \in Y$, we can define the empirical measure
\[
	\probP_{\mathrm{emp}} = \frac{1}{N} \sum_{i=1}^N \delta_{x_i, y_i}
\]
by using the Dirac measure.
\begin{definition} % Definition 40
\label{thm:40}
The \keydef{empirical $\ell$-risk} of a function $f: \Omega \to \R$ is defined as
\begin{IEEEeqnarray*}{rCl}
\riskRemp{\ell}(f) &=& \int_{\Omega \times Y} \ell\left(x, y, f(x\right)) \dd \probP_{\mathrm{emp}} \\
&=& \frac{1}{N} \sum_{i=1}^N \ell\left(x_i, y_i, f(x_i) \right).
\end{IEEEeqnarray*}
\end{definition}
For inverse problems one uses Tikhonov-regularization to restrict the class of admissible functions.
In particular, we use function space regularization.
An alternative is to penalize on the coefficients $\alpha$ when
\[
	f = \sum_{i=1}^N \alpha_j K(x_j, \cdot),
\]
e.g.\ to obtain sparsity one can use $\| \alpha \|_{1}$.
For support vector machines, one can observe that a ``small'' number of the coefficients $\alpha$ are nonzero.

In the following, we will assume that $\riskRemp{\ell}(f)$ is continuous in $f$.
The regularization can be seen as restricting the class of possible minimizers to some compact set $\spH$.
For a continuous function on a compact set $\spH$, we can apply the operator inversion lemma to get that the inverse map from the minimum of $\riskRemp{\ell} : \spH \to \R$ to its minimizer $\hat{f}$ is continuous and the optimization problem is well-posed.

Directly minimizing $\riskRemp{\ell}$ in $\spH$ is typically a difficult constrained optimization problem.
Instead we add a regularization term to the empirical loss:
\[
	\riskRreg{\ell}(f) \coloneqq \riskRemp{\ell}(f) + \lambda \tilde{s}(f).
\]
Here, the $\lambda$ is the regularization parameter, which balances the empirical error and the smoothness or simplicity enforced by the regularization term $\tilde{s}(f)$.
\begin{theorem} % Theorem 41
\label{thm:41}
Let $s : [0, \infty) \to \R$ be a strictly monotone increasing function, $\lambda > 0$, $\Omega$ be a set, $\spH$ a \ac{RKHS} over $\Omega$, and let $\ell : \Omega \times Y \times \R$ be a loss function.
Then, for the data $D \coloneqq \left\{ \left( x_i, y_i \right ) \right\}_{i=1}^N$, $x_i \in \Omega$, $y_i \in Y$, each minimizer $f \in \spH$ of the regularized empirical risk
\begin{equation}
\label{eq:RegRiskMinimizer}
\opttag{$\star$}
\riskRreg\ell (f) = \frac{1}{N} \sum_{i=1}^N \ell(x_i, y_i, f(x_i)) + \lambda s\left( \| f \|_{\spH} \right) 
\end{equation}
admits a representation
\[
	f(x) = \sum_{i=1}^N \alpha_i K(x_i, x),
\]
or $f \in \spH_{X}$, $X = \{ x_1, \ldots, x_N \}$.
\end{theorem}
\begin{proof}
Without loss of generality, we assume
\[
	s \left( \| f \|_{\spH} \right) = \bar{s} \left( \| f \|_{\spH}^2 \right).
\]
We decompose any $f \in \spH$ into $f_X \in \spH_X$ and $f_{X^\perp} \in \spH_{X}^\perp$ after \cref{thm:14}.
Then,
\[
	f = \sum_{i=1}^N \alpha_i K(x_i, \cdot) + f_{X^\perp}, \quad \alpha_i \in \R.
\]
We know
\[
	\langle f_{X^\perp}, K(x_i, x) \rangle_{\spH} = 0 \quad \text{for all } i = 1, \ldots, N
\]
with \cref{eq:starRE} this yields:
\begin{IEEEeqnarray*}{rCl}
f(x_j) = \langle f(\cdot), K(x_j, \cdot) \rangle &=& \sum_{i=1}^N \alpha_i K(x_i, x_j) + \langle f_{X^\perp}, K(x_j, \cdot) \rangle_{\spH} \\
&=& \sum_{i=1}^N \alpha_i K(x_i, x_j).
\end{IEEEeqnarray*}
Furthermore, for all $f_{X^\perp}$,
\begin{IEEEeqnarray*}{rCl}
	s(\| f \|_{\spH}) &=& \bar{s} \left( \| f_X \|_{\spH}^2 + \| f_{X^\perp} \|_{\spH}^2 \right) \\
	&\geq& \bar{s} \left( \| f_X \|_{\spH}^2 \right)
\end{IEEEeqnarray*}
Therefore, \cref{eq:RegRiskMinimizer} is for any fixed $\alpha \in \R^N$ minimal if $f_{X^\perp} = 0$.
This also has to hold for minimizing $f$.
\end{proof}
\end{document}