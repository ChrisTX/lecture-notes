\documentclass[../lecture-notes.tex]{subfiles}

\begin{document}
\begin{theorem} % Theorem 10
\label{thm:10}
The dual space $\spHstar$ of a \ac{RKHS} of functions on some set $\Omega$ is the closure of the span of all point evaluation functionals $\delta_x$ for $x \in \Omega$.
For each pair $\lambda, \mu$ of functionals from the dual $\spHstar$ of $\spH$, one can define $\lambda^x(\mu^y(K(x, y)))$ uniquely via Cauchy sequences to yield
\begin{equation}
\label{eq:three-star}
\opttag{$\star\!\star\!\star$}
	\lambda^x(\mu^y(K(x, y))) = \langle \lambda, \mu \rangle_{\spHstar} \quad \text{for all } \lambda, \mu \in \spHstar
\end{equation}
as a generalization of \cref{eq:two-star}.
In particular, the Riesz representer of $\lambda \in \spHstar$ is $\lambda^x(K(\cdot, x)) \in \spH$.
\end{theorem}
\begin{proof}
The first part is the dual form of \cref{thm:7}.
The identity \cref{eq:three-star} holds for all linear combinations of point evaluation functionals and by collinearity it carries our to all limits of Cauchy sequences, i.e.\ to all functionals in the dual.
For linear combinations of point evaluations (and then later for all functionals) the reproduction equation is
\[
	\lambda(f) = \langle f, \lambda^x K(\cdot, x) \rangle_{\spH}
\]
proving
\[
	R(\lambda) = \lambda^x(K(\cdot, x)).
\]
\end{proof}
Roughly spoken, the dual space consist of functionals which can be obtained via sequences of linear combinations of point evaluation functionals such that their application in the limit is possible for both arguments of $K$ independently.

We aim for functionals like
\[
	f \mapsto (\Delta f)(x)
\]
or
\[
	f \mapsto \int_\Omega f(t) \dt.
\]
We are aiming for a sufficient condition for these to lie in $\spHstar$.
\begin{theorem} % Theorem 11
\label{thm:11}
Assume that the kernel $K$ of a \ac{RKHS} $\spH$ of functions on some set $\Omega$ is explicitly known as a function on $\Omega \times \Omega$ and assume it allows the action of a general functional $\lambda$ to both arguments, i.e.\ $\lambda^y(\lambda^x(K(y, x))) \in \R$ exists.
Furthermore, assume that there is a sequence $\{ \lambda_n \}_{n \in \N}$ of linear combinations of point evaluation functionals on points of $\Omega$ such that for all $\varepsilon > 0$ there is a $N \in \N$ such that for all $n, m \geq N$ we have
\[
	\left\| \lambda_n^y \left(\lambda_m^x\left(K\left(y,x\right)\right)\right) - \lambda^y\left(\lambda^x\left(K\left(y, x\right)\right)\right) \right\| \leq \varepsilon.
\]
Finally assume
\[
	\lim_{n \to \infty} \lambda_n^y(K(x,y)) = \lambda^y(K(x,y)) \quad \text{for all } x \in \Omega.
\]
Then $\lambda \in \spHstar$ and it is the limit of the Cauchy sequence $\{ \lambda_n \}_{n \in \N} \subset \spHstar$.
\end{theorem}
\begin{proof}
We start showing that $\{\lambda_n\}_{n \in \N}$ is a Cauchy sequence in $\spHstar$:
\begin{IEEEeqnarray*}{rCl}
	\| \lambda_n - \lambda_m \|_{\spHstar}^2 &=& \| \lambda_n \|_{\spHstar}^2 + \| \lambda_m \|_{\spHstar}^2 - 2 \langle \lambda_n, \lambda_m \rangle_{\spHstar} \\
	&\overset{\text{\cref{thm:10}}}{=}& \lambda_n^y \lambda_n^x K(y, x) + \lambda_m^y \lambda_m^x K(y, x) - 2 \lambda_n^y \lambda_m^x K(y, x) \\
	&\overset{\pm \lambda^y \lambda^x K(y, x)}{\leq}& 4 \varepsilon \quad \text{for all } n, m > N.
\end{IEEEeqnarray*}
Therefore the $\{ \lambda_n \}$ must have a limit $\tilde{\lambda} \in \spHstar$ and we need to show that $\tilde{\lambda} = \lambda$ as functionals on $\spH$.
From the assumption on the limit we set % Was "From (**)".. ?
\[
	\lim_{n \to \infty} \lambda_n^y K(x, y) = \tilde{\lambda}^y K(x, y) = \lambda^y K(x, y) \quad \text{for all } x \in \Omega.
\]
By \cref{thm:7} this extends to all of $\spH$.
\end{proof}
\subsection*{Kernels for subspaces}
Let $\spH$ be a Hilbert space of functions on $\Omega$ and let $\spH_0$ be a closed subspace of $\spH$, which is a Hilbert space again with its own reproducing kernel $K_0$. With the projector $\Pi_0 : \spH \to \spH_0$ we set
\begin{theorem} % Theorem 12
\label{thm:12}
The subspace kernel is $K_0(x, \cdot) = \Pi_0(K(x, \cdot))$ for all $x \in \Omega$ and the reproducing kernel for the orthogonal complement $\spH_0^\perp$ is $K - K_0$.
\end{theorem}
\begin{proof}
The identity on $\spH$ can be decomposed into
\[
	I = \Pi_0 + (I - \Pi_0) = \Pi_0 + \Pi_0^\perp.
\]
Thus:
\[
	f(y) = (\Pi_0 f)(y) + \left( \Pi_0^\perp f \right)(y).
\]
Insert into \cref{eq:starRE}:
\begin{IEEEeqnarray*}{rCl}
f(y) &=& \langle f, K(y, \cdot) \rangle_{\spH} \\
&=& \langle \Pi_0 f + \Pi_0^\perp f, \Pi_0 K(y, \cdot) + \Pi_0^\perp K(y, \cdot) \rangle_{\spH} \\
&=& \langle \Pi_0 f, \Pi_0 K(y, \cdot) \rangle_{\spH} + \langle \Pi_0^\perp f, \Pi_0^\perp K(y, \cdot) \rangle_{\spH}
\end{IEEEeqnarray*}
with $f \in \spH_0$ we see the first result and then easily see the second.
\end{proof}
\begin{remark}
Orthogonal space decompositions relate additive kernel decompositions using the corresponding projectors.
\end{remark}
\paragraph*{Subspaces from point sets}
We fix a non-empty subset $X \subseteq \Omega$ and look at
\[
	\spH_X \coloneqq \overline{\spn \left\{ K(x, \cdot) \mid x \in X \right\}} \subseteq \spH.
\]
\begin{theorem} % Theorem 13
\label{thm:13}
Then
\[
	\spH_X^\perp = \left\{ f \mid f \in \spH, f(X) = \{ 0 \} \right\}.
\]
\end{theorem}
\begin{proof}
If $f(X) = \{ 0 \}$ then $f \in \spH_X^\perp$ by \cref{eq:starRE} and conversely.
\end{proof}
Now take the projector $\Pi_X$ from $\spH$ to $\spH_X$ and denote
\[
	f_X \coloneqq \Pi_X(f).
\]
Standard results in Hilbert spaces give us
\begin{theorem} % Theorem 14
\label{thm:14}
Each function $f \in \spH$ has an orthogonal decomposition
\[
	f = f_X + f_{X^\perp}
\]
with $f_X \in \spH_X$ and $f_{X^\perp} \in \spH_{X^\perp}$.
In particular, each $f \in \spH$ has an interpolant $f_X \in \spH_X$ recovering the values of $f$ on $X$.
Additionally,
\[
	\| f - f_X \|_{\spH} = \inf_{g \in \spH_X} \| f - g \|_{\spH}
\]
and 
\[
	\| f_X \|_{\spH} = \inf_{\substack{f(x) = g(x) \; \forall x \in X \\ g \in \spH}} \| g \|_{\spH} = \inf_{v \in \spH_X^\perp} \| f - v \|_{\spH}
\]
due to the orthogonality of the decomposition.
\end{theorem}
\begin{corollary} % Corollary 15
\label{thm:15}
The interpolant $f_X \in \spH_X$ to a function $f$ on $X$ is at the same time the best approximation to $f$ from all functions in $\spH_X$.
\end{corollary}
\begin{corollary} % Corollary 16
\label{thm:16}
The interpolant $f_X \in \spH_X$ to a function $f$ on $X$ minimizes the norm under all interpolants from the full space $\spH$.
\end{corollary}
With $f_0 = 0$, $f_0^\perp = f$ and 
for completeness, we easily see
\begin{corollary} % Corollary 17
\label{thm:17}
For all sets $X \subseteq Y \subseteq \Omega$ and all $f \in \spH$ we have
\[
	\| f_X \|_{\spH} \leq \| f_Y \|_{\spH} \leq \| f \|_{\spH}
\]
and
\[
	\| f \|_{\spH} \geq \| f - f_X \|_{\spH} \geq \| f - f_Y \|_{\spH}.
\]
\end{corollary}
\subsection*{Interpolate on finite sets}
We now restrict ourselves to finite sets $X = \{ x_1, \ldots, x_N \} \subseteq \Omega$. For each $f \in \spH$ we can write
\[
	f_X(\cdot) = \sum_{j=1}^N \alpha_j K(x_j, \cdot)
\]
with $\alpha_j \in \R$.
The coefficients $\alpha_j$ might not be unique since we do not assume that the kernels $K(x_j, \cdot)$ are linearly independent.
We know that $f_X$ interpolates $f$ on $X$, therefore it holds
\addtocounter{dummythm}{2} % 18 and 19 follow next lecture
\begin{theorem} % Theorem 20
\label{thm:20}
For each $f \in \spH$, the linear system
\[
	\sum_{j=1}^N \alpha_j K(x_j, x_k) = f_k, \quad 1 \leq k \leq N
\]
with $f_k = f(x_k)$ and the symmetric kernel matrix
\[
	A = (K(x_j, x_k))_{1 \leq j, k \leq N}
\]
is solvable.
\end{theorem}
\begin{remark}
The kernel matrix can be singular under the assumptions made, therefore this result is not obvious.
\end{remark}
\begin{theorem} % Theorem 21
\label{thm:21}
In an \ac{RKHS}, the kernel matrix for a finite set $X$ is positive semidefinite. It is positive definite if the point evaluation functionals $\delta_X$ for $x \in X$ or equivalently the functions $K(x, \cdot)$ for $x \in X$ are linearly independent.
\end{theorem}
\begin{proof}
This follows because any kernel matrix on a finite set $X$ is a Gram matrix for the functionals $\delta_{x_j}$ or the functions $K(x_j, \cdot)$, due to \cref{eq:starRE}.
\end{proof}
\begin{remark}
The \cref{thm:20} therefore says that the right hand side of point evaluations is always in the span of the columns of the matrix.
For general right hand sides the system in \cref{thm:20} might not be solvable.
\end{remark}
\begin{definition} % Definition 22
\label{thm:22}
A kernel on $\Omega \times \Omega$ is symmetric and positive semidefinite if all kernel matrices for all finite point sets of $\Omega$ are symmetric and positive semidefinite.
\end{definition}
The last theorem gives
\begin{theorem} % Theorem 23
\label{thm:23}
All reproducing kernels of Hilbert spaces are symmetric and positive semidefinite.
\end{theorem}
\end{document}