\documentclass{scrbook}
\KOMAoptions{twoside=false,parskip=yes}

% This document is only designed for
% XeLaTeX, LuaLaTeX or pdfLaTeX
% 
\usepackage{iftex}

\newif\ifHaveUnicodeTeX

\ifXeTeX
	\global\HaveUnicodeTeXtrue
\else
	\ifLuaTeX
		\global\HaveUnicodeTeXtrue
	\fi
\fi

\ifHaveUnicodeTeX
	\usepackage{mathtools}
	\usepackage{unicode-math}
	\usepackage{polyglossia}
	\setdefaultlanguage[variant=us]{english}
\else
	\usepackage[USenglish]{babel}
	\usepackage[utf8]{inputenc}
	\usepackage[T1]{fontenc}
	\usepackage{mathtools}
	\usepackage{amssymb}
	\usepackage{stmaryrd}
\fi

\usepackage{subfiles}
\usepackage{csquotes}
\usepackage[backend=biber,style=alphabetic]{biblatex}

\usepackage{relsize}
\usepackage[dvipsnames]{xcolor}
\usepackage{enumerate}
\usepackage{acro}
\usepackage[xindy]{imakeidx}

\usepackage{tikz}
\usepackage{chngcntr}
\usepackage{caption}

\usepackage{needspace} 	% Needed so we can glue theorems right after sections

\usepackage{ifpdf}
\ifpdf
	\pdfminorversion=7
	\usepackage[pdfpagelabels]{hyperref}
\else
	\usepackage{hyperref}
\fi

\usepackage{microtype}

% Proprietary Adobe font packages
% WARNING:
%	These are *NOT* included with normal, free LaTeX redistributions
% 	In order to acquire these fonts, one has to install "FontPro" and provide the appropriate OpenType font files.
% NOTE: This is not supported with XeLaTeX or LuaLaTeX because MnSymbol does not work in a reasonable way with unicode-math
%		There is a font called Minion Math for this purpose, but it's also proprietary		
\ifHaveUnicodeTeX
	\ifx\adobefonts\undefined
		\usepackage{lmodern}								% Without Adobe fonts, lmodern is preferable over the default
	\else
		\usepackage{CronosPro} 								% Replaces sans-serif fonts only
		\usepackage[minionint,mathlf,swash]{MinionPro} 		% Replaces serif text fonts and math fonts. Also pulls in MnSymbol.
	\fi
\fi

% Document structure
\setcounter{secnumdepth}{2}
\setcounter{tocdepth}{2}
\counterwithout{equation}{chapter}

\newcommand\myshade{85}
\colorlet{mylinkcolor}{NavyBlue}
\colorlet{mycitecolor}{YellowOrange}
\colorlet{myurlcolor}{violet}

\hypersetup{
  linkcolor  = mylinkcolor!\myshade!black,
  citecolor  = mycitecolor!\myshade!black,
  urlcolor   = myurlcolor!\myshade!black,
  colorlinks = true,
}

% These have to be loaded after hyperref!
\usepackage[amsmath,hyperref,thmmarks]{ntheorem}
\usepackage[ntheorem,framemethod=tikz]{mdframed}

\usepackage[capitalize,nameinlink]{cleveref}
\usepackage[a4paper]{geometry}
\usepackage{IEEEtrantools}

% Given (*) works badly in lecture notes, we default to numbered equations. Define \originaltags to change the behavior
\ifx\originaltags\undefined
	\newcommand{\opttag}[1]{}
\else
	\newcommand{\opttag}[1]{\tag{#1}}
\fi

% Theorems

\definecolor{theoremcolor}{rgb}{1,0.96,0.94}
\definecolor{lemmacolor}{rgb}{1,0.96,0.91}
\definecolor{corollarycolor}{rgb}{1,1,0.91}
\definecolor{propositioncolor}{rgb}{0.93,0.97,1}

\definecolor{definitioncolor}{rgb}{0.95,0.95,1}
\definecolor{examplecolor}{rgb}{0.93,1,1}
\definecolor{algorithmcolor}{rgb}{0.93,1,0.96}

\definecolor{remarkcolor}{rgb}{0.95,0.95,0.95}
\definecolor{addendumcolor}{rgb}{0.96,0.91,0.91}

\definecolor{problemcolor}{rgb}{0.97,1,0.93}
\definecolor{assumptioncolor}{rgb}{0.93,0.97,1}

\newcounter{dummythm}
\numberwithin{dummythm}{chapter}
\mdfdefinestyle{theoremframing}{skipabove=\topskip,skipbelow=\topskip,nobreak=true}

\theoremstyle{break}
\newmdtheoremenv[backgroundcolor=theoremcolor,style=theoremframing]{theorem}[dummythm]{Theorem}
\crefname{theorem}{Theorem}{Theorems}
\newmdtheoremenv[backgroundcolor=lemmacolor,style=theoremframing]{lemma}[dummythm]{Lemma}
\crefname{lemma}{Lemma}{Lemmata}
\newmdtheoremenv[backgroundcolor=corollarycolor,style=theoremframing]{corollary}[dummythm]{Corollary}
\crefname{corollary}{Corollary}{Corollaries}
\newmdtheoremenv[backgroundcolor=propositioncolor,style=theoremframing]{proposition}[dummythm]{Proposition}
\crefname{proposition}{Proposition}{Propositions}

\theoremstyle{nonumberbreak}
\newmdtheoremenv[backgroundcolor=theoremcolor,style=theoremframing]{theoremnonumb}{Theorem}
\crefname{theoremnonumb}{Theorem}{Theorems}
\newmdtheoremenv[backgroundcolor=lemmacolor,style=theoremframing]{lemmanonumb}{Lemma}
\crefname{lemmanonumb}{Lemma}{Lemmata}
\newmdtheoremenv[backgroundcolor=corollarycolor,style=theoremframing]{corollarynonumb}{Corollary}
\crefname{corollarynonumb}{Corollary}{Corollaries}
\newmdtheoremenv[backgroundcolor=propositioncolor,style=theoremframing]{propositionnonumb}{Proposition}
\crefname{propositionnonumb}{Proposition}{Propositions}

\theoremstyle{nonumberbreak}
\theoremseparator{.}
\newtheorem{example}{Example}
\crefname{example}{Example}{Examples}
\theoremstyle{break}
\newtheorem{examplenumb}[dummythm]{Example}
\crefname{examplenumb}{Example}{Examples}
\theoremstyle{plain}
\theorembodyfont{\normalfont}
\newmdtheoremenv[backgroundcolor=definitioncolor,style=theoremframing]{definition}[dummythm]{Definition}
\crefname{definition}{Definition}{Definitions}
\newmdtheoremenv[backgroundcolor=problemcolor,style=theoremframing]{problem}{Problem}[chapter]
\crefname{problem}{Problem}{Problems}
\theoremstyle{empty}
\newmdtheoremenv[backgroundcolor=problemcolor,style=theoremframing]{problemnonumb}{Problem}
\crefname{problemnonumb}{Problem}{Problems}
\theoremstyle{nonumberplain}
\newmdtheoremenv[backgroundcolor=definitioncolor,style=theoremframing]{definitionnonumb}{Definition}
\crefname{definitionnonumb}{Definition}{Definitions}
\theoremheaderfont{\itshape}
\newtheorem{remark}[dummythm]{Remark}
\crefname{remark}{Remark}{Remarks}

\theoremstyle{break}
\theorembodyfont{\normalfont}
\theoremseparator{.}
\newmdtheoremenv[backgroundcolor=algorithmcolor,style=theoremframing]{algorithm}{Algorithm}
\crefname{algorithm}{Algorithm}{Algorithms}

\theoremstyle{nonumberplain}
\theoremheaderfont{\itshape}
\theorembodyfont{\normalfont}
\theoremseparator{.}
\newmdtheoremenv[backgroundcolor=addendumcolor,style=theoremframing]{addendum}{Addendum}
\crefname{addendum}{Addendum}{Addenda}

\theoremstyle{nonumberplain}
\theoremheaderfont{\sc}
\theorembodyfont{\upshape}
\theoremseparator{.}
\theoremsymbol{\rule{1ex}{1ex}}
\newtheorem{proof}{Proof}
\crefname{proof}{Proof}{Proofs}

% Abbreviations
% Differentiale
\newcommand{\dx}{\:\mathrm{d}x}
\newcommand{\dy}{\:\mathrm{d}y}
\newcommand{\dz}{\:\mathrm{d}z}
\newcommand{\ds}{\:\mathrm{d}s}
\newcommand{\dt}{\:\mathrm{d}t}
\newcommand{\du}{\:\mathrm{d}u}
\newcommand{\dxi}{\:\mathrm{d}\xi}
\newcommand{\dxy}{\:\mathrm{d}(x,y)}
\newcommand{\dxyz}{\:\mathrm{d}(x,y,z)}

% Differentialoperatoren
\DeclareMathOperator{\grad}{grad}
\DeclareMathOperator{\dive}{div}
\DeclareMathOperator{\curl}{curl}

% ess sup, ess inf
\DeclareMathOperator*{\esssup}{ess\,sup}
\DeclareMathOperator*{\essinf}{ess\,inf}

% arg min/max
\DeclareMathOperator*{\argmin}{arg\;min}
\DeclareMathOperator*{\argmax}{arg\;max}

% Maß
\DeclareMathOperator{\meas}{meas}

% Im
\DeclareMathOperator{\im}{Im}

% Zahlen
\newcommand{\Q}{\ensuremath{\mathbb{Q}}}
\newcommand{\R}{\ensuremath{\mathbb{R}}}
\newcommand{\N}{\ensuremath{\mathbb{N}}}
\newcommand{\Z}{\ensuremath{\mathbb{Z}}}
\newcommand{\C}{\ensuremath{\mathbb{C}}}

% Weitere Operatoren
\DeclareMathOperator*{\spn}{span}
\DeclareMathOperator*{\supp}{supp} 
\DeclareMathOperator*{\dist}{dist}
\DeclareMathOperator*{\diam}{diam}
\DeclareMathOperator{\conv}{conv}
\DeclareMathOperator{\Int}{int}

\DeclareMathOperator{\sgn}{sgn}
\DeclareMathOperator{\codim}{codim}
\DeclareMathOperator{\diag}{diag}

% Transpose symbol
\newcommand{\tp}{\ensuremath{\mathsmaller T}}
\newcommand{\smallo}{\ensuremath{\mathit{o}}}

% Acronyms
\DeclareAcronym{wlog}{short={w.l.o.g.}, long={without loss of generality}}
\DeclareAcronym{st}{short={s.t.}, long={such that}}
\DeclareAcronym{iff}{short={iff}, long={if and only if}}
\DeclareAcronym{wrt}{short={w.r.t.}, long={with respect to}}
\DeclareAcronym{resp}{short={resp.}, long={respectively}}
\DeclareAcronym{ae}{short={a.e.}, long={almost everywhere}}

\DeclareMathOperator{\loc}{loc}
\newcommand{\lapl}{\ensuremath{\Delta}}

\bibliography{literature}

\makeindex[name=keydefinitions, title=Index of key definitions, intoc]
\newcommand{\keydef}[1]{\emph{#1}\index[keydefinitions]{#1}}

\newif\ifbuggedpgf
\begingroup
\def\getmainversion#1.#2\getmainversion{#1}
\ifnum\expandafter\getmainversion\pgfversion\getmainversion<3
  \global\buggedpgftrue
\fi
\endgroup

\ifbuggedpgf
	\GenericWarning{}{Using PGF bug fix for broken pow function}
	\makeatletter
	\def\pgfmathfloatpow@#1#2{%
	    \begingroup%
	    \expandafter\pgfmathfloat@decompose@tok#1\relax\pgfmathfloat@a@S\pgfmathfloat@a@Mtok\pgfmathfloat@a@E
	    \ifcase\pgfmathfloat@a@S\relax
	        % 0 ^ #2 = 0
	        \pgfmathfloatcreate{0}{0.0}{0}%
	    \else
	        \expandafter\pgfmathfloat@decompose@tok#2\relax\pgfmathfloat@a@S\pgfmathfloat@a@Mtok\pgfmathfloat@a@E
	        \ifcase\pgfmathfloat@a@S\relax
	            % #1 ^ 0 = 1
	            \pgfmathfloatcreate{1}{1.0}{0}%
	        \or
	            % #2 > 0
	            \pgfmathfloatpow@@{#1}{#2}%
	        \or
	            % #2 < 0
	            \pgfmathfloatpow@@{#1}{#2}%
	        \or
	            % #2 = nan
	            \edef\pgfmathresult{#2}%
	        \or
	            % #2 = inf
	            \edef\pgfmathresult{#2}%
	        \or
	            % #2 = -inf
	            \pgfmathfloatcreate{0}{0.0}{0}%
	        \fi
	    \fi
	    \pgfmath@smuggleone\pgfmathresult
	    \endgroup
	}%
	\makeatother
\fi

\begin{document}
\frontmatter

\begin{titlepage}

	\centering

	\definecolor{c001c4a}{RGB}{0,28,74}
\definecolor{ce7d900}{RGB}{231,217,0}

\begin{tikzpicture}[y=0.80pt, x=0.80pt, yscale=-2, xscale=2, inner sep=0pt, outer sep=0pt]
  \path[fill=c001c4a,nonzero rule] (34.2578,23.2578) .. controls (34.2578,24.8984)
    and (32.9258,26.2344) .. (31.2812,26.2344) .. controls (29.6367,26.2344) and
    (28.3047,24.8984) .. (28.3047,23.2578) .. controls (28.3047,21.6133) and
    (29.6367,20.2812) .. (31.2812,20.2812) .. controls (32.9258,20.2812) and
    (34.2578,21.6133) .. (34.2578,23.2578);
  \path[fill=c001c4a,nonzero rule] (60.8828,41.0430) .. controls (60.8828,42.6836)
    and (59.5508,44.0156) .. (57.9062,44.0156) .. controls (56.2617,44.0156) and
    (54.9297,42.6836) .. (54.9297,41.0430) .. controls (54.9297,39.3984) and
    (56.2617,38.0664) .. (57.9062,38.0664) .. controls (59.5508,38.0664) and
    (60.8828,39.3984) .. (60.8828,41.0430);
  \path[fill=c001c4a,nonzero rule] (74.8203,52.2852) .. controls (74.8203,53.9297)
    and (73.4883,55.2617) .. (71.8438,55.2617) .. controls (70.1992,55.2617) and
    (68.8672,53.9297) .. (68.8672,52.2852) .. controls (68.8672,50.6406) and
    (70.1992,49.3086) .. (71.8438,49.3086) .. controls (73.4883,49.3086) and
    (74.8203,50.6406) .. (74.8203,52.2852);
  \path[fill=c001c4a,nonzero rule] (25.1797,15.6758) .. controls (25.1797,17.3203)
    and (23.8477,18.6523) .. (22.2031,18.6523) .. controls (20.5586,18.6523) and
    (19.2266,17.3203) .. (19.2266,15.6758) .. controls (19.2266,14.0312) and
    (20.5586,12.6992) .. (22.2031,12.6992) .. controls (23.8477,12.6992) and
    (25.1797,14.0312) .. (25.1797,15.6758);
  \path[fill=c001c4a,nonzero rule] (5.9531,2.9766) .. controls (5.9531,4.6211) and
    (4.6172,5.9531) .. (2.9766,5.9531) .. controls (1.3320,5.9531) and
    (0.0000,4.6211) .. (0.0000,2.9766) .. controls (0.0000,1.3320) and
    (1.3320,0.0000) .. (2.9766,0.0000) .. controls (4.6172,0.0000) and
    (5.9531,1.3320) .. (5.9531,2.9766);
  \path[fill=c001c4a,nonzero rule] (76.2539,41.0195) .. controls (76.2539,42.6641)
    and (74.9180,43.9961) .. (73.2773,43.9961) .. controls (71.6328,43.9961) and
    (70.3008,42.6641) .. (70.3008,41.0195) .. controls (70.3008,39.3789) and
    (71.6328,38.0469) .. (73.2773,38.0469) .. controls (74.9180,38.0469) and
    (76.2539,39.3789) .. (76.2539,41.0195);
  \path[fill=c001c4a,nonzero rule] (57.8594,20.0391) .. controls (57.8594,21.6836)
    and (56.5273,23.0156) .. (54.8828,23.0156) .. controls (53.2422,23.0156) and
    (51.9102,21.6836) .. (51.9102,20.0391) .. controls (51.9102,18.3945) and
    (53.2422,17.0625) .. (54.8828,17.0625) .. controls (56.5273,17.0625) and
    (57.8594,18.3945) .. (57.8594,20.0391);
    \path[fill=c001c4a,nonzero rule] (78.8008,75.2383) .. controls (78.8008,76.8828)
      and (77.4688,78.2148) .. (75.8242,78.2148) .. controls (74.1797,78.2148) and
      (72.8477,76.8828) .. (72.8477,75.2383) .. controls (72.8477,73.5938) and
      (74.1797,72.2617) .. (75.8242,72.2617) .. controls (77.4688,72.2617) and
      (78.8008,73.5938) .. (78.8008,75.2383);
  \path[fill=c001c4a,nonzero rule] (62.1836,30.7383) .. controls (62.1836,32.3789)
    and (60.8477,33.7109) .. (59.2070,33.7109) .. controls (57.5625,33.7109) and
    (56.2305,32.3789) .. (56.2305,30.7383) .. controls (56.2305,29.0938) and
    (57.5625,27.7617) .. (59.2070,27.7617) .. controls (60.8477,27.7617) and
    (62.1836,29.0938) .. (62.1836,30.7383);
    \path[fill=c001c4a,nonzero rule] (59.8789,80.9062) .. controls (59.8789,85.8281)
      and (55.8906,89.8164) .. (50.9648,89.8164) .. controls (46.0430,89.8164) and
      (42.0547,85.8281) .. (42.0547,80.9062) .. controls (42.0547,75.9844) and
      (46.0430,71.9922) .. (50.9648,71.9922) .. controls (55.8906,71.9922) and
      (59.8789,75.9844) .. (59.8789,80.9062);
  \path[fill=c001c4a,nonzero rule] (70.4883,78.1680) .. controls (70.4883,80.6289)
    and (68.4922,82.6250) .. (66.0312,82.6250) .. controls (63.5703,82.6250) and
    (61.5742,80.6289) .. (61.5742,78.1680) .. controls (61.5742,75.7070) and
    (63.5703,73.7109) .. (66.0312,73.7109) .. controls (68.4922,73.7109) and
    (70.4883,75.7070) .. (70.4883,78.1680);
  \path[fill=c001c4a,nonzero rule] (64.1484,68.5117) .. controls (64.1484,70.9766)
    and (62.1523,72.9688) .. (59.6953,72.9688) .. controls (57.2305,72.9688) and
    (55.2383,70.9766) .. (55.2383,68.5117) .. controls (55.2383,66.0547) and
    (57.2305,64.0586) .. (59.6953,64.0586) .. controls (62.1523,64.0586) and
    (64.1484,66.0547) .. (64.1484,68.5117);
  \path[fill=c001c4a,nonzero rule] (43.5547,53.0781) .. controls (43.5547,55.5391)
    and (41.5586,57.5352) .. (39.0977,57.5352) .. controls (36.6367,57.5352) and
    (34.6445,55.5391) .. (34.6445,53.0781) .. controls (34.6445,50.6172) and
    (36.6367,48.6211) .. (39.0977,48.6211) .. controls (41.5586,48.6211) and
    (43.5547,50.6172) .. (43.5547,53.0781);
  \path[fill=c001c4a,nonzero rule] (46.6094,41.2773) .. controls (46.6094,43.7383)
    and (44.6133,45.7305) .. (42.1523,45.7305) .. controls (39.6914,45.7305) and
    (37.6953,43.7383) .. (37.6953,41.2773) .. controls (37.6953,38.8164) and
    (39.6914,36.8203) .. (42.1523,36.8203) .. controls (44.6133,36.8203) and
    (46.6094,38.8164) .. (46.6094,41.2773);
  \path[fill=c001c4a,nonzero rule] (43.7188,31.4570) .. controls (43.7188,33.9180)
    and (41.7266,35.9102) .. (39.2656,35.9102) .. controls (36.8047,35.9102) and
    (34.8086,33.9180) .. (34.8086,31.4570) .. controls (34.8086,28.9961) and
    (36.8047,27.0000) .. (39.2656,27.0000) .. controls (41.7266,27.0000) and
    (43.7188,28.9961) .. (43.7188,31.4570);
  \path[fill=c001c4a,nonzero rule] (58.3047,50.6250) .. controls (58.3047,53.0859)
    and (56.3125,55.0820) .. (53.8516,55.0820) .. controls (51.3906,55.0820) and
    (49.3945,53.0859) .. (49.3945,50.6250) .. controls (49.3945,48.1641) and
    (51.3906,46.1680) .. (53.8516,46.1680) .. controls (56.3125,46.1680) and
    (58.3047,48.1641) .. (58.3047,50.6250);
  \path[fill=ce7d900,nonzero rule] (70.9062,60.7656) .. controls (70.9062,63.2266)
    and (68.9141,65.2188) .. (66.4531,65.2188) .. controls (63.9922,65.2188) and
    (61.9961,63.2266) .. (61.9961,60.7656) .. controls (61.9961,58.3047) and
    (63.9922,56.3086) .. (66.4531,56.3086) .. controls (68.9141,56.3086) and
    (70.9062,58.3047) .. (70.9062,60.7656);
  \path[fill=c001c4a,nonzero rule] (54.1367,60.7148) .. controls (54.1367,63.1758)
    and (52.1406,65.1719) .. (49.6797,65.1719) .. controls (47.2188,65.1719) and
    (45.2227,63.1758) .. (45.2227,60.7148) .. controls (45.2227,58.2539) and
    (47.2188,56.2578) .. (49.6797,56.2578) .. controls (52.1406,56.2578) and
    (54.1367,58.2539) .. (54.1367,60.7148);

\end{tikzpicture}
\par\vspace{2.2cm}
	
	\textsc{\LARGE Institut für Numerische Simulation}\par\vspace{0.3cm}
	\textsc{\small Rheinische Friedrich-Wilhelms-Universität Bonn}\par\vspace{0.9cm}
	
	\vfill

	\begin{mdframed}[rightline=false,linewidth=0.5mm,leftline=false,innerbottommargin=\baselineskip,innertopmargin=\baselineskip]
	\centering\textbf{\huge Scientific Computing II}
	\end{mdframed}
	
	\vspace{1.5cm}
	
	\begin{minipage}{0.4\textwidth}
		\centering\large
		\textit{held by:} \\
		Prof.\ Dr.\ Jochen \textsc{Garcke}
	\end{minipage}
	\vfill
	\vfill
	\vfill
	\vfill
	\begin{minipage}{0.4\textwidth}
		\centering\large
		\textit{lectures notes created by:}\\
		Christian \textsc{Pfeiffer}
	\end{minipage}
	
	\vfill

	{\large \today}

\end{titlepage}

\printacronyms
\tableofcontents

\mainmatter

\subfile{Lectures/VL-12-4-2016}
\subfile{Lectures/VL-14-4-2016}
\subfile{Lectures/VL-19-4-2016}
\subfile{Lectures/VL-21-4-2016}
\subfile{Lectures/VL-26-4-2016}
\subfile{Lectures/VL-28-4-2016}
\subfile{Lectures/VL-3-5-2016}
\subfile{Lectures/VL-10-5-2016}
\subfile{Lectures/VL-12-5-2016}

\backmatter
\printindex[keydefinitions]
\printbibliography[heading=bibintoc]

\end{document}