\documentclass[../skript.tex]{subfiles}

\begin{document}
\chapter{Linear-quadratic elliptic control problems} % 2
\label{sec:c2}
\section{Sobolev spaces} % 2.3
\label{sec:c2e1}
In the following $\Omega \subseteq \R^N$ will be a \emph{domain} (i.e.\ open and connected), and $\Gamma = \partial \Omega$.
\begin{definition}[spaces of continuous functions] % Def 2.1
\label{def:c2e1}
Let $\Omega$ be a bounded domain. The space $C(\bar{\Omega})$ consists out of all continuous functions $y : \bar{\Omega} \to \R$. The space $C^k(\bar{\Omega})$ consists out of all functions$y \in C^k(\Omega)$ such that $D^\alpha y$ can be continuously extended to $\bar{\Omega}$ for all multi-indices $|\alpha| \leq k$. We identify $D^\alpha y$ with its extension.
These spaces are Banach spaces under the norms
\[
	\| y \|_{C(\bar{\Omega})} = \max_{x \in \bar{\Omega}} |y(x)|, \quad \| y \|_{C^k(\bar{\Omega})} = \sum_{|\alpha|\leq k} \| D^\alpha y \|_{C(\bar{\Omega})}.
\]
The \emph{set $C_0^\infty(\Omega)$} encompasses all infinitely continuously differentiable functions with compact support.
\end{definition}
\pagebreak
\paragraph{Regular domains}
\begin{definition} % Def 2.2
\label{def:c2e2}
A bounded domain $\Omega \subseteq \R^N$, $N \geq 2$, with boundary $\Gamma$ is said to be of class $C^{k, 1}$, $k \in \N \cap \{ \infty \}$ if there exist finitely many local coordinate systems $S_1, \ldots, S_M$, functions $h_1, \ldots, h_M$, and constants $a, b > 0$ such that
\begin{enumerate}[(i)]
\item $h_i \in C^k(\bar{Q}_{N-1})$, where 
\[
	\bar{Q}_{N-1} = \{ \eta = (y_1, \ldots, y_{N-1}) \mid |y_i| \leq a, i = 1, \ldots, N-1 \}
\]
and all $D^\alpha h_i$ are Lipschitz-continuous.
\item For $P \in \Gamma$ there exists a coordinate system $S_i$ such that
\[
	P = (\eta, h_i(\eta)), \quad \eta \in Q_{N-1} = \Int \bar{Q}_{N-1}
\]
\item In the coordinate system $S_i$ it holds
\begin{IEEEeqnarray*}{rCl"C"rCl'rClCl}
(\eta, y_N) &\in& \Omega & \Longleftrightarrow & \eta &\in& \bar{Q}_{N-1}, & h_i(\eta) &<& y_N &<& h_i(\eta) + b, \\
(\eta, y_N) &\notin& \Omega & \Longleftrightarrow & \eta &\in& \bar{Q}_{N-1}, & h_i(\eta) -b &<& y_N &<& h_i(\eta).
\end{IEEEeqnarray*}
\end{enumerate}
A domain of class $C^{0, 1}$ is called \emph{Lipschitz domain}. The \emph{Lebesgue measure} on $\Gamma$ will be denotes by $\mathrm{d}s$.
\end{definition}
\paragraph{Sobolev spaces}
If $\Omega$ is a bounded Lipschitz domain, and $y, v \in C^1(\bar{\Omega})$, then the \emph{partial integration formula}
\[
	\int_\Omega v(x) D_i y(x) \dx = \int_\Gamma v(x) y(x) \nu_i(x) \ds - \int_\Omega y(x) D_i v(x) \dx
\]
holds for every $i = 1, \ldots, n$ ($\nu_i$ is the outer normal vector of $\Omega$). If $v \in C_0^\infty(\Omega)$, and $y \in C^k(\bar{\Omega})$, then
\[
	\int_\Omega v(x) D^\alpha y(x) \dx = (-1)^{|\alpha|} \int_\Omega y(x) D^\alpha v(x) \dx \quad \forall |\alpha| \leq k.
\]
\begin{definition}[weak derivatives] % Def 2.3
\label{def:c2e3}
Let $y \in L_{1, \loc}(\Omega)$ (i.e.\ the restrictions to compact sets are integrable), and $\alpha$ a multi-index. A function $w \in L_{1, \loc}$ is called \emph{weak derivative of $y$ of order $\alpha$} if
\[
	\int_\Omega v(x) \cdot w(x) \dx = (-1)^{|\alpha|} \int_\Omega y(x) D^\alpha v(x) \dx \quad \forall v \in C_0^\infty(\Omega).
\]
We write $D^\alpha y = w$.
\end{definition}
\begin{definition}[Sobolev spaces] % Def 2.4
\label{def:c2e4}
Let $1 \leq p \leq \infty$ and $k \in \N$. The \emph{Sobolev space $W^{k,p}(\Omega)$} consists out of all functions $y \in L_p(\Omega)$ possessing weak derivatives $D^\alpha y$ with $D^\alpha y \in L_p(\Omega)$ for all $|\alpha| \leq k$. It is a Banach space under the norm
\begin{IEEEeqnarray*}{u'rCl"l}
	& \| y \|_{W^{k,p}(\Omega)} &=& \left( \sum_{|\alpha| \leq k} \int_\Omega |D^\alpha y(x)|^p \dx \right)^\frac{1}{p} & \text{if } p < \infty, \\
	or & \| y \|_{W^{k, \infty}(\Omega)} &=& \max_{|\alpha| \leq k} \|D^\alpha y \|_{L^\infty(\Omega)} & \text{if } p = \infty.
\end{IEEEeqnarray*}
If $p = 2$, we write $H^k(\Omega) \coloneqq W^{k, 2}(\Omega)$. The spaces $H^k$ are Hilbert spaces. For instance, the inner product in $H^1(\Omega)$ is
\[
	\langle y, v \rangle_{H^1(\Omega)} = \int_\Omega y \cdot v \dx + \int_\Omega \nabla y \nabla v \dx.
\]
\end{definition}
\paragraph{Embedding theorems}
\begin{theorem}[Sobolev embedding theorem] % Thm 2.5
\label{thm:c2e5}
Let $\Omega \subseteq \R^N$ be a bounded Lipschitz domain, $1 < p < \infty$, and $m \in \N \cup \{ \infty \}$. Then the following embeddings are continuous:
\begin{IEEEeqnarray*}{r"rCl"l}
mp < N: & W^{m,p}(\Omega) &\hookrightarrow& L^q(\Omega), & \text{if  } 1 \leq q \leq \frac{Np}{N - mp} \\ 
mp = N: & W^{m,p}(\Omega) &\hookrightarrow& L^q(\Omega), & \text{for all  } 1 \leq q \leq \infty \\ 
mp > N: & W^{m,p}(\Omega) &\hookrightarrow& C(\bar{\Omega}). \\ 
\end{IEEEeqnarray*}
\end{theorem}
\begin{example}
By the theorem, the following embeddings hold:
\begin{IEEEeqnarray*}{r"rCl}
\Omega \subseteq \R^2: & H^1(\Omega) = W^{1,2}(\Omega) &\hookrightarrow& L^q \quad \text{for all } 1 \leq q < \infty \\
\Omega \subseteq \R^3: & H^1(\Omega) &\hookrightarrow& L^6
\end{IEEEeqnarray*}
\end{example}
\begin{theorem}[Rellich] % Thm 2.6 
\label{thm:c2e6}
Let $\Omega \subseteq \R^N$ be a bounded Lipschitz domain, $1 \leq p < \infty$, and $m \geq 1$. Then bounded sets in $W^{m,p}(\Omega)$ are precompact in $W^{m-1,p}(\Omega)$.
\end{theorem}
\paragraph{Trace operator}
\begin{definition} % Def 2.7
\label{def:c2e7}
The \emph{space $W_0^{k,p}(\Omega)$} is the closure of $C_0^\infty(\Omega)$ in $W^{k,p}(\Omega)$. As such it is a closed subspace of $W^{k,p}(\Omega)$.
For $p = 2$, we write $H_0^k(\Omega) \coloneqq W_0^{k,2}(\Omega)$.
The functions $y \in W_0^{k,p}(\Omega)$ are considered to be zero on the boundary $\Gamma = \partial \Omega$.
\end{definition}
For inhomogeneous boundary conditions one uses the trace operator.
\begin{theorem}[trace operator] % Thm 2.8
\label{thm:c2e8}
Let $\Omega \subseteq \R^N$ be a bounded Lipschitz domain, and $1 \leq p \leq \infty$. There exists a bounded and linear operator $\tau : W^{1, p}(\Omega) \to L^p(\Gamma)$, called the \emph{trace operator}, such that for all $y \in W^{1,p}(\Omega) \cap C(\bar{\Omega})$ it holds
\[
	\tau y = y|_\Gamma.
\]
We always write $\tau y \eqqcolon y|_\Gamma$ and call it the \emph{trace of $y$ on $\Gamma$}.
One can show $H_0^1(\Omega) = \{ y \in H^1(\Omega) \mid y|_\Gamma = 0 \}$.
\end{theorem}
\paragraph{Generalized Friedrich inequality}
\begin{theorem} % Thm 2.9
\label{thm:c2e9}
Let $\Omega \subseteq \R^N$ be a bounded Lipschitz domain and $\Gamma_1 \subseteq \Gamma$ be measurable with positive measure in $\Gamma$. There exists a constant $c_{\Gamma_1} > 0$ such that
\[
	\| y \|_{H^1(\Omega)}^2 \leq c_{\Gamma_1} \left( \int_\Omega \| \nabla y \|^2 \dx + \left( \int_{\Gamma_1} y \ds \right)^2 \right) \quad \forall y \in H^1(\Omega).
\]
\end{theorem}
\section{Weak solution of elliptic PDEs} % 2.2
\label{sec:c2e2}
\paragraph{The Lax-Milgram Lemma}
Consider a \emph{variational equation}:
\begin{equation}
\label{eq:c2e2-star}
	\boxed{a(y, v) = F(v) \quad \forall v \in V}
	\tag{$\star$}
\end{equation}
Here $a : V \times V \to \R$ is a bilinear form on a real Hilbert space $V$, and $F \in V^*$.
\begin{theorem}[Lax-Milgram] % Thm 2.10
\label{thm:c2e10}
Assume there exists constants $\alpha_0, \beta_0 > 0$, \ac{st}
\begin{enumerate}[(i)]
\item $|a(y, v)| \leq \alpha_0 \| y \|_V \cdot \| v \|_V \quad \forall y, v \in V$ (``\emph{Boundedness}'')
\item $a(y,y) \geq \beta_0 \| y \|_V^2 \quad \forall y \in V$ (``\emph{coercivity}'')
\end{enumerate}
Then \cref{eq:c2e2-star} admits a unique solution for every $F \in V^*$, and there exists $c > 0$ (independent of $F$), \ac{st}
\[
	\| y \|_V \leq c \| F \|_{V^*} \quad \text{(``\emph{continuous dependence on data}'')}.
\]
\end{theorem}
\begin{remark} % Rem 2.11
\label{rem:c2e11}
If $a$ is symmetric, the results follow instantly from the results in \cref{sec:c1e2}, since \cref{eq:c2e2-star} is equivalent to
\[
	\nabla f(y) = 0, \quad \text{where } f(y) = \frac{1}{2} a(y, y) - F(y).
\]
The function $J$ is strictly convex and coercive, and hence possesses a unique critical point, its minimizer on V, existing by \cref{thm:c1e14}.
The ``nontrivial'' assertion is that a $a$ does not bee to be symmetric.
However, in many cases, e.g.\ the Poisson equation considered next, we have a symmetric and bilinear $a$, and thus weak solutions have an interpretation as ``minimization of energy''. 
\end{remark}
\paragraph{Model problem: Poisson equation}
\begin{equation}
\label{eq:c2e2-star-Poisson}
\tag{$\star$}
\begin{IEEEeqnarraybox}[][c]{rCl"l}
-\lapl y &=& f & \text{in } \Omega \\
y &=& 0 \text{on } \Gamma
\end{IEEEeqnarraybox}
\end{equation}
Here, $f \in L^2(\Omega)$.\\
\underline{Weak formulation:} multiply by $v \in C_0^\infty(\Omega)$ and integrate by parts:
\[
	\int_\Omega \nabla y \nabla v \dx = \int_\Omega fv \dx.
\]
By density, this transfers to $v \in H_0^1(\Omega)$. Also the formula makes sense if $y \in H_0^1(\Omega)$. Then the boundary condition is ``naturally'' satisfied.
\paragraph{Weak\slash{}variational formulation}
Find $y \in H_0^1(\Omega)$ such that
\begin{equation}
\label{eq:c2e2-starstar}
\tag{$\star\star$}
\boxed{\underbrace{ \int_\Omega \nabla y \nabla v \dx }_{a(y, v)} = \underbrace{\int_\Omega fv \dx}_{= F(v)} \quad \forall v \in H_0^1(\Omega)}
\end{equation}
This is equivalent to
\[
	\min \frac{1}{2} \int_\Omega \| \nabla y \|^2 \dx - \int_\Omega f \cdot y \dx.
\]
\begin{definition} % Def 2.12
\label{def:c2e12}
A solution $y$ of \cref{eq:c2e2-starstar} is called weak solution of \cref{eq:c2e2-star-Poisson}.
\end{definition}
\begin{theorem} % Def 2.13
\label{def:c2e13}
For every $f \in L^2(\Omega)$ there exists a unique weak solution $y \in H_0^1(\Omega)$ and
\[
\| y \|_{H^1(\Omega)} \leq c_p \| f \|_{L^2(\Omega)} \quad \text{($c_p > 0$ independent of $f$)}
\]
\end{theorem}
\begin{proof}
Apply Lax-Milgram for $V = H_0^1(\Omega)$, $a(y, v) = \int_\Omega \nabla y \nabla v \dx$, $F(v) = \int f v \dx$.
\begin{enumerate}[(i)]
\item $|a(y, v)| \leq \| \nabla y \|_{L_2} \cdot \| \nabla v \|_{L_2} \leq \| y \|_{H^1} \cdot \| v \|_{H^1}$ \checkmark
\item For coercivity, we have
\begin{IEEEeqnarray*}{rCl}
a(y, y) &=& \frac{1}{2} \| \nabla y \|_{L_2}^2 + \frac{1}{2} \| \nabla y \|_{L_2}^2 \\
&\overset{\text{\cref{thm:c2e9}}, \; y \in H_0^1}{\geq}& \frac{1}{2} \| \nabla y \|_{L_2}^2 + \frac{1}{2c} \| y \|_{L_2}^2 \\
&\geq& \min \left( \frac{1}{2}, \frac{1}{2c} \right) \| y \|_{H^1}^2.
\end{IEEEeqnarray*}
Also $F \in V^*$, since
\[
|F(v)| \leq \| f \|_{L^2} \cdot \| v \|_{L^2} \leq \| f\|_{L^2} \cdot \| v \|_{H^1}.
\]
\end{enumerate}
\end{proof}
\end{document}