\documentclass[../skript.tex]{subfiles}

\begin{document}
\section{Lagrange multipliers for smooth equality constraints}\label{sec:c1e3}
\underline{Aim:} Lagrange multiplier rule for a problem of the type
\begin{problemnonumb}
	minimize $f(u)$, subject to $g(u) = 0$.
\end{problemnonumb}
Here, $f,g$ are assumed to be Fréchet-differentiable. To do so, one needs to show that the \emph{tangent space} of
\[
	M=\{u\in U\,:\,g(u)=0\}
\]
at a point $u_0$ consists of $h$, \ac{st} $g'(u_0)h = 0$. This turns out to be nontrivial in infinite-dimensional spaces.
\begin{theorem}[Implicit function theorem]\label{thm:c1e26}
	Let $U,V,W$ be Banach-spaces, and let $\Phi:\mathcal{O}(u_0,v_0)\subseteq U\times V\to W$ %O = Neighborhood of u_0,v_0
	be Fréchet-differentiable and defined on an open neighborhood $\mathcal{O}(u_0,v_0)$ of $(u_0,v_0)\subseteq U\times V$ with $\Phi(u_0,v_0) = 0$.
	Suppose that 
	\[
		\Phi_v(u_0,v_0)=\frac{\partial\Phi}{\partial v}(u_0,v_0)\,:\,V\to W
	\]
	is bijective. The, there exist $r>0, R>0$ \ac{st} for $\|u-u_0\|<r$ there exists a \emph{unique} $v=v(u)\in V$ satisfying $(u,v)\in\mathcal{O}(u_0,v_0),\,\|v-v_0\|<R$ and $\Phi(u,v) = 0$. The map $u\mapsto v(u)$ is Fréchet-differentiable in $N_r=\{u\in U:\|u-u_0\|<r\}$. 
\end{theorem}
\begin{proof}
	Assume $u_0=0, v_0=0, U\not=\{0\}, V\not=\{0\}, W\not=\{0\}$. Define 
	\begin{IEEEeqnarray*}{rCl}
		F(u,v) &\coloneqq& \Phi_v(0,0)v - \Phi(u,v) \nonumber \\
		A(u)v &\coloneqq& \left[\Phi_v(0,0)\right]^{-1} F(u,v) 
	\end{IEEEeqnarray*}
	Then, the problem $\Phi(u,v) = 0$ is equivalent to 
	\begin{equation}
	\label{eq:c1e3-star}
		v = A(u)v
		\opttag{$\star$}
	\end{equation}
	 (fix point-problem). By the \emph{Banach inverse mapping theorem} we have $\left[\Phi_v(0,0)\right]^{-1}$ is a bounded linear map. Since the set of invertible operators on $V$ is open (can be shown by \emph{Neumann series}) in the space $\mathcal{L}(V,W)$, $\left[\Phi_v(u,v)\right]^{-1}$ also exists and $\left[\Phi_v(u,v)\right]^{-1}\leq \text{const}$ for $u,v$ small enough.

	We observe that 
	\[
		F_v(u,v) = \Phi_v(0,0) - \Phi_v(u,v)
	\]
	Since $F_v$ is continuous in $(0,0)$ by assumption and $F_v(0,0) = 0$, we can apply the (one dimensional) \emph{Taylor theorem} that
	\begin{IEEEeqnarray*}{rCl}
		\|F(u,v)-F(u,w)\| &\leq& \sup_{0\leq \Theta\leq 1} \|F_v(u,v+ \Theta(w-v))\|\cdot\|v-w\| \\
		&\leq& \smallo(1)\|v-w\| \quad \text{ for } u,v,w\to 0.
	\end{IEEEeqnarray*}
	Since $F$ is continuous at $(0,0)$ and also $F(0,0) = 0$, we also get
	\begin{IEEEeqnarray*}{rCl}
		\|F(u,v)\| &\leq& \|F(u,v)-F(u,0)\| + \|F(u,0)\| \\
		&\leq& \smallo(1)\|v\| + \|F(u,0)\|.
	\end{IEEEeqnarray*}
	Where $F(u,0)\to 0$ for $u\to 0$. Let now $M_R=\{v\in V:\|v\| < R\}$. Then for $r>0$ and $R>0$ small enough we get
	\[
		\|A(u)v\|\leq R,\quad\|A(u)v-A(u)w\|\leq\frac{1}{2}\|v-w\|
	\] 
	for all $v,w\in M_R$ and a fixed (but arbitrary) $u\in N_r$. The \emph{Banach fixed-point theorem} now yields a unique fixed point $v = v(u)$ of the \cref{eq:c1e3-star}.
	
	We can choose $r,R$ so small, that $(u,v)\mapsto\left[\Phi_v(u,v)\right]^{-1}$ is continuous on $N_r\times M_R$ to $\mathcal{L}(W,V)$. To show that $u\mapsto v(u)$ is continuous, we know that
	\begin{IEEEeqnarray*}{rCl}
		\|v(u)-v(z)\| &=& \|A(u)v(u)-A(z)v(z)\| \\
		&\leq& \underbrace{\|A(u)(v(u)-v(z))\|}_{\leq\frac{1}{2}\|v(u)-v(z)\|} + \underbrace{\|(A(u)-A(z))v(z)\|}_{\to 0\text{ for }u\to z} 
	\end{IEEEeqnarray*}
	Then $\|v(u)-v(z)\|\to 0$ for $u\to z$. To show that it is $C^1$, consider for fixed $u$ and $h$ small enough
	\[
		k= v(u+h)-v(u).
	\]
	Since $\Phi\in C^1$, it follows from
	\[
		0 = \Phi(u+h,v+k) - \Phi(u,v),
	\]
	that
	\[
		0 = \Phi_u(u,v)h + \Phi_v(u,v)k.
	\]
	Hence 
	\[
		k = - \left[\Phi_v(u,v)\right]^{-1}\left(\Phi_u(u,v) + \smallo(\|h\|+\|k\|)\right)
	\]
	for $h$ small enough (since $k\to 0$ for $h\to 0$). This implies
	\[
		\|k\|\leq \text{const}\|h\|
	\]
	and therefore 
	\[
		k = v(u+h)-v(u) = -\left[\Phi_v(u,v(u))\right]^{-1}\left(\Phi_u(u,v(u))h+\smallo(\|h\|)\right).
	\]
	Hence $v'(u) = -\left[\Phi_v(u,v(u))\right]^{-1}\Phi_u(u,v(u))$. 
\end{proof}

\subsection{The submersion theorem (generalized implicit functions)}
\begin{definition}
\label{def:c1e27}
	Let $M\subseteq U$ and $u_0\in M$. Then $h\in U$ is called a \emph{tangent vector to $M$ at $u_0$} if there exists an \emph{admissible curve} $\gamma:[0,\varepsilon)\to U$ \ac{st} 
	\[
		\gamma(t) = u_0 + th + \smallo(t)\in M,\quad\forall t\in[0,\varepsilon)
	\]
	($\gamma'(0) = h$). Then, the set of all tangent vectors is a closed cone, the \emph{tangent cone at $u_0$}. If the tangent cone is a \emph{linear space} then it is called the \emph{tangent space to $M$ at $u_0$} and denoted by $T_{u_0}(M)$. 
\end{definition}
\begin{definition}[Submersion]
\label{def:c1e28}
	Let $U,V$ be Banach spaces. A mapping $g:\mathcal{D}(g)\subseteq U\to V$ is called a \emph{submersion} at $u_0\in\mathcal{D}(g)$, if the following statements hold:
	\begin{enumerate}[(i)]
		\item \label[equation]{def:c1e28-i} $g\in C^1$ in a neighborhood of $u_0$,
		\item \label[equation]{def:c1e28-ii} $g'(u_0)\,:\,U\to V$ is surjective,
		\item \label[equation]{def:c1e28-iii} The null-space $\mathcal{N}(g'(u_0))$ splits $U$: There exists a continuous projection $P$ on $U$, \ac{st}
				\[
					P(U) = \mathcal{N}(g'(u_0))
				\]
				and
				\[
					U = \mathcal{N}(g'(u_0)) \oplus (I-P)(u).
				\]
	\end{enumerate}
\end{definition}

\begin{remarknonumb}
	Condition \labelcref{def:c1e28-iii} is nontrivial in general Banach spaces. However, it holds in the following cases:
	\begin{enumerate}[(i)]
		\item $U$ is a Hilbert space (closed subsets of Hilbert spaces have a complement),
		\item $\dim(\mathcal{N}(g'(u_0))) < \infty$,
		\item $\codim(\mathcal{N}(g'(u_0))) < \infty$ (derivative has finite rank).
	\end{enumerate}

\end{remarknonumb}

\begin{theorem}[Ljusternik, 1934] % Thm 1.29
\label{thm:c1e29}
	Let $U,V$ be Banach spaces, $\mathcal{O}$ an open neighborhood of $u_0\in U$ and $g:\mathcal{O}\subseteq U\to V$ be a submersion at $u_0$. Then the tangent cone to the set
	\[
		M=\{u\in\mathcal{O}:\,g(u)=0\}
	\]
	at $u_0$ is a tangent space, namely
	\[
		T_{u_0}M = \mathcal{N}(g'(u_0)).
	\]
	Furthermore there exists a homeomorphism $\varphi\,:\,\tilde{\mathcal{O}}\subseteq T_{u_0}M\to M$ from an \emph{open neighborhood} $\tilde{\mathcal{O}}$ of zero in $T_{u_0}M$ to $M$, \ac{st}
	\[
		\varphi(h) = u_0+h + \smallo(\|h\|).
	\]
\end{theorem}
\begin{proof}
	Thanks to the splitting property, we can identify $U$ with $U_1\times U_2$, where $U_1=\mathcal{N}(g'(u_0))$ and $U_2 = (I-P)(u)$. We want to construct the homeomorphism first. Consider
	\[
		\Phi(u_1,u_2) = g(u_0+u_1+u_2)
	\]
	then $\Phi(0,0) = 0$. By the chain rule, $\Phi\in C^1$ in a neighborhood of $(0,0)$ and 
	\[
		A \coloneqq \frac{\partial\Phi}{\partial u_2}(0,0) = \left.g'(u_0)\right|_{U_2}.
	\]
	As $U_2$ is complimentary to $U_1 = \mathcal{N}(g'(u_0))$, it follows that $A$ is injective, as well as
	\[
		\mathcal{R}(A) = \mathcal{R}(g'(u_0)) = V.
	\]
	By the \emph{implicit function theorem} there exists a $C^1$ map $\Psi$, \ac{st} locally $\Phi(u_1,u_2) = 0$ \ac{iff} $u_2=\Psi(u_1)$. By the chain rule 
	\[
		\frac{\partial\Phi}{\partial u_1}(0,0) + A \Psi'(0) = 0.
	\]
	Since
	\[
		\frac{\partial\Phi}{\partial u_1}(0,0) = \left.g'(u_0)\right|_{U_1} = 0,
	\]
	it follows $\Psi'(0) = 0$. Since $\Psi(0) = 0$ we get
	\[
		\Psi(u_1) = \smallo(\|u_1\|).
	\]
	The required homeomorphism is
	\[
		\varphi(u_1) \coloneqq u_0 + u_1 + \Psi(u_1).
	\]
	It remains to show that the inverse of $\varphi$ is continuous. We already know that it is locally bijective from a neighborhood of zero in $\mathcal{N}(g'(u_0))$ to a (relative) neighborhood of $u_0$ in $M$. To see that it's a homeomorphism, note that the inverse map is simply given by $u_1 = P(u-u_0),\,g(u) = 0$.

	If $h$ is a tangent vector and $\gamma$ the corresponding admissible curve, then $g(\gamma(t))=0$ (for $t$ small enough) implies by chain rule that $g'(u_0)\gamma'(0)=h=0$.
	Conversely, if $h\in\mathcal{N}(g'(u_0))$ then $t\mapsto \varphi(th)$ is an admissible curve for $t$ small enough.   
\end{proof}

\subsection{Generalized Ljusternik theorem}

\begin{definition} % Thm 1.30
\label{def:c1e30}
	Let $g\,:\,\mathcal{D}(g)\subseteq U\to V$ be a $C^1$ function. A point $u_0\in\mathcal{D}(g)$ is called \emph{regular point of $g$}, if $g'(u_0)\,:\,U\to V$ is surjective: $\mathcal{R}(g'(u_0)) = V$.
\end{definition}

\begin{theorem} % Thm 1.31
\label{thm:c1e31}
	Let $U$, $V$ be Banach spaces and $\mathcal{O}$ be an open neighborhood of a regular point $u_0$ of $g:\mathcal{D}(g)\subseteq U\to V$. Then the tangent cone to the set $M = \{u\in\mathcal{O}:\,g(u) = 0\}$ at $u_0$ is a tangent space and
	\[
		T_{u_0}M = \mathcal{N}(g'(u_0)).
	\]
\end{theorem}
\begin{proof}
	\cite[Chapter 0.2.4]{Ioffe} and 
	\cite[Chapter 8, § 10]{Ljusternik}.
\end{proof}
\end{document}