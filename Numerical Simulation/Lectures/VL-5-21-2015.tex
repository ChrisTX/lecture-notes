\documentclass[../skript.tex]{subfiles}

\begin{document}
\pagebreak
\section{Necessary and sufficient optimality conditions} % Sec 2.4
\label{sec:c2e4}
\begin{problem}[Abstract setting]
\label{prb:c2e4-star}
Let $U$, $V$ be Hilbert spaces.
\begin{IEEEeqnarray*}{u"l}
minimize & J(y, u) = \frac{1}{2} \| y - y_\Omega \|_{V}^2 + \frac{\lambda}{2} \| u \|_U^2 \\
subject to & y \in V \quad a(y, v) = F_u(v) \quad \forall v \in V \\
and & u \in U_{ad} \;\; \text{convex, closed}.
\end{IEEEeqnarray*}
Under the assumption that $(u, v) \mapsto F_u(v)$ is bilinear and bounded, there exists $\tilde{F} : V \to U$ \ac{st}\ $F_u(v) = \langle u, \tilde{F} v \rangle_U$.

\underline{Control-to-solution:} $S : U \to V$ bounded \ac{st}\ $a(Su, v) = F_u(v) \quad \forall v \in V$.

\underline{Reduced problem:}
\begin{IEEEeqnarray*}{u"l}
minimize & f(u) \coloneqq \frac{1}{2} \| Su - y_\Omega \|_V^2 + \frac{\lambda}{2} \| u \|_U^2 \\
subject to & u \in U_{ad}.
\end{IEEEeqnarray*}
\end{problem}
Recall \cref{thm:c1e25}:
\begin{theoremnonumb}[Variational inequality] % Theorem 1.25
$u_0$ solves the problem if and only if
\[
	\langle \nabla f(u), u - u_0 \rangle_U = \langle S^*(Su_0 - y_\Omega) + \lambda u, u - u_0 \rangle_{U} \geq 0 \quad \forall u \in U_{ad}.
\]
The solution is unique if $\lambda > 0$ or $S$ is injective.
\end{theoremnonumb}
\underline{Lagrange function and adjoint state}
\[
	\mathcal{L}(y, u, p) \coloneqq J(y, u) + a(y, p) - F_u(p), \quad y, p \in V, u \in U
\]
This functions eliminates the equality constraint.
The function $\mathcal{L}$ is Fréchet-differentiable with partial derivatives:
\begin{IEEEeqnarray*}{rCl"l}
	\langle \mathcal{L}_y(y, u, p), \psi \rangle &=& \langle y - y_\Omega, \psi \rangle - a(\psi, p) & \forall \psi \in V, \\
	\langle \mathcal{L}_p(y, u, p), \xi \rangle &=& a(y, \xi) - F_u(\xi) & \forall \xi \in V, \\
	\langle \mathcal{L}_u(y, u, p), \varphi \rangle &=& \lambda \langle u, \varphi \rangle + F_\varphi(p) & \forall \varphi \in U.
\end{IEEEeqnarray*}
\begin{theorem} % Thm 2.19
\label{thm:c2e19}
A pair $(y_0, u_0) \in V \times U_{ad}$ solves \cref{prb:c2e4-star} if and only if there exists a $p_0 \in V$ such that
\begin{IEEEeqnarray*}{rCl"l}
	\langle \mathcal{L}_y(y_0, u_0, p_0), v \rangle &=& 0 & \forall v \in V, \\
	\langle \mathcal{L}_p(y_0, u_0, p_0), v \rangle &=& 0 & \forall v \in V, \\
	\langle \mathcal{L}_u(y_0, u_0, p_0), u - u_0 \rangle &\geq& 0 & \forall u \in U_{ad}.
\end{IEEEeqnarray*}
\end{theorem}
\begin{proof}
We show that the three conditions are equivalent to the variational inequality using $\tilde{F} p_0 = S^*(S u_0 - y_\Omega)$.

Assume the conditions hold at some point $(u_0, y_0, p_0)$. The second one means $y_0 = S u_0$. From the first one it follows 
\[
	\langle S u_0- y_\Omega, v \rangle = a(v, p_0) \quad \forall v \in V.
\]
Choosing $v = Sq$ with $q \in U$ arbitrary, then by definition of $S$
\begin{IEEEeqnarray*}{rCl"l}
	\langle S u_0 - y_\Omega, S q \rangle = a(S q, p_0) &=& F_q(p_0) & \forall q \in U. \\
	&=& \langle q, \tilde{F} p_0 \rangle.
\end{IEEEeqnarray*}
Therefore,
\[
	S^*(S u_0 - y_\Omega) = \tilde{F} p_0.
\]
Now the last inequality shows
\[
	0 \leq \lambda \langle u, u - u_0 \rangle + F_{u - u_0}(p_0) = \langle \lambda u + S^*(S u_0 - y_\Omega), u - u_0 \rangle \quad \forall u \in U_{ad}.
\]
So $u_0$ solves the problem by \cref{thm:c1e25}.
The converse direction is similar.
\end{proof}
\begin{remark} % Rem 2.20
\label{rem:c2e20}
The proof shows that
\[
	\nabla f(u) = S^*(S u - y_\Omega) + \lambda u = \tilde{F} p + \lambda u,
\]
where $p \in V$ is found by solving
\begin{IEEEeqnarray*}{rCl"l}
a(y, v) &=& F_u(v) & \forall v \in V, \\
a(v, p) &=& \langle y - y_\Omega, v \rangle & \forall v \in V.
\end{IEEEeqnarray*}
By Lax-Milgram, the second problem always has a solution $p$.
\end{remark}
\begin{definition}[Adjoint state] % Def 2.21
\label{def:c2e21}
The second variational equality in \cref{rem:c2e20} is called \emph{adjoint equation} and the solution $p$ is called the \emph{adjoint state} for $y$.
If $a$ is self-adjoint, the adjoint equation is ``of the same form'' as the state equation.
\end{definition}
\underline{Optimal stationary heating}
\begin{IEEEeqnarray*}{u"l}
minimize & \frac{1}{2} \| y - y_\Omega \|_{L^2(\Omega)}^2 + \frac{\lambda}{2} \| u \|_{L^2(\Omega)}^2 \\
subject to &
\left\{ 
\begin{IEEEeqnarraybox}[][c]{rCl"l}
- \lapl y &=& \beta \cdot u & \text{in } \Omega \\
y &=& 0 & \text{on } \Gamma
\end{IEEEeqnarraybox} \right. \\
and & u_a(x) \leq u(x) \leq u_b(x) \quad \text{\ac{ae}} 
\end{IEEEeqnarray*}
Here,
\[
	F_u(v) = \int_\Omega \beta(x) u(x) v(x) \dx \quad  \text{and thus} \quad \tilde{F} v = \beta \cdot v.
\]
From \cref{thm:c2e19,rem:c2e20} we obtain: \\
\underline{Necessary and sufficient optimality}
\begin{IEEEeqnarray*}{c}
\begin{IEEEeqnarraybox}{rCl"rCl}
- \lapl y &=& \beta \cdot u & - \lapl p &=& y - y_\Omega \\
y|_\Gamma &=& 0 & p|_\Gamma &=& 0
\end{IEEEeqnarraybox} \\
u \in U_{ad} \\
\langle \beta \cdot p + \lambda u, \varphi - u \rangle_{L^2(\Omega)} \geq 0 \quad \forall \varphi \in U_{ad}
\end{IEEEeqnarray*}
\paragraph{Point-wise discussion}
\begin{lemma} % Lemma 2.22
\label{thm:c2e22}
The variational inequality
\[
	\langle \beta p + \lambda u, \varphi - u \rangle \geq 0 \quad \forall \varphi \in U_{ad}
\]
is satisfied if any only if for almost every $x \in \Omega$ it holds
\[
	\left( \beta(x) p(x) + \lambda u(x) \right) \cdot \left( w - u(x) \right) \geq 0 \quad \forall w \in [u_a(x), u_b(x)].
\]
\end{lemma}
\begin{proof}
Fix representatives of all functions.
The assertion is equivalent to
\[
	u(x) = \begin{cases}
	u_a(x) & \text{if } \beta(x) p(x) + \lambda u(x) > 0 \\
	\in [u_a(x), u_b(x)] & \text{if } \beta(x) p(x) + \lambda u(x) = 0 \\
	u_b(x) & \text{if } \beta(x) p(x) + \lambda u(x) < 0
	\end{cases}.
\]
Assume this was not true. \ac{wlog}\ the measurable set
\[
	A_+(u) = \{ x \in \Omega \midcolon \beta(x) p(x) + \lambda u(x) > 0 \}
\]
then contains a set $E_+ \subseteq A_+(u)$ of positive measure such that 
\[
	u(x) > u_a(x) \quad \forall x \in E_+.
\]
For the function
\[
	\varphi(x) = \begin{cases}
	u_a(x) & \text{if } x \in E_+ \\
	u(x) & \text{if } x \notin E_+
	\end{cases} \;\; \in U_{ad}
\]
it follows
\[
	\int_\Omega (\beta p + \lambda u ) (\varphi -u ) \dx = \int_{E_+} (\beta p + \lambda u )\underbrace{(\varphi - u)}_{< 0} < 0.
\]
The reverse direction can be shown similarly.
\end{proof}
\begin{theorem} % Thm 2.23
\label{thm:c2e23}
$u_0 \in U_{ad}$ is an optimal control for the stationary heating problem if and only if it satisfies one of the following minimum principles for almost all $x \in \Omega$.
\begin{itemize}
\item \emph{weak minimum principle}
\[
	\inf_{w \in [u_a(x), u_b(x)]} \left( \beta(x) p_0(x) + \lambda u_0(x) \right) w = \left( \beta(x) p_0(x) + \lambda u_0(x) \right) u_0(x).
\]
\item \emph{minimum principle}
\[
	\inf_{w \in [u_a(x), u_b(x)]} \beta(x) p_0(x) w + \frac{\lambda}{2} w^2 = \beta(x) p_0(x) u_0(x) + \frac{\lambda}{2} u_0(x)^2
\]
\end{itemize}
\end{theorem}
\begin{proof}
The weak minimum principle is stated in \cref{thm:c2e22}.
The necessary and sufficient condition for the minimum principle is
\[
	\left( \beta(x) p(x) + \lambda w_0 \right) \cdot (w - w_0) \geq 0 \quad \forall w \in [u_a(x), u_b(x)].
\]
Therefore, we can apply \cref{thm:c2e22} once again.
\end{proof}
\paragraph{Bang-bang control}
If $\lambda = 0$ and $\beta(x) p_0(x) \neq 0$ \ac{ae}, then \cref{thm:c2e22} states that \ac{ae}\ we have $u_0(x) = u_a(x)$ or $u_0(x) = u_b(x)$.
This is called a \emph{bang-bang control} and is likely \underline{discontinuous}.

\underline{Regularized case:}
\begin{theorem} % Thm 2.24
\label{thm:c2e24}
If $\lambda > 0$, then $u_0$ is an optimal control if and only if
\[
	u_0(x) = \mathbb{P}_{[u_a(x), u_b(x)]} \left\{ - \frac{1}{\lambda} \beta(x) p_0(x) \right\},
\]
where $\mathbb{P}_{[u_a(x), u_b(x)]}$ is the standard metric projection.
\end{theorem}
\begin{proof}
The theorem follows from \cref{thm:c2e23}.
\end{proof}
\underline{KKT conditions}
By \cref{thm:c1e50} and the conclusion afterwards, a solution of the KKT conditions for convex programs solves the program.
The KKT condition for $u_a(x) \leq u(x) \leq u_b(x)$ \ac{ae}\ reads (c.f.\ \cref{sec:c1e6} for $G$ and $K^+$):
\begin{IEEEeqnarray*}{rCl}
\nabla f(u) - \mu^a + \mu^b &=& 0 \\
\IEEEeqnarraymulticol{3}{c}{ u_a(x) \leq u(x) \leq u_b(x) \; \text{\ac{ae}}, \; \mu^a(x) \geq 0, \; \mu^b(x) \geq 0 \; \text{a.e.} } \\
\langle -u - u_a, \mu^a \rangle_{L^2} + \langle u - u_b, \mu^b \rangle_{L^2(\Omega)} &=& 0 \quad \text{(slack condition)}
\end{IEEEeqnarray*}
We have also seen that the slack conditions is equivalent to the point-wise conditions
\[
	(-u(x) - u_a(x)) \mu^a(x) = 0, \;\; (u(x) - u_b(x)) \mu^b(x) = 0 \quad \text{\ac{ae}}.
\]
The existence of multipliers is not implied by any results we have, but given $u_0$, the choices
\[
	\mu_0^a = [\nabla f(u_0)]_+, \quad \mu_0^b = - [\nabla f(u_0)]_-
\]
work. Finally, since
\[
	\nabla f(u) = \beta \cdot p + \lambda u
\]
(\cref{rem:c2e20}).
\begin{theorem}[Necessary and sufficient KKT conditions] % Thm 2.25
\label{thm:c2e25}
\begin{IEEEeqnarray*}{c}
\begin{IEEEeqnarraybox}{rCl"rCl}
- \lapl y &=& \beta u & -\lapl p &=& y - y_\Omega \\
y|_\Gamma &=& 0 & p|_\Gamma &=& 0
\end{IEEEeqnarraybox} \\
\beta p + \lambda u - \mu^a + \mu^b = 0\\
u_a \leq u \leq u_b, \;\; \mu^a \geq 0, \;\; \mu^b \geq 0 \quad \text{\ac{ae}} \\
\mu^a(u_a - u) = \mu^b(u - u_b) = 0 \quad \text{\ac{ae}}
\end{IEEEeqnarray*}
\end{theorem}
\end{document}