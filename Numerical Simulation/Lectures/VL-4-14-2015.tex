\documentclass[../skript.tex]{subfiles}

\begin{document}
\chapter{Variational principles in Banach spaces} % Chapter 1
\label{sec:c1}
\section{Existence results} % Section 1.1
\label{sec:c1e1}
\begin{problem} % Problem (1)
\label{prb:c1e1}
minimize $f(u)$, \ac{st} $u \in C$, where $C$ is a subset of a Banach space $U$.
\end{problem}
\underline{Question:} Do solutions exist?
\subsection{Sequential Weierstraß existence principle}
One is given
\begin{enumerate}[(i)]
\item a notion of ``convergence'' $u_n \rightsquigarrow u$ with respect to which the set $C$ is ``sequentially compact'', namely every sequence $(u_n) \subseteq C$ possesses a subsequence $(u_{n_k})$ \ac{st} there exists $u \in C: u_{n_k} \rightsquigarrow u$.
\item the function $f$ is ``sequentially lower semi-continuous'': $u_n \rightsquigarrow u$ in $C$ implies $f(u) \leq \liminf_{n \to \infty} f(u_n)$.
\end{enumerate}
Then, \cref{prb:c1e1} has a solution.
\begin{proof}
Let $(u_n)$ be a \emph{minimizing sequence}, that is, $f(u_n) \to \inf_{u \in C} f(u)$.
\Ac{wlog} we can assume $u_n \rightsquigarrow u_* \in C$.
Now
\[
	f(u_*) \leq \liminf f(u_n) = \inf_{u \in C} f(u)
\]
implies equality.
\end{proof}
\begin{examplenumb} % Example 1.1
\label{ex:c1e1}
\begin{enumerate}[(i)]
\item $C$ is compact in the norm topology,
\item $f$ is continuous in the norm topology.
\end{enumerate}
\end{examplenumb}
\subsection{Weak convergence}
The \emph{dual space} of $U$ consisting of all continuous linear functionals $\ell : U \to \R$ is denoted by $U^*$.
\begin{definition} % Definition 1.2
\label{def:c1e2}
A sequence $(u_n) \subseteq U$ is said to be \emph{weakly convergent to $u \in U$} if
\[
	\ell(u_n) \to \ell(u) \quad \forall \ell \in U^*.
\]
\emph{Notation:} $u_n \rightharpoonup u$.
\end{definition}
\begin{definition} % Definition 1.3
\label{def:c1e3}
A subset $C \subseteq U$ is called \emph{weakly sequentially closed}, if $(u_n) \subseteq C, u_n \rightharpoonup u$ implies $u \in C$. And it is called \emph{weakly sequentially compact} if additionally every sequence $(u_n) \subseteq C$ has a weakly convergent subsequence.
\end{definition}
\begin{definition} % Definition 1.4
\label{def:c1e4}
A function $f : C \to \R$ is called \emph{weakly sequentially lower \\ semi-continuous} if it holds
\[
	f(u) \leq \liminf_{n \to \infty} f(u_n)
\]
for all $(u_n) \subset C$ with $u_n \rightharpoonup u$.
\end{definition}
\begin{theorem} % Theorem 1.5
\label{thm:c1e5}
If $C$ is weakly sequentially compact and $f$ is weakly sequentially lower \\ semi-continuous, then \cref{prb:c1e1} has at least one solution.
\end{theorem}
\subsection*{Separation of convex sets}
\begin{remark*}[Sublinear functional]
	Let $p\in X^*$, so $p:X\to\R$. $p$ is called a \emph{sublinear functional}, if
	\begin{IEEEeqnarray*}{rCl}
		p(\xi + \eta) & \leq & p(\xi) + p(\eta),\quad  \forall \xi,\eta\in X\\
		p(\alpha\xi) &=& \alpha p(\xi), \quad \text{for some }\alpha \geq 0 
	\end{IEEEeqnarray*}
\end{remark*}
\begin{theoremnonumb}[Hahn-Banach-Theorem] % unnumbered
Let $X$ be a real linear space, $p : X \to \R$ be a sublinear functional and $X_0\subseteq X$ a linear subspace. If $\ell_0$ is a linear functional on $X_0$ such that 
\[
	\ell_0(\xi) \leq p(\xi) \quad \forall \xi \in X_0,
\]
then there exists a linear function $\ell$ on $X$ (``\emph{extension}'') such that
\[
	\ell(x) \leq p(x) \quad \forall x \in X, \quad \ell|_{X_0} = \ell_0.
\]
\end{theoremnonumb}
\begin{definitionnonumb}[Minkowski functional] % unnumbered
Let $A$ be a convex subset of a \emph{normed} $X$, such that $0 \in \Int A = A^\circ$. Then
\[
	p_A:X\to \R_0^+
\]
with 
\[
	p_A(\xi) = \inf \left\{ r \geq 0 : \frac{\xi}{r} \in A \right\}
\]
is sublinear functional and is called \emph{Minkowski functional}.
\end{definitionnonumb}
\begin{remark*}
	The Minkowski functional can be seen as a generalization to the norm on the space. In general, it can be built for each \emph{absorbing subset} $A$ of $X$ (that is: $\forall x\in X\exists r\geq 0:\,\alpha x\in A,\,\forall\|\alpha\|\leq r$).
\end{remark*}
\begin{definition} % Definition 1.6
\label{def:c1e6}
A linear functional $\ell$ on a real linear space $X$ \emph{separates the sets $A, B \subseteq X$}, if it holds that
\[
	\ell(a) \leq \ell(b) \quad \forall a \in A, b \in B.
\]
It strictly separates $A$ and $B$, if
\[
	\sup_{a \in A} \ell(a) < \inf_{b \in B} \ell(b).
\]
\end{definition}
\begin{theorem}[Separation theorem] % Definition 1.7
\label{thm:c1e7}
Let $A$ and $B$ be disjoint convex sets in a real \emph{normed} space $X$. If $\Int A = A^\circ \neq \emptyset$ then there exists a \emph{continuous} linear functional $\ell \in X^*$, $\ell \neq 0$, that separates $A$ and $B$.
\end{theorem}
\begin{proof}
\Ac{wlog} we assume $0 \in \Int A$. Let $b_0 \in B$, then $-b_0$ is an interior point of 
\[
	A - B = \{ a - b \; : \; a \in A, b \in B \},
\]
and so $0$ is an interior point of the convex set
\[
	K = A - B + \{ b_0 \}.
\]
Since $A \cap B = \emptyset$, we have $0 \notin A - B$, and so $b_0 \notin K$. Then, letting $p$ be the Minkowski functional of the set $K$, it holds that
\[
	p(b_0) \geq 1.
\]
Consider the linear functional
\[
\ell_0(\alpha b_0) = \alpha p(b_0)
\]
on the subspace $\spn\{b_0\}$.
One verifies
\[
	\ell_0(\alpha b_0) = \begin{cases}
	p(\alpha b_0) & \text{for } \alpha \geq 0, \\
	\alpha p(b_0) < 0 \leq p(\alpha b_0) & \text{for } \alpha < 0 
	\end{cases} \leq p(\alpha b_0).
\]
By the Hahn-Banach theorem, there exists an extension $\ell$ on $X$ such that
\[
	\ell(\xi) \leq p(\xi) \quad \forall \xi \in X.
\]
This implies
\[
	\ell(\xi) \leq p(\xi) \leq 1 \quad \forall \xi \in K,
\]
while $\ell(b_0) \geq 1$.
Therefore $\ell$ separates $K$ and $\{ b_0 \}$, which is equivalent to separating $A$ and $B$.
Since $0 \in \Int K$, there exists $\varepsilon > 0$ such that $\xi \in K$ for all $\xi$ with $\| \xi \| \leq \varepsilon$. By definition of the Minkowski functional
\[
	\ell(\xi) \leq p(\xi) \leq \frac{1}{\varepsilon} \| \xi \| \quad \forall \xi \in X,
\]
so $\ell$ is continuous.
\end{proof}
\begin{theorem}[strict separation of single points] % Theorem 1.8
\label{thm:c1e8}
Let $A$ be a closed convex subset of a normed space $X$ and let $\xi \notin A$. Then there exists a continuous linear functional that strictly separates $A$ and $\{ \xi \}$.
\end{theorem}
\begin{proof}
Since $X \setminus A$ is open, there exists an open ball $B$ around $\xi$ such that $A \cap B \neq \emptyset$. Since $B^\circ \neq \emptyset$, we can apply \cref{thm:c1e7} to obtain $\ell \in X^*$ such that
\[
	\ell(a) \leq \ell(\xi) + \ell(b - \xi) \quad \forall b \in B, a \in A.
\]
Since $\ell$ is continuous, there exists $\delta > 0$ such that
\[
	\ell(b - \xi) > - \delta \quad \forall b \in B.
\]
This shows
\[
	\sup_{a \in A} \ell(a) \leq \ell(\xi) - \delta < \ell(\xi).
\]
\end{proof}
\paragraph{Interpretation:} Given $\ell \in X^*$, the set
\[
	H(\ell, \alpha) = \{ \xi \in X \mid \ell(\xi) = \alpha \}
\]
is called a hyperplane. The sets
\begin{IEEEeqnarray*}{rCl}
	H^+(\ell, \alpha) &=& \{ \xi \in X \mid \ell(\xi) \leq \alpha \} \\
	H^-(\ell, \alpha) &=& \{ \xi \in X \mid \ell(\xi) \geq \alpha \} \\
\end{IEEEeqnarray*}
are called the generated \emph{half-spaces}. $\ell$ separates $A$ and $B$ \ac{iff} there exists $\alpha$ such that
\[
	\sup_{a \in A} \ell(a) \leq \alpha \leq \inf_{b \in B} \ell(b).
\]
So $A$ and $B$ belong to different half-spaces. One says the hyperplane $H(\ell, \alpha)$ separates $A$ and $B$.
\begin{theorem} % Theorem 1.9
\label{thm:c1e9}
A closed convex subset A of a normed space $X$ is \emph{weakly sequentially closed}.
\end{theorem}

\begin{proof}
Let $(\xi_n) \subseteq A$, with
\[
\ell(\xi_n) \to \ell(\xi) \quad \forall \ell \in X^*.
\]
This immediately implies that $A$ and $\xi$ cannot be strictly separated. By \cref{thm:c1e8}, $\xi \in A$.
\end{proof}
\begin{corollary}[Mazur's lemma] % Cor 1.10
\label{cor:c1e10}
Let $X$ be a normed space and $(\xi_n) \subseteq X, \xi_n \rightharpoonup \xi$. Then $\xi \in \conv\{ \xi_n \; : \; n \in \N \}$.
\end{corollary}
\subsection{Reflexive Banach spaces}
Recall that a Banach space $U$ is called \emph{reflexive} if the canonical map $U \to U^{**}$, $u \mapsto u''$, $u''(\ell) = \ell(u)$ is surjective (in which case it is an isometric isomorphism, because the canonical embedding of $U$ into $U^{**}$ is always linear and injective).
\begin{example}
\begin{itemize}
\item Hilbert spaces, 
\item $L^p$ spaces, $1 < p < \infty$
\end{itemize}
\end{example}
\begin{theorem} % Theorem 1.11
\label{thm:c1e11}
Closed balls in reflexive Banach spaces are weakly sequentially compact.
\end{theorem}
\begin{proof}
\cite{Alt}
\end{proof}
\begin{corollary} % Cor 1.12
\label{cor:c1e12}
A bounded, closed and convex set in a reflexive Banach space is weakly sequentially compact.
\end{corollary}
\begin{proof}
\cref{thm:c1e9,thm:c1e11}.
\end{proof}
\subsection{Main result}
\begin{lemma} % Lemma 1.13
\label{thm:c1e13}
Let $U$ be a Banach space, $C \subseteq U$ be closed and convex, and let $f : C \to R$ be convex and continuous.
Then $f$ is weakly sequentially lower semi-continuous.
\end{lemma}
\begin{proof}
Let $u_n \rightharpoonup u$, but assume there exists $\varepsilon > 0$ such that
\[
	f(u) \geq f(u_n) + \varepsilon
\]
for large enough $n$.
Hence,
\[
	u_n \in \mathcal{L} = \{ v \in C \mid f(v) \leq f(u) - \varepsilon \}
\]
for large enough $n$.
This set is closed and convex.
By \cref{thm:c1e9}, $u \in \mathcal{L}$, which is a contradiction.
\end{proof}
\begin{definition}[coercive function]
	Let $X$ normed space, $f:X\to\R$. $f$ is called \emph{coercive}, if for all sequences $\{x_n\}_n\subset X$ with $\lim_{n\to\infty}\|x_n\| = \infty$ it holds that 
	\[
		\lim_{n\to\infty} f(x_n) = \infty
	\]
\end{definition}
\begin{theorem} % Theorem 1.14
\label{thm:c1e14}
Let $U$ be a reflexive Banach space, $C \subseteq U$ be closed and convex, and $f : C \to R$ be convex and continuous. Assume one of the following holds:
\begin{enumerate}[(i)]
\item \label[equation]{thm:c1e14-i} $C$ is bounded,
\item \label[equation]{thm:c1e14-ii} $f$ is coercive.
\end{enumerate}
Then \cref{prb:c1e1} has at least one solution. If $f$ is strictly convex, then in either case the solution is unique.
\end{theorem}
\begin{proof}
In case \labelcref{thm:c1e14-i} use \cref{thm:c1e13,cor:c1e12,thm:c1e5}. In case \labelcref{thm:c1e14-ii} restrict the problem to
\[
	\tilde{C} = C \cap \{ u \mid f(u) \leq f(u_0) \}, u_0 \in C,
\]
which is closed, convex and bounded.
Uniqueness follows as usual: if two minimizers existed, the midpoint of the connecting line segment, which entirely belongs to $C$, takes a smaller function value - a contradiction.
\end{proof}
\end{document}