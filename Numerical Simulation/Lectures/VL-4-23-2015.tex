\documentclass[../skript.tex]{subfiles}

\begin{document}
\subsection{Lagrange multiplier rule}
\begin{problem} % (*)
\label{prob:c1e3-star}
minimize $f(u)$ subject to $g(u) = 0$,
where $f : U \to \R$, $g : U \to V$.
\end{problem}
Then
\[
	\mathcal{L}(u, \lambda_0, \lambda_1^*) = \lambda_0 f(u) + \lambda_1^* (g(u))
\]
is the Lagrange function and $\lambda_0 \in \R, \lambda_1^* \in V^*$ are the ``\emph{Lagrange multipliers}''.
\begin{theorem}[Lagrange multiplier rule] % Theorem 1.32 
\label{thm:c1e32}
Let $u_0$ be a local minimum of \cref{prob:c1e3-star} such that
\begin{enumerate}[(i)]
\item $f$ and $g$ are Fréchet-differentiable in a neighborhood of $u_0$,
\item $\mathcal{R}(g'(u_0))$ is closed.
\end{enumerate}
Then there exists $\lambda_0 \in \R$ and $\lambda_1^* \in V^*$ which do not vanish simultaneously, \ac{st}
\begin{IEEEeqnarray*}{rCl"l}
\mathcal{L}_u(u_0, \lambda_0, \lambda_1^*) = \lambda_0 f'(u) + g'(u_0)^* \lambda_1^* &=& 0 & \text{in } U^*, \\
\noalign{that is \vspace{\jot}} \lambda_0 f'(u) h + \lambda_1^* (g'(u_0)h) &=& 0 & \forall h \in U.
\end{IEEEeqnarray*}
($g'(u_0)^* \, : \, V^* \to U^*$ is denoting the adjoint operator).
If $g$ is in $C^1$ and $u_0$ is a regular point of $g$, then $\lambda_0 \neq 0$, so that \ac{wlog}\ $\lambda_0 = 1$.
\end{theorem}
\begin{proof}
We first consider the case that $u_0$ is a regular point, then $\mathcal{R}(g'(n_0)) = V$. In this case it holds with $M = \{ u \in U \, : \, g(u) = 0\}$ that $T_{u_0} M = \mathcal{N}(g'(n_0))$ by \cref{thm:c1e31}. Let $h \in \mathcal{N}(g'(u_0))$. Since $T_{u_0} M$ is a linear space, there exists an admissible curve $\gamma : (- \varepsilon, \varepsilon) \to M$ \ac{st} $\dot\gamma(0) = h$. Then the function $t \mapsto f(\gamma(t))$ attains a local minimum at $t=0$. By the chain rule,
\[
	f(u_0) \dot\gamma(0) = f'(u_0) h = 0.
\]
Since this holds for every $h \in \mathcal{N}(g'(u_0))$, this means that
\[
	f'(u_0) \in \left(\mathcal{N}(g'(u_0))\right)^\perp,
\]
(with $(\cdot)^\perp$ denoting the annihilator). By the closed range theorem,
\[
	\left(\mathcal{N}(g'(u_0))\right)^\perp = \mathcal{R}\left((g'(u_0))^*\right).
\]
In summary,
\[
	f'(u_0) = -g'(u_0)^* \lambda_1^*
\]
for some $\lambda_1 \in V^*$.
Assume now $\mathcal{R}(g'(u_0)) \neq V$, then there exists $\lambda_1^* \in V^*$, $\lambda_1^* \neq 0$, \ac{st}
\[
	\lambda_1^*(V) = 0 \quad \forall v \in \mathcal{R}(g'(u_0))
\]
(by Hahn-Banach), that is
\[
	\lambda_1^*(g'(u_0)h) = 0 \quad \forall h \in U.
\]
Setting $\lambda_0 = 0$, the Lagrange multiplier rule holds.
\end{proof}
\begin{examplenumb} % Ex 1.33
\label{ex:c1e33}
Consider
\begin{problemnonumb}
minimize $f(u)$ \ac{st} $g_1(u) = 0, \ldots, g_n(u) = 0$, $g_i : U \to \R$.
\end{problemnonumb}
\emph{Lagrange function:}
\[
	\mathcal{L}(u, \lambda_0, \lambda_1, \ldots, \lambda_n) = \lambda_0 f(u) + \sum_{i=1}^n \lambda_i g_i(u).
\]
\emph{Necessary condition:}
If $u_0$ is a local minimum, and all $f$ and $g_i$ are locally $C^1$, then there exists $\lambda_0, \lambda_1, \ldots, \lambda_n \in \R$ which do not vanish simultaneously \ac{st}
\[
	\mathcal{L}_u(u_0, \lambda_0, \lambda_1, \ldots, \lambda_n) = \lambda_0 f'(u) + \sum_{i=1}^n \lambda_i g_i'(u) = 0.
\]
If the $g'(u_0)$ are linearly independent, then $\lambda_0 \neq 0$ (and we choose it to be equal to 1).
\end{examplenumb}
\section{General Lagrange multipliers} % 1.4
\label{sec:c1e4}
\underline{Idea:} Let $f : C \to \R$ have a minimum at $u_0 \in \partial C$. Let $K_0$ be the cone of directions in which $f$ is (locally) decreasing and $K_1$ be the cone of directions that point from $u_0$ into $C$. Then we need
\[
	K_0 \cap K_1 = \{ 0 \} \text{ or } = \emptyset.
\]
Hence, we need conditions under which cones do not intersect. Thus, we need a separation theorem.
\begin{proposition}[Krein's extension theorem] % Prop 1.34
\label{prop:c1e34}
Let $U$ be a Banach space, $K$ a convex cone, and $L$ a linear subspace of $U$ \ac{st}
\[
	L \cap \Int K = \emptyset.
\]
If $\ell : L \to \R$ is a continuous linear functional such that
\[
	\ell(u) \geq 0 \quad \forall u \in L \cap K.
\]
Then $\ell$ can be extended to a continuous functional on $U$ \ac{st}
\[
	\ell(u) \geq 0 \quad \forall u \in K.
\]
\end{proposition}
\begin{proof}
Exercise.
\end{proof}
\subsection{Cones and dual cones}
\begin{definition} % Def 1.35
\label{def:c1e35}
Let $U$ be a Banach space. A subset $K \subseteq U$ is called a \emph{cone}, if $u \in K$ implies $\alpha u \in K$ for all $\alpha > 0$. The \emph{dual cone $K^+$} is defined as
\[
	K^+ = \{ \ell \in U^* \, : \, \ell(u) \geq 0 \quad \forall u \in K \}.
\]
\end{definition}
\begin{proposition} % Prop 1.36
\label{prop:c1e36}
Let $K$ be a cone in a Banach space.
\begin{enumerate}[(i)]
\item\label[equation]{prop:c1e36-i} $K^+$ is closed, convex and nonempty.
\item\label[equation]{prop:c1e36-ii} If $K \neq \emptyset$, then
\[
	\inf_{u \in K} \ell(u) = \begin{cases}
	0 & \text{if } \ell \in K^+ \\
	-\infty & \text{if } \ell \notin K^+
	\end{cases}.
\]
\end{enumerate}
\end{proposition}
\begin{proof}
\labelcref{prop:c1e36-i} is trivial.

\labelcref{prop:c1e36-ii}: If $\ell \in K^+$, then
\[
	\alpha \underbrace{ \ell(u) }_{\geq 0} = \ell\underbrace{(\alpha u)}_{\in K} \to 0 \quad \text{for } \alpha \to 0.
\]
If $\ell \notin K^+$, then there exists $u \in K$, \ac{st}
\[
	\ell(\alpha u) = \alpha \underbrace{\ell(u)}_{< 0} \to -\infty \quad \text{for } \alpha \to \infty.
\]
\end{proof}
\subsection{Dubovickii-Miljutin Lemma}
\begin{lemma}[Dubovickii-Miljutin] % Lemma 1.37
\label{thm:c1e37}
Let $K_0, K_1, \ldots, K_{n+1}$ be convex cones in a Banach space $U$, \ac{st} \\
$K_0, \ldots, K_n$ are open and $K_0\not=\emptyset$. Then the following statements are equivalent:
\begin{enumerate}[(i)]
\item\label[equation]{thm:c1e37-i} It holds $\bigcap_{i=0}^{n+1} K_i = \emptyset$.
\item \label[equation]{thm:c1e37-ii} There exists $\ell_i \in K_i^+$, $i = 0, \ldots, n+1$ which are not simultaneously zero, \ac{st}
\[
	\ell_0 + \ell_1 + \ldots + \ell_{n+1} = 0.
\]
\end{enumerate}
\end{lemma}
\begin{proof}
\labelcref{thm:c1e37-ii} $\Rightarrow$ \labelcref{thm:c1e37-i}: Assume the opposite: There exists
\[
	u \in \bigcap_{i=0}^{n+1} K_i.
\]
By \labelcref{thm:c1e37-ii} we cannot have
\[
	\ell_0 = \ell_1 = \ldots = \ell_n = 0.
\]
\ac{wlog} assume $\ell_0(v) \neq 0$ for some $v \in U$. Since $K_0$ is open, it follows
\[
	\ell_0(u) + \lambda \ell_0(v) = \ell_0\underbrace{(u + \lambda v)}_{\in K} \geq 0
\]
for $|\lambda|$ small enough. This implies $\ell_0(u) > 0$.
Since all $\ell_i \in K_i^+$, we get the contradiction
\[
	0 = \ell_0(u) + \underbrace{\ell_1(u)}_{\geq 0} + \ldots + \underbrace{\ell_{n+1}(u)}_{\geq 0} \geq \ell_0(u) > 0.
\]

\labelcref{thm:c1e37-i} $\Rightarrow$ \labelcref{thm:c1e37-ii}: Since $K_0 \neq 0$, but \labelcref{thm:c1e37-i} holds, there must be $m \leq n$, \ac{st}
\[
	K \coloneqq \bigcap_{i = 0}^{m} K_i \neq \emptyset, \quad K \cap K_{m + 1} = \emptyset.
\]
By assumption, $K$ is open, and $K$ and $K_{m + 1}$ are convex.
We thus can apply the Separation Theorem \labelcref{thm:c1e7}: There exists $\ell \in U^*$, $\ell \neq 0$, \ac{st}
\[
	\sup_{u \in K_{m + 1}} \ell(u) \leq \inf_{v \in K}\ell(v)
\]
By \cref{prop:c1e36}, \labelcref{prop:c1e36-ii}, thus implies $\ell \in K^+$ and $-\ell \in K_{m +1}^+$. We choose $\ell_{m + 1} = -\ell$, $\ell_{m+2} = \ldots = \ell_{n+1} = 0$. Then,
\[
	\ell + \ell_{m + 1} + \ell_{m + 2} + \ldots + \ell_{n + 1} = 0.
\]
We now claim that
\begin{equation}
\label{eq:thm-c1e37-star}
\tag{$\star$} K^+ = \left( \bigcap_{i=0}^m K_i \right)^+ = K_0^+ + K_1^+ + \ldots + K_m^+,
\end{equation}
which then proves \labelcref{prop:c1e36-ii}, since then $\ell = \ell_0 + \ldots + \ell_m$ for some $\ell_i \in K_i^+$, $i = 0, \ldots, m$.
Note that ``$\supseteq$'' in \cref{eq:thm-c1e37-star} is trivial by definition of dual cones.
To prove ``$\subseteq$'', let
\begin{IEEEeqnarray*}{rCl}
	X &=& \bigtimes_{i=0}^m U = U^{m + 1}, \\
	L &=& \{ (u, u, \ldots, u) \, : \, u \in U \} \subseteq X, \\
	C &=& K_0 \times K_1 \times \ldots \times K_m,
\end{IEEEeqnarray*}
then $C$ is an open convex cone in the Banach space in $X$, and $L \cap C \neq \emptyset$ because $K\neq \emptyset$.
Every $F \in X^*$ has the form
\[
	F(x) = \sum_{i=0}^m \ell_i(u_i),
\]
where $x = (u_0, \ldots, u_{m})$ and $\ell_i \in U^*$.

Now pick $\ell \in K^+$. Then we can define $F$ on $L$ as
\[
	F(x) = \ell(u) \quad \text{for } x = (u, u, \ldots, u) \in L.
\]
Since $\ell \in \left( \bigcap_{i=0}^m K_i \right)^+$, it holds that
\[
	F(x) \geq 0 \quad \forall x \in L \cap C.
\]
By \cref{prop:c1e34}, we can extend $F$ to $X$ \ac{st}
\begin{IEEEeqnarray*}{l"rCl"l}
	& F(x) &\geq& 0 & \forall x \in C, \\
	\Leftrightarrow & \sum_{i=0}^m \ell_i(u_i) &\geq& 0 & \forall (u_0, \ldots, u_m) \in K_0 \times \ldots \times K_m \text{ for some } \ell_i \in U^*. 
\end{IEEEeqnarray*}
Since the $K_i$ are cones, it follows
\[
	\ell_i(u_i) \geq 0 \quad \forall u_i \in K_i
\]
that is, $\ell_i \in K_i^+$. By construction
\[
	\ell(u) = \sum_{i=0}^m \ell_i(u) \quad \forall u \in U,
\]
which proves ``$\subseteq$'' in \cref{eq:thm-c1e37-star}.
\end{proof}
\end{document}