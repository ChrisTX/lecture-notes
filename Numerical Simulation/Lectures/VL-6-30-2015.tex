\documentclass[../skript.tex]{subfiles}

\begin{document}
\begin{proof}[\cref{thm:c4e3}]
We would life to apply \cref{thm:c4e4} with $V = H^1(\Omega)$. We claim that the variational formulation is an operator equation
\[
	(Ty)[v] \coloneqq (Ay + Dy + Ey)[v] = F[v] = \langle f, v \rangle_{L^2(\Omega)} + \langle g, v \rangle_{L^2(\Sigma)} \quad \forall v \in H^1(\Omega),
\]
where
\begin{IEEEeqnarray*}{rCl}
(Ay)[v] &=& a(y, v) \\
(Dy)[v] &=& \int_\Omega d(x, y(x)) v(x) \dx \\
(Ey)[v] &=& \int_\Gamma e(x, y(x)) v(x) \ds 
\end{IEEEeqnarray*}
Indeed, \cref{as:c4e2} implies $y \mapsto d(x, y(x)) \in L^2(\Omega)$ and $x \mapsto e(x, y(x)) \in L^2(\Sigma)$ and therefore $Dy \in V^*$, $Ey \in V^*$.
For $Ay$ we know this already and $F \in V^*$ is clear.
The operators $A, D, E : V \to V^*$ are already monotone. Indeed, $A$ is strongly monotone (linear and coercive):
\[
	\langle Ay, y \rangle_{V^*, V} = a(y, y) \geq \beta_0 \| y \|_{V}^2.
\]
By assumption, $\eta \mapsto d(x, \eta)$ is monotone for \ac{ae} $x \in \Omega$. This implies
\[
	\left( d(x, \eta_1) - d(x, \eta_2) \right) \left( \eta_1 - \eta_2 \right) \geq 0 \quad \forall \eta_1, \eta_2 \in \R
\]
and hence
\[
	\langle D y_1 - D y_2, y_1 - y_2 \rangle_{V^*, V} = \int_\Omega \left( d(x, y_1(x)) - d(x, y_2(x)) \right) \left(y_1(x) - y_2(x) \right) \dx \geq 0.
\]
In particular, since $d(x, 0) = 0$, it holds $\langle Dy, y \rangle_{V^*, V} \geq 0$.
The arguments for $E$ are the same. As a result, the operator $T : V \to V^*$ is strongly monotone with coercivity constant of $A$:
\[
	\langle T y_1 - T y_2, y_1 - y_2 \rangle_{V^*, V} \geq \beta_0 \| y_1 - y_2 \|_V^2.
\]

We show hemi-continuity. $A$ is linear and continuous, therefore hemi-continuous. For $D$, fix $y, v, w \in V$ and consider
\[
	\varphi(t) = \langle D(y + tv), w \rangle_{V^*, V} = \int_\Omega d(x, y(x) + t v(x)) w(x) \dx.
\]
If $t_n \to t_*$, then for \ac{ae} $x \in \Omega$,
\[
	d(x, y(x) + tv(x)) \to d(x, y(x) + t_* v(x)).
\]
Furthermore,
\[
	|d(x, y(t) + tv(x)) w(x) | \leq M |w(x)|
\]
and $w \in L^1(\Omega)$. By Lebesgue's dominated convergence theorem, it follows $\varphi(t_n) \to \varphi(t_*)$ that is, $D$ is hemi-continuous. The arguments for $E$ are the same.

Applying \cref{thm:c4e4}, we find that for every $F \in V^*$, in particular for every $(f, g) \in L^2(\Omega) \times L^2(\Gamma)$ there exists a unique weak solution $y \in V$ and
\[
	\| y_1 - y_2 \|_V \leq \frac{1}{\beta_0} \| T y_1 - T y_2 \|_{V^*} \quad \forall y_1, y_2 \in V.
\]

The stability estimate follows from $T(0) = 0$ ($D(0) = 0, E(0) = 0$!):
\[
	\| y \|_V \leq \frac{1}{\beta_0} \| Ty \|_{V^*} = \frac{1}{\beta_0} \| F \|_{V^*} \overset{\text{Exercise}}{\leq} \frac{1}{\beta_0} \left( \| f \|_{L^2(\Omega)} + \| g \|_{L^2(\Sigma)} \right).
\]
\end{proof}
\begin{remark} % Rem 4.5
\label{rem:c4e5}
The result is not applicable to the nonlinearity $d(x, y) = y^3$, since it does not fulfill \cref{as:c4e2}, \labelcref{as:c4e2-iv}. The proof reveals, that we can replace \labelcref{as:c4e2-iv} by
\begin{enumerate}[(i')]
\setcounter{enumi}{3}
\item $y \mapsto d(\cdot, y(\cdot))$ is continuous from $H^1(\Omega) \to L^2(\Omega)$ and $y \mapsto e(\cdot, y(\cdot))$ is continuous from $H^1(\Omega) \to L^2(\Gamma)$.
\end{enumerate}
Then $D : H^1(\Omega) \to \left( H^1(\Omega) \right)^*$ is continuous, in particular, hemi-continuous.
From the Sobolev embedding \cref{thm:c2e5}, in the case $N = 3$, we get $H^1 \hookrightarrow L^6(\Omega)$, which proves (iv') for $y \mapsto y^3$.
\end{remark}
\begin{remark} % Rem 4.6
\label{rem:c4e6}
We see that $F \in V^*$ is enough in \cref{thm:c4e3}. Based on Hölder's inequality and the Sobolev embedding \cref{thm:c2e5}, we can require less integrability from the right hand side: \cref{thm:c4e3} remains true if $f \in L^r(\Omega)$, $g \in L^s(\Gamma)$ if $r > \frac{N}{2}$, $s > N -1$. In this case
\[
	\| y \|_{H^1(\Omega)} \leq c \| F \|_{V^*} \leq c \left( \| f \|_{L^r(\Omega)} + \| g \|_{L^s(\Gamma)} \right).
\]
(compare discussion on $L^p$ data in \cref{sec:c2e2}).
\end{remark}
\subsection{Continuity of solutions}
\begin{theorem} % Thm 4.7
\label{thm:c4e7}
Let $r > \frac{N}{2}$, $s > N - 1$ and $f \in L^r(\Omega)$, $g \in L^s(\Gamma)$. The unique solution of the model problem \cref{eq:c4e1-star} (which exists by \cref{rem:c4e6}) belongs to $L^\infty(\Omega)$, and there exists $c_\infty > 0$ such that
\[
	\| y \|_{L^\infty(\Omega)} \leq c_\infty \left( \| f \|_{L^r(\Omega)} + \| g \|_{L^s(\Gamma)} \right).
\]
\end{theorem}
\begin{lemma}[Casas \lbrack{}1993\rbrack{}] % Lemma 4.8
\label{thm:c4e8}
Let $\Omega \subseteq \R^N$ be a bounded Lipschitz domain, $r > \frac{N}{2}$, $s > N - 1$ and $f \in L^r(\Omega)$, $g \in L^s(\Gamma)$.
Then the unique weak solution $y \in H^1(\Omega)$ of the Neumann problem
\begin{IEEEeqnarray*}{rCl"l}
\mathcal{A} y + y &=& f & \text{on } \Omega \\
\partial_{\nu_\mathcal{A}} y &=& g & \text{on } \Gamma
\end{IEEEeqnarray*}
(which exists by \cref{thm:c4e3}) is continuous on $\bar{\Omega}$, and there exists $c(r, s) > 0$ such that
\[
	\| y \|_{C(\bar{\Omega})} \leq c(r, s) \left( \| f \|_{L^r(\Omega)} + \| g \|_{L^s(\Gamma)} \right).
\]
\end{lemma}
The proofs for \cref{thm:c4e7,thm:c4e8} can be found in \cite{Troeltzsch}.
\begin{theorem}[Dropping the annoying \cref{as:c4e2}, \labelcref{as:c4e2-iii}, \labelcref{as:c4e2-iv}] % Thm 4.9
\label{thm:c4e9}
Suppose \cref{as:c4e1} holds but with $f \in L^r(\Omega)$, $g \in L^s(\Gamma)$, where $r > \frac{N}{2}$, $s > N - 1$. Suppose further that \cref{as:c4e2}, \labelcref{as:c4e2-i}, \labelcref{as:c4e2-ii} hold. Then the model problem \cref{eq:c4e1-star} admits a unique weak solution $y \in H^1(\Omega) \cap L^\infty(\Omega)$. This solution is continuous on $\bar{\Omega}$, and there exists a constant $c > 0$ (independent of $d, e, f, g$) such that
\[
	\| y \|_{H^1(\Omega)} + \| y \|_{C(\bar{\Omega})} \leq c \left( \| f - d(\cdot, 0) \|_{L^r(\Omega)} + \| g + e(\cdot, 0) \|_{L^s(\Gamma)} \right).
\]
\end{theorem}
\begin{proof}
\begin{enumerate}
\item\label{thm:c4e9-proof-case1} We first consider that \cref{as:c4e2}, \labelcref{as:c4e2-iii} still holds, i.e.\ $d(\cdot, 0) = 0$, $e(\cdot, 0) = 0$.
We cut off $d$ by defining
\[
	d_k(x, y) = \begin{cases}
	d(x, k) & \text{if } y > k \\
	d(x, y) & \text{if } |y| \leq k \\
	d(x, -k) & \text{if } y < -k 
	\end{cases}.
\]
By \cref{as:c4e2}, \labelcref{as:c4e2-i}, \labelcref{as:c4e2-ii} it holds that
\[
	|d_k(x, y)| \leq \max \left( |d(x, k)|, |d(x, -k)| \right) \leq M_k
\]
for some $M_k > 0$. Analogously, we define $e_k$.
By \cref{thm:c4e7}, the problem
\begin{IEEEeqnarray*}{rCl}
\mathcal{A} y + c_0(x) y + d_k(x, y) &=& f \\
\partial_{\nu_\mathcal{A}} y + \alpha(x) y + e_k(x, y) &=& g
\end{IEEEeqnarray*}
has exactly one solution $y_k \in H^1(\Omega)$ that satisfies
\[
	\| y_k \|_{L^\infty(\Gamma)} \overset{\text{Exercise}}{\leq} \| y_k \|_{L^\infty(\Omega)} \leq c_\infty \left( \| f \|_{L^r(\Omega)} + \| g \|_{L^s(\Gamma)} \right),
\]
where $c_\infty$ is independent of $k$!
Choosing
\[
	k > c_\infty \left( \| f \|_{L^r(\Omega)} + \| g \|_{L^s(\Gamma)} \right),
\]
we get $|y_k(x)| \leq k$ \ac{ae} and therefore
\[
	d_k(x, y_k(x)) = d(x, y_k(x)).
\]
hence, $y = y_k$ is a weak solution of the model problem \cref{eq:c4e1-star}, and by applying \cref{thm:c4e3} with $d_k$ and $e_k$, we get
\[
	\| y \|_{H^1(\Omega)} \leq c_1 \left( \| f \|_{L^r(\Omega)} + \| g \|_{L^s(\Gamma)} \right).
\]
We can rewrite the model problem \cref{eq:c4e1-star} for $y$ as
\begin{IEEEeqnarray*}{rCl"l}
\mathcal{A} y + y &=& f + y - c_0(x) y + d_k(x, y) & \text{in } \Omega \\
\partial_{\nu_\mathcal{A}} y &=& g - \alpha(x) y - e_k(x, y) & \text{on } \Gamma
\end{IEEEeqnarray*}
Since $y \in L^\infty(\Omega)$, $y|_\Gamma \in L^\infty(\Gamma)$, the right hand sides are in $L^r(\Omega)$, $L^s(\Gamma)$, respectively.
Thus, \cref{thm:c4e8} applies, that is $y \in C(\bar{\Omega})$ and
\[
	\| y \|_{C(\bar{\Omega})} = \| y \|_{L^\infty(\Omega)} \leq c_1 \left( \| f \|_{L^r(\Omega)} + \| g \|_{L^s(\Gamma)} \right)
\]
for the same $c_1 > 0$.
To prove uniqueness, let $\tilde{y} \in H^1(\Omega) \cap L^\infty(\Omega)$ be another weak solution. By the same argument it then holds that $\tilde{y}$ is continuous on $\bar{\Omega}$. One verifies that both $y$ and $\tilde{y}$ solve the cutoff problem with
\[
	k \coloneqq \max \left\{ \| y \|_{C(\bar{\Omega})}, \| \tilde{y}(x) \|_{C(\bar{\Omega})} \right\},
\]
whose solution is unique $y = \tilde{y}$.
\item In the general case, we rewrite the model problem \cref{eq:c4e1-star} as
\begin{IEEEeqnarray*}{rCl}
\mathcal{A} y + c_0(x) y + \underbrace{ d(x, y) - d(x, 0) }_{\tilde{d}(x, y)} &=& \underbrace{ f(x) - d(x, 0) }_{\tilde{f}(x) \in L^r(\Omega)} \\
\partial_{\nu_{\mathcal{A}}} y + \alpha(x) y + \underbrace{ e(x, y) - e(x, 0) }_{\tilde{e}(x, y)} &=& \underbrace{ g(x) - e(x, 0) }_{\tilde{g}(x) \in L^s(\Gamma)}
\end{IEEEeqnarray*}
and apply \cref{thm:c4e9-proof-case1}.
\end{enumerate}
\end{proof}
\paragraph{Further generalization}
Currently we cannot treat e.g. 
\begin{IEEEeqnarray*}{rCl"l}
- \lapl y + y^3 &=& f & \text{on } \Omega \\
\partial_{\nu_\mathcal{A}} y &=& 0 \text{on } \Gamma
\end{IEEEeqnarray*}
in which both $c_0 = 0$ and $\alpha = 0$. \\
\underline{Generalized model problem:}
\begin{equation}
\opttag{$\star\star$}
\label{eq:c4e1-starstar}
\begin{IEEEeqnarraybox}[][c]{rCl"l}
\mathcal{A} y + d(x, y) &=& 0 & \text{on } \Omega \\
\partial_{\nu_\mathcal{A}} y + e(x, y) &=& 0 & \text{on } \Gamma
\end{IEEEeqnarraybox}
\end{equation}
\begin{assumption} % As 4.10
\label{as:c4e10}
\begin{itemize}
\item $\Omega \subseteq \R^N$ and $\mathcal{A}$ satisfy the same assumptions as in \cref{as:c4e1}.
\item $d$ and $e$ satisfy \cref{as:c4e2}, \labelcref{as:c4e2-i} and \labelcref{as:c4e2-ii}.
\item Furthermore, there exists for any $M > 0$ functions $\psi_M \in L^r(\Omega)$, $\phi_M \in L^s(\Gamma)$ where $r > \frac{N}{2}$, $s > N -1$, such that
\begin{IEEEeqnarray*}{rCl"s}
|d(x, y)| &\leq& \psi_M(x) & for almost every $x \in \Omega$, and all $|y| \leq M$ \\
|e(x, y)| &\leq& \psi_M(x) & for almost every $x \in \Gamma$, and all $|y| \leq M$
\end{IEEEeqnarray*}
and one of the following two conditions hold:
\begin{enumerate}[(i)]
\item\label[equation]{as:c4e10-i} $\exists E_d \subset \Omega$, $|E_d| > 0$, and $M_d > 0$, $\lambda_d > 0$ such that
\begin{IEEEeqnarray*}{rCl"l}
d(x, y_1) &<& d(x, y_2) & \forall x \in E_d, \; y_1 < y_2 \\
(d(x, y) - d(x, 0))y &\geq& \lambda_d |y|^2 & \forall x \in E_d, \; \forall |y| > M_d
\end{IEEEeqnarray*}
\item\label[equation]{as:c4e10-ii} $\exists E_e \subset \Gamma$, $|E_e| > 0$ (on $\Gamma$), $M_e > 0$, $\lambda_e > 0$
\begin{IEEEeqnarray*}{rCl"l}
e(x, y_1) &<& e(x, y_2) & \forall x \in E_e, \; y_1 < y_2 \\
(e(x, y) - e(x, 0))y &\geq& \lambda_e |y|^2 & \forall x \in E_e, \; \forall |y| > M_e
\end{IEEEeqnarray*}
\end{enumerate}
\end{itemize}
\end{assumption}
\begin{theorem} % Thm 4.11
\label{thm:c4e11}
Given \cref{as:c4e10}, the generalized model problem \cref{eq:c4e1-starstar} admits a unique solution in $H^1(\Omega) \cap L^\infty(\Omega)$. This solution is continuous on $\bar{\Omega}$.
\end{theorem}
\begin{proof}
\cite[Theorem 4.10]{Troeltzsch}.
\end{proof}
\begin{remark} % Rem 4.12
\label{rem:c4e12}
If $y \mapsto d(x, y)$ and\slash{}or $y \mapsto e(x, y)$ are differentiable for almost every $x \in \Omega$. Then the crucial assumptions in \cref{as:c4e10} can be replaced by
\[
\frac{\partial}{\partial y}d(x, y) \geq \lambda_d \quad \forall x \in E_d, \forall y \in \R
\]
and\slash{}or
\[
\frac{\partial}{\partial y}e(x, y) \geq \lambda_e \quad \forall x \in E_e, \forall y \in \R.
\]
\end{remark}
\begin{example} % Ex 4.13
\label{ex:c4e13}
\begin{IEEEeqnarray*}{rCl}
\mathcal{A} y + y + y^3 &=& \underbrace{ f }_{\in L^r(\Omega)} \\
\partial_{\nu_\mathcal{A}} y &=& 0
\end{IEEEeqnarray*}
This implies
\begin{IEEEeqnarray*}{rCl}
d(x, y) &=& y + y^3 - f(x) \\
e(x, y) &=& 0
\end{IEEEeqnarray*}
and this satisfies \cref{rem:c4e12}. Thus
\[
	\frac{\partial}{\partial y} d(x, y) = 1 + y^2 \geq \lambda_d = 1 \quad \forall x \in E_d \coloneqq \Omega.
\]

On the other hand,
\begin{IEEEeqnarray*}{rCl}
\mathcal{A} y + y^3 &=& \underbrace{ f }_{\in L^r(\Omega)} \\
\partial_{\nu_\mathcal{A}} y &=& 0
\end{IEEEeqnarray*}
This implies
\begin{IEEEeqnarray*}{rCl}
d(x, y) &=& y^3 - f(x) \\
e(x, y) &=& 0
\end{IEEEeqnarray*}
This does not satisfy \cref{rem:c4e12}, but it still satisfies \cref{as:c4e10}, \labelcref{as:c4e10-i}: Either $d(x, y_1) < d(x, y_2)$:
\[
	\left( d(x, y) - d(x, 0) \right) = y^4 \geq y^2 \quad \forall x \in \Omega, |y| \geq 1
\]
and therefore $\lambda_d = 1$, $M_d = 1$, $E_d = \Omega$.
\end{example}
\end{document}