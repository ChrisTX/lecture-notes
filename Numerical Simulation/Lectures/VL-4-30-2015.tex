\documentclass[../skript.tex]{subfiles}

\begin{document}
\section{Saddle points and generalized KKT conditions} % 1.5
\label{sec:c1e5}
\begin{definition} % Def 1.41
\label{def:c1e41}
Let $K$ be a convex cone in a vector space $Z$ such that $0 \in K$. We write $z \geq_K \zeta$ \ac{iff} $z - \zeta \in K$.
Analogously, $z \leq_K \zeta$ if $\zeta - z \in K$. The cone $K$ defining this relation is called the \emph{nonnegative cone in $Z$}.
\end{definition}
\begin{example}
$Z = \R^3$, $K = \{ z \in \R^3 \; : \; z_1 = 0, z_2 \leq 0, z_3 \geq 0\}$.
\end{example}
\begin{definition} % Def 1.42
\label{def:c1e42}
Let $U$, $Z$ be linear spaces, and $K$ a nonnegative cone in $Z$. A function $G : U \to Z$ is called \emph{convex} (\ac{wrt} $\leq_K$) if
\[
	G(tu + (1-t)v) \leq_K tG(u) + (1-t) G(v) \quad \forall t \in (0, 1)
\]
\end{definition}
\begin{problem}[General convex program]
\label{prob:c1e5-star}
\begin{IEEEeqnarray*}{u"l}
minimize & f(u) \\
subject to & G(u) \leq_K 0, \; u \in C
\end{IEEEeqnarray*}
\end{problem}
\paragraph{Assumptions:}
\begin{itemize}
\item $C \subseteq U$ convex subset of the linear space $U$
\item $f : C \to \R$ convex
\item $G : U \to Z$ convex \ac{wrt} $\leq_K$
\end{itemize}
\underline{Observation:} The set
\[
\{ u \; : \; G(u) \leq_K 0, u \in C \}
\]
is convex.

\textbf{Lagrange multipliers:} \\
\underline{Idea:} In the setting of \cref{prob:c1e5-star}, let
\[
	\Gamma = \{ z \in Z \; : \; \exists u \in C \text{ with } G(u) \leq_K z \}.
\]
This set $\Gamma$ is convex. Define the \emph{primal function}
\begin{IEEEeqnarray*}{rCl}
\omega : \Gamma &\to& \R \cup \{ - \infty \}, \\
\omega(z) &=& \inf \{  f(u) \; : \; u \in C, G(u) \leq_K z \}.
\end{IEEEeqnarray*}
We are interested in $\omega(0)$.
\begin{proposition} % Prop 1.43
\label{prop:c1e43}
The function $\omega$ is convex and decreasing, that is $z_1, z_2 \in \Gamma$ with $z_1 \geq_K z_2$, then $\omega(z_1) \leq \omega(z_2)$.
\end{proposition}
\begin{proof}
\begin{IEEEeqnarray*}{rCl}
\omega(tz_1 + (1-t) z_2) &=& \inf \{ f (u) \; : \; u \in C, G(u) \leq_K tz_1 + (1-t) z_2\} \\
&\leq& \inf \{ f(u) \; : \; u = tu_1 + (1-t) u_2;\; u_1, u_2 \in C, \\
&& G(u_1) \leq_K z_1, G(u_2) \leq_K z_2 \} \\
&\leq& \inf \{ f(u_1) \; : \; u_1 \in C, G(u_1) \leq_K z_1 \} \\
&& {} + (1-t) \inf \{ f(u_2) \; : \; u_2 \in C, G(u_2) \leq_K z_2 \} \\
&=& t\omega(z_1) + (1-t) \omega(z_2).
\end{IEEEeqnarray*}
If $z_2 \leq_K z_1$, then $G(u) \leq_K z_2$ implies
\[
z_1 - G(u) = \underbrace{ z_1 - z_2 }_{\in K} + \underbrace{ z_2 - G(u) }_{\in K} \in K \quad \Leftrightarrow \quad G(u) \leq_K z_1.
\]
Therefore, the infimum in $\omega(z_2)$ is taken over a subset compared to $\omega(z_1)$.
\end{proof}
\underline{Idea:} There exist $z^* \in Z^*$ such that $z \mapsto \omega(z) + z^*(z)$ is minimized in 0.

In the following, $Z$ is a normed space. \\
\emph{Slater condition:} There exists $u_1 \in C$ \ac{st} 
\[
	G(u_1) <_K 0 \; \Leftrightarrow \; -G(u_1) \in \Int K.
\]
\addtocounter{dummythm}{1} % Another numeration error
\begin{theorem}[Generalized KKT] % Thm 1.45
\label{thm:c1e45}
Let $U$ be a linear space, $Z$ a normed space, $C \subseteq U$ convex, and $K \subseteq Z$ a nonnegative cone.
Let $f : C \to \R$ and $G : U \to Z$ be convex, and assume the Slater condition holds:
Let
\[
	\mu_0 = \inf f(u) \quad \text{\ac{st}} \; u \in C, G(u) \leq_K 0
\]
and assume $\mu_0 > -\infty$.
Then there exists $z_0^* \in K^+$ such that
\[
	\mu_0 = \inf_{u \in C} \left[ f(u) + \langle z_0^*, G(u) \rangle \right].
\]
If $u_0 \in C$, $G(u_0) \leq_K 0$ solves the convex program \cref{prob:c1e5-star} (i.e.\ $f(u_0) = \mu_0$), then the \emph{complementary slack condition}
\[
	\langle z_0^*, G(u_0) \rangle = 0
\]
holds.
\end{theorem}
\begin{proof}
In $W = Z \times R$, we define
\begin{IEEEeqnarray*}{rCl}
A &=& \{ (z, r) \; : \; r \geq f(u), G(u) \leq_K z \text{ for some } u \in C \}, \\
B &=& \{ (z, r) \; : \; r \leq \mu_0, z \leq_K 0 \}.
\end{IEEEeqnarray*}
Both $A$ and $B$ are easily shown to be convex.
The definition of $\mu_0$ implies that $\Int A \cap B = \emptyset$.
Also, since $-K$ contains an interior point, $B$ contains an interior point.
By the separation \cref{thm:c1e7}, there exists
\[
	\omega_0^* = (z_0^*, r_0) \in W^*, \; \omega_0^* \neq 0,
\]
such that
\[
	r_0 r_1 + \langle z_0^*, z_1 \rangle \geq r_0 r_2 + \langle z_0^*, z_2 \rangle \quad \forall (z_1, r_1) \in \Int A, (z_2, r_2) \in B.  
\]
By continuity of $\omega_0^*$ this in fact holds for all $(z_1, r_1) \in A$, $(z_2, r_2) \in B$.
By the nature of $B$, this can only hold, if
\[
	r_0 \geq 0 \; \text{ and } \; z_0^*(z_2) \leq 0 \quad \forall z_2 \leq_K 0 \quad \Leftrightarrow \quad - z_2 \in K \quad \Leftrightarrow \quad z_0^* \in K^+.
\]
We show $r_0 > 0$. The point $(0, \mu_0)$ is in $B$, hence:
\begin{equation}
\tag{+}
\label{eq:thm-c1e45-plus}
r_0 r + \langle z_0^*, z \rangle \geq r_0 \mu_0 \quad \forall (r, z) \in A.
\end{equation}
If $r_0 = 0$, then we would have $\langle z_0^*, G(u_1) \rangle \geq 0$, but $z_0^* \neq 0$.
As $G(u_1)$ is an interior point of $-K$, this implies $\langle z_0^*, z \rangle > 0$ for some $z \in - K$, which contradicts $z_0^* \in K^+$ as already shown. Hence $r_0 > 0$, we can assume $r_0 = 1$.

Now we use that $(0, \mu_0) \in \partial A$. Therefore, using $r_0 = 1$ in \cref{eq:thm-c1e45-plus}:
\begin{IEEEeqnarray*}{rCl}
	\mu_0 &\overset{\text{\cref{eq:thm-c1e45-plus}}}{=}& \inf_{(z, r) \in A} \left[r + \langle z_0^*, z \rangle \right] \\
	&\overset{\text{restricting to } (f(u),G(u)) \cap A}{\leq}& \inf_{u \in C} \left[ f(u) + \langle z_0^*, G(u) \rangle \right] \\
	&\leq& \inf_{\substack{u \in C \\ G(u) \leq_K 0}} [ f(u) + \underbrace{ \langle z_0^*, G(u) \rangle }_{\leq 0}] \\
	&\leq& \inf_{\substack{u \in C \\ G(u) \leq_K 0}} f(u) = \mu_0.
\end{IEEEeqnarray*}

\underline{Slack condition:} Given $u_0 \in C$, $G(u_0) \leq_K 0$ \ac{st} $f(u_0) = \mu_0$, then
\begin{IEEEeqnarray*}{rCl}
\mu_0 &\leq& f(u_0) + \underbrace{ \langle z_0^*, G(u_0) \rangle }_{\leq 0} \\
&\leq& f(u_0) = \mu_0.
\end{IEEEeqnarray*}
Thus,
\[
	\langle z_0^*, G(u_0) \rangle = 0.
\]
\end{proof}
\paragraph{Saddle point formulation}
\begin{definition} % Def 1.46
\label{def:c1e46}
The function $L : C \times Z^* \to \R$,
\[
	L(u, z^*) = f(u) + \langle z^*, G(u) \rangle
\]
is called the \emph{Lagrange function} of the convex program \cref{prob:c1e5-star}.

A point $(u_0, z_0^*) \in C \times K^+$ is called a \emph{saddle point} of the Lagrange function $L$, if it satisfies
\[
	L(u_0, z^*) \leq L(u_0, z_0^*) \leq L(u, z_0^*) \quad \forall u \in C, z^* \in K^+.
\]
In this case, $z_0^*$ is called the \emph{Lagrange multiplier} for $u_0$.
\end{definition}
\begin{theorem}[Saddle point formulation] % Thm 1.47
\label{thm:c1e47}
Under the same assumptions as in \cref{thm:c1e45}, if a point $u_0 \in C$, $G(u_0) \leq_K 0$ solves the convex program \cref{prob:c1e5-star}, then there exists a Lagrange multiplier $z_0^* \in K^+$ such that $(u_0, z_0^*)$ is a saddle point of $L$ satisfying the complementary slack condition $\langle z_0^*, G(u_0) \rangle = 0$.
\end{theorem}
\begin{proof}
From \cref{thm:c1e45} we get $z_0^* \in K^+$ with $\langle z_0^*, G(u_0) \rangle = 0$ and
\[
	L(u_0, z_0^*) \leq L(u, z_0^*) \quad \forall u \in C.
\]
Since $G(u_0) \leq_K 0$, we have 
\[
	z^*(G(u_0)) \leq 0 \quad \forall z^* \in K^+.
\]
Therefore,
\[
	L(u_0, z^*) = f(u_0) + \langle z^*, G(u_0) \rangle \leq f(u_0) = L(u_0, z_0^*) \quad \forall z^* \in K^+.
\]
\end{proof}
\begin{theorem} % Thm 1.48
\label{thm:c1e48}
Let the cone $K$ be closed. If there exists $u_0 \in C$ and a Lagrange multiplier $z_0 \in K^+$ such that $(u_0, z_0^*)$ is a saddle point of the Lagrange function,
\[
	L(u_0, z^*) \leq L(u_0, z_0^*) \leq L(u, z_0^*) \quad \forall u \in C, z^* \in K^+.
\]
Then $u_0$ solves the convex program \cref{prob:c1e5-star} and the complementary slack condition
\[
	\langle z^*_0, G(u_0) \rangle = 0
\]
holds.
\end{theorem}
\begin{proof}
We get from the first inequality
\begin{equation}
\tag{++}
\label{eq:thm-c1e48-plusplus}
\langle z^*, G(u_0) \rangle \leq \langle z_0^*, G(u_0) \rangle \quad \forall z^* \in K^+.
\end{equation}
In particular,
\[
	\forall z_1^* \in K^+ \; : \; \langle z_0^* + z_1^*, G(u_0) \rangle \leq \langle z_0^*, G(u_0) \rangle.
\]
Thus,
\[
	\langle z_1^*, G(u_0) \rangle \leq 0 \quad \forall z_1 \in K^+.
\]
This implies $-G(u_0) \in K^{++} = K$ (Exercise) ($K^{++} = \{ z \in Z : z^*(z) \geq 0 \; \forall z \in K\}$),
that is $G(u_0) \leq_K 0$. Now \cref{eq:thm-c1e48-plusplus} implies
\[
	\langle z_0^*, G(u_0) \rangle = 0.
\]
For $z_1 \in C$, $G(z_1) \leq_K 0$ it follows
\[
	f(u_0) = L(u_0, z_0^*) \leq L(u_1, z_0^*) = f(u_1) + \underbrace{ \langle z_0^*, G(u_1) \rangle }_{\leq 0} \leq f(u_1).
\]
\end{proof}
\end{document}