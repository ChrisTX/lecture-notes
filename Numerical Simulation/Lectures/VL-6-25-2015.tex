\documentclass[../skript.tex]{subfiles}

\begin{document}
\paragraph{Gradient projection method}
\begin{IEEEeqnarray*}{u"l}
minimize & f(u) \coloneqq \frac{1}{2} \| Su - y_\Omega \|_{L^2(\Omega)}^2 + \frac{\lambda}{2} \| u \|_{L^2(\Sigma)}^2 \\
subject to & u \in U_{ad}
\end{IEEEeqnarray*}
\begin{IEEEeqnarray*}{c}
Su
\left\{
\begin{IEEEeqnarraybox}[][c]{rCl}
y_t - \lapl y &=& 0 \\
\partial_\nu y + \alpha y &=& \beta u \\
y(0) &=& y_0
\end{IEEEeqnarraybox}
\right. \quad\quad
\begin{IEEEeqnarraybox}[][c]{rCl}
- p_t - \lapl p &=& 0 \\
\partial_\nu p + \alpha p &=& 0 \\
p(T) &=& y(T) - y_\Omega
\end{IEEEeqnarraybox} \\
u = \PP_{U_{ad}} \left\{ - \frac{1}{\lambda} \beta \cdot p \right\}
\end{IEEEeqnarray*}
For this formulation, we have assume that $\lambda > 0$.
Our previous calculations \cref{rem:c3e20,thm:c3e23} reveal that
\begin{IEEEeqnarray*}{rCl"l}
	\langle \nabla f(u), v \rangle_{L^2(\Sigma)} &=& \langle Su - \hat{y}_\Omega, Sv \rangle_{L^2(\Omega)} + \lambda \langle u, v \rangle_{L^2(\Sigma)} \\
	&=& \langle \beta \cdot p + \lambda u, v \rangle_{L^2(\Sigma)} & \forall u \in U = L^2(\Sigma).
\end{IEEEeqnarray*}
Thus,
\[
	\nabla f(u) = \beta \cdot p + \lambda u.
\]
Similar to \cref{thm:c2e27} it holds that $u_0$ is an optimal control if and only if
\[
	u_0 = \PP_{U_{ad}} \left( u_0 - \sigma \nabla f(u_0) \right).
\]
Here $\sigma > 0$ is arbitrary. For $\sigma = \frac{1}{\lambda}$, we recover
\[
	u_0 = \PP_{U_{ad}} \left\{ -\frac{1}{\lambda} \beta \cdot p \right\}.
\]
The gradient projection method proceeds as usual (for the model problem):
Given a control $u_k \in L^2(\Sigma)$:
\begin{enumerate}
\item\label[equation]{eq:c3e6-PGM-1} Calculate the state $y_k$ for the control $u_k$.
\item Calculate the adjoint state $p_k$ and obtain the anti-gradient
\[
	v_k = - \nabla f(u_k) = - \left( \beta p_k + \lambda u_k \right).
\]
\item Find step-size $\sigma_k$ (e.g.\ by backtracking) such that
\[
	f\left(\PP\left\{u_k + \sigma_k v_k\right\}\right) \approx \min_{\sigma > 0} f\left( \PP_{U_{ad}} \left\{ u_k + \sigma v_k \right\}\right).
\]
\item Update $u_{k+1} = u_k + \sigma_k v_k$, $k \leftarrow k + 1$, back to \labelcref{eq:c3e6-PGM-1}.
\end{enumerate}
Alternatives (see \cite[Section 3.7.2]{Troeltzsch}):
\begin{itemize}
\item primal-dual active set method
\item direct solution of optimality system
\item full discretization of non-reduced problem
\end{itemize}
\chapter{Semilinear elliptic problems} % Ch 4
\label{sec:c4}
\begin{example}[an optimal control problem with a semilinear state equation]
\begin{IEEEeqnarray*}{u"l}
minimize & J(y, u) \coloneqq \frac{1}{2} \| y - y_\Omega \|_{L^2(\Omega)}^2 + \frac{\lambda}{2} \| u \|_{L^2(\Omega)}^2 \\
subject to & \left\{
\begin{IEEEeqnarraybox}[][c]{rCl"l}
- \lapl y + y + y^3 &=& u & \text{in $\Omega$} \\
\partial_\nu y &=& 0 & \text{on $\Gamma = \partial \Omega$}
\end{IEEEeqnarraybox}
\right. \\
and & u_a \leq u(x) \leq u_b \quad \text{\ac{ae} on $\Omega$}
\end{IEEEeqnarray*}
The state equation is called \emph{semilinear}, since the derivatives of highest order (here two) enter linearly.
\end{example}
\underline{Outline}
\begin{itemize}
\item the existence of a continuous control-to-state operator $y = Su$ is derived in the framework of monotone operators.
\item continuity of $y$ can be derived for controls $u$ having higher integrability than 2.
\item the reduced formulation might be nonconvex (since $S$ will be nonlinear)
\begin{itemize}
\item not clear whether optimal controls exist
\item necessary conditions are not sufficient
\item no globally convergent optimization methods
\end{itemize}
\end{itemize}
\pagebreak
\section{Solvability of a model class of semilinear elliptic PDEs} % Sec 4.1
\label{sec:c4e1}
\begin{problemnonumb}[Model problem]
\begin{equation}
\opttag{$\star$}
\label{eq:c4e1-star}
\left\{ \begin{IEEEeqnarraybox}[][c]{rCl"l}
\mathcal{A} y + c_0(x) + d(x, y) &=& f & \text{on $\Omega$} \\
\partial_{\nu_{\mathcal{A}}} y + \alpha(x) y + e(x, y) &=& g & \text{on $\Gamma$}
\end{IEEEeqnarraybox} \right.
\end{equation}
\end{problemnonumb}
\begin{assumption} % Ass 4.1
\label{as:c4e1}
\begin{itemize}
\item $\Omega \subseteq \R^N$ bounded Lipschitz domain
\item $(\mathcal{A}y)(x) = \dive(A(x) \cdot \nabla y(x))$, where $A \in (L^\infty(\Omega))^{N \times N}$ is symmetric and uniformly elliptic
\item $\nu_{\mathcal{A}} = \mathcal{A} \nu$ conormal vector
\item $c_0 \in L^\infty_+(\Omega)$, $\alpha \in L^\infty_+(\Gamma)$, such that $\| c \|_{L^\infty(\Omega)} + \| \alpha \|_{L^\infty(\Gamma)} > 0$
\item $f \in L^2(\Omega)$, $g \in L^2(\Gamma)$
\end{itemize}
\end{assumption}
\cref{as:c4e1} is the same as in \cref{sec:c2e2}.
\begin{assumption} % Ass 4.2
\label{as:c4e2}
$d : \Omega \times \R \to \R$, $e : \Gamma \times \R \to \R$ satisfy:
\begin{enumerate}[(i)]
\item\label[equation]{as:c4e2-i} $x \mapsto d(x, \eta) \in L^\infty(\Omega)$, $x \mapsto e(x, \eta) \in L^\infty(\Gamma) \quad \forall \eta \in \R$
\item\label[equation]{as:c4e2-ii} $\eta \mapsto d(x, \eta)$ and $\eta \to e(x, \eta)$ are \emph{continuous} and \emph{monotonically increasing} for almost every $x \in \Omega$ and $x \in \Gamma$, respectively.
\item\label[equation]{as:c4e2-iii} $d(x, 0) = 0$ and $e(x, 0) = 0$ for almost every $x \in \Omega$ and $x \in \Gamma$, respectively.
\item\label[equation]{as:c4e2-iv} There exists $M > 0$ such that $|d(x, \eta)| \leq M$ and $|e(x, \eta)| \leq M$ for almost every $x \in \Omega$ and $x \in \Gamma$, respectively, and all $\eta \in \R$.
\end{enumerate}
\end{assumption}
\paragraph{Weak solution in \texorpdfstring{$H^1(\Omega)$}{H1(Ω)}}
A function $y \in H^1(\Omega)$ is called \emph{weak solution} of \cref{eq:c4e1-star} if it satisfies the \emph{variational equation}:
\[
	a(y, v) + \int_\Omega d(x, y(x)) v(x) \dx + \int_\Gamma e(x, y(x)) v(x) \ds = \int_\Omega f \cdot v \dx + \int_\Gamma g \cdot v \ds \quad \forall v \in H^1(\Omega),
\]
where
\[
	a(y, v) = \int_\Omega \nabla y \cdot A \nabla v \dx + \int_\Omega c_0 y v \dx + \int_\Gamma \alpha y v \ds.
\]
We know that $a$ is bounded and coercive on $H^1(\Omega) \times H^1(\Omega)$. Therefore it induces a linear, bounded and continuously invertible operator
\begin{IEEEeqnarray*}{rCl}
	A : H^1(\Omega) &\to& \left( H^1(\Omega) \right)^* \\
	y &\mapsto& a(y, \cdot)
\end{IEEEeqnarray*}
\begin{theorem} % Thm 4.3
\label{thm:c4e3}
Under \cref{as:c4e1,as:c4e2} the model problem \cref{eq:c4e1-star} admits a unique weak solution $y \in H^1(\Omega)$.
There exists a constant $c > 0$ (independent of $d, e, f, g$) such that
\[
	\| y \|_{H^1(\Omega)} \leq c \left( \| f \|_{L^2(\Omega)} + \| g \|_{L^2(\Sigma)} \right)
\]
In fact, we can choose $c = \frac{1}{\beta_0}$, where
\[
	a(v, v) \geq \beta_0 \| v \|_{H^1(\Omega)}^2 \quad \forall v \in H^1(\Omega).
\]
\end{theorem}
The proof uses a standard existence result for \emph{monotone operators}.
\begin{definitionnonumb}
Let $V$ be a real Banach space. A (nonlinear) operator $T : V \to V^*$ is called
\begin{enumerate}[(i)]
\item \emph{monotone}, if $\langle T y_1 - T y_2, y_1 - y_2 \rangle_{V^*, V} \geq 0 \quad \forall y_1, y_2 \in V$.
\item \emph{strictly monotone}, if the inequality is always strict for $y_1 \neq y_2$.
\item \emph{coercive}, if
\[
	\frac{\langle Ty, y \rangle_{V^*, V}}{\| y \|_V} \to + \infty \quad \text{for $\| y \|_V \to \infty$}.
\]
\item \emph{hemi-continuous}, if $t \mapsto \langle T(y + tv), w \rangle_{V^*, V}$ is continuous on $[0, 1]$ for all $y, v, w \in V$.
\item \emph{strongly monotone}, if there exists $\beta_0 > 0$ such that
\[
	\langle Ty_1 - Ty_2, y_1 - y_2 \rangle_{V^*, V} \geq \beta_0 \| y_1 - y_2 \|_V^2 \quad \forall y_1, y_2 \in V.
\]
\end{enumerate}
\end{definitionnonumb}
\begin{theorem}[Main theorem for monotone operators] % Thm 4.4
\label{thm:c4e4}
Let $V$ be a real and separable Hilbert space, and let $T : V \to V^*$ be a monotone, coercive and hemi-continuous operator. Then for every $F \in V^*$ the equation
\[
	Ty = F
\]
has at least one solution $y \in V$. The set of solutions is bounded, convex and closed. If $T$ is strictly monotone, then a solution $y$ is unique. If $T$ is strongly monotone, then $T^{-1} : V^* \to V$ is Lipschitz-continuous: 
\[
	\frac{1}{\beta} \| T y_1 - T y_2 \|_{V^*} \geq \| y_1 - y_2 \|_V.
\]
\end{theorem}
\begin{proof}
See \cite{Zeidler2B}. % Zeidler, "Nonlinear Functional Analysis", Vol. II/B
A simple proof via Banach's fixed point theorem is possible if $T$ is Lipschitz-continuous (Exercise).
\end{proof}
\end{document}