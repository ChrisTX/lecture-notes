\documentclass[../skript.tex]{subfiles}

\begin{document}
The point in \cref{thm:c1e48} is that it is only assumed $u_0 \in C$. If we assume $G(u_0) \leq_K 0$ from the beginning then the second inequality in the saddle-point condition together with the slack condition is of course already sufficient.
\begin{theorem} % Thm 1.49
\label{thm:c1e49}
Let the cone $K$ in the convex program be closed. If there exists $u_0 \in C$ satisfying $G(u_0) \leq_K 0$, and a $z_0^* \in K^+$ such that
\[
	L(u_0, z_0^*) \leq L(u, z_0^*) \; \forall u \in C \quad \text{and} \quad \langle z_0^*, G(u_0) \rangle = 0.
\]
Then $u_0$ solves the convex program and $(u_0, z_0^*)$ is a saddle point.
\end{theorem}
\begin{remarknonumb}
These statements remain true if $f$ is nonconvex.
\end{remarknonumb}
\begin{proof}
Let $u_1 \in C$, $G(u_1 \leq_K 0)$ and $z^* \in K^+$, then
\begin{IEEEeqnarray*}{rCl}
	L(u_0, z^*) &=& f(u_0) + \langle z^*, G(u_0) \rangle \\
	&\leq& f(u_0) \\
	&=& L(u_0, z_0^*) \\
	&\leq& L(u_1, z_0^*) \\
	&=& f(u_1) + \underbrace{ \langle z_0^*, G(u_1) \rangle }_{\leq 0} \\
	&\leq& f(u_1).
\end{IEEEeqnarray*}
This proves both statements.
\end{proof}
We arrive at the following sufficient condition for the convex program:
\begin{mdframed}
\begin{IEEEeqnarray*}{rLl"l}
L(u_0, z_0^*) &\leq& L(u, z_0^*) & \forall u \in C \\
u_0 \in C, \; G(u_0) &\leq_K& 0, & z_0^* = K^+ \\
\langle z_0^*, G(u_0) \rangle &=& 0
\end{IEEEeqnarray*}
\end{mdframed}
So far we can only claim the condition to be necessary if the Slater condition holds.
\pagebreak
\subsection{Smooth problems}
\begin{theorem} % Thm 1.50
\label{thm:c1e50}
Let $U$, $Z$ be Banach spaces, and $f, G$ be convex and Fréchet-differentiable. Then the 
sufficient conditions are equivalent to the \emph{KKT conditions:}
\begin{IEEEeqnarray*}{rCl}
f'(u_0)(u - u_0) + \langle z_0^*, G'(u_0)(u - u_0) \rangle &\geq& 0 \quad \forall u \in C \\
u_0 \in C, \; G(u_0) &\leq_K& 0, z_0^* \in K^+ \\
\langle z_0^*, G(u_0) \rangle = 0
\end{IEEEeqnarray*}
Specifically, if $C = U$, then the first inequality becomes
\[
	f'(u_0) + z_0^* \circ G'(u_0)^* = 0 \quad \text{in } U^*.
\]
($f'(u_0) h + \langle z_0^*, G'(u_0) h \rangle = 0 \; \forall h \in U$).
\end{theorem}
\begin{proof}
\cref{thm:c1e24}.
\end{proof}
\begin{remark} % Rem 1.51
\label{rem:c1e51}
In the case that $f$ is not convex (but $C$ and $G$ are), the KKT conditions are only necessary for \cref{eq:c1starstar}, and should be used to defined Lagrange multipliers (as only global solutions if existent, can be saddle points).
\end{remark}
\subsection{Construction for smooth problems}
In the case $C = U$, a solution $(u_0, z_0^*)$ of the ``algebraic'' KKT conditions
\begin{IEEEeqnarray*}{rCl}
f'(u) + z^* G'(u) &\geq& 0 \\
G(u) &\leq_K& 0, z^* \in K^+ \\
\langle z^*, G(u) \rangle = 0
\end{IEEEeqnarray*}
is a saddle point of the Lagrange function, that is
\begin{enumerate}[(i)]
\item $u_0$ is a solution to the ``variational'' problem
\begin{IEEEeqnarray*}{u"l}
minimize & f(u) \\
subject to & G(u) \leq_K 0
\end{IEEEeqnarray*}
\item $z_0^*$ is a Lagrange multiplier.
\end{enumerate}
The question whether a solution to the KKT conditions exists, which is obviously important, contains two questions:
\begin{enumerate}[(i)]
\item Does a solution $u_0$ to the variational problem exist?
\item Does a Lagrange multiplier exist?
\end{enumerate}
For (i) we use \cref{thm:c1e14}. For (ii) we use \cref{thm:c1e47} if possible, or a different argument.
\section{Example: Point-wise inequality constraints} % Sec 1.6
\label{sec:c1e6}
\subsection{One-sided box constraints} % (a)
\label{sec:c1e6-a}
Given $u_d \in L_2(0, 1)$, consider the problem
\begin{IEEEeqnarray*}{u"rCl}
minimize & f(u) &\coloneqq& \frac{1}{2} \int_0^1 [u(x) - u_d(x)]^2 \dx = \frac{1}{2} \| u - u_d \|_{L_2}^2 \\
subject to & u(x) &\leq& 0 \quad \text{a.e.}
\end{IEEEeqnarray*}
Setting $U = Z = C = L_2(0, 1)$, $G = I$ (identity), and
\[
	K = L_2(0, 1)_+ = \{ u \in L_2(0, 1) \mid u(x) \geq 0 \quad \text{a.e.} \},
\]
then the task can be written as
\begin{IEEEeqnarray*}{u"l}
minimize & f(u) \\
subject to & G(u) \leq_K 0
\end{IEEEeqnarray*}
As $L_2(0, 1)$ is reflexive and $f$ is coercive and strictly convex, there is exactly one solution $u_0$ (\cref{thm:c1e14}). In fact, it is simply seen to be
\[
	u_0(x) = \min (u_d(x), 0).
\]
As already seen in \cref{ex:c1e40}, the cone $K$ contains no interior points. Therefore, we cannot use \cref{thm:c1e47}.
To show that a Lagrange multiplier nevertheless exists, we observe that
\[
	Z^* = U^* = U = L_2(0, 1).
\]
Hence, the Lagrange function is
\[
	L(u, \mu) = f(u) + \langle \mu, u \rangle_{L_2}.
\]
A Lagrange multiplier $\mu_0$ has to satisfy
\[
	0 = L_u(u_0, \mu_0) = \nabla f(u_0) + \mu = u_0 - u_d + \mu_0.
\]
Thus,
\[
	\mu_0 = u_d - u_0.
\]
\begin{lemmanonumb} % Lemma no number.
In this example, $K^+ = K$.
\end{lemmanonumb}
\begin{proof}
If
\[
	\int_0^1 \mu(x) u(x) \dx \geq 0, \quad \forall u \in K,
\]
but $\{ \mu < 0 \}$ has positive measure, then we obtain a contradiction choosing $u = \mathds{1}_{\{ \mu < 0\}}$.
\end{proof}
Hence $\mu_0 \in K^+$. The complementary slack condition holds, since
\[
	\langle \mu_0, G(u_0) \rangle = \int_0^1 (u_d(x) - u_0(x)) u(x) \dx = 0.
\]
Hence, $\mu_0$ is a Lagrange multiplier. By the considerations of \cref{sec:c1e5}, we could retrieve $(u_0, \mu_0)$ by solving
\begin{IEEEeqnarray*}{l}
u - u_d + \mu = 0 \\
u \leq 0 \; \text{ a.e.}, \; \mu \geq 0 \; \text{a.e.} \\
\langle \mu, u \rangle_{L_2} = 0
\end{IEEEeqnarray*}
\begin{lemmanonumb} % Lemma No numb
For $u \leq 0$ and $\mu \geq 0$, the condition
\[
	\langle \mu, u \rangle_{L_2} = \int_0^1 u(x) \mu(x) \dx = 0
\]
is equivalent to
\[
	\mu(x) u(x) = 0 \quad \text{a.e.}
\]
\end{lemmanonumb}
\underline{Result:}
\begin{IEEEeqnarray*}{l}
u - u_d + \mu = 0 \\
u \leq 0 \; \text{ a.e.}, \; \mu \geq 0 \; \text{a.e.} \\
\mu \cdot u = 0 \; \text{a.e.}
\end{IEEEeqnarray*}
This is called ``formal'' Lagrange technique.
\subsection{Two-sided box constraints in \texorpdfstring{$L_2(0, 1)$}{L2(0,1)}} % (b)
\label{sec:c1e6-b}
Some $f$ as before, but now
\begin{IEEEeqnarray*}{u"l}
minimize & f(u) \\
subject to & -1 \leq u(x) \leq 1 \; \Leftrightarrow \; u(x) - 1 \leq 0, \, u(x) - 1 \leq 0 \; \text{ a.e.}
\end{IEEEeqnarray*}
This time $C = U = L_2(0, 1)$, $Z = L_2(0, 1) \times L_2(0, 1)$, $K = L_2(0, 1)_+ \times L_2(0, 1)_+$ and
\[
	G : U \to Z \; : \;  u \mapsto \begin{pmatrix}
	u - 1 \\ - u - 1
	\end{pmatrix}.
\]
Then $G$ is convex, and the problems reads
\begin{IEEEeqnarray*}{u"l}
minimize & f(u) \\
subject to & G(u) \leq_K 0.
\end{IEEEeqnarray*}
Again, a unique solution $u_0$ exists by \cref{thm:c1e14}, but $\Int K = \emptyset$, and \cref{thm:c1e47} is thus not applicable.
We apply the ``formal'' Lagrange technique: \\
The Lagrange function is
\[
	L(u, \mu^a, \mu^b) = \frac{1}{2} \| u - u_d \|_{L_2}^2 + \langle - u - 1, \mu^a \rangle_{L_2} + \langle u - 1, \mu^b \rangle_{L_2}.
\]
We need
\[
	L_u(u_0, \mu_0^a, \mu_0^b) = u_0 - u_d - \mu_0^a + \mu_0^b = 0,
\]
with $(\mu_0^a, \mu_0^b) \in K^+$, that is $\mu_0^a \geq 0$, $\mu_0^b \geq 0$. Hence, we take
\[
	\mu_0^a = (u_0 - u_d)_+, \quad \mu_0^b(u_0 - u_d)_-.
\]
As before, one verifies 
\[
	\langle \mu_0^a, u_0 - 1 \rangle + \langle \mu_0^b, u_0 - 1 \rangle = 0.
\]
Altogether, this shows that $\mu_0^a$ and $\mu_0^b$ are Lagrange multipliers.
\emph{KKT:}
\begin{IEEEeqnarray*}{l}
u - u_d + \mu^a + \mu^b = 0 \\
-1 \leq u \leq 1 \; \text{ a.e.}, \; \mu^a \geq 0, \; \mu^b \geq 0 \\
\mu^a(-u - 1) = 0, \; \mu^b(u-1) = 0 \; \text{ a.e.}
\end{IEEEeqnarray*}
Note that if $u_d \in L_\infty(0, 1)$, our considerations revealed that $\mu^a, \mu^b \in L_\infty(0, 1)$, which would be not obvious, if we used the following ``a-priori'' trick.
\subsection{Two-sided box constraints in \texorpdfstring{$L_\infty(0, 1)$}{L∞(0, 1)}} % (c)
\label{sec:c1e6-c}
We consider the same task as in \cref{sec:c1e6-b}, but now $U = L_\infty(0, 1)$, $Z = L_\infty(0, 1) \times L_\infty(0, 1)$, $K = L_\infty(0, 1)_+ \times L_\infty(0, 1))+$.
This is no restriction after convincing oneself that the solution in \cref{sec:c1e6-b} must be bounded almost everywhere.
In particular, a solution exists in $L_\infty(0, 1)$!
This time, we \emph{can} use \cref{thm:c1e47} to prove existence of Lagrange multipliers, since
\[
	-G(0) = \begin{pmatrix}
	1 \\ 1
	\end{pmatrix} \in \Int K,
\]
i.e.\ the Slater condition is satisfied.
The price we pay is that we only know $\mu_0^a, \mu_0^b \in (L_\infty(0, 1))^* \cap K^+$. This is not very helpful, since $(L_\infty(0, 1))^*$ contains linear functionals which are not necessarily measures.

The explicit construction in the larger space $L_2$ is superior!
\end{document}