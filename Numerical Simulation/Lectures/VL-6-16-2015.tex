\documentclass[../skript.tex]{subfiles}

\begin{document}
\section{Variational formulation of parabolic PDEs} % 3.2
\label{sec:c3e2}
The state equation could be of the form
\begin{IEEEeqnarray*}{rCl"l}
	y_t(x, t) + (\mathcal{A} y(\cdot, t))(x) &=& b(t, x) & \text{on } Q = \Omega \times [0, T] \\
	y(x, t) &=& y_0(x) & \text{in } \Omega
\end{IEEEeqnarray*}
Here $\mathcal{A}$ is an elliptic differential operator on a suitable function space (boundary conditions and etc.). For instance, $y(\cdot, t) \in H^1(\Omega)$ for almost every $t$.

How to deal with the time derivative? \\
Tread the problem as a function-space valued differential equation:
\begin{IEEEeqnarray*}{rCl"l}
	y'(t) + \mathcal{A} y(t) &=& b(t) & \in H \\
	y(0) &=& y_0 & \in H
\end{IEEEeqnarray*}
\underline{Aim:} Variational formulation
\begin{IEEEeqnarray*}{rCl"l}
\frac{\mathrm{d}}{\mathrm{d} t} \langle y(t), v \rangle_H + a(y(t), v) &=& \langle b(t), v \rangle_V & \forall v \in V \\
y(0) &=& y_0 & \in H
\end{IEEEeqnarray*}
Thus $y \in W^{1, 2}(0, T; V)$ with the Gelfand triple $V \subseteq H \subseteq V^*$.
\paragraph{Bochner spaces}
\addtocounter{dummythm}{1} % 7 and 8 have not been in the lecture
% Professor noticed the mistake partially and retroactively renamed all of the following things
\begin{definition}[classes of vector-valued functions] % Def 3.9 - renamed to Def 3.8
\label{def:c3e8}
Let $X$ be a real Banach space and $0 < T < \infty$. A function $y : [0, T] \to X$ is called
\begin{enumerate}[(i)]
\item \emph{continuous}, if $\tau \to t$ implies $y(\tau) \to y(t)$. The space of continuous functions $y : [0, T] \to X$ is a Banach space denoted by $C([0, T], X)$ using the norm
\[
	\| y \|_{C([0, T], X)} = \sup_{t \in [0, T]} \| y \|_X.
\]
\item \emph{step-function}, if it takes only finitely many values $z_1, \ldots, z_m \in X$ and \\ $y^{-1}(z; \cdot) \subseteq [0, T]$ is measurable for $i = 1, \ldots, m$.
\item \emph{measurable}, if there exists a sequence $(y_n)$ of step-functions, such that
\[
	y(t) = \lim_{n \to \infty} y_n(t) \;\; \text{\ac{ae}}
\]
\end{enumerate}
\end{definition}
\begin{definition}[space \texorpdfstring{$L^p(0, T; X)$}{Lp(0, T; X)}] % Def 3.10 - renamed to Def 3.9
\label{def:c3e9}
Let $X$ be a real Banach space. The \emph{space $L^p(0, T; X)$} with $1 \leq p < \infty$ consists out of all equivalence classes of measurable functions $y : [0, T] \to X$ for which
\[
	\| y \|_p \coloneqq \left( \int_0^T \| y(t) \|_X^p \dt \right)^{\frac{1}{p}} < \infty.
\]
The space $L^\infty(0, T; X)$ consists of all measurable functions $y : [0, T] \to X$ for which
\[
	\| y \|_{\infty} \coloneqq \esssup_{t \in [0, T]} \| y(t) \|_X < \infty.
\]
\end{definition}
\begin{proposition} % Prop 3.11 - renamed Prop 3.10
\label{thm:c3e10}
Let $X$ and $Y$ be real Banach spaces, and $1 \leq p < \infty$.
\begin{enumerate}[(i)]
\item\label[equation]{thm:c3e11-i} The space $L^p(0, T; X)$ with the norm $\| y \|_p$ is a Banach space. The step-functions are dense in that space.
\item\label[equation]{thm:c3e11-ii} The space $C([0, T]; X)$ is dense in $L^p(0, T; X)$, and the embedding
\[
	C([0, T]; X) \subseteq L^p(0, T; X)
\]
is continuous.
\item\label[equation]{thm:c3e11-iii} If $X$ is a Hilbert space, then $L^2(0, T; X)$ is also a Hilbert space with the scalar product
\[
	\langle y, v \rangle = \int_0^T \langle y(t), v(t) \rangle_X \dt.
\]
\item\label[equation]{thm:c3e11-iv} If $X$ is a separable space, then $L^p(0, T; X)$ is separable.
\item\label[equation]{thm:c3e11-v} If the embedding $X \subseteq Y$ is continuous, then the embeddings
\[
	L^r(0, T; X) \subseteq L^q(0, T; Y), \quad 1 \leq q \leq r \leq \infty
\]
are continuous.
\end{enumerate}
\end{proposition}
\begin{proof}
Exercise.
\end{proof}
\paragraph{Bochner integral}
For step-functions $y = \sum_{i=1}^m z_1 \mathds{1}_{M_i}$ we define
\[
	\int_0^T y(t) \dt \coloneqq \sum_{i=1}^m z_i |M_i| \quad \in X.
\]
For measurable functions $y(t) = \lim_{n \to \infty} y_n(t)$ \ac{ae}, where $(y_n)$ are step-functions we define
\[
	\int_0^T y(t) \dt \coloneqq \lim_{n \to \infty} \int_0^T y_n(t) \dt.
\]
One can show that this is independent of the choice of $(y_n)$.
\paragraph{Dual spaces of \texorpdfstring{$L^p(0, T; X)$}{Lp(0, T; X)}}
\begin{proposition} % Prop 3.12 - renamed to Prop 3.11
\label{thm:c3e11}
Let $X$ be a real reflexive and separable Banach space, and let $1 < p < \infty$, $\frac{1}{p} + \frac{1}{q} = 1$. Then the spaces
\[
	L^p(0, T; X)^* \quad \text{and} \quad L^q(0, T; X^*)
\]
are isometrically isomorphic. These spaces are reflexive and separable.
\end{proposition}
\begin{proof}
The isometric embedding $L^q(0, T; X^*) \subseteq L^p(0, T; X)^*$ is ``easy''. The reverse relation is nontrivial, see \cite[8.20.5]{Edwards}. % Edwards ``Functional Analysis'', 8.20.5.
Reflexivity is then clear. Separability follows from \cref{thm:c3e11}, \labelcref{thm:c3e11-iv}.
\end{proof}
\underline{Identification:} $L^p(0, T; X)^* = L^q(0, T; X^*)$.
In particular $L^2(0, T; X)^* = L^2(0, T; X^*)$.
\paragraph{Gelfand (Evolution) triplets}
\begin{definition} % Def 3.13 - renamed to Def 3.12
\label{def:c3e12}
We understand a \emph{Gelfand triple} ``$V \subseteq H \subseteq V^*$'' to be the following
\begin{enumerate}[(i)]
\item $V$ is a real, separable and reflexive Banach space.
\item $H$ is a real, separable Hilbert space.
\item The embedding $V \subseteq H$ is continuous, and $V$ is dense in $H$.
\end{enumerate}
\end{definition}
Note that the embedding $H = H^* \subseteq V^*$ is then injective, continuous and dense (Exercise). In this sense, we can identify $V \subseteq H \subseteq V^*$ and
\begin{equation}
\opttag{$\star$}
\label{eq:c3e2-star}
\langle \underbrace{h}_{\mathclap{\in V^*}}, v \rangle_V = \langle \underbrace{h}_{\mathclap{\in H}}, v \rangle_H \quad \forall h \in H, v \in V.
\end{equation}
\paragraph{Generalized derivatives}
\begin{definition} % Def 3.14 - renamed to Def 3.13
\label{def:c3e13}
Let $y \in L^1(0, T; X)$ and $w \in L^1(0, T; Z)$. Then $w$ is called the \emph{$n$th generalized derivative of $y$} on $(0, T)$ if
\[
	\int_0^T \underbrace{\varphi^{(n)}(t) y(t)}_{\in L^1 !} \dt = (-1)^n \int_0^T \underbrace{ \varphi(t) w(t) }_{\in L^1 !} \dt \quad \forall \varphi \in C_0^\infty(0, T).
\]
Notation: $w = y^{(n)}$.
\end{definition}
One can show uniqueness using
\[
	\int_0^T \varphi(t) w(t) \dt = 0 \quad \forall \varphi \in C_0^\infty.
\]
implying $w = 0$ (fundamental Lemma).
\begin{proposition}[Existence] % Prop 3.15 - renamed to Prop 3.14
\label{thm:c3e14}
Let ``$V \subseteq H \subseteq V^*$'' be a Gelfand triple, and let $1 \leq p, q \leq \infty$, $0 < T < \infty$.
Then for $y \in L^p(0, T; V)$ there exists a generalized derivative $y^{(n)} \in L^q(0, T; V^*)$ if and only if there exists $w \in L^q(0, T; V^*)$ such that
\[
	\int_0^T \langle y(t), v \rangle_H \varphi^{(n)}(t) \dt = (-1)^n \int_0^T \langle w(t), v \rangle_V \varphi(t) \dt \quad \forall v \in V, \; \forall \varphi \in C_0^\infty(0, T).
\]
In this case $y^{(n)} = w$ and
\[
	\frac{\mathrm{d}^n}{\mathrm{d} t^n} \langle y(t), v \rangle_H = \langle y^{(n)}(t), v \rangle_V \quad \forall v \in V \; \text{\ac{ae} on } [0, T],
\]
where $\frac{\mathrm{d}^n}{\mathrm{d} t^n}$ denotes the $n$th weak derivative for real functions on $(0, T)$.
\end{proposition}
\begin{proof}
The definition of the generalized derivative is equivalent to 
\[
	\left\langle \int_0^T \varphi^{(n)}(t) y(t) \dt, v \right\rangle_V = (-1)^n \left\langle \int_0^T \varphi(t) w(t) \dt, v \right\rangle_V \quad \forall v \in V.
\]
It is an exercise to prove that this equivalent to
\[
	\int_0^T \langle \varphi^{(n)} (t) y(t), v \rangle_V \dt = (-1)^n \int_0^T \langle \varphi(t) w(t), v \rangle_V \dt.
\]
The assertion follows from \cref{eq:c3e2-star}.
\end{proof}
\pagebreak
\paragraph{The Sobolev space \texorpdfstring{$W^1_p(0, T; V, H)$}{W1p(0, T; V, H)}}
\begin{propdef} % Prop & Def 3.16 - renumbered to 3.15
\label{thm:c3e15}
Let ``$V \subseteq H \subseteq V^*$'' be a Gelfand triple and let $1 < p < \infty$, $\frac{1}{p} + \frac{1}{q} = 1$, $0 < T < \infty$. The following hold:
\begin{enumerate}[(i)]
\item\label[equation]{thm:c3e15-i} The set of all $y \in L^p(0, T; V)$ that have a generalized derivative
\[
	y' \in L^q(0, T; V^*) = \left( L^p(0, T; V) \right)^*
\]
forms a real Banach space with the norm
\[
	\| y \|_{W^{1,p}} = \| y \|_{L^p(0, T; V)} + \| y' \|_{L^q(0, T; V)}.
\]
This space is denoted by $W^{1, p}(0, T; V, H)$.
\item\label[equation]{thm:c3e15-ii} The embedding
\[
	W^{1, p}(0, T; V, H) \subseteq C([0, T]; H)
\]
is continuous.
\item\label[equation]{thm:c3e15-iii} For all $y, v \in W^{1, p}(0, T; V, H)$ and arbitrary $t, s, 0 \leq s \leq t \leq T$ it holds
\[
	\langle y(t), v(t) \rangle_H - \langle y(s), v(s) \rangle_H = \int_s^t \langle y'(\tau), v(\tau) \rangle_V + \langle v'(\tau), y(\tau) \rangle_V \: \mathrm{d} \tau
\]
(\emph{integration by parts}). The point evaluations on the left side make sense by \labelcref{thm:c3e15-ii}.
\end{enumerate}
\end{propdef}
\underline{Special case:}
\[
	W(0, T) \coloneqq W^{1, 2}(0, T; V, H) = \left\{ y \in L^2(0, T; V) \midcolon y' \in L^2(0, T; V^*) \right\}.
\]
\end{document}