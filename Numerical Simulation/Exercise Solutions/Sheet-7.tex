%Corrected!

\documentclass{article}
\usepackage[left=3cm,right=3cm,top=0cm,bottom=2cm]{geometry} % page settings
\usepackage{amsmath} % provides many mathematical environments & tools
\usepackage{IEEEtrantools}
\usepackage{amssymb}
\usepackage{mathtools}

\setlength{\parindent}{0mm}

\begin{document}

\textbf{Exercise 1 (Pointwise inequality constraints)}\newline\noindent
Let $\Omega$ bounded Lipschitz domain and $u_a,u_b\in L_2(\Omega)$ s.t. $u_a\leq u_b$ almost everywhere. Then
\[
	U_{ad}\coloneqq\{u\in L_2(\Omega)|\,u_a\leq u\leq u_b\,a.e.\}
\]
is a \emph{convex}, \emph{bounded} and \emph{closed} subset of $L_2(\Omega)$.\newline\newline\noindent
\textbf{Proof. a) Convexity of $U_{ad}$: } Let $u_1,u_2\in U_{ad}$ and $\lambda\in[0,1]$. Then we have
\[
	\lambda u_1+(1-\lambda)u_2 	\begin{cases}
									\geq \lambda+(1-\lambda)\min{\{u_1,u_2\}} = \min{\{u_1,u_2\}} \geq u_a &a.e.\\
									\leq \lambda+(1-\lambda)\max{\{u_1,u_2\}} = \max{\{u_1,u_2\}} \leq u_b &a.e.
								\end{cases}
\]
and therefore $\lambda u_1 + (1-\lambda)u_2\in U_{ad}$. This shows convexity.\newline\newline\noindent
\textbf{b) Boundedness of $U_{ad}$: }
We need a distance function on $L_2(\Omega)$. As this space is normed, we already have the induced distance function
\[
	d(f,g):L_2(\Omega)\times L_2(\Omega)\to\mathbb{R}
\]
given by
\[
	d(f,g) \coloneqq \|f-g\|_{L_2(\Omega)}.
\]
As all $u\in U_{ad}$ satisfy $u_a\leq u\leq u_b$ almost everywhere, we see that $U_{ad}$ is bounded, because it holds that
\[
	|f-g| \leq |u_b-u_a|\quad\,a.e.
\]
and therefore
\[
	\sup_{f,g\in U_{ad}}d(f,g) \leq d(u_a,u_b) = \|u_a-u_b\|_{L_2(\Omega)}\overset{Minkowski}\leq \underbrace{\|u_a\|_{L_2(\Omega)}}_{<\infty}+\underbrace{\|u_b\|_{L_2(\Omega)}}_{<\infty} <\infty
\]
and this gives us the boundedness(?).\newline\newline\noindent
\textbf{c) Closedness of $U_{ad}$: } $L^2$ is a Hilbert-space (metric), so sequentially closedness is enough. Let therefore $(u_n)_n\subset U_{ad}$ a Cauchy-sequence. Let $u\in L_2(\Omega)$ be the limit w.r.t the $L^2$-norm.\newline\noindent
In this case, there is a subsequence $\left(u_{n_k}\right)_k$ s.t. $u_{n_k}\to u$ almost everywhere. That's why $u_a\leq u\leq u_b$ almost everywhere because we have pointwise convergence. Therefore $u\in U_{ad}$ and $U_{ad}$ is closed. \newline\newline\noindent

\textbf{Exercise 2 (Convergence of the Uzawa Algorithm)}\newline\noindent
Let $A\in\mathbb{R}^{n\times n}$ symmetric, positive definite, $b\in\mathbb{R}^n,d\in\mathbb{R}^m$ and $C\in\mathbb{R}^{m\times n}$ with rank $m\leq n$. Consider the system
\begin{equation}\label{system1}
	\begin{bmatrix}A&C^T\\C&0\end{bmatrix}\begin{bmatrix}u\\\lambda\end{bmatrix} = \begin{bmatrix}b\\d\end{bmatrix}
\end{equation}

If $\varepsilon=0$ and $\alpha<\frac{2}{\|S\|}$, where $S=CA^{-1}C^T$ is the Schur complement, then the \emph{Uzawa Algorithm} converges to the true solution.\newline\newline\noindent
\textbf{Proof. } The first equation of the system above reads
\[
	Au + C^T\lambda = b \Rightarrow u = A^{-1}b-A^{-1}C^T\lambda
\]
and by inserting this into the second equation we get
\[
	d = Cu = CA^{-1}b - CA^{-1}C^T\lambda.
\]
So we have a system $S\lambda = h$ where
\begin{IEEEeqnarray*}{rCl}
	S &\coloneqq& CA^{-1}C^T\\
	h &\coloneqq& CA^{-1}b-d.
\end{IEEEeqnarray*}

Let $q^{(k)} \coloneqq d-Cu^{(k)}$ the \emph{defect} and $(u,\lambda)$ the exact solution of the saddle point problem. Then it holds that
\begin{IEEEeqnarray*}{rCl}
	q^{(k)} &=& d-CA^{-1}(b-C^T\lambda^{k-1})\\
	&=& CA^{-1}C^T\lambda^{(k-1)} - \underbrace{(CA^{-1}b-d)}_{=h=S\lambda=CA^{-1}b-d}\\
	&=& S(\lambda^{(k-1)}-\lambda).
\end{IEEEeqnarray*}
Moreover:
\begin{IEEEeqnarray*}{rCl}
	\lambda^{(k)}-\lambda &=& \lambda^{(k)}-\lambda^{(k-1)}+\lambda^{(k-1)}-\lambda \\
	&=& \alpha \underbrace{Cu^{(k)}-d}_{=-q^{(k)}=-S(\lambda^{(k-1)}-\lambda)} + \lambda^{(k-1)}-\lambda\\
	&=& (I-\alpha S)(\lambda^{(k-1)}-\lambda)
\end{IEEEeqnarray*}
Using this we have
\begin{IEEEeqnarray*}{rCl}
	\|\lambda^{(k)}-\lambda\| &\leq& \underbrace{\|I-\alpha S\|}_{\eqqcolon\varrho} \|\lambda^{(k-1)}-\lambda\|\\
	&\leq& ...\\
	&\leq& \varrho^k \|\lambda^{(0)}-\lambda\|
\end{IEEEeqnarray*}
By hypothesis we have $\varrho < 1$ so we have convergence. In fact, Uzawa's algorithm is nothing else than Richardson applied to the Schur complement $S$.\newline\newline\noindent

\textbf{Exercise 3 (Optimal stationary heat equation with boundary control)}\newline\noindent

...\newline\noindent
For the adjoint state consider the Lagrange function of the problem
\[
	\nabla_y L(y,u,\underbrace{p}_{\text{multiplier}}) \coloneqq \|y-y_\Omega\|_{L^2(\Omega)} + \frac{\lambda}{2}\|u\|_{L^2(\Omega)}^2 - a(y,p) + F_u(p)
\]
and calculate the derivate
\begin{IEEEeqnarray*}{rCll}
	\nabla_y L(\xi) &=& 2\langle y-y_d,\xi\rangle_{L^2(\Omega)} - a(\xi,p) &\overset{!}= 0,\quad\forall \xi\in H^1(\Omega)\\
	\nabla_p L(\beta) &=& -a(y,\beta) + F_u(\beta) &\overset{!}= 0,\quad\forall \beta\in H^1(\Omega)\\
	\nabla_u L(\gamma) &=& \lambda\langue u,\gamma\rangle_{L^2(\Omega)} + F_\gamma(p) &\geq 0,\quad\forall \gamma\in U_{ad}-u\\
\end{IEEEeqnarray*}
And this gives the variational inequality
\[
	\langle \alpha p|_\Gamma + \lambda u, \tilde{u} - u\rangle_{L^2(\Omega)}\geq 0,\quad\forall \tilde{u}\in U_{ad}
\]
in weak form we get:
\begin{IEEEeqnarray*}{rCl}
	-\Delta p &=& 2y - y_\Omega,\quad\text{in }\Omega\\ 
	\partial_\nu p + \alpha p &=& 0,\quad\text{on }\Gamma
\end{IEEEeqnarray*}



\end{document}